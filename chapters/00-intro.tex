Second harmonic generation (SHG) is a powerful spectroscopic tool for studying
the optical properties of surfaces and interfaces since it has the advantage
of being surface sensitive. Within the dipole approximation, inversion
symmetry forbids SHG from the bulk of controsymmetric materials. SHG is
allowed at the surface of these materials where the inversion symmetry is
broken and should necessarily come from the localized surface region. SHG
allows the study of the structural atomic arrangement and phase transitions of
clean and adsorbate covered surfaces. Since it is also an optical probe it can
be used out of UHV conditions and is non-invasive and non-destructive.
Experimentally, new tunable high intensity laser systems have made SHG
spectroscopy readily accessible and applicable to a wide range of
systems.\cite{downerSIA01,lupkeSSR99}

However, theoretical development of the field is still an ongoing subject of
research. Some recent advances for the cases of semiconducting and metallic
systems have appeared in the literature, where the use of theoretical models
with experimental results have yielded correct physical interpretations for
observed SHG spectra. \cite{downerSIA01, mendozaPRB01, limPRB00,
gavrilenkoTSF00, mendozaPRB99, mendozaPRL98, mendozaPRB96b, mendozaPRB97,
guyotPRB90}

In a previous article\cite{mendozaEPI04} we reviewed some of the recent
results in the study of SHG using the velocity gauge for the coupling between
the electromagnetic field and the electron. In particular, we demonstrated a
method to systematically analyze the different contributions to the observed
SHG peaks.\cite{arzatePRB01} This approach consists of separating the
different contributions to the nonlinear susceptibility according to 1$\omega$
and 2$\omega$ transitions, and the surface or bulk nature of the states among
which the transitions take place.

To compliment those results, in this article we review the calculation of the
nonlinear susceptibility using the longitudinal gauge. We show that it is
posible to clearly obtain the ``layer-by-layer'' contribution for a slab
scheme used for surface calculations.
