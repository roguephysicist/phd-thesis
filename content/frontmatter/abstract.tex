%!TEX root = ../../main.tex
\begin{abstract}
In this thesis formulate a theoretical approach of surface second-harmonic generation from
semiconductor surfaces based on the length gauge and the electron density
operator. Within the independent particle approximation the nonlinear
second-order surface susceptibility tensor
$\chi^{\mathrm{a}\mathrm{b}\mathrm{c}}(-2\omega;\omega,\omega)$ is calculated,
including in one unique formulation (i) the scissors correction, needed to have
the correct value of the energy band gap, (ii) the contribution of the nonlocal
part of the pseudopotentials, routinely  used in \textit{ab initio} band
structure calculations, and (iii) the derivation for the inclusion of the cut
function, used to extract the surface response. The first two contributions are
described by spatially nonlocal quantum mechanical operators and are fully taken
into account in the present formulation. As a test-case of the approach we
calculate $\chi^{xxx}(-2\omega;\omega,\omega)$ for the clean Si(001)$2\times 1$
reconstructed surface. The effects of the scissors correction and of the
nonlocal part of the pseudopotentials are discussed in surface nonlinear optics.
The scissors correction shifts the spectrum to higher energies though the
shifting is not rigid and mixes the $1\omega$ and $2\omega$ resonances, and has
a strong influence in the line-shape. The effects of the nonlocal part of the
pseudopotentials keeps the same line-shape of $|\chi^{xxx}_{2\times
1}(-2\omega;\omega,\omega)|$, but reduces its value by 15-20\%. Therefore, the
inclusion of the three aforementioned contributions is very important and makes
our scheme unprecedented and opens the possibility to study surface
second-harmonic generation with more versatility and providing more accurate
results.

We carry out an improved \emph{ab initio} calculation of surface second-harmonic
generation from the Si(111)(1$\times$1):H surface. This calculation includes
three new features in one unique formulation: (i) the scissors correction, (ii)
the contribution of the nonlocal part of the pseudopotentials, and (iii) the
inclusion of a cut function to extract the surface response, all within the
independent particle approximation. We apply these improvements on the
Si(111)(1$\times$1):H surface and compare with various experimental spectra from
several different sources. We also revisit the three layer model for the SSHG
yield and demonstrate that it provides more accurate results over several, more
common, two layer models. We demonstrate the importance of using properly
relaxed coordinates for the theoretical calculations. We conclude that this new
approach to the calculation of the second-harmonic spectra is versatile and
accurate within this level of approximation. This well-characterized surface
offers an excellent platform for comparison with theory, and allows us to offer
this study as an efficient benchmark for this type of calculation.
\end{abstract}
