%!TEX root = ../../main.tex
\begin{abstract}
In this thesis I formulate a theoretical approach of surface second-harmonic
generation from semiconductor surfaces based on the length gauge and the
electron density operator. Within the independent particle approximation the
surface nonlinear second-order surface susceptibility tensor
$\chi^{\mathrm{abc}}_{\mathrm{surface}}(-2\omega;\omega,\omega)$ is calculated,
including in one unique formulation (i) the scissors correction, needed to have
the correct value of the energy band gap, (ii) the contribution of the nonlocal
part of the pseudopotentials, and (iii) the derivation for the inclusion of the
cut function, used to extract the surface response. The first two contributions
are described by spatially nonlocal quantum mechanical operators and are fully
taken into account in the present formulation. I also revisit the three layer
model for the SSHG yield and demonstrate that it provides more accurate results
over several, more common, two layer models. This entire framework is
implemented in the TINIBA software suite, which I helped develop over the course
of this doctoral project. I apply this framework for the clean Si(001)$2\times
1$ and Si(111)(1$\times$1):H surfaces, and compare with various experimental
spectra from several different sources. I conclude that this new approach to the
calculation of the second-harmonic spectra is versatile and accurate within this
level of approximation. These surfaces provide an excellent platform for
comparison with theory, and allows us to offer this study as an efficient
benchmark for this type of calculation.
\end{abstract}
