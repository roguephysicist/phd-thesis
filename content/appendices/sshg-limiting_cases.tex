%!TEX root = ../../main.tex
\chapter{Some limiting cases of interest}\label{app:limiting_cases}
\partialtoc

In this section, we derive the expresions for $\mathcal{R}_{pP}$ for different
limiting cases. We evaluate $\mathcal{P}(2\omega)$ and the fundamental fields in
different regions. It is worth noting that the first case, the three layer
model, can be reduced to any of the other cases by simply considering where we
want to evaluate the $1\omega$ and $2\omega$ terms.


\section{The two layer model}

In order to reduce above result to that of Ref. \cite{mizrahiJOSA88} and
\cite{sipePRB87}, we now consider that $\mathcal{P}(2\omega)$ is evaluated in
the vacuum region, while the fundamental fields are evaluated in the bulk
region. To do this, we take the $2\omega$ radiations factors for vacuum by
taking $\ell=v$, thus $\epsilon_{\ell}(2\omega)=1$, $T^{\ell v}_{p}=1$, $T^{\ell
b}_{p}=T^{vb}_{p}$, and the fundamental field inside medium $b$ by taking
$\ell=b$, thus $\epsilon_{\ell}(\omega)=\epsilon_{b}(\omega)$,
$t^{v\ell}_{p}=t^{vb}_{p}$, and $t^{\ell b}_{p}=1$. With these choices
\begin{equation*}\label{m800}
\mathbf{e}^{\,2\omega}_{v}\cdot\boldsymbol{\chi}:
\mathbf{e}^\omega_{b}\mathbf{e}^\omega_{b}
\equiv\Gamma^{vb}_{pP}\,r^{vb}_{pP}
,
\end{equation*}
where,
\begin{equation*}\label{m82}
\begin{split}
r^{vb}_{pP}
&= \epsilon_{b}(2\omega)\sin\theta_{0}
\Big(
\sin^2\theta_{0}\chi_{zzz} + k^{2}_{b}\chi_{zxx}
\Big)\\
&- k_{b}K_{b}
\Big(
2\sin\theta_{0}\chi_{xxz} + k_{b}\chi_{xxx}\cos(3\phi) 
\Big) 
,
\end{split}
\end{equation*}
and 
\begin{equation*}\label{m78}
\Gamma^{vb}_{pP}
= \frac{T^{v b}_{p}(t^{vb}_{p})^2}
       {\epsilon_{b}(\omega)\sqrt{\epsilon_{b}(2\omega)}}.
\end{equation*}


\section{Taking \texorpdfstring{$\mathcal{P}(2\omega)$}{P(2w)} and the
fundamental fields in the bulk}

To consider the $2\omega$ fields in the bulk, we start with Eq. \eqref{r9} but
substitute $\ell\rightarrow b$, thus
\begin{equation*}
\mathbf{H}_{b}
= \hat{\mathbf{s}}\,T_s^{b v}\left(1+R_{s}^{b b}\right)\hat{\mathbf{s}}
+ \hat{\mathbf{P}}_{v+}T_{p}^{b v}
\left(
\hat{\mathbf{P}}_{b+} + R_{p}^{b b}\hat{\mathbf{P}}_{b-}
\right).
\end{equation*}
$R_{p}^{b b}$ and $R_{s}^{b b}$ are zero, so we are left with
\begin{equation*}
\begin{split}
\mathbf{H}_{b}
&= \hat{\mathbf{s}}\,T_s^{b v}\hat{\mathbf{s}}
 + \hat{\mathbf{P}}_{v+}T_{p}^{b v}\hat{\mathbf{P}}_{b+}\\
&= \frac{K_{b}}{K_{v}}\left(\hat{\mathbf{s}}\,T_s^{vb}\hat{\mathbf{s}}
 + \hat{\mathbf{P}}_{v+}T_{p}^{vb}\hat{\mathbf{P}}_{b+}\right)\\
&= \frac{K_{b}}{K_{v}}
   \left[
   \hat{\mathbf{s}}\,T_s^{vb}\hat{\mathbf{s}}
 + \hat{\mathbf{P}}_{v+}
   \frac{T_{p}^{vb}}{\sqrt{\epsilon_{b}(2\omega)}}
   (\sin\theta_{0}\hat{\mathbf{z}}
 - K_{b}\cos\phi\hat{\mathbf{x}} 
 - K_{b}\sin\phi\hat{\mathbf{y}})
   \right],
\end{split}
\end{equation*}
and we define
\begin{equation*}
\mathbf{e}^{\,2\omega}_{b}
= \frac{K_{b}}{K_{v}}\,\hat{\mathbf{e}}^{\mathrm{out}}\cdot
\left[
   \hat{\mathbf{s}}\,T_s^{vb}\hat{\mathbf{s}}
 + \hat{\mathbf{P}}_{v+}
   \frac{T_{p}^{vb}}{\sqrt{\epsilon_{b}(2\omega)}}
   (\sin\theta_{0}\hat{\mathbf{z}}
 - K_{b}\cos\phi\hat{\mathbf{x}} 
 - K_{b}\sin\phi\hat{\mathbf{y}})
   \right].
\end{equation*}
For $\mathcal{R}_{pP}$, we require
$\hat{\mathbf{e}}^{\mathrm{out}}=\hat{\mathbf{P}}_{v+}$, so we have that
\begin{equation*}
\mathbf{e}^{\,2\omega}_{b}
= \frac{K_{b}}{K_{v}}
  \frac{T_{p}^{vb}}{\sqrt{\epsilon_{b}(2\omega)}}
  (\sin\theta_{0}\hat{\mathbf{z}}
- K_{b}\cos\phi\hat{\mathbf{x}} 
- K_{b}\sin\phi\hat{\mathbf{y}}).
\end{equation*}

The $1\omega$ fields will still be evaluated inside the bulk, so we have
\begin{equation*}
\mathbf{e}^{\omega}_{b}
= \left[
\hat{\mathbf{s}}t_{s}^{vb}\hat{\mathbf{s}}
+ \frac{t^{vb}_{p}}{\sqrt{\epsilon_{b}(\omega)}}
\left(
  \sin\theta_{0}\hat{\mathbf{z}}
+ k_{b}\cos\phi\hat{\mathbf{x}}
+ k_{b}\sin\phi\hat{\mathbf{y}}
\right) 
\hat{\mathbf{p}}_{v-}
\right]
\cdot\hat{\mathbf{e}}^{\mathrm{in}},  
\end{equation*}
and for our particular case of
$\hat{\mathbf{e}}^{\mathrm{in}}=\hat{\mathbf{p}}_{v-}$,
\begin{equation*}
\mathbf{e}^{\omega}_{b}
= \frac{t^{vb}_{p}}{\sqrt{\epsilon_{b}(\omega)}}
\left(
  \sin\theta_{0}\hat{\mathbf{z}}
+ k_{b}\cos\phi\hat{\mathbf{x}}
+ k_{b}\sin\phi\hat{\mathbf{y}}
\right),
\end{equation*}
and
\begin{equation*}
\begin{split}
\mathbf{e}^{\omega}_{b}\mathbf{e}^{\omega}_{b}
&= \frac{\left(t^{vb}_{p}\right)^{2}}{\epsilon_{b}(\omega)}
\left(
  \sin\theta_{0}\hat{\mathbf{z}}
+ k_{b}\cos\phi\hat{\mathbf{x}}
+ k_{b}\sin\phi\hat{\mathbf{y}}
\right)^{2}\\
&= \frac{\left(t^{vb}_{p}\right)^{2}}{\epsilon_{b}(\omega)}
\big(
  \sin^{2}\theta_{0}\hat{\mathbf{z}}\hat{\mathbf{z}}
+ k^{2}_{b}\cos^{2}\phi\hat{\mathbf{x}}\hat{\mathbf{x}}
+ k^{2}_{b}\sin^{2}\phi\hat{\mathbf{y}}\hat{\mathbf{y}}\\
&\qquad\qquad
+ 2k_{b}\sin\theta_{0}\cos\phi\hat{\mathbf{z}}\hat{\mathbf{x}}
+ 2k_{b}\sin\theta_{0}\sin\phi\hat{\mathbf{z}}\hat{\mathbf{y}}
+ 2k^{2}_{b}\sin\phi\cos\phi\hat{\mathbf{x}}\hat{\mathbf{y}}
\big)
\end{split}
\end{equation*}

So lastly, we have that
\begin{equation*}
\begin{split}
\mathbf{e}^{\,2\omega}_{b}\cdot
\boldsymbol{\chi}:\mathbf{e}^{\omega}_{b}\mathbf{e}^{\omega}_{b} =
\frac{K_{b}}{K_{v}}
\frac{T_{p}^{vb}\left(t^{vb}_{p}\right)^{2}}
     {\epsilon_{b}(\omega)\sqrt{\epsilon_{b}(2\omega)}}
&\big(
   \sin^{3}\theta_{0}\chi_{zzz}\\
&+ k^{2}_{b}\sin\theta_{0}\cos^{2}\phi\chi_{zxx}\\
&+ k^{2}_{b}\sin\theta_{0}\sin^{2}\phi\chi_{zyy}\\
&+ 2k_{b}\sin^{2}\theta_{0}\cos\phi\chi_{zzx}\\
&+ 2k_{b}\sin^{2}\theta_{0}\sin\phi\chi_{zzy}\\
&+ 2k^{2}_{b}\sin\theta_{0}\sin\phi\cos\phi\chi_{zxy}\\
&- K_{b}\sin^{2}\theta_{0}\cos\phi\chi_{xzz}\\
&- k^{2}_{b}K_{b}\cos^{3}\phi\chi_{xxx}\\
&- k^{2}_{b}K_{b}\sin^{2}\phi\cos\phi\chi_{xyy}\\
&- 2k_{b}K_{b}\sin\theta_{0}\cos^{2}\phi\chi_{xzx}\\
&- 2k_{b}K_{b}\sin\theta_{0}\sin\phi\cos\phi\chi_{xzy}\\
&- 2k^{2}_{b}K_{b}\sin\phi\cos^{2}\phi\chi_{xxy}\\
&- K_{b}\sin^{2}\theta_{0}\sin\phi\chi_{yzz}\\
&- k^{2}_{b}K_{b}\sin\phi\cos^{2}\phi\chi_{yxx}\\
&- k^{2}_{b}K_{b}\sin^{3}\phi\chi_{yyy}\\
&- 2k_{b}K_{b}\sin\theta_{0}\sin\phi\cos\phi\chi_{yzx}\\
&- 2k_{b}K_{b}\sin\theta_{0}\sin^{2}\phi\chi_{yzy}\\
&- 2k^{2}_{b}K_{b}\sin^{2}\phi\cos\phi\chi_{yxy}
\big),
\end{split}
\end{equation*}
and we can eliminate many terms since
$\chi_{zzx}=\chi_{zzy}=\chi_{zxy}=\chi_{xzz}=\chi_{xzy}=\chi_{xxy}=\chi_{yzz}
=\chi_{yxx}=\chi_{yyy}=\chi_{yzx}=0$, and substituting the equivalent
components of $\boldsymbol{\chi},$
\begin{equation*}
\begin{split}
=
\frac{K_{b}}{K_{v}}
\Gamma^{b}_{pP}
&\big(
   \sin^{3}\theta_{0}\chi_{zzz}\\
&+ k^{2}_{b}\sin\theta_{0}\cos^{2}\phi\chi_{zxx}\\
&+ k^{2}_{b}\sin\theta_{0}\sin^{2}\phi\chi_{zxx}\\
&- 2k_{b}K_{b}\sin\theta_{0}\cos^{2}\phi\chi_{xxz}\\
&- 2k_{b}K_{b}\sin\theta_{0}\sin^{2}\phi\chi_{xxz}\\
&- k^{2}_{b}K_{b}\cos^{3}\phi\chi_{xxx}\\
&+ k^{2}_{b}K_{b}\sin^{2}\phi\cos\phi\chi_{xxx}\\
&+ 2k^{2}_{b}K_{b}\sin^{2}\phi\cos\phi\chi_{xxx}
\big),
\end{split}
\end{equation*}
and reducing,
\begin{equation*}
\begin{split}
=
\frac{K_{b}}{K_{v}}
\Gamma^{b}_{pP}
&\big(
   \sin^{3}\theta_{0}\chi_{zzz}\\
&+ k^{2}_{b}\sin\theta_{0}(\sin^{2}\phi + \cos^{2}\phi)\chi_{zxx}\\
&- 2k_{b}K_{b}\sin\theta_{0}(\sin^{2}\phi + \cos^{2}\phi)\chi_{xxz}\\
&+ k^{2}_{b}K_{b}(3\sin^{2}\phi\cos\phi - \cos^{3}\phi)\chi_{xxx}
\big)\\\\
=
\frac{K_{b}}{K_{v}}
\Gamma^{b}_{pP}
&\big(
  \sin^{3}\theta_{0}\chi_{zzz} 
+ k^{2}_{b}\sin\theta_{0}\chi_{zxx}
- 2k_{b}K_{b}\sin\theta_{0}\chi_{xxz}
- k^{2}_{b}K_{b}\chi_{xxx}\cos3\phi
\big),
\end{split}
\end{equation*}
where,
\begin{equation*}
\Gamma^{b}_{pP} =
\frac{T_{p}^{vb}\left(t^{vb}_{p}\right)^{2}}
     {\epsilon_{b}(\omega)\sqrt{\epsilon_{b}(2\omega)}}.
\end{equation*}

We find the equivalent expression for $\mathcal{R}$ evaluated inside the bulk
as
\begin{equation*}
R(2\omega) =
\frac{32\pi^{3} \omega^{2}}{c^{3}K^{2}_{b}}
\left\vert
\mathbf{e}^{\,2\omega}_{b}\cdot\boldsymbol{\chi}:
\mathbf{e}^{\omega}_{b}\mathbf{e}^{\omega}_{b}
\right\vert^{2} 
,
\end{equation*}
and we can remove the $K_{b}/K_{v}$ factor completely and reduce to the
standard form of
\begin{equation*}
R(2\omega) =
\frac{32\pi^{3} \omega^{2}}{c^{3}\cos^{2}\theta_{0}}
\left\vert
\mathbf{e}^{\,2\omega}_{b}\cdot\boldsymbol{\chi}:
\mathbf{e}^{\omega}_{b}\mathbf{e}^{\omega}_{b}
\right\vert^{2}.
\end{equation*}


\section{Taking \texorpdfstring{$\mathcal{P}(2\omega)$}{P(2w)} and the
fundamental fields in the vacuum}

To consider the $1\omega$ fields in the vacuum, we start with Eq. \eqref{m2}
but substitute $\ell\rightarrow v$, thus
\begin{equation*}
\mathbf{E}_{v}(\omega)= E_{0}
\left[
  \hat{\mathbf{s}} t^{vv}_s(1+r^{v b}_s)\hat{\mathbf{s}}
+ \hat{\mathbf{p}}_{v-}t^{vv}_{p}\hat{\mathbf{p}}_{v-}
+ \hat{\mathbf{p}}_{v+}t^{vv}_{p}r^{v b}_{p}\hat{\mathbf{p}}_{v-}
\right]
\cdot\hat{\mathbf{e}}^{\mathrm{in}}
,
\end{equation*}
$t_{p}^{vv}$ and $t_{s}^{vv}$ are one, so we are left with
\begin{equation*}
\begin{split}
\mathbf{e}^{\omega}_{v}
&= \left[
  \hat{\mathbf{s}}(1+r^{v b}_s)\hat{\mathbf{s}}
+ \hat{\mathbf{p}}_{v-}\hat{\mathbf{p}}_{v-}
+ \hat{\mathbf{p}}_{v+}r^{v b}_{p}\hat{\mathbf{p}}_{v-}
\right]
\cdot\hat{\mathbf{e}}^{\mathrm{in}}\\
&= \left[
  \hat{\mathbf{s}}(t^{vb}_s)\hat{\mathbf{s}}
+ (\hat{\mathbf{p}}_{v-} + \hat{\mathbf{p}}_{v+}r^{v b}_{p})
  \hat{\mathbf{p}}_{v-}
\right]
\cdot\hat{\mathbf{e}}^{\mathrm{in}}\\
&= \left[
  \hat{\mathbf{s}}(t^{vb}_s)\hat{\mathbf{s}}
+ \frac{1}{\sqrt{\epsilon_{v}(\omega)}}
\big(
  k_{v}(1 - r^{vb}_{p})\hat{\boldsymbol{\kappa}}
+ \sin\theta_{0}(1 + r^{vb}_{p})
\hat{\mathbf{z}}
\big)
  \hat{\mathbf{p}}_{v-}
\right]\\
&= \left[
  \hat{\mathbf{s}}(t^{vb}_s)\hat{\mathbf{s}}
+ \left(
  \frac{k_{b}}{\sqrt{\epsilon_{b}(\omega)}}
  t^{v b}_{p}\hat{\boldsymbol{\kappa}} 
+ \sqrt{\epsilon_{b}(\omega)}\sin\theta_{0}
  t^{v b}_{p}\hat{\mathbf{z}}
  \right)
  \hat{\mathbf{p}}_{v-}
\right]
\cdot\hat{\mathbf{e}}^{\mathrm{in}}\\
&= \left[
  \hat{\mathbf{s}}(t^{vb}_s)\hat{\mathbf{s}}
+ \frac{t^{v b}_{p}}{\sqrt{\epsilon_{b}(\omega)}}\left(
  k_{b}\cos\phi\hat{\mathbf{x}}
+ k_{b}\sin\phi\hat{\mathbf{y}}
+ \epsilon_{b}(\omega)\sin\theta_{0}\hat{\mathbf{z}}
  \right)
  \hat{\mathbf{p}}_{v-}
\right]
\cdot\hat{\mathbf{e}}^{\mathrm{in}}.
\end{split}
\end{equation*}
For $\mathcal{R}_{pP}$ we require that $\hat{\mathbf{e}}^{\mathrm{in}} =
\hat{\mathbf{p}}_{v-}$, so
\begin{equation*}
\mathbf{e}^{\omega}_{v} =
\frac{t^{v b}_{p}}{\sqrt{\epsilon_{b}(\omega)}}
\left(
  k_{b}\cos\phi\hat{\mathbf{x}}
+ k_{b}\sin\phi\hat{\mathbf{y}}
+ \epsilon_{b}(\omega)\sin\theta_{0}\hat{\mathbf{z}}
  \right),
\end{equation*}
and
\begin{equation*}
\begin{split}
\mathbf{e}^{\omega}_{v}\mathbf{e}^{\omega}_{v} =
\left(\frac{t^{v b}_{p}}{\sqrt{\epsilon_{b}(\omega)}}\right)^{2}
&\big[
   k^{2}_{b}\cos^{2}\phi\hat{\mathbf{x}}\hat{\mathbf{x}}\\
&+ k^{2}_{b}\sin^{2}\phi\hat{\mathbf{y}}\hat{\mathbf{y}}\\
&+ \epsilon^{2}_{b}(\omega)\sin^{2}\theta_{0}
   \hat{\mathbf{z}}\hat{\mathbf{z}}\\
&+ 2k^{2}_{b}\sin\phi\cos\phi\hat{\mathbf{x}}\hat{\mathbf{y}}\\
&+ 2\epsilon_{b}(\omega)k_{b}\sin\theta_{0}\sin\phi
   \hat{\mathbf{y}}\hat{\mathbf{z}}\\
&+ 2\epsilon_{b}(\omega)k_{b}\sin\theta_{0}\cos\phi
   \hat{\mathbf{x}}\hat{\mathbf{z}}
\big].
\end{split}
\end{equation*}

We also require the $2\omega$ fields evaluated in the vacuum, so
\begin{equation}
\mathbf{e}^{\,2\omega}_{v} = \hat{\mathbf{e}}^{\mathrm{out}}
\cdot\left[
\hat{\mathbf{s}}T_s^{v b}\hat{\mathbf{s}} + \hat{\mathbf{P}}_{v+}
\frac{T^{v b}_{p}}{\sqrt{\epsilon_{b}(2\omega)}}
\left(
  \epsilon_{b}(2\omega)\sin\theta_{0}\hat{\mathbf{z}}
  - K_{b}\hat{\boldsymbol{\kappa}}
\right) 
\right],
\end{equation}
and with $\hat{\mathbf{e}}^{\mathrm{out}} = \hat{\mathbf{P}}_{v+}$ we have
\begin{equation}
\mathbf{e}^{\,2\omega}_{v} =
\frac{T^{v b}_{p}}{\sqrt{\epsilon_{b}(2\omega)}}
\left(
\epsilon_{b}(2\omega)\sin\theta_{0}\hat{\mathbf{z}}
- K_{b}\cos\phi\hat{\mathbf{x}}
- K_{b}\sin\phi\hat{\mathbf{y}}
\right).
\end{equation}

So lastly, we have that
\begin{equation*}
\begin{split}
\mathbf{e}^{\,2\omega}_{v}\cdot
\boldsymbol{\chi}:\mathbf{e}^{\omega}_{v}\mathbf{e}^{\omega}_{v}
=\qquad\qquad&\\
\frac{T^{v b}_{p}}{\sqrt{\epsilon_{b}(2\omega)}}
\left(\frac{t^{v b}_{p}}{\sqrt{\epsilon_{b}(\omega)}}\right)^{2}
&\big[
    \epsilon_{b}(2\omega)k^{2}_{b}
    \sin\theta_{0}\cos^{2}\phi\chi_{zxx}\\
&+  \epsilon_{b}(2\omega)k^{2}_{b}
    \sin\theta_{0}\sin^{2}\phi\chi_{zyy}\\
&+  \epsilon^{2}_{b}(\omega)\epsilon_{b}(2\omega)
    \sin^{3}\theta_{0}\chi_{zzz}\\
&+ 2\epsilon_{b}(2\omega)k^{2}_{b}\sin\theta_{0}
    \sin\phi\cos\phi\chi_{zxy}\\
&+ 2\epsilon_{b}(\omega)\epsilon_{b}(2\omega)k_{b}
    \sin^{2}\theta_{0}\sin\phi\chi_{zyz}\\
&+ 2\epsilon_{b}(\omega)\epsilon_{b}(2\omega)k_{b}
    \sin^{2}\theta_{0}\cos\phi\chi_{zxz}\\
&-  k^{2}_{b}K_{b}\cos^{3}\phi\chi_{xxx}\\
&-  k^{2}_{b}K_{b}\sin^{2}\phi\cos\phi\chi_{xyy}\\
&-  \epsilon^{2}_{b}(\omega)K_{b}
    \sin^{2}\theta_{0}\cos\phi\chi_{xzz}\\
&- 2k^{2}_{b}K_{b}\sin\phi\cos^{2}\phi\chi_{xxy}\\
&- 2\epsilon_{b}(\omega)k_{b}K_{b}
    \sin\theta_{0}\sin\phi\cos\phi\chi_{xyz}\\
&- 2\epsilon_{b}(\omega)k_{b}K_{b}
    \sin\theta_{0}\cos^{2}\phi\chi_{xxz}\\
&-  k^{2}_{b}K_{b}\sin\phi\cos^{2}\phi\chi_{yxx}\\
&-  k^{2}_{b}K_{b}\sin^{3}\phi\chi{yyy}\\
&-  \epsilon^{2}_{b}(\omega)K_{b}
    \sin^{2}\theta_{0}\sin\phi\chi_{yzz}\\
&- 2k^{2}_{b}K_{b}\sin^{2}\phi\cos\phi\chi_{yxy}\\
&- 2\epsilon_{b}(\omega)k_{b}K_{b}
    \sin\theta_{0}\sin^{2}\phi\chi_{yyz}\\
&- 2\epsilon_{b}(\omega)k_{b}K_{b}
    \sin\theta_{0}\sin\phi\cos\phi\chi_{yxz}
\big],
\end{split}
\end{equation*}
and after eliminating components,
\begin{equation*}
\begin{split}
= \Gamma^{v}_{pP}&\big[
    \epsilon^{2}_{b}(\omega)\epsilon_{b}
    (2\omega)\sin^{3}\theta_{0}\chi_{zzz}\\
&+  \epsilon_{b}(2\omega)k^{2}_{b}
    \sin\theta_{0}\cos^{2}\phi\chi_{zxx}\\
&+  \epsilon_{b}(2\omega)k^{2}_{b}
    \sin\theta_{0}\sin^{2}\phi\chi_{zxx}\\
&- 2\epsilon_{b}(\omega)k_{b}K_{b}
    \sin\theta_{0}\cos^{2}\phi\chi_{xxz}\\
&- 2\epsilon_{b}(\omega)k_{b}K_{b}
    \sin\theta_{0}\sin^{2}\phi\chi_{xxz}\\
&+ 3k^{2}_{b}K_{b}\sin^{2}\phi\cos\phi\chi_{xxx}\\
&-  k^{2}_{b}K_{b}\cos^{3}\phi\chi_{xxx}
\big]\\\\
= \Gamma^{v}_{pP}&\big[
    \epsilon^{2}_{b}(\omega)\epsilon_{b}(2\omega)
    \sin^{3}\theta_{0}\chi_{zzz}
 +  \epsilon_{b}(2\omega)k^{2}_{b}\sin\theta_{0}\chi_{zxx}\\
&- 2\epsilon_{b}(\omega)k_{b}K_{b}\sin\theta_{0}\chi_{xxz}
 -  k^{2}_{b}K_{b}\chi_{xxx}\cos3\phi
\big],
\end{split}
\end{equation*}
where
\begin{equation*}
\Gamma^{v}_{pP} =
\frac{T^{v b}_{p}\left(t^{v b}_{p}\right)^{2}}
     {\epsilon_{b}(\omega)\sqrt{\epsilon_{b}(2\omega)}}.
\end{equation*}


\section{Taking \texorpdfstring{$\mathcal{P}(2\omega)$}{P(2w)} in
\texorpdfstring{$\ell$}{l} and the fundamental fields in the bulk}

For this scenario with $\hat{\mathbf{e}}^{\mathrm{in}}=\hat{\mathbf{p}}_{v-}$
and $\hat{\mathbf{e}}^{\mathrm{out}}=\hat{\mathbf{P}}_{v+}$ we have,
\begin{equation*}\label{ri12}
\mathbf{e}^{2\omega}_{\ell} =
\frac{T^{v\ell}_{p}T^{\ell b}_{p}}
     {\epsilon_{\ell}({2\omega})\sqrt{\epsilon_{b}(2\omega)}}
\left(
  \epsilon_{b}(2\omega)\sin\theta_{0}\hat{\mathbf{z}}
- \epsilon_{\ell}(2\omega)K_{b}\cos\phi\hat{\mathbf{x}}
- \epsilon_{\ell}(2\omega)K_{b}\sin\phi\hat{\mathbf{y}}
\right),
\end{equation*}
and
\begin{equation*}
\begin{split}
\mathbf{e}^{\omega}_{b}\mathbf{e}^{\omega}_{b}
&= \frac{\left(t^{vb}_{p}\right)^{2}}{\epsilon_{b}(\omega)}
\big(
  \sin^{2}\theta_{0}\hat{\mathbf{z}}\hat{\mathbf{z}}
+ k^{2}_{b}\cos^{2}\phi\hat{\mathbf{x}}\hat{\mathbf{x}}
+ k^{2}_{b}\sin^{2}\phi\hat{\mathbf{y}}\hat{\mathbf{y}}\\
&\qquad\qquad
+ 2k_{b}\sin\theta_{0}\cos\phi\hat{\mathbf{z}}\hat{\mathbf{x}}
+ 2k_{b}\sin\theta_{0}\sin\phi\hat{\mathbf{z}}\hat{\mathbf{y}}
+ 2k^{2}_{b}\sin\phi\cos\phi\hat{\mathbf{x}}\hat{\mathbf{y}}
\big).
\end{split}
\end{equation*}
Thus,
\begin{equation*}
\begin{split}
\mathbf{e}^{\,2\omega}_{\ell}\cdot
\boldsymbol{\chi}:\mathbf{e}^{\omega}_{b}\mathbf{e}^{\omega}_{b} = 
\frac{T^{v\ell}_{p}T^{\ell b}_{p}\left(t^{vb}_{p}\right)^{2}}
     {\epsilon_{\ell}({2\omega})\epsilon_{b}(\omega)\sqrt{\epsilon_{b}(2\omega)}}
\bigg[
&+ \epsilon_{b}(2\omega)\sin^{3}\theta_{0}\chi_{zzz}\\
&+ \epsilon_{b}(2\omega)k^{2}_{b}\sin\theta_{0}\cos^{2}\phi\chi_{zxx}\\
&+ \epsilon_{b}(2\omega)k^{2}_{b}\sin\theta_{0}\sin^{2}\phi\chi_{zyy}\\
&+ 2\epsilon_{b}(2\omega)k_{b}\sin^{2}\theta_{0}\cos\phi\chi_{zzx}\\
&+ 2\epsilon_{b}(2\omega)k_{b}\sin^{2}\theta_{0}\sin\phi\chi_{zzy}\\
&+ 2\epsilon_{b}(2\omega)k^{2}_{b}\sin\theta_{0}\sin\phi\cos\phi\chi_{zxy}\\
%%%%%%%%%%%%%%%%%%%%%%%%%%%%%%%%%%%%%%%%%%%%%%%%%%%%%%%%%%%%%%%%%%%
&- \epsilon_{\ell}(2\omega)\sin^{2}\theta_{0}K_{b}\cos\phi\chi_{xzz}\\
&- \epsilon_{\ell}(2\omega)k^{2}_{b}K_{b}\cos^{3}\phi\chi_{xxx}\\
&- \epsilon_{\ell}(2\omega)k^{2}_{b}K_{b}\sin^{2}\phi\cos\phi\chi_{xyy}\\
&- 2\epsilon_{\ell}(2\omega)k_{b}K_{b}\sin\theta_{0}\cos^{2}\phi\chi_{xzx}\\
&- 2\epsilon_{\ell}(2\omega)k_{b}K_{b}\sin\theta_{0}\sin\phi\cos\phi\chi_{xzy}\\
&- 2\epsilon_{\ell}(2\omega)k^{2}_{b}K_{b}\sin\phi\cos^{2}\phi\chi_{xxy}\\
%%%%%%%%%%%%%%%%%%%%%%%%%%%%%%%%%%%%%%%%%%%%%%%%%%%%%%%%%%%%%%%%%%%
&- \epsilon_{\ell}(2\omega)K_{b}\sin^{2}\theta_{0}\sin\phi\chi_{yzz}\\
&- \epsilon_{\ell}(2\omega)k^{2}_{b}K_{b}\cos^{2}\phi\sin\phi\chi_{yxx}\\
&- \epsilon_{\ell}(2\omega)k^{2}_{b}K_{b}\sin^{3}\phi\chi_{yyy}\\
&- 2\epsilon_{\ell}(2\omega)k_{b}K_{b}\sin\theta_{0}\cos\phi\sin\phi\chi_{yzx}\\
&- 2\epsilon_{\ell}(2\omega)k_{b}K_{b}\sin\theta_{0}\sin^{2}\phi\chi_{yzy}\\
&- 2\epsilon_{\ell}(2\omega)k^{2}_{b}K_{b}\sin^{2}\phi\cos\phi\chi_{yxy}
\bigg].
\end{split}
\end{equation*}
We eliminate and replace components,
\begin{equation*}
\begin{split}
\mathbf{e}^{\,2\omega}_{\ell}\cdot
\boldsymbol{\chi}:\mathbf{e}^{\omega}_{b}\mathbf{e}^{\omega}_{b} = 
\Gamma^{\ell b}_{pP}
\bigg[
&+ \epsilon_{b}(2\omega)\sin^{3}\theta_{0}\chi_{zzz}\\
&+ \epsilon_{b}(2\omega)k^{2}_{b}\sin\theta_{0}\cos^{2}\phi\chi_{zxx}\\
&+ \epsilon_{b}(2\omega)k^{2}_{b}\sin\theta_{0}\sin^{2}\phi\chi_{zxx}\\
&- 2\epsilon_{\ell}(2\omega)k_{b}K_{b}\sin\theta_{0}\cos^{2}\phi\chi_{xxz}\\
&- 2\epsilon_{\ell}(2\omega)k_{b}K_{b}\sin\theta_{0}\sin^{2}\phi\chi_{xxz}\\
&- \epsilon_{\ell}(2\omega)k^{2}_{b}K_{b}\cos^{3}\phi\chi_{xxx}\\
&+ \epsilon_{\ell}(2\omega)k^{2}_{b}K_{b}\sin^{2}\phi\cos\phi\chi_{xxx}\\
&+ 2\epsilon_{\ell}(2\omega)k^{2}_{b}K_{b}\sin^{2}\phi\cos\phi\chi_{xxx}
\bigg],
\end{split}
\end{equation*}
so lastly
\begin{equation*}
\begin{split}
\mathbf{e}^{\,2\omega}_{\ell}\cdot
\boldsymbol{\chi}:\mathbf{e}^{\omega}_{b}\mathbf{e}^{\omega}_{b} = 
\Gamma^{\ell b}_{pP}&
\bigg[
  \epsilon_{b}(2\omega)\sin^{3}\theta_{0}\chi_{zzz}
+ \epsilon_{b}(2\omega)k^{2}_{b}\sin\theta_{0}\chi_{zxx}\\
&- 2\epsilon_{\ell}(2\omega)k_{b}K_{b}\sin\theta_{0}\chi_{xxz}
- \epsilon_{\ell}(2\omega)k^{2}_{b}K_{b}\chi_{xxx}\cos3\phi
\bigg],
\end{split}
\end{equation*}
where
\begin{equation*}
\Gamma^{\ell b}_{pP}=
\frac{T^{v\ell}_{p}T^{\ell b}_{p}\left(t^{vb}_{p}\right)^{2}}
  {\epsilon_{\ell}({2\omega})\epsilon_{b}(\omega)\sqrt{\epsilon_{b}(2\omega)}}.
\end{equation*}

\stopcontents[chapters]
