\documentclass[aps,letterpaper]{revtex4}
\usepackage{amsmath}
\usepackage[colorlinks=true]{hyperref}

%%%%% defs
%%%% acronimos
\def\copyr{$^\copyright$}
\def\reg{\textsuperscript{\textregistered}}
\def\tm{\textsuperscript{\texttrademark}}
\def\lufac{LUFAC${^\textsuperscript{\textregistered}}$}
%\def\depe{DP${^\textsuperscript{\texttrademark}}$}
\def\depe{DP${\texttrademark}$}
\def\ps{\mathrm{ps}}
\def\lda{\mathrm{LDA}}
\def\rpa{\mathrm{RPA}}
\def\nl{\mathrm{nl}}
\def\acu{Accu-Check\textsuperscript{\textregistered}~Performa}
\def\goni{Glucometro \'Optico No Invasivo}
\def\Reg{\textsuperscript{\textregistered}}
\def\tiniba{TINIBA\textsuperscript{\textregistered}}
\def\gw{{\it GW}}
\def\gsa{Generaci\'on del Segundo Arm\'onico}
\def\shg{Second Harmonic Generation}
\def\sfg{Sum Frequency Generation}
\def\sdf{Generaci\'on de Suma de Frecuencias}
%%%%% accent of i
\def\'#1{\if#1i{\accent19\i}\else{\accent19#1}\fi}
%%%%% compa\~nias
\def\micro{{\it Supermicro}}
%%%%% lugares
\def\lou{Laboratorio de \'Optica Ultrar\'apida}
\def\roma{Universidad de Roma II}
\def\tor{``Tor Vergata''}
\def\pssb{Physica Satatus Solidi B}
\def\dti{Direcci\'on de Tecnolog\'{\i}a e Innovaci\'on}
\def\di{Direcci\'on de Investigaci\'on}
\def\dfa{Direcci\'on de Formaci\'on Acd\'emica}
\def\dg{Direcci\'on General}
\def\da{Direcci\'on Administrativa}
\def\ifug{Insituto de F\'isica de la U. de Guanajuato}
\def\icf{Instituto de Ciencias F\'isicas}
\def\unam{Universidad Nacional Aut\'onoma de M\'exico}
\def\uguille{Universidad del Nordeste, Argentina}
\def\fotonica{Departamento de Fotonica}
\def\grupo{Propiedades \'Opticas de Nano-Sistemas, Interfases y Superficies}
\def\grupoa{PRONASIS}
%\def\grupo{Propiedades \'Opticas de Superficies e Interfases y Sistemas Nanosc\'opicos}
%\def\grupoa{POSISNA}
\def\di{Direcci\'on de Investigaci\'on}
\def\dfa{Direcci\'on de Formaci\'on Acad\'emica}
\def\cio{Centro de Investigaciones en \'Optica}
\def\ciod{Centro de Investigaciones en \'Optica, León, Guanajuato.}
\def\Conacyt{Consejo Nacional de Ciencia y Tecnolog\'ia}
\def\Concyteg{Consejo  de Ciencia y Tecnolog\'ia del Estado de Guanajuato}
\def\conacyt{CONACyT}
\def\concyteg{CONCyTEG}
\def\lagos{Centro Universitario de los Lagos}
\def\udeg{Universidad de Guadalajara}
\def\dinv{Direcci\'on de Investigaci\'on}
\def\dop{Department of Physics}
\def\uoft{University of Toronto}
\def\ua{University of Texas at Austin}
\def\icf{Instituto de Ciencias Físicas, UNAM, Cuernavaca}
%%%%% gente
%% grupo
\def\gabriel{Gabriel Ramos Ortíz}
\def\ramon{Ram\'on~ Carriles~ Jaimes}
\def\ramonm{Ram\acute{o}n~ Carriles~ Jaimes}
\def\enrique{Enrique~ Castro~ Camus}
\def\raul{Ra\'ul Alfonso V\'azquez Nava}
\def\raulm{Ra\acute{u}l~ Alfonso~ V\acute{a}zquez~ Nava}
\def\beto{Norberto~ Arzate~ Plata}
\def\bmsa{Bernardo S. Mendoza}
\def\bms{Bernardo~ Mendoza~ Santoyo}
%% alumnos
\def\cesar{C\'esar Castillo Quevedo}
\def\cabellos{Jos\'e Luis Cabellos Quiroz}
\def\tona{Tonatiuh Rangel Gordillo}
\def\temok{Juan Cuauhtemoc Salazar Gonz\'alez}
\def\adan{Luis Adan Mart\'inez Jim\'enez}
\def\sean{Sean Martin Anderson}
\def\reinaldo{Reinaldo Zapata Pe\~na}
\def\miguel{Miguel Angel Honorato Colin}
\def\oscar{Oscar Andrés Naranjo Montoya}
%%% alumnos del grupo
%% enrique
%Maestria:
\def\jorgee{Jorge Alberto Caballero Mendoza}
\def\sofia{Sofía Carolina Corzo García}
\def\ruth{Ruth Julieta Medina López} 
%Doctorado: 
\def\juane{Juan Jes\'us S\'anchez S\'anchez}
%Licenciatura
\def\alma{Alma Gabriela González Patlán}
%(con Ramon): 
\def\sergioer{Sergio Augusto Romero Serv\'{\i}n}
%% Raul
%Maestria:
\def\ramses{Ramses Valente Salazar Aparicio}
\def\enriquer{Enrique Arag\'on Navarro}%udg
\def\salomonr{Salom\'on Rodr\'{\i}guez Carrera}
\def\hectorr{H\'ector Santiago Hern\'andez}
\def\victor{Victor Manuel Villanueva Reyes}
%% Ramon
%Maestria:
\def\alfredora{Alfredo Campos Mej\'{\i}a}
%% Beto
%Doctorado
\def\noe{No\'e Gonz\'alez Baquedano}
%% otros
\def\liliana{Liliana Wilson Herr\'an}
\def\gerardo{Gerardo E. S\'anchez Garc\'{\i}a Rojas}
\def\amalia{Amalia Mart\'inez Garc\'{\i}a}
\def\gabriel{Gabriel Ramos Ort\'{\i}z}
\def\nacho{Ing. José Ignacio Diego Manrique}
\def\tere{Teresita del Niño Jesús Pérez Hernández}
\def\elder{Elder de la Rosa Cruz}
\def\gonzalo{Gonzalo P\'aez Padilla}
\def\wlm{W. Luis Moch\'an Backal}
\def\oracio{Oracio C. Barbosa Garc\'ia}
\def\hector{H\'ector Hugo S\'anchez Hern\'andez}
\def\marco{Marco Antonio Escobar-Acevedo}
\def\gil{Alejandro Gil-Villegas Montiel}
\def\ernesto{Ernesto Carlos Cort\'es Morales}
\def\fms{Fernando Mendoza Santoyo}
\def\cuevas{Francisco Javier Cuevas de la Rosa}
\def\brenda{Brenda Esmeralda Matr\'inez Z\'erega}
\def\guille{Guillermo Ortiz}
\def\cesar{Cesar Castillo Quevedo}
\def\sipe{Prof. John Sipe}
\def\mike{Prof. Michael Downer}
\def\jems{Jorge Enrique Mej\'ia S\'anchez}
\def\lamon{Ram\'on Rodr\'iguez Vera}
\def\ldp{Luis de la Pe\~na}
\def\sole{Rodolfo Del Sole}
\def\lucia{Lucia Reining}
\def\sch{Schr\"odinger}
\def\Cuevas{Francisco J. Cuevas de la Rosa}
%%%%% categorias
\def\ita{Investigador Titular A}
\def\itb{Investigador Titular B}
\def\itc{Investigador Titular C}
\def\itd{Investigador Titular D}
\def\ite{Investigador Titular E}
\def\sr{Senior Researcher}
\def\iac{Investigador Asociado C}    
\def\alm{Alumno de Maestr\'ia}
\def\ald{Alumno de Doctorado}
\def\all{Alumno de Licenciatura}
\def\adei{Asistente de Investigaci\'on}
\def\sniIII{S.N.I. nivel III}
\def\sni{S.N.I.}
\def\cv{Currículum Vitae}
%%%%%% fonts
\def\tit{\sf}
\def\col{\sc}
\def\alu{\it} % for students
\def\cual{2$^{do}$}
\def\anno{2005}
\def\spe{\vspace{.12cm}}
%%%%%% cosas
\def\capa{capa-a-capa}
\def\espin{espintr\'onica}
\def\oespin{optoespintr\'onica}
\def\proyecto{Photon Assisted Spintronics}
\def\npro{48915}
\def\cvk{cv\mathbf{k}}
\def\cvkp{c'v'\mathbf{k}'}
%%%%%% revistas
\def\prb{Physical Review B}
\def\prl{Physical Review Letters}
\def\ol{Optics Letters}
\def\opn{Optics and Photonics News}
\def\pssc{physica status solidi (c)}
%%%%%%%%%%%%%%%%%%%%%%%%%%%%%%%%%%%%%%%
%%%%%% griegas
\def\ga{\alpha}
\def\gb{\beta}
\def\gga{\gamma}
\def\gGa{\Gamma}
\def\go{\omega}
\def\got{\tilde\omega}
\def\gO{\Omega}
\def\gr{{\rho}}
\def\ge{\epsilon}
\def\ve{\varepsilon}
\def\gve{\varepsilon}
\def\gd{\delta}
\def\gD{\Delta}
\def\gl{\lambda}
\def\gs{\sigma}
\def\gS{\Sigma}
\def\gbs{\overline{\sigma}}
\def\gka{\kappa}
%%%%%% griegas with tilde
\def\gta{\tilde{\alpha}}
\def\gtb{\tilde{\beta}}
\def\gtga{\tilde{\gamma}}
\def\gto{\tilde{\omega}}
\def\gtO{\tilde{\Omega}}
\def\gtr{\tilde{\rho}}
\def\gte{\tilde{\epsilon}}
\def\vte{\tilde{\varepsilon}}
\def\gtd{\tilde{\delta}}
\def\gtD{\tilde{\Delta}}
\def\gtl{\tilde{\lambda}}
\def\gts{\tilde{\sigma}}
\def\gtS{\tilde{\Sigma}}
%%%%%% romans with tilde
\def\bftr{\tilde{\mathbf{r}}}
\def\bftp{\tilde{\mathbf{p}}}
\def\bftv{\tilde{\mathbf{v}}}
\def\ta{\tilde{a}}
\def\tb{\tilde{b}}
\def\tr{\tilde{r}}
\def\tp{\tilde{p}}
\def\tV{\tilde{V}}
\def\tv{\tilde{v}}
%%
\newcommand{\ham}{\hat{\mathcal H}}
%%%%%% bra kets
\newcommand{\la}{\langle}
\newcommand{\ra}{\rangle}
\newcommand{\ket}[1]{| #1 \rangle}
\newcommand{\bra}[1]{\langle #1 |}
\newcommand{\braket}[2]{\langle {#1} | {#2} \rangle}
\newcommand{\ketbra}[2]{| {#1} \rangle {#1} \langle {#2} |}
\newcommand{\ave}[1]{\langle {#1} \rangle}
\newcommand{\dotp}[2]{\mathbf{#1} \cdot \mathbf{#2}}
%%%%%% averages
\newcommand{\prom}[1]{\langle {#1} \rangle}
%%%%%% creation and annihilation operators
\newcommand{\oa}{\hat a^{\tiny\strut}}
\newcommand{\oad}{\hat a^\dagger}
\newcommand{\oadk}{\hat a^\dagger_{\mathbf k}}
\newcommand{\oak}{\hat a^{\tiny\strut}_{\mathbf k}}
\newcommand{\obd}[1]{\hat b^\dagger_{#1}}
\newcommand{\ob}[1]{\hat b^{\tiny\strut}_{#1}}
%%%%%% Caligraphic
\newcommand{\cala}{{\cal A}}
\newcommand{\calb}{{\cal B}}
\newcommand{\calc}{{\cal C}}
\newcommand{\cald}{{\cal D}}
\newcommand{\cale}{{\cal E}}
\newcommand{\calf}{{\cal F}}
\newcommand{\calh}{{\cal H}}
\newcommand{\cali}{{\cal I}}
\newcommand{\calp}{{\cal P}}
\newcommand{\calg}{{\cal G}}
\newcommand{\calv}{{\cal V}}
\newcommand{\call}{{\cal L}}
\newcommand{\calo}{{\cal O}}
\newcommand{\calm}{{\cal M}}
\newcommand{\caln}{{\cal N}}
\newcommand{\calr}{{\cal R}}
\newcommand{\cals}{{\cal S}}
\newcommand{\calt}{{\cal T}}
\newcommand{\calu}{{\cal U}}
\newcommand{\calw}{{\cal W}}
\newcommand{\calbd}{\boldsymbol{\mathcal{\cal D}}}
\newcommand{\calbe}{\boldsymbol{\mathcal{\cal E}}}
\newcommand{\calbj}{\boldsymbol{\mathcal{\cal J}}}
\newcommand{\calbm}{\boldsymbol{\mathcal{\cal M}}}
\newcommand{\calbp}{\boldsymbol{\mathcal{\cal P}}}
\newcommand{\calbv}{\boldsymbol{\mathcal{\cal V}}}
\newcommand{\calbs}{\boldsymbol{\mathcal{\cal S}}}
\newcommand{\calbg}{\boldsymbol{\mathcal{\cal G}}}
%%%%%% mathematicla bold roman & greek
\newcommand{\mbf}[1]{\mathbf{#1}}
\newcommand{\mbg}[1]{\boldsymbol{\mathcal {#1}}}
\newcommand{\bfA}{\mathbf{A}}
\newcommand{\bfB}{\mathbf{B}}
\newcommand{\bfC}{\mathbf{C}}
\newcommand{\bfD}{\mathbf{D}}
\newcommand{\bfE}{\mathbf{E}}
\newcommand{\bfF}{\mathbf{F}}
\newcommand{\bfG}{\mathbf{G}}
\newcommand{\bfH}{\mathbf{H}}
\newcommand{\bfI}{\mathbf{I}}
\newcommand{\bfJ}{\mathbf{J}}
\newcommand{\bfK}{\mathbf{K}}
\newcommand{\bfL}{\mathbf{L}}
\newcommand{\bfM}{\mathbf{M}}
\newcommand{\bfN}{\mathbf{N}}
\newcommand{\bfP}{\mathbf{P}}
\newcommand{\bfR}{\mathbf{R}}
\newcommand{\bfS}{\mathbf{S}}
\newcommand{\bfT}{\mathbf{T}}
\newcommand{\bfU}{\mathbf{U}}
\newcommand{\bfV}{\mathbf{V}}
\newcommand{\bfW}{\mathbf{W}}
\newcommand{\bfX}{\mathbf{X}}
\newcommand{\bfY}{\mathbf{Y}}
\newcommand{\bfZ}{\mathbf{Z}}
\newcommand{\bfa}{\mathbf{a}}
\newcommand{\bfb}{\mathbf{b}}
\newcommand{\bfc}{\mathbf{c}}
\newcommand{\bfd}{\mathbf{d}}
\newcommand{\bfe}{\mathbf{e}}
\newcommand{\bff}{\mathbf{f}}
\newcommand{\bfg}{\mathbf{g}}
\newcommand{\bfh}{\mathbf{h}}
\newcommand{\bfi}{\mathbf{i}}
\newcommand{\bfj}{\mathbf{j}}
\newcommand{\bfk}{\mathbf{k}}
\newcommand{\bfn}{\mathbf{n}}
\newcommand{\bfp}{\mathbf{p}}
\newcommand{\bfq}{\mathbf{q}}
\newcommand{\bfr}{\mathbf{r}}
\newcommand{\bfs}{\mathbf{s}}
\newcommand{\bft}{\mathbf{t}}
\newcommand{\bfu}{\mathbf{u}}
\newcommand{\bfv}{\mathbf{v}}
\newcommand{\bfx}{\mathbf{x}}
\newcommand{\bfy}{\mathbf{y}}
\newcommand{\bfz}{\mathbf{z}}
\newcommand{\bfzero}{\mathbf{0}}
\newcommand{\bfone}{\mathbf{1}}
%
\newcommand{\bfgeta}{\boldsymbol{\eta}}
\newcommand{\bfSig}{\boldsymbol{\Sigma}}
\newcommand{\bfsig}{\boldsymbol{\sigma}}
\newcommand{\bfgS}{\boldsymbol{\Sigma}}
\newcommand{\bfgs}{\boldsymbol{\sigma}}
\newcommand{\bfga}{\boldsymbol{\alpha}}
\newcommand{\bfgb}{\boldsymbol{\beta}}
\newcommand{\bfge}{\boldsymbol{\epsilon}}
\newcommand{\bfgvare}{\boldsymbol{\varepsilon}}
\newcommand{\bfgg}{\boldsymbol{\gamma}}
\newcommand{\bfgG}{\boldsymbol{\Gamma}}
\newcommand{\bfgphi}{\boldsymbol{\phi}}
\newcommand{\bfgpsi}{\boldsymbol{\psi}}
\newcommand{\bfgD}{\boldsymbol{\Delta}}
\newcommand{\bfgPhi}{\boldsymbol{\Phi}}
\newcommand{\bfgPsi}{\boldsymbol{\Psi}}
\newcommand{\bfgtau}{\boldsymbol{\tau}}
\newcommand{\bfgxi}{\boldsymbol{\xi}}
\newcommand{\bfgchi}{\boldsymbol{\chi}}
\newcommand{\bfgnabla}{\boldsymbol{\nabla}}
\newcommand{\bfgnu}{\boldsymbol{\nu}}
\newcommand{\bfgmu}{\boldsymbol{\mu}}
\newcommand{\bfgrho}{\boldsymbol{\rho}}
\newcommand{\bfgRho}{\boldsymbol{\Rho}}
%%%%%% nabla
\newcommand{\nablak}{\frac{\partial}{\partial\mathbf{k}} }
%%%%%% ; derivative
\def\gk{{;\mathbf k}}
%%%%%% k derivative
\newcommand{\deriv}[2] {\frac{\partial {#1}} {\partial {#2} }}
%%%%%% prime for \sum
\def\prima{\strut^{_{'}}}
%%%%%% subindices
%\def\eti{n\bfk}
\newcommand{\eti}[1]{_{#1 \bfk}}
\newcommand{\etiup}[1]{_{#1 \bfk s}}
\newcommand{\etidn}[1]{_{#1 \bfk \bar{s}}}
%%%%% superindice to push down the subindex in greeks!
\def\pd{^{\strut}}
%%%%% gauges
\def\rde{$\bfr\cdot\bfE$~}
\def\rder{length-gauge}
\def\pda{$\bfp\cdot\bfA$~}
\def\vda{$\bfv\cdot\bfA$~}
\def\vdar{velocity-gauge}
%%%%% integral over k
\def\intk{\int\frac{d^3k}{8\pi^3}}
%%%%% roman indices
\def\rmi{\mathrm{i}}
\def\rmj{\mathrm{j}}
\def\rmk{\mathrm{k}}
\def\rml{\mathrm{l}}
\def\rmr{\mathrm{r}}
\def\rma{\mathrm{a}}
\def\rmb{\mathrm{b}}
\def\rmc{\mathrm{c}}
\def\rmd{\mathrm{d}}
\def\rme{\mathrm{e}}
\def\rmv{\mathrm{v}}
\def\rmz{\mathrm{z}}
\def\rmx{\mathrm{x}}
\def\rmy{\mathrm{y}}
\def\rmH{\mathrm{H}}
\def\rmI{\mathrm{I}}
\def\rmG{\mathrm{G}}
\def\rmW{\mathrm{W}}
%%%%% functions
\def\erf{\mathrm{erf}}
\def\erfc{\mathrm{erfc}}
\def\erfi{\mathrm{erfi}}

%iave: C01243171 y 0124317111
%%%%% Green's one point model
\def\tyv{\tilde y_v}
\def\ty{\tilde y}
\def\tv{\tilde v}


\begin{document}

\section{\texorpdfstring{$\mathbf{r}_{e}$ and $\mathbf{r}_{i}$}{re and ri}}
\label{app:re_ri}

In this appendix, we derive the expressions for the matrix elements of the
electron position operator $\mathbf{r}$. The $r$ representation of the Bloch
states is given by
\begin{equation}\label{bloch}
\psi_{n\mathbf{k}}(\mathbf{r})=\langle\mathbf{r}\vert n\mathbf{k}\rangle =
\sqrt{\frac{\Omega}{8\pi^{3}}}
e^{i\mathbf{k} \cdot \mathbf{r}}u_{n\mathbf{k}}(\mathbf{r}),
\end{equation}
where $u_{n\mathbf{k}}(\mathbf{r}) = u_{n\mathbf{k}}(\mathbf{r} + \mathbf{R})$
is cell periodic, and
\begin{equation}\label{normal}
\int_{\Omega}
u_{n\mathbf{k}}^{*}(\mathbf{r})u_{m\mathbf{k}'}(\mathbf{r})\,d^{3}r
= \delta_{nm}\delta_{\mathbf{\mathbf{k},\mathbf{k}'}},
\end{equation}
and $\Omega$ is the unit cell volume.

The key ingredient in the calculation are the matrix elements of the position
operator $\mathbf{r}$. We start from the basic relation
\begin{equation}\label{nbraket}
\langle n\mathbf{k}\vert m\mathbf{k}'\rangle =
\delta_{nm}\delta(\mathbf{k} - \mathbf{k}'),
\end{equation}
and take its derivative with respect to $\mathbf{k}$ as follows. On one hand,
\begin{equation}\label{ddk1}
\frac{\partial}{\partial\mathbf{k}}
\langle n\mathbf{k}\vert m\mathbf{k}'\rangle =
\delta_{nm}\frac{\partial}{\partial\mathbf{k}}\delta(\mathbf{k} - \mathbf{k}'),
\end{equation}
and on the other,
\begin{equation}\label{dkbraket}
\frac{\partial}{\partial\mathbf{k}}\braket{n\mathbf{k}}{m\mathbf{k}'}
= \frac{\partial}{\partial\mathbf{k}}\int
\langle n\mathbf{k}\vert \mathbf{r}\rangle
\langle \mathbf{r}\vert m\mathbf{k}'\rangle
\,d\mathbf{r} 
= \int
\left(
\frac{\partial}{\partial\mathbf{k}}
\psi^{*}_{n\mathbf{k}}(\mathbf{r})
\right)
\psi_{m\mathbf{k}'}(\mathbf{r})\,d\mathbf{r}.
\end{equation}
The derivative of the wavefunction is simply given by
\begin{equation}\label{dpsi}
\frac{\partial}{\partial\mathbf{k}}\psi^{*}_{n\mathbf{k}}(\mathbf{r})
= \sqrt{\frac{\Omega}{8\pi^{3}}}
\left(
\frac{\partial}{\partial\mathbf{k}}
u^{*}_{n\mathbf{k}}(\mathbf{r})
\right)
e^{-i\mathbf{k}\cdot\mathbf{r}} - i\mathbf{r}\psi^{*}_{n\mathbf{k}}(\mathbf{r}).
\end{equation}
Substituting into Eq. \eqref{dkbraket}, we obtain
\begin{align}\label{dkbraket2}
\frac{\partial}{\partial\mathbf{k}}\braket{n\mathbf{k}}{m\mathbf{k}'}
&= \sqrt{\frac{\Omega}{8\pi^{3}}}\int
\left(
\frac{\partial}{\partial\mathbf{k}}
u^{*}_{n\mathbf{k}}(\mathbf{r})\right)e^{-i\mathbf{k}\cdot\mathbf{r}}
\psi_{m\mathbf{k}'}(\mathbf{r})\,d\mathbf{r} 
- i\int\psi^{*}_{n\mathbf{k}}(\mathbf{r})\mathbf{r}
  \psi_{m\mathbf{k}'}(\mathbf{r})\,d\mathbf{r}\nonumber \\
&= \frac{\Omega}{8\pi^{3}}\int
  e^{-i(\mathbf{k}-\mathbf{k}')\cdot\mathbf{r}}
  \left(
  \frac{\partial}{\partial\mathbf{k}}u^{*}_{n\mathbf{k}}(\mathbf{r})
  \right)
u_{m\mathbf{k}'}(\mathbf{r})\,d\mathbf{r}
- i \bra{n\mathbf{k}}\hat{\mathbf{r}}\ket{m\mathbf{k}'}.
\end{align}
Restricting $\mathbf{k}$ and $\mathbf{k}'$ to the first Brillouin zone, we use
the following result that is valid for any periodic function
$f(\mathbf{r}) = f(\mathbf{r} + \mathbf{R})$,
\begin{align}\label{periodic}
\int e^{i(\mathbf{q}-\mathbf{k})\cdot\mathbf{r}}f(\mathbf{r})\,d^{3}r =
\frac{8\pi^{3}}{\Omega}\delta(\mathbf{q} - \mathbf{k})
\int_{\Omega}f(\mathbf{r})\,d^{3}r,
\end{align}
to finally write \cite{blountSSP62}
\begin{align}\label{dkbraket3}
\frac{\partial}{\partial\mathbf{k}}
\langle n\mathbf{k}\vert m\mathbf{k}'\rangle
&= \delta(\mathbf{k}-\mathbf{k}')\int_{\Omega}
\left(
\frac{\partial}{\partial\mathbf{k}} u^{*}_{n\mathbf{k}}(\mathbf{r})
\right)
u_{m\mathbf{k}}(\mathbf{r})\,d\mathbf{r}
-i\bra{n\mathbf{k}}\hat{\mathbf{r}}\ket{m\mathbf{k}'}.
\end{align}
From
\begin{align}\label{dnm1}
\int_{\Omega}u_{m\mathbf{k}} u^{*}_{n\mathbf{k}}\,d\mathbf{r} = \delta_{nm},
\end{align}
we easily find that
\begin{align}\label{dnm2}
\int_{\Omega}
\left(
\frac{\partial}{\partial\mathbf{k}} u_{m\mathbf{k}}(\mathbf{r})
\right)
u^{*}_{n\mathbf{k}}(\mathbf{r})\,d\mathbf{r}
= -\int_{\Omega}u_{m\mathbf{k}}(\mathbf{r})
\left(
\frac{\partial}{\partial\mathbf{k}} u^{*}_{n\mathbf{k}}(\mathbf{r})
\right)\,d\mathbf{r}.
\end{align}
Therefore, we define
\begin{align}\label{zeta}
\boldsymbol{\xi}_{nm}(\mathbf{k}) \equiv
i\int_{\Omega}u^{*}_{n\mathbf{k}}(\mathbf{r})\nabla_{\mathbf{k}}
u_{m\mathbf{k}}(\mathbf{r})\,d\mathbf{r},
\end{align} 
with $\nabla_{\mathbf{k}} = \partial/\partial\mathbf{k}$. Now, from Eqs.
\eqref{ddk1}, \eqref{dkbraket2}, and  \eqref{zeta}, we have that the matrix
elements of the position operator of the electron are given by
\begin{align}\label{erre}
\langle n\mathbf{k}\vert \hat{\mathbf{r}} \vert m\mathbf{k}'\rangle 
= \delta(\mathbf{k}-\mathbf{k}')\boldsymbol{\xi}_{nm}(\mathbf{k})
+ i\delta_{nm}\nabla_{\mathbf{k}}\delta(\mathbf{k}-\mathbf{k}'),
\end{align}
Then, from Eq. \eqref{erre} and writing $\hat{\mathbf{r}} = \hat{\mathbf{r}}_{e}
+ \hat{\mathbf{r}}_{i}$, with $\hat{\mathbf{r}}_{e}$ ($\hat{\mathbf{r}}_{i}$)
the interband (intraband) part, we obtain that
\begin{align}\label{rnmi}
\langle n\mathbf{k}\vert \hat{\mathbf{r}}_{i} \rangle m\mathbf{k}'\vert
&= \delta_{nm}
\left[
  \delta(\mathbf{k}-\mathbf{k}')\boldsymbol{\xi}_{nn}(\mathbf{k})
+ i\nabla_{\mathbf{k}}\delta(\mathbf{k} - \mathbf{k}')
\right],\\
\langle n\mathbf{k}\vert \hat{\mathbf{r}}_e \rangle m\mathbf{k}'\vert
&= (1 - \delta_{nm})\delta(\mathbf{k}-\mathbf{k}')
   \boldsymbol{\xi}_{nm}(\mathbf{k}).
\label{rnme}
\end{align} 

To proceed, we relate Eq. \eqref{rnme} to the matrix elements of the momentum
operator as follows. For the intraband part, we derive the following general
result,
\begin{align}\label{conmri}
\bra{n\mathbf{k}}[\hat{\mathbf{r}}_{i},\hat\calo]\ket{m\mathbf{k}'}
&=
\sum_{\ell,\mathbf{k}''}
\left(
\bra{n\mathbf{k}}
\hat{\mathbf{r}}_{i}
\ket{\ell\mathbf{k}''}
\bra{\ell\mathbf{k}''}
\hat\calo
\ket{m\mathbf{k}'}
\right.
\nonumber \\
&
\left.
-
\bra{n\mathbf{k}}
\hat\calo
\ket{\ell\mathbf{k}''}
\bra{\ell\mathbf{k}''}
\hat{\mathbf{r}}_{i}
\ket{m\mathbf{k}'}
\right)
\nonumber \\
&=
\sum_{\ell}
\left(
\bra{n\mathbf{k}}
\hat{\mathbf{r}}_{i}
\ket{\ell\mathbf{k}'}
\calo_{\ell m}(\mathbf{k}')
\right.
\nonumber \\
&
\left.
-
\calo_{n\ell}(\mathbf{k})
\ket{\ell\mathbf{k}}
\bra{\ell\mathbf{k}}
\hat{\mathbf{r}}_{i}
\ket{m\mathbf{k}'}
\right)
,
\end{align}
where we have taken
$\bra{n\mathbf{k}}\hat\calo\ket{\ell\mathbf{k}''}=\delta(\mathbf{k}-\mathbf{k}'')\calo_{n\ell}(\mathbf{k})$.
We substitute Eq. \eqref{rnmi}, to obtain
\begin{align}\label{conmri2}
\sum_{\ell}&
\left(
\delta_{n\ell}\left[\delta(\mathbf{k}-\mathbf{k}')\boldsymbol{\xi}_{nn}(\mathbf{k})
+i\nabla_{\mathbf{k}}\delta(\mathbf{k}-\mathbf{k}')\right]
\calo_{\ell m}(\mathbf{k}')
\right.
\nonumber \\
&
\left.
-
\calo_{n\ell}(\mathbf{k})
\delta_{\ell m}\left[\delta(\mathbf{k}-\mathbf{k}')\boldsymbol{\xi}_{mm}(\mathbf{k})
+i\nabla_{\mathbf{k}}\delta(\mathbf{k}-\mathbf{k}')\right]
\right)
\nonumber\\
&=
\left(
\left[\delta(\mathbf{k}-\mathbf{k}')\boldsymbol{\xi}_{nn}(\mathbf{k})
+i\nabla_{\mathbf{k}}\delta(\mathbf{k}-\mathbf{k}')\right]
\calo_{n m}(\mathbf{k}')
\right.
\nonumber \\
&-
\left.
\calo_{nm}(\mathbf{k})
\left[\delta(\mathbf{k}-\mathbf{k}')\boldsymbol{\xi}_{mm}(\mathbf{k})
+i\nabla_{\mathbf{k}}\delta(\mathbf{k}-\mathbf{k}')\right]
\right)
\nonumber\\
&=
\delta(\mathbf{k}-\mathbf{k}')
\calo_{nm}(\mathbf{k})
\left(
\boldsymbol{\xi}_{nn}(\mathbf{k})
-
\boldsymbol{\xi}_{mm}(\mathbf{k})
\right)
+i\calo_{n m}(\mathbf{k}')
\nabla_{\mathbf{k}}
\delta(\mathbf{k}-\mathbf{k}')
\nonumber \\
&+i
\delta(\mathbf{k}-\mathbf{k}')
\nabla_{\mathbf{k}}
\calo_{n m}(\mathbf{k})
-
i\calo_{n m}(\mathbf{k}')
\nabla_{\mathbf{k}}\delta(\mathbf{k}-\mathbf{k}')
\nonumber \\
&=
i\delta(\mathbf{k}-\mathbf{k}')
\bigg(
\nabla_{\mathbf{k}}
\calo_{n m}(\mathbf{k})
-
i
\calo_{nm}(\mathbf{k})
\left(
\boldsymbol{\xi}_{nn}(\mathbf{k})
-
\boldsymbol{\xi}_{mm}(\mathbf{k})
\right)
\bigg)
\nonumber \\
&\equiv
i\delta(\mathbf{k}-\mathbf{k}')(\calo_{nm})_{;\mathbf{k}}
.
\end{align}
Then,
\begin{align}\label{conmri3}
\bra{n\mathbf{k}}[\hat{\mathbf{r}}_{i},\hat\calo]\ket{m\mathbf{k}'}
=i\delta(\mathbf{k}-\mathbf{k}')(\calo_{nm})_{;\mathbf{k}}
,
\end{align}   
with
\begin{align}\label{gendev}
(\calo_{nm})_{;\mathbf{k}}=
\nabla_{\mathbf{k}}
\calo_{n m}(\mathbf{k})
-  
i
\calo_{nm}(\mathbf{k})
\left(
\boldsymbol{\xi}_{nn}(\mathbf{k})
-
\boldsymbol{\xi}_{mm}(\mathbf{k})
\right)
,
\end{align}  
the generalized derivative of $\calo_{nm}$ with respect to $\mathbf{k}$.
Note that the highly singular term $\nabla_{\mathbf{k}}\delta(\mathbf{k}-\mathbf{k}')$
cancels in Eq. \eqref{conmri2}, thus giving a well defined commutator
of the intraband position operator with an arbitrary operator $\hat\calo$.
We use Eq. \eqref{chon.98} and \eqref{conmri3}  in the next section.

\bibliographystyle{unsrt}
\bibliography{/Users/sma/Dropbox/docs/academics/master}

\end{document}
