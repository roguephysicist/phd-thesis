%!TEX root = ../../main.tex
\chapter{Useful derivations for the nonlinear surface susceptibility}
\label{app:chi2deriv}
\partialtoc


%%%%%%%%%%%%%%%%%%%%%%%%%%%%%%%%%%%%%%%%%%%%%%%%%%%%%%%%%%%%%%%%%%%%%%%%%%%%%%%%
%%%%%%%%%%%%%%%%%%%%%%%%%%%%%%%%%%%%%%%%%%%%%%%%%%%%%%%%%%%%%%%%%%%%%%%%%%%%%%%%

\section{\texorpdfstring{$\mathbf{r}_{e}$ and $\mathbf{r}_{i}$}{re and ri}}
\label{app:re_ri}

In this appendix, we derive the expressions for the matrix elements of the
electron position operator $\mathbf{r}$. The $r$ representation of the Bloch
states is given by
\begin{equation}\label{bloch}
\psi_{n\mathbf{k}}(\mathbf{r})=\langle\mathbf{r}\vert n\mathbf{k}\rangle =
\sqrt{\frac{\Omega}{8\pi^{3}}}
e^{i\mathbf{k} \cdot \mathbf{r}}u_{n\mathbf{k}}(\mathbf{r}),
\end{equation}
where $u_{n\mathbf{k}}(\mathbf{r}) = u_{n\mathbf{k}}(\mathbf{r} + \mathbf{R})$
is cell periodic, and
\begin{equation}\label{normal}
\int_{\Omega}
u_{n\mathbf{k}}^{*}(\mathbf{r})u_{m\mathbf{k}'}(\mathbf{r})\,d^{3}r
= \delta_{nm}\delta_{\mathbf{\mathbf{k},\mathbf{k}'}},
\end{equation}
and $\Omega$ is the unit cell volume.

The key ingredient in the calculation are the matrix elements of the position
operator $\mathbf{r}$. We start from the basic relation
\begin{equation}\label{nbraket}
\langle n\mathbf{k}\vert m\mathbf{k}'\rangle =
\delta_{nm}\delta(\mathbf{k} - \mathbf{k}'),
\end{equation}
and take its derivative with respect to $\mathbf{k}$ as follows. On one hand,
\begin{equation}\label{ddk1}
\frac{\partial}{\partial\mathbf{k}}
\langle n\mathbf{k}\vert m\mathbf{k}'\rangle =
\delta_{nm}\frac{\partial}{\partial\mathbf{k}}\delta(\mathbf{k} - \mathbf{k}'),
\end{equation}
and on the other,
\begin{equation}\label{dkbraket}
\frac{\partial}{\partial\mathbf{k}}\langle n\mathbf{k}\vert m\mathbf{k}'\rangle
= \frac{\partial}{\partial\mathbf{k}}\int
\langle n\mathbf{k}\vert \mathbf{r}\rangle
\langle \mathbf{r}\vert m\mathbf{k}'\rangle
\,d\mathbf{r} 
= \int
\left(
\frac{\partial}{\partial\mathbf{k}}
\psi^{*}_{n\mathbf{k}}(\mathbf{r})
\right)
\psi_{m\mathbf{k}'}(\mathbf{r})\,d\mathbf{r}.
\end{equation}
The derivative of the wavefunction is simply given by
\begin{equation}\label{dpsi}
\frac{\partial}{\partial\mathbf{k}}\psi^{*}_{n\mathbf{k}}(\mathbf{r})
= \sqrt{\frac{\Omega}{8\pi^{3}}}
\left(
\frac{\partial}{\partial\mathbf{k}}
u^{*}_{n\mathbf{k}}(\mathbf{r})
\right)
e^{-i\mathbf{k}\cdot\mathbf{r}} - i\mathbf{r}\psi^{*}_{n\mathbf{k}}(\mathbf{r}).
\end{equation}
Substituting into Eq. \eqref{dkbraket}, we obtain
\begin{align}\label{dkbraket2}
\frac{\partial}{\partial\mathbf{k}}\langle n\mathbf{k}\vert m\mathbf{k}'\rangle
&= \sqrt{\frac{\Omega}{8\pi^{3}}}\int
\left(
\frac{\partial}{\partial\mathbf{k}}
u^{*}_{n\mathbf{k}}(\mathbf{r})\right)e^{-i\mathbf{k}\cdot\mathbf{r}}
\psi_{m\mathbf{k}'}(\mathbf{r})\,d\mathbf{r} 
- i\int\psi^{*}_{n\mathbf{k}}(\mathbf{r})\mathbf{r}
  \psi_{m\mathbf{k}'}(\mathbf{r})\,d\mathbf{r}\nonumber \\
&= \frac{\Omega}{8\pi^{3}}\int
  e^{-i(\mathbf{k}-\mathbf{k}')\cdot\mathbf{r}}
  \left(
  \frac{\partial}{\partial\mathbf{k}}u^{*}_{n\mathbf{k}}(\mathbf{r})
  \right)
u_{m\mathbf{k}'}(\mathbf{r})\,d\mathbf{r}
- i \langle n\mathbf{k}\vert\hat{\mathbf{r}}\vert m\mathbf{k}'\rangle.
\end{align}
Restricting $\mathbf{k}$ and $\mathbf{k}'$ to the first Brillouin zone, we use
the following result that is valid for any periodic function
$f(\mathbf{r}) = f(\mathbf{r} + \mathbf{R})$,
\begin{align}\label{periodic}
\int e^{i(\mathbf{q}-\mathbf{k})\cdot\mathbf{r}}f(\mathbf{r})\,d^{3}r =
\frac{8\pi^{3}}{\Omega}\delta(\mathbf{q} - \mathbf{k})
\int_{\Omega}f(\mathbf{r})\,d^{3}r,
\end{align}
to finally write \cite{blountSSP62}
\begin{align}\label{dkbraket3}
\frac{\partial}{\partial\mathbf{k}}
\langle n\mathbf{k}\vert m\mathbf{k}'\rangle
&= \delta(\mathbf{k}-\mathbf{k}')\int_{\Omega}
\left(
\frac{\partial}{\partial\mathbf{k}} u^{*}_{n\mathbf{k}}(\mathbf{r})
\right)
u_{m\mathbf{k}}(\mathbf{r})\,d\mathbf{r}
-i\langle n\mathbf{k}\vert\hat{\mathbf{r}}\vert m\mathbf{k}'\rangle.
\end{align}
From
\begin{align}\label{dnm1}
\int_{\Omega}u_{m\mathbf{k}} u^{*}_{n\mathbf{k}}\,d\mathbf{r} = \delta_{nm},
\end{align}
we easily find that
\begin{align}\label{dnm2}
\int_{\Omega}
\left(
\frac{\partial}{\partial\mathbf{k}} u_{m\mathbf{k}}(\mathbf{r})
\right)
u^{*}_{n\mathbf{k}}(\mathbf{r})\,d\mathbf{r}
= -\int_{\Omega}u_{m\mathbf{k}}(\mathbf{r})
\left(
\frac{\partial}{\partial\mathbf{k}} u^{*}_{n\mathbf{k}}(\mathbf{r})
\right)\,d\mathbf{r}.
\end{align}
Therefore, we define
\begin{align}\label{zeta}
\boldsymbol{\xi}_{nm}(\mathbf{k}) \equiv
i\int_{\Omega}u^{*}_{n\mathbf{k}}(\mathbf{r})\nabla_{\mathbf{k}}
u_{m\mathbf{k}}(\mathbf{r})\,d\mathbf{r},
\end{align} 
with $\nabla_{\mathbf{k}} = \partial/\partial\mathbf{k}$. Now, from Eqs.
\eqref{ddk1}, \eqref{dkbraket2}, and  \eqref{zeta}, we have that the matrix
elements of the position operator of the electron are given by
\begin{align}\label{erre}
\langle n\mathbf{k}\vert \hat{\mathbf{r}} \vert m\mathbf{k}'\rangle 
= \delta(\mathbf{k}-\mathbf{k}')\boldsymbol{\xi}_{nm}(\mathbf{k})
+ i\delta_{nm}\nabla_{\mathbf{k}}\delta(\mathbf{k}-\mathbf{k}'),
\end{align}
Then, from Eq. \eqref{erre} and writing $\hat{\mathbf{r}} = \hat{\mathbf{r}}_{e}
+ \hat{\mathbf{r}}_{i}$, with $\hat{\mathbf{r}}_{e}$ ($\hat{\mathbf{r}}_{i}$)
the interband (intraband) part, we obtain that
\begin{align}\label{rnmi}
\langle n\mathbf{k}\vert \hat{\mathbf{r}}_{i} \rangle m\mathbf{k}'\vert
&= \delta_{nm}
\left[
  \delta(\mathbf{k}-\mathbf{k}')\boldsymbol{\xi}_{nn}(\mathbf{k})
+ i\nabla_{\mathbf{k}}\delta(\mathbf{k} - \mathbf{k}')
\right],\\
\langle n\mathbf{k}\vert \hat{\mathbf{r}}_e \rangle m\mathbf{k}'\vert
&= (1 - \delta_{nm})\delta(\mathbf{k}-\mathbf{k}')
   \boldsymbol{\xi}_{nm}(\mathbf{k}).
\label{rnme}
\end{align} 

To proceed, we relate Eq. \eqref{rnme} to the matrix elements of the momentum
operator as follows. For the intraband part, we derive the following general
result,
\begin{equation}\label{conmri}
\begin{split}
\langle n\mathbf{k}\vert
\left[\hat{\mathbf{r}}_{i},\hat{\mathcal{O}}\right]
\vert m\mathbf{k}'\rangle
&= \sum_{\ell,\mathbf{k}''}
\left(
\langle n\mathbf{k}\vert\hat{\mathbf{r}}_{i}\vert\ell\mathbf{k}''\rangle
\langle\ell\mathbf{k}''\vert\hat{\mathcal{O}}\vert m\mathbf{k}'\rangle
-
\langle n\mathbf{k}\vert\hat{\mathcal{O}}\vert\ell\mathbf{k}''\rangle
\langle\ell\mathbf{k}''\vert\hat{\mathbf{r}}_{i}\vert m\mathbf{k}'\rangle
\right)\\
&=
\sum_{\ell}
\left(
\langle n\mathbf{k}\vert\hat{\mathbf{r}}_{i}\vert \ell\mathbf{k}'\rangle
\mathcal{O}_{\ell m}(\mathbf{k}')
-
\mathcal{O}_{n\ell}(\mathbf{k})
\vert \ell\mathbf{k}\rangle\langle \ell\mathbf{k}\vert\hat{\mathbf{r}}_{i}
\vert m\mathbf{k}'\rangle
\right),
\end{split}
\end{equation}
where we have taken $\langle
n\mathbf{k}\vert\hat{\mathcal{O}}\vert\ell\mathbf{k}''\rangle =
\delta(\mathbf{k} - \mathbf{k}'')\mathcal{O}_{n\ell}(\mathbf{k})$. We substitute
Eq. \eqref{rnmi} to obtain
\begin{align}\label{conmri2}
\sum_{\ell}&
\left(
\delta_{n\ell}[
  \delta(\mathbf{k}-\mathbf{k}')\boldsymbol{\xi}_{nn}(\mathbf{k})
+ i\nabla_{\mathbf{k}}\delta(\mathbf{k}-\mathbf{k}')
]\mathcal{O}_{\ell m}(\mathbf{k}')
- \mathcal{O}_{n\ell}(\mathbf{k})\delta_{\ell m}
\left[
  \delta(\mathbf{k}-\mathbf{k}')\boldsymbol{\xi}_{mm}(\mathbf{k})
+ i\nabla_{\mathbf{k}}\delta(\mathbf{k}-\mathbf{k}')
\right]
\right)\nonumber\\
%%%%%%%%%%%%%%%%%%%%%%%%%%%%%%%%%%%%%%%%%%%%%%%%%%%%%%%%%%%%%%%%
&=
\left(
\left[
\delta(\mathbf{k}-\mathbf{k}')\boldsymbol{\xi}_{nn}(\mathbf{k})
+ i\nabla_{\mathbf{k}}\delta(\mathbf{k}-\mathbf{k}')
\right]
\mathcal{O}_{n m}(\mathbf{k}')
- \mathcal{O}_{nm}(\mathbf{k})
\left[
  \delta(\mathbf{k}-\mathbf{k}')\boldsymbol{\xi}_{mm}(\mathbf{k})
+ i\nabla_{\mathbf{k}}\delta(\mathbf{k}-\mathbf{k}')
\right]
\right)\nonumber\\
%%%%%%%%%%%%%%%%%%%%%%%%%%%%%%%%%%%%%%%%%%%%%%%%%%%%%%%%%%%%%%%%
&= \delta(\mathbf{k}-\mathbf{k}')\mathcal{O}_{nm}(\mathbf{k})
\left(
\boldsymbol{\xi}_{nn}(\mathbf{k})-\boldsymbol{\xi}_{mm}(\mathbf{k})
\right)
+ i\mathcal{O}_{n m}(\mathbf{k}')\nabla_{\mathbf{k}}
  \delta(\mathbf{k}-\mathbf{k}')
+ i\delta(\mathbf{k}-\mathbf{k}')\nabla_{\mathbf{k}}
  \mathcal{O}_{n m}(\mathbf{k})
- i\mathcal{O}_{n m}(\mathbf{k}')\nabla_{\mathbf{k}}
  \delta(\mathbf{k}-\mathbf{k}')\nonumber\\
%%%%%%%%%%%%%%%%%%%%%%%%%%%%%%%%%%%%%%%%%%%%%%%%%%%%%%%%%%%%%%%%
&=
i\delta(\mathbf{k}-\mathbf{k}')
\big(
\nabla_{\mathbf{k}}\mathcal{O}_{nm}(\mathbf{k}) - i\mathcal{O}_{nm}(\mathbf{k})
\left(
\boldsymbol{\xi}_{nn}(\mathbf{k}) - \boldsymbol{\xi}_{mm}(\mathbf{k})
\right)
\big)\nonumber\\
%%%%%%%%%%%%%%%%%%%%%%%%%%%%%%%%%%%%%%%%%%%%%%%%%%%%%%%%%%%%%%%%
&\equiv i\delta(\mathbf{k}-\mathbf{k}')(\mathcal{O}_{nm})_{;\mathbf{k}}.
\end{align}
Then,
\begin{equation}\label{conmri3}
\langle n\mathbf{k}\vert
[\hat{\mathbf{r}}_{i},\hat{\mathcal{O}}]\vert m\mathbf{k}'\rangle
= i\delta(\mathbf{k}-\mathbf{k}')(\mathcal{O}_{nm})_{;\mathbf{k}},
\end{equation}   
where
\begin{equation}\label{gendev}
(\mathcal{O}_{nm})_{;\mathbf{k}}=
\nabla_{\mathbf{k}}\mathcal{O}_{nm}(\mathbf{k}) - i\mathcal{O}_{nm}(\mathbf{k})
\left(
\boldsymbol{\xi}_{nn}(\mathbf{k}) - \boldsymbol{\xi}_{mm}(\mathbf{k})
\right),
\end{equation}  
is the generalized derivative of $\mathcal{O}_{nm}$ with respect to
$\mathbf{k}$. Note that the highly singular term
$\nabla_{\mathbf{k}}\delta(\mathbf{k}-\mathbf{k}')$ cancels in Eq.
\eqref{conmri2}, thus giving a well defined commutator of the intraband position
operator with any arbitrary operator $\hat{\mathcal{O}}$.


%%%%%%%%%%%%%%%%%%%%%%%%%%%%%%%%%%%%%%%%%%%%%%%%%%%%%%%%%%%%%%%%%%%%%%%%%%%%%%%%
%%%%%%%%%%%%%%%%%%%%%%%%%%%%%%%%%%%%%%%%%%%%%%%%%%%%%%%%%%%%%%%%%%%%%%%%%%%%%%%%

\section{Matrix elements of
\texorpdfstring{$\mathbf{v}^\mathrm{nl}_{nm}(\mathbf{k})$}{vnl} and 
\texorpdfstring{$\boldsymbol{\mathcal{V}}^{\mathrm{nl},\ell}_{nm}(\mathbf{k})$}
{calVnl}}
\label{app:vnlme}

From Eq. \eqref{conhr}, we have that
\begin{align}\label{vnln.0}
\mathbf{v}^\mathrm{nl}_{nm}(\mathbf{k})
 = \langle n\mathbf{k}\vert\hat{\mathbf{v}}^\mathrm{nl}\vert m\mathbf{k}'\rangle
&= \frac{i}{\hbar}\langle n\mathbf{k}\vert
   \left[\hat{V}^\mathrm{nl},\hat{\mathbf{r}}\right]
   \vert m\mathbf{k}'\rangle\nonumber\\
&= \frac{i}{\hbar}\int
   \langle n\mathbf{k}\vert\mathbf{r}\rangle
   \langle\mathbf{r}\vert
   \left[\hat{V}^\mathrm{nl},\hat{\mathbf{r}}\right]
   \vert\mathbf{r}'\rangle
   \langle\mathbf{r}'\vert m\mathbf{k}'\rangle
   \,d\mathbf{r}\,d\mathbf{r}'\nonumber\\
&= \frac{i}{\hbar}\delta(\mathbf{k}-\mathbf{k}')\int 
   \psi^{*}_{n\mathbf{k}}(\mathbf{r})
   \langle\mathbf{r}\vert
   \left[\hat{V}^\mathrm{nl},\hat{\mathbf{r}}\right]
   \vert\mathbf{r}'\rangle
   \psi_{m\mathbf{k}'}(\mathbf{r}')
   \,d\mathbf{r}\,d\mathbf{r}',
\end{align}   
where $\mathbf{k}=\mathbf{k}'$ due to the fact that the integrand is periodic in
real space, and $\mathbf{k}$ is restricted to the Brillouin Zone. Now,
\begin{align}\label{vnln.1}
\langle\mathbf{r}\vert
\left[\hat{V}^\mathrm{nl},\hat{\mathbf{r}}\right]
\vert\mathbf{r}'\rangle
&= \langle\mathbf{r}\vert\hat{V}^\mathrm{nl}\hat{\mathbf{r}}
   - \hat{\mathbf{r}}\hat{V}^\mathrm{nl}\vert\mathbf{r}'\rangle
 = \langle\mathbf{r}\vert
   \hat{V}^\mathrm{nl}\hat{\mathbf{r}}
   \vert\mathbf{r}'\rangle
   - \langle\mathbf{r}\vert
     \hat{\mathbf{r}}\hat{V}^\mathrm{nl}
     \vert\mathbf{r}'\rangle\nonumber\\
&= \langle\mathbf{r}\vert\hat{V}^\mathrm{nl} \mathbf{r}'\vert\mathbf{r}'\rangle
   - \langle\mathbf{r}\vert\mathbf{r}\hat{V}^\mathrm{nl}\vert\mathbf{r}'\rangle
 = \langle\mathbf{r}\vert\hat{V}^\mathrm{nl}\vert\mathbf{r}'\rangle
   (\mathbf{r}'-\mathbf{r})
 = V^\mathrm{nl}(\mathbf{r},\mathbf{r}')(\mathbf{r}'-\mathbf{r}),
\end{align}
where we used $\hat{r}\langle\mathbf{r}\vert = r\langle\mathbf{r}\vert$,
$\langle\mathbf{r}'\vert\hat{r} = \langle\mathbf{r}\vert r'$, and
$V^\mathrm{nl}(\mathbf{r},\mathbf{r}') =
\langle\mathbf{r}\vert\hat{V}^\mathrm{nl}\vert\mathbf{r}'\rangle$
(Eq. \eqref{ache.3n}). Also, we have the following identity which will be used
shortly,
\begin{align}\label{cn51}
(\nabla_\mathbf{K}+\nabla_{\mathbf{K}'})\frac{1}{\Omega}\int e^{-i\mathbf{K}\cdot\mathbf{r}}V^\mathrm{nl}(\mathbf{r},\mathbf{r}')e^{i\mathbf{K}'\cdot\mathbf{r}'}\,d\mathbf{r}\,d\mathbf{r}'
&= -i \frac{1}{\Omega} \int e^{-i\mathbf{K}\cdot\mathbf{r}} \left(\mathbf{r} V^\mathrm{nl}(\mathbf{r},\mathbf{r}') - V^\mathrm{nl}(\mathbf{r},\mathbf{r}') \mathbf{r}'\right) e^{i\mathbf{K}'\cdot\mathbf{r}'} \,d\mathbf{r}\,d\mathbf{r}'\nonumber\\ (\nabla_\mathbf{K}+\nabla_{\mathbf{K}'}) \langle\mathbf{K}\vert V^\mathrm{nl} \vert\mathbf{K}'\rangle
&= \frac{i}{\Omega} \int e^{-i\mathbf{K}\cdot\mathbf{r}} V^\mathrm{nl}(\mathbf{r},\mathbf{r}') \big(\mathbf{r}'- \mathbf{r} \big) e^{i\mathbf{K}'\cdot\mathbf{r}'} \,d\mathbf{r}\,d\mathbf{r}',
\end{align}
where $\Omega$ is the volume of the unit cell, and we defined
\begin{align}\label{cn52}
V^\mathrm{nl}(\mathbf{K},\mathbf{K}') 
\equiv
\langle\mathbf{K}\vert V^\mathrm{nl} \vert\mathbf{K}'\rangle
= \frac{1}{\Omega} \int e^{-i\mathbf{K}\cdot\mathbf{r}} V^\mathrm{nl}(\mathbf{r},\mathbf{r}') e^{i\mathbf{K}'\cdot\mathbf{r}'} \,d\mathbf{r}\,d\mathbf{r}',
\end{align}
where 
$V^\mathrm{nl}(\mathbf{K}',\mathbf{K}) =V^{\mathrm{nl} *}(\mathbf{K},\mathbf{K}')$, since $V^\mathrm{nl}(\mathbf{r}',\mathbf{r})=V^{\mathrm{nl}*}(\mathbf{r},\mathbf{r}')$ due to the fact that $\hat V^\mathrm{nl}$ is a hermitian operator. Using the plane wave expansion
\begin{align}\label{cn3}
\langle\mathbf{r}\vert n\mathbf{k}\rangle=\psi_{n\mathbf{k}}(\mathbf{r})=\frac{1}{\sqrt{\Omega}} \sum_{\mathbf{G}} A_{n\mathbf{k}}(\mathbf{G})e^{i\mathbf{K}\cdot\mathbf{r}} ,
\end{align}
with $\mathbf{K}=\mathbf{k}+\mathbf{G}$, we obtain from Eq. \eqref{vnln.0} and Eq. \eqref{cn51}, that
\begin{align}\label{vnln.2}
\mathbf{v}^\mathrm{nl}_{nm}(\mathbf{k})
&= \frac{i}{\hbar} \delta(\mathbf{k}-\mathbf{k}') \sum_{\mathbf{G},\mathbf{G}'} A^{*}_{n\mathbf{k}}(\mathbf{G}) A_{m\mathbf{k}'}(\mathbf{G}') \frac{1}{\Omega} \int d\mathbf{r}\,d\mathbf{r}'e^{-i\mathbf{K}\cdot\mathbf{r}} \langle\mathbf{r}\vert [\hat{V}^\mathrm{nl},\hat{\mathbf{r}}] \vert\mathbf{r}'\rangle e^{i\mathbf{K}'\cdot\mathbf{r}'} \nonumber\\ 
&= \frac{1}{\hbar} \delta(\mathbf{k}-\mathbf{k}') \sum_{\mathbf{G},\mathbf{G}'} A^{*}_{n\mathbf{k}}(\mathbf{G}) A_{m\mathbf{k}'}(\mathbf{G}') \frac{i}{\Omega} \int d\mathbf{r}\,d\mathbf{r}'e^{-i\mathbf{K}\cdot\mathbf{r}} V^\mathrm{nl}(\mathbf{r},\mathbf{r}') \big(\mathbf{r}'-\mathbf{r}\big) e^{i\mathbf{K}'\cdot\mathbf{r}'} \nonumber\\
&= \frac{1}{\hbar} \delta(\mathbf{k}-\mathbf{k}') \sum_{\mathbf{G},\mathbf{G}'} A^{*}_{n\mathbf{k}}(\mathbf{G}) A_{m\mathbf{k}'}(\mathbf{G}') (\nabla_\mathbf{K}+\nabla_{\mathbf{K}'}) V^\mathrm{nl}(\mathbf{K},\mathbf{K}') .
\end{align}  

For fully  separable pseudopotentials in the Kleinman-Bylander (KB) form,\cite{mottaCMS10,kleinmanPRL82,adolphPRB96} the matrix elements $\langle\mathbf{K}\vert V^\mathrm{nl} \vert\mathbf{K}'\rangle =V^\mathrm{nl}(\mathbf{K},\mathbf{K}') $ can be readily calculated. \cite{mottaCMS10} Indeed, the Fourier representation assumes the form,\cite{adolphPRB96,gordienkoRPJ04,fuchsCPC99}
\begin{align}\label{ji.1} 
V^\mathrm{nl}_{\mathrm{KB}}(\mathbf{K},\mathbf{K}')  
 &= 
\sum_s e^{i(\mathbf{K}-\mathbf{K}')\cdot\boldsymbol{\tau}_s}
\sum_{l=0}^{l_s}\sum_{m=-l}^{l}E_lF_{lm}^s(\mathbf{K})F_{lm}^{s*}(\mathbf{K}')  
\nonumber\\
 &= 
\sum_s 
\sum_{l=0}^{l_s}\sum_{m=-l}^{l}E_lf_{lm}^s(\mathbf{K})f_{lm}^{s*}(\mathbf{K}')  
,
\end{align} 
with $f^s_{lm}(\mathbf{K})=e^{i\mathbf{K}\cdot\boldsymbol{\tau}_s}F^s_{lm}(\mathbf{K})$, and
\begin{align}\label{ji.2}
F^s_{lm}(\mathbf{K})=\int d\mathbf{r}\,e^{-i\mathbf{K}\cdot\mathbf{r}}
\delta V^S_l(\mathbf{r})
\Phi^\mathrm{ps}_{lm}(\mathbf{r}) 
.
\end{align}
Here $\delta V^S_l(\mathbf{r})$ is the non-local contribution of the ionic
pseudopotential centered at the atomic position $\boldsymbol{\tau}_s$ located in
the unit cell, 
$\Phi^\mathrm{ps}_{lm}(\mathbf{r})$ is the pseudo-wavefunction of the corresponding
atom, while $E_l$ is the so called 
Kleinman-Bylander energy. Further details can be found in
Ref. \cite{fuchsCPC99}.
From Eq. \eqref{ji.1} we find
\begin{align}\label{ji.1n}
(\nabla_\mathbf{K}+\nabla_{\mathbf{K}'})  
V^\mathrm{nl}_{\mathrm{KB}}(\mathbf{K},\mathbf{K}') 
 &= 
\sum_s 
\sum_{l=0}^{l_s}\sum_{m=-l}^{l}E_l 
(\nabla_\mathbf{K}+\nabla_{\mathbf{K}'})   
f_{lm}^s (\mathbf{K})f_{lm}^{s*}(\mathbf{K}') 
\nonumber\\
 &= 
\sum_s 
\sum_{l=0}^{l_s}\sum_{m=-l}^{l}E_l 
\left(\left[\nabla_\mathbf{K} f_{lm}^s (\mathbf{K})\right]f_{lm}^{s*}(\mathbf{K}') 
+
f_{lm}^s (\mathbf{K}) \left[\nabla_{\mathbf{K}'}  f_{lm}^{s*}(\mathbf{K}') \right]
\right),
\end{align}
and using this in Eq. \eqref{vnln.2} leads to
\begin{align}\label{forg}
\mathbf{v}^\mathrm{nl}_{nm}(\mathbf{k})
&=
\frac{1}{\hbar}
\sum_s
\sum_{l=0}^{l_s}\sum_{m=-l}^{l}E_l \sum_{\mathbf{G}\mathbf{G}'}
A^{*}_{n,\vec{k}}(\mathbf{G})A_{n',\vec{k}}(\mathbf{G}')
\times
( \nabla_{\mathbf{K}}f_{lm}^s(\mathbf{K})f_{lm}^{s*}(\mathbf{K}') +
f_{lm}^s(\mathbf{K})\nabla_{\mathbf{K}'}f_{lm}^{s*}(\mathbf{K}') ) \nonumber\\
&
=\frac{1}{\hbar}
 \sum_s \sum_{l=0}^{l_s}\sum_{m=-l}^{l}E_l \Bigg[
\Bigg(\sum_{\mathbf{G}}A^{*}_{n,\vec{k}}(\mathbf{G})\nabla_{\mathbf{K}}f_{lm}^s(\mathbf{K})\Bigg)
\Bigg(\sum_{\mathbf{G}'}A_{n',\vec{k}}(\mathbf{G}')
f_{lm}^{s*}(\mathbf{K}')\Bigg) \nonumber\\
&\qquad\qquad\qquad\qquad\qquad+
\Bigg(\sum_{\mathbf{G}}A^{*}_{n,\vec{k}}(\mathbf{G})
f_{lm}^s(\mathbf{K})\Bigg)\Bigg
(\sum_{\mathbf{G}'}A_{n',\vec{k}}(\mathbf{G}')
\nabla_{\mathbf{K}'}f_{lm}^{s*}(\mathbf{K}')\Bigg) \Bigg]
,
\end{align}
where there are only single sums over $\mathbf{G}$. Above is implemented in the
DP code \cite{olevanoDP}.

Now we derive $\boldsymbol{\mathcal{V}}^{\mathrm{nl},\ell}_{nm}(\mathbf{k})$. First we prove that
\begin{align}\label{cvnl.1}
\sum_{\mathbf{G}}
\vert\mathbf{k}+\mathbf{G}\rangle\langle\mathbf{k}+\mathbf{G}\vert
=1
.
\end{align}
\noindent Proof:
\begin{align}\label{cvnl.2}
\langle n\mathbf{k}\vert 1\vert n'\mathbf{k}\rangle=\delta_{nn'}
,
\end{align}
take
\begin{align}\label{cvnl.3}
\sum_\mathbf{G} \langle n\mathbf{k}\vert \vert\mathbf{k}+\mathbf{G}\rangle\langle\mathbf{k}+\mathbf{G}\vert\vert n'\mathbf{k}\rangle
&=
\int d\mathbf{r}\,d\mathbf{r}' 
\sum_\mathbf{G} \langle n\mathbf{k}\vert 
\vert\mathbf{r}\rangle\langle\mathbf{r}\vert
\vert\mathbf{k}+\mathbf{G}\rangle\langle\mathbf{k}+\mathbf{G}\vert
\vert\mathbf{r}'\rangle\langle\mathbf{r}'\vert
\vert n'\mathbf{k}\rangle
\nonumber\\
&=
\int d\mathbf{r}\,d\mathbf{r}' 
\sum_\mathbf{G} 
\psi^{*}_{n\mathbf{k}}(\mathbf{r}) 
\frac{1}{\sqrt{\Omega}}e^{i(\mathbf{k}+\mathbf{G})\cdot\mathbf{r}}
\frac{1}{\sqrt{\Omega}}e^{-i(\mathbf{k}+\mathbf{G})\cdot\mathbf{r}'}
\psi_{m\mathbf{k}}(\mathbf{r}') 
\nonumber\\
&=
\int d\mathbf{r}\,d\mathbf{r}' 
\psi^{*}_{n\mathbf{k}}(\mathbf{r}) 
\psi_{m\mathbf{k}}(\mathbf{r}') 
\frac{1}{V}
\sum_\mathbf{G} 
e^{i(\mathbf{k}+\mathbf{G})\cdot(\mathbf{r}-\mathbf{r}')}
\nonumber\\
&=
\int d\mathbf{r}\,d\mathbf{r}' 
\psi^{*}_{n\mathbf{k}}(\mathbf{r}) 
\psi_{m\mathbf{k}}(\mathbf{r}') 
\delta(\mathbf{r}-\mathbf{r}')
=
\int d\mathbf{r}
\psi^{*}_{n\mathbf{k}}(\mathbf{r}) 
\psi_{m\mathbf{k}}(\mathbf{r})=\delta_{nn'} ,
\end{align}
and thus Eq. \eqref{cvnl.1} follows. We used
\begin{align}\label{vnl.4}
\langle\mathbf{r}\vert\mathbf{k}+\mathbf{G}\rangle=\frac{1}{\sqrt{\Omega}}e^{i(\mathbf{k}+\mathbf{G})\cdot\mathbf{r}}.
\end{align}

From Eq. \eqref{nl.4}, we would like to calculate
\begin{align}\label{vnl.5}
\boldsymbol{\mathcal{V}}^{\mathrm{nl},\ell}_{nm}(\mathbf{k})=\frac{1}{2}
\langle n\mathbf{k}\vert  
C^{\ell}(z)\mathbf{v}^\mathrm{nl}+\mathbf{v}^\mathrm{nl} C^{\ell}(z)  
\vert m\mathbf{k}\rangle.
\end{align}  
We work out the first term on the right hand side,
\begin{align}\label{vnl.6}
\langle n\mathbf{k}\vert C^{\ell}(z) 
\mathbf{v}^\mathrm{nl}\vert m\mathbf{k}\rangle
&=\sum_{\mathbf{G}}
\langle n\mathbf{k}\vert C^{\ell}(z)\vert\mathbf{k}+\mathbf{G}\rangle
\langle\mathbf{k}+\mathbf{G}\vert\mathbf{v}^\mathrm{nl}\vert m\mathbf{k}\rangle
\nonumber\\
&=\sum_{\mathbf{G}}
\int d\mathbf{r} 
\int d\mathbf{r}' 
\langle n\mathbf{k}\vert 
\vert\mathbf{r}\rangle\langle\mathbf{r}\vert  
C^{\ell}(z) 
\vert\mathbf{r}'\rangle\langle\mathbf{r}'\vert
\vert\mathbf{k}+\mathbf{G}\rangle
\times 
\int d\mathbf{r}'' 
\int d\mathbf{r}''' 
\langle\mathbf{k}+\mathbf{G}\vert
\vert\mathbf{r}''\rangle\langle\mathbf{r}''\vert 
\mathbf{v}^\mathrm{nl} 
\vert\mathbf{r}'''\rangle\langle\mathbf{r}'''\vert
\vert m\mathbf{k}\rangle
\nonumber\\
&=\sum_{\mathbf{G}}
\int d\mathbf{r} 
\int d\mathbf{r}' 
\langle n\mathbf{k}\vert\mathbf{r}\rangle
C^{\ell}(z) 
\delta(\mathbf{r}-\mathbf{r}') 
\langle\mathbf{r}'\vert\mathbf{k}+\mathbf{G}\rangle
\times 
\int d\mathbf{r}'' 
\int d\mathbf{r}''' 
\langle\mathbf{k}+\mathbf{G}\vert\mathbf{r}''\rangle
\langle\mathbf{r}''\vert
\mathbf{v}^\mathrm{nl} 
\vert\mathbf{r}'''\rangle
\langle\mathbf{r}'''\vert m\mathbf{k}\rangle
\nonumber\\
&=\sum_{\mathbf{G}}
\int d\mathbf{r} 
\langle n\mathbf{k}\vert\mathbf{r}\rangle
C^{\ell}(z) 
\langle\mathbf{r}\vert\mathbf{k}+\mathbf{G}\rangle
\times 
\frac{i}{\hbar}
\int d\mathbf{r}'' 
\int d\mathbf{r}''' 
\langle\mathbf{k}+\mathbf{G}\vert\mathbf{r}''\rangle
V^\mathrm{nl}(\mathbf{r}'',\mathbf{r}''')(\mathbf{r}'''-\mathbf{r}'') 
\langle\mathbf{r}'''\vert m\mathbf{k}\rangle
,
\end{align}
where we used Eq. \eqref{vnln.1} and \eqref{conhr}. 
We use Eq. \eqref{cn3}, \eqref{vnl.4} and \eqref{cn51}  to obtain
\begin{align}\label{vnl.7}
\langle n\mathbf{k}\vert C^{\ell}(z) 
\mathbf{v}^\mathrm{nl}\vert m\mathbf{k}\rangle
&=\sum_{\mathbf{G}}
\sum_{\mathbf{G}'}
A^{*}_{n\mathbf{k}}(\mathbf{G}') 
\frac{1}{\Omega}
\int d\mathbf{r} 
e^{-i(\mathbf{k}+\mathbf{G}')\cdot\mathbf{r}}
C^{\ell}(z) 
e^{i(\mathbf{k}+\mathbf{G})\cdot\mathbf{r}}
\nonumber\\
&\times 
\sum_{\mathbf{G}''}
A_{m\mathbf{k}}(\mathbf{G}'') 
\frac{i}{\hbar\Omega}
\int d\mathbf{r}'' 
\int d\mathbf{r}''' 
e^{-i(\mathbf{k}+\mathbf{G})\cdot\mathbf{r}''} 
V^\mathrm{nl}(\mathbf{r}'',\mathbf{r}''')(\mathbf{r}'''-\mathbf{r}'') 
e^{i(\mathbf{k}+\mathbf{G}'')\cdot\mathbf{r}'''}
\nonumber\\
&=
\frac{1}{\hbar}
\sum_{\mathbf{G}}
\sum_{\mathbf{G}'}
A^{*}_{n\mathbf{k}}(\mathbf{G}') 
\delta_{\mathbf{G}_\parallel \mathbf{G}'_\parallel}f_\ell(\mathbf{G}_\perp-\mathbf{G}'_\perp) 
\sum_{\mathbf{G}''}
A_{m\mathbf{k}}(\mathbf{G}'') 
(\nabla_{\mathbf{K}}+\nabla_{\mathbf{K}''}) 
V^\mathrm{nl}(\mathbf{K},\mathbf{K}'') 
,
\end{align}
where
\begin{align}\label{vnl.8}
\frac{1}{\Omega}
\int d\mathbf{r}\, 
C^{\ell}(z) 
e^{i(\mathbf{G}-\mathbf{G}')\cdot\mathbf{r}}
=\delta_{\mathbf{G}_\parallel \mathbf{G}'_\parallel}f_\ell(\mathbf{G}_\perp-\mathbf{G}'_\perp)
,
\end{align} 
and
\begin{align}\label{vnl.9}
f_\ell(g)=\frac{1}{L}\int_{z_\ell-\Delta^b_\ell}^{z_\ell+\Delta^f_\ell} e^{igz}dz  
 ,
\end{align}
where $f^{*}(g)=f(-g)$.
We define
\begin{align}\label{vnl.10}
\mathcal{F}^{\ell}_{n\mathbf{k}}(\mathbf{G}) 
=
\sum_{\mathbf{G}'}
A_{n\mathbf{k}}(\mathbf{G}') 
\delta_{\mathbf{G}_\parallel \mathbf{G}'_\parallel}f_\ell(\mathbf{G}'_\perp-\mathbf{G}_\perp) 
,
\end{align}
and
\begin{align}\label{vnl.11}
\mathcal{H}_{n\mathbf{k}}(\mathbf{G})&=
\sum_{\mathbf{G}'}
A_{n\mathbf{k}}(\mathbf{G}') 
(\nabla_{\mathbf{K}}+\nabla_{\mathbf{K}'}) 
V^\mathrm{nl}(\mathbf{K},\mathbf{K}') 
,
\end{align}
thus we can compactly write,
\begin{align}\label{vn.12}
\langle n\mathbf{k}\vert C^{\ell}(z) 
\mathbf{v}^\mathrm{nl}\vert m\mathbf{k}\rangle
&=
\frac{1}{\hbar}
\sum_{\mathbf{G}}
\mathcal{F}^{\ell*}_{n\mathbf{k}}(\mathbf{G}) 
\mathcal{H}_{m\mathbf{k}}(\mathbf{G}) 
.
\end{align}
Now, the second term of Eq. \eqref{vnl.5}
\begin{align}\label{vnl.12}
\langle n\mathbf{k}\vert 
\mathbf{v}^\mathrm{nl}
C^{\ell}(z) \vert m\mathbf{k}\rangle
&=\sum_{\mathbf{G}}
\langle n\mathbf{k}\vert 
\mathbf{v}^\mathrm{nl} 
\vert\mathbf{k}+\mathbf{G}\rangle
\langle\mathbf{k}+\mathbf{G}\vert C^{\ell}(z)
\vert m\mathbf{k}\rangle
\nonumber\\
&=\sum_{\mathbf{G}}
\int d\mathbf{r}'' 
\int d\mathbf{r}''' 
\langle n\mathbf{k}\vert 
\vert\mathbf{r}''\rangle\langle\mathbf{r}''\vert 
\mathbf{v}^\mathrm{nl} 
\vert\mathbf{r}'''\rangle\langle\mathbf{r}'''\vert
\vert\mathbf{k}+\mathbf{G}\rangle
\nonumber\\
&\times 
\int d\mathbf{r} 
\int d\mathbf{r}' 
\langle\mathbf{k}+\mathbf{G}\vert
\vert\mathbf{r}\rangle\langle\mathbf{r}\vert  
C^{\ell}(z) 
\vert\mathbf{r}'\rangle\langle\mathbf{r}'\vert
\vert m\mathbf{k}\rangle
\nonumber\\
&=\sum_{\mathbf{G}}
\frac{i}{\hbar}
\int d\mathbf{r}'' 
\int d\mathbf{r}''' 
\langle n\mathbf{k}\vert\mathbf{r}''\rangle
V^\mathrm{nl}(\mathbf{r}'',\mathbf{r}''')(\mathbf{r}'''-\mathbf{r}'') 
\langle\mathbf{r}'''\vert\mathbf{k}+\mathbf{G}\rangle
\nonumber\\
&\times 
\int d\mathbf{r} 
\langle\mathbf{k}+\mathbf{G}\vert\mathbf{r}\rangle
C^{\ell}(z) 
\langle\mathbf{r}\vert m\mathbf{k}\rangle
\nonumber\\
&=\sum_{\mathbf{G}}
\sum_{\mathbf{G}'}
A^{*}_{n\mathbf{k}}(\mathbf{G}') 
\frac{i}{\hbar\Omega}
\int d\mathbf{r}'' 
\int d\mathbf{r}''' 
e^{-i(\mathbf{k}+\mathbf{G}')\cdot\mathbf{r}''} 
V^\mathrm{nl}(\mathbf{r}'',\mathbf{r}''')(\mathbf{r}'''-\mathbf{r}'') 
e^{i(\mathbf{k}+\mathbf{G})\cdot\mathbf{r}'''}
\nonumber\\
&\times 
\sum_{\mathbf{G}''}
A_{m\mathbf{k}}(\mathbf{G}'') 
\frac{1}{\Omega}
\int d\mathbf{r} 
e^{-i(\mathbf{k}+\mathbf{G})\cdot\mathbf{r}} 
C^{\ell}(z) 
e^{i(\mathbf{k}+\mathbf{G}'')\cdot\mathbf{r}}
\nonumber\\
&=
\frac{1}{\hbar}
\sum_{\mathbf{G}}
\sum_{\mathbf{G}'}
A^{*}_{n\mathbf{k}}(\mathbf{G}') 
(\nabla_\mathbf{K}+\nabla_{\mathbf{K}'})
V^\mathrm{nl}(\mathbf{K}',\mathbf{K})
\nonumber\\
&\times 
\sum_{\mathbf{G}''}
A_{m\mathbf{k}}(\mathbf{G}'') 
\delta_{\mathbf{G}_\parallel \mathbf{G}''_\parallel}f_\ell(\mathbf{G}''_\perp-\mathbf{G}_\perp)
\nonumber\\
&=
\frac{1}{\hbar}
\sum_{\mathbf{G}}
\mathcal{H}^{*}_{n\mathbf{k}}(\mathbf{G}) 
\mathcal{F}^{\ell}_{m\mathbf{k}}(\mathbf{G}) 
.
\end{align}
Therefore Eq. \eqref{vnl.5} is compactly given by
\begin{align}\label{vnl.13}
\boldsymbol{\mathcal{V}}^{\mathrm{nl},\ell}_{nm}(\mathbf{k})=\frac{1}{2 \hbar}
\sum_{\mathbf{G}}
\left(
\mathcal{F}^{\ell*}_{n\mathbf{k}}(\mathbf{G}) 
\mathcal{H}_{m\mathbf{k}}(\mathbf{G}) 
+
\mathcal{H}^{*}_{n\mathbf{k}}(\mathbf{G}) 
\mathcal{F}^{\ell}_{m\mathbf{k}}(\mathbf{G}) 
\right)
.
\end{align} 
For fully  separable pseudopotentials in the Kleinman-Bylander (KB) form
\cite{mottaCMS10,kleinmanPRL82,adolphPRB96}, we can use Eq. \eqref{ji.1n} and
evaluate above expression, that we have implemented in the DP code
\cite{olevanoDP}. Explicitly,
\begin{align}\label{vnl.14}
\boldsymbol{\mathcal{V}}^{\mathrm{nl},\ell}_{nm}(\mathbf{k})=\frac{1}{2 \hbar}
\sum_s\sum_{l=0}^{l_s}\sum_{m=-l}^{l}E_l \nonumber\\
\Bigg[ \Bigg(\sum_{{\mathbf{G}''}}
\nabla_{{\mathbf{G}''}}f_{lm}^s({\mathbf{G}''})
\sum_{\mathbf{G}}A^{*}_{n{\mathbf{k}}}(\mathbf{G})\delta_{\mathbf{G}_{||}{\mathbf{G}''}_{||}}
f_\ell(G_z-G''_z) \Bigg)  
\Bigg( \sum_{{\mathbf{G}'}}A_{m{\mathbf{k}}}({\mathbf{G}'})f_{lm}^{s*}({\mathbf{K}'}) \Bigg) \nonumber\\
%
+ \Bigg(\sum_{{\mathbf{G}''}}
f_{lm}^s({\mathbf{G}''})\sum_{\mathbf{G}}A^{*}_{n{\mathbf{k}}}(\mathbf{G})\delta_{\mathbf{G}_{||}{\mathbf{G}''}_{||}}
f_\ell(G_z-G''_z) \Bigg)  
\Bigg( \sum_{{\mathbf{G}'}}A_{m{\mathbf{k}}}({\mathbf{G}'})\nabla_{{\mathbf{K}'}}f_{lm}^{s*}({\mathbf{K}'}) \Bigg)  \nonumber\\
+
\Bigg(\sum_{\mathbf{G}}A^{*}_{n{\mathbf{k}}}(\mathbf{G})\nabla_{\mathbf{G}}f_{lm}^s(\mathbf{G})\Bigg)\Bigg(
\sum_{{\mathbf{G}''}}
f_{lm}^{s*}({\mathbf{G}''})\sum_{{\mathbf{G}'}}A_{m{\mathbf{k}}}({\mathbf{G}'})\delta_{{\mathbf{G}'}_{||}{\mathbf{G}''}_{||}}
f_\ell(G''_z-G'_z) \Bigg) \nonumber\\ 
+
\Bigg(\sum_{\mathbf{G}}A^{*}_{n{\mathbf{k}}}(\mathbf{G})f_{lm}^s(\mathbf{G})
\Bigg) \Bigg(
\sum_{{\mathbf{G}''}}\nabla_{{\mathbf{G}''}}f_{lm}^{s*}({\mathbf{G}''})
\sum_{{\mathbf{G}'}}A_{m{\mathbf{k}}}({\mathbf{G}'})\delta_{{\mathbf{G}'}_{||}{\mathbf{G}''}_{||}}
f_\ell(G''_z-G'_z) \Bigg) \Bigg].\nonumber\\ 
\end{align} 
For a full slab calculation, equivalent to a bulk calculation,
$C^{\ell}(z)=1$ and then 
$f_\ell(g) =\delta_{g0}$, and Eq. \eqref{vnl.14}
reduces to 
Eq. \eqref{forg}. 


%%%%%%%%%%%%%%%%%%%%%%%%%%%%%%%%%%%%%%%%%%%%%%%%%%%%%%%%%%%%%%%%%%%%%%%%%%%%%%%%
%%%%%%%%%%%%%%%%%%%%%%%%%%%%%%%%%%%%%%%%%%%%%%%%%%%%%%%%%%%%%%%%%%%%%%%%%%%%%%%%

\section{Explicit expressions for
\texorpdfstring{$\mathcal{V}^{a,\ell}_{nm}(\mathbf{k})$ and
$\mathcal{C}^{\ell}_{nm}(\mathbf{k})$}{Vnm and Cnm}}
\label{app:calpcalc}

Expanding the wave function in planewaves, we obtain
\begin{equation}\label{eni.1}
\psi_{n\mathbf{k}}(\mathbf{r})
= \sum_{\mathbf{G}}
A_{n\mathbf{k}}(\mathbf{G})e^{i(\mathbf{k}+\mathbf{G})\cdot\mathbf{r}},
\end{equation}
where $\{\mathbf{G}\}$ are the reciprocal basis vectors satisfying
$e^{\mathbf{R}\cdot\mathbf{G}} = 1$, $\{\mathbf{R}\}$ are the translation
vectors in real space, and $A_{n\mathbf{k}}(\mathbf{G})$ are the expansion
coefficients. Using $m_{e}\mathbf{v} = -i\hbar\boldsymbol{\nabla}$ into Eqs.
\eqref{vcali} and \eqref{nl.3} we obtain \cite{mendozaPRB06},
\begin{equation}\label{eni.2}
\boldsymbol{\mathcal{V}}^{\ell}_{nm}(\mathbf{k})=
\frac{\hbar}{2m_{e}}\sum_{\mathbf{G},\mathbf{G}'}
A^*_{n\mathbf{k}}(\mathbf{G}')A_{m\mathbf{k}}(\mathbf{G})
(2\mathbf{k}+\mathbf{G}+\mathbf{G}')
\delta_{\mathbf{G}_\parallel \mathbf{G}'_\parallel}f_{\ell}(G_\perp-G'_\perp),
\end{equation}   
with
\begin{align}\label{eni.3}
f_{\ell}(g) = \frac{1}{L}
\int_{z_{\ell}-\Delta^{b}_{\ell}}^{z_{\ell}+\Delta^{f}_{\ell}}
e^{igz}\,dz,
\end{align}
where the reciprocal lattice vectors $\mathbf{G}$ are decomposed into components
parallel to the surface $\mathbf{G}_{\parallel}$, and perpendicular to the
surface $G_{\perp}\hat{z}$, so that $\mathbf{G} = \mathbf{G}_{\parallel} +
G_{\perp}\hat{z}$. Likewise we obtain that
\begin{align*}
\mathcal{C}_{nm}(\mathbf{k})
&=  \int\psi^{*}_{n\mathbf{k}}(\mathbf{r})f(z)
    \psi_{m\mathbf{k}}(\mathbf{r})\,d\mathbf{r}\\
&=  \sum_{\mathbf{G},\mathbf{G^{\prime}}}
    A^{*}_{n\mathbf{k}}(\mathbf{G^{\prime}})
    A_{m\mathbf{k}}(\mathbf{G})
   \int f(z)e^{-i(\mathbf{G}-\mathbf{G^{\prime}})\cdot\mathbf{r}}\,d\mathbf{r}\\
&=  \sum_{\mathbf{G},\mathbf{G^{\prime}}}
    A^{*}_{n\mathbf{k}}(\mathbf{G^{\prime}})
    A_{m\mathbf{k}}(\mathbf{G})\,
    \underbrace{
    \int e^{-i(\mathbf{G}_{\parallel}-\mathbf{G}^{\prime}_{\parallel})
    \cdot\mathbf{R}_{\parallel}}\,d\mathbf{R}_{\parallel}
    }_{\delta_{\mathbf{G}_{\parallel}\mathbf{G}^{\prime}_{\parallel}}}
    \,\underbrace{
    \int e^{-i(g-g^{\prime})z}f(z)\,dz
    }_{f_{\ell}(G_{\perp} - G^{\prime}_{\perp})},
\end{align*}
which we can express compactly as,
\begin{align}\label{eni.4}
\mathcal{C}^{\ell}_{nm}(\mathbf{k})=
\sum_{\mathbf{G},\mathbf{G}'} A^*_{n\mathbf{k}}(\mathbf{G}')  A_{m\mathbf{k}}(\mathbf{G})
\delta_{\mathbf{G}_\parallel \mathbf{G}'_\parallel} 
f_{\ell}(G_\perp-G'_\perp)
.
\end{align}  
The double summation over the $\mathbf{G}$ vectors can be efficiently done by
creating a pointer array to identify all the plane-wave coefficients associated
with the same $G_{\parallel}$. We take $z_{\ell}$ at the center of an atom that
belongs to layer $\ell$, so the equations above give the $\ell$-th atomic-layer
contribution to the optical response \cite{mendozaPRB06}.

If $\mathcal{C}^{\ell}(z) = 1$ from Eqs. \eqref{eni.2} and \eqref{eni.4}, we
recover the well known results
\begin{align}\label{eni.21}
v_{nm}(\mathbf{k})
&= \frac{\hbar}{m_{e}}\sum_{\mathbf{G}}
A^*_{n\mathbf{k}}(\mathbf{G})A_{m\mathbf{k}}(\mathbf{G})(\mathbf{k}+\mathbf{G}),
\nonumber\\
\mathcal{C}^{\ell}_{nm} &= \delta_{nm},
\end{align}  
since for this case, $f_{\ell}(g) = \delta_{g0}$.

We remark that $\boldsymbol{\mathcal{V}}^{\ell}_{nm}(\mathbf{k})$ of Eq.
\eqref{eni.2} does not contain the contribution coming from the scissors
operator. As commented in the paragraph after Eq. \eqref{vopii},
$\boldsymbol{\mathcal{V}}^{\Sigma,\ell}_{nm}(\mathbf{k})\ne
(\omega^{\Sigma}_{nm}/\omega_{nm})
\boldsymbol{\mathcal{V}}^{\mathrm{LDA},\ell}_{nm}(\mathbf{k})$ and
$\boldsymbol{\mathcal{V}}^{\Sigma,\ell}_{nn}(\mathbf{k})\ne
\boldsymbol{\mathcal{V}}^{\mathrm{LDA},\ell}_{nn}(\mathbf{k})$, relations that
are correct whether or not the contribution of $\mathbf{v}^\mathrm{nl}$ is taken
into account. We will learn how to correctly implement the scissors correction
in the next section, Sec. \ref{app:calvs}.


%%%%%%%%%%%%%%%%%%%%%%%%%%%%%%%%%%%%%%%%%%%%%%%%%%%%%%%%%%%%%%%%%%%%%%%%%%%%%%%%

\subsection{Time-reversal relations}
The following relations hold for time-reversal symmetry.
\begin{equation*}
A_{n\mathbf{k}}^{*}(\mathbf{G}) = A_{n-\mathbf{k}}(\mathbf{G}),
\end{equation*}
\begin{equation*}
\mathbf{P}_{n\ell }(-\mathbf{k}) 
=   \hbar\sum_{\mathbf{G}}
    A_{n-\mathbf{k}}^{*}(\mathbf{G})
    A_{\ell -\mathbf{k}}(\mathbf{G})(-\mathbf{k}+\mathbf{G}),
\end{equation*}
\begin{equation*}
(\mathbf{G}\rightarrow-\mathbf{G})
=   -\hbar\sum_{\mathbf{G}}
    A_{n\mathbf{k}}(\mathbf{G})
    A_{\ell\mathbf{k}}^{*}(\mathbf{G})(\mathbf{k}+\mathbf{G})
=   -\mathbf{P}_{\ell n}(\mathbf{k}),
\end{equation*}
\begin{align*}
\mathcal{C}_{nm} (L;-\mathbf{k}) 
&=  \sum_{\mathbf{G}_{\parallel},g,g'}
    A_{n-\mathbf{k}}^{*}(\mathbf{G}_{\parallel},g)
    A_{m-\mathbf{k}}(\mathbf{G}_{\parallel},g')
    f_{\ell}(g-g') \\
&=  \sum_{\mathbf{G}_{\parallel},g,g'}
    A_{n\mathbf{k}} (\mathbf{G}_{\parallel},g)
    A_{m\mathbf{k}}^{*} (\mathbf{G}_{\parallel},g')
    f_{\ell}(g-g') \\
&=  \mathcal{C}_{mn}(L;\mathbf{k}).
\end{align*}


%%%%%%%%%%%%%%%%%%%%%%%%%%%%%%%%%%%%%%%%%%%%%%%%%%%%%%%%%%%%%%%%%%%%%%%%%%%%%%%%
%%%%%%%%%%%%%%%%%%%%%%%%%%%%%%%%%%%%%%%%%%%%%%%%%%%%%%%%%%%%%%%%%%%%%%%%%%%%%%%%

\section{
\texorpdfstring{$\mathcal{V}^{\Sigma,\mathrm{a},\ell}_{nm}$}{Vnm} and 
\texorpdfstring{
$\left(\mathcal{V}^{\Sigma,\mathrm{a},\ell}_{nm}\right)_{;k^\mathrm{b}}$}
{(Vnm);kb}}
\label{app:calvs}

From Eq. \eqref{vopii}
\begin{equation}\label{a.1}
\left(\mathcal{V}^{\Sigma,\mathrm{a},\ell}_{nm}\right)_{;k^\mathrm{b}}
= \left(\mathcal{V}^{\mathrm{LDA},\mathrm{a},\ell}_{nm}\right)_{;k^\mathrm{b}}
+ \left(\mathcal{V}^{{\mathcal{S}},\mathrm{a},\ell}_{nm}\right)_{;k^\mathrm{b}}.
\end{equation} 
For the LDA term we have
\begin{equation}\label{a.2}
\mathcal{V}^{\mathrm{LDA},\mathrm{a},\ell}_{nm}
= \frac{1}{2}
\left(v^{\mathrm{LDA},\mathrm{a}}\mathcal{C}^{\ell}
      + \mathcal{C}^{\ell} v^{\mathrm{LDA},\mathrm{a}}\right)_{nm} 
= \frac{1}{2}\sum_{q}
\left(
  v^{\mathrm{LDA},\mathrm{a}}_{nq}\mathcal{C}^{\ell}_{qm}
+ \mathcal{C}^{\ell}_{nq} v^{\mathrm{LDA},\mathrm{a}}_{qm}
\right)
\end{equation}
and
\begin{align}\label{a.2a}
\left(\mathcal{V}^{\mathrm{LDA},\mathrm{a}}_{nm}\right)_{;k^\mathrm{b}}
&= \frac{1}{2}\sum_{q}\left(  
  v^{\mathrm{LDA},\mathrm{a}}_{nq}\mathcal{C}^{\ell}_{qm}
+ \mathcal{C}^{\ell}_{nq}v^{\mathrm{LDA},\mathrm{a}}_{qm}
\right)_{;k^\mathrm{b}}\nonumber\\
&= \frac{1}{2}\sum_{q}\left(
  (v^{\mathrm{LDA},\mathrm{a}}_{nq})_{;k^\mathrm{b}}\mathcal{C}^{\ell}_{qm}
+  v^{\mathrm{LDA},\mathrm{a}}_{nq}(\mathcal{C}^{\ell}_{qm})_{;k^\mathrm{b}}
+ (\mathcal{C}^{\ell}_{nq})_{;k^\mathrm{b}} v^{\mathrm{LDA},\mathrm{a}}_{qm}
+ \mathcal{C}^{\ell}_{nq} (v^{\mathrm{LDA},\mathrm{a}}_{qm})_{;k^\mathrm{b}}
\right),
\end{align}   
where we omit the $\mathbf{k}$ argument in all terms. From Eq. \eqref{vnln.2} we
know that $\mathbf{v}^\mathrm{nl}_{nm}(\mathbf{k})$ can be readily calculated,
and from Sec. \ref{app:calpcalc}, both $v^{a}_{nm}$ and $\mathcal{C}_{nm}^{\ell}$
are also known quantities. Thus, $\mathbf{v}^\mathrm{LDA}_{nm}(\mathbf{k})$ is
known, and in turn $\mathcal{V}^{\mathrm{LDA},\mathrm{a},\ell}_{nm}$ is also
known. For the generalized derivative
$(\mathbf{v}^\mathrm{LDA}_{nm}(\mathbf{k}))_{;\mathbf{k}}$ we use Eq.
\eqref{chon.98} to write
\begin{align}\label{a.3}
(v^{\mathrm{LDA},\mathrm{a}}_{nm})_{;k^\mathrm{b}}
&= im_{e}(\omega^\mathrm{LDA}_{nm}r^\mathrm{a}_{nm})_{;k^\mathrm{b}}\nonumber\\
&= im_{e}(\omega^\mathrm{LDA}_{nm})_{;k^\mathrm{b}} r^\mathrm{a}_{nm}
 + im_{e}\omega^\mathrm{LDA}_{nm}(r^\mathrm{a}_{nm})_{;k^\mathrm{b}}\nonumber\\
&= im_{e}\Delta^b_{nm}r^\mathrm{a}_{nm}
 + im_{e}\omega^\mathrm{LDA}_{nm}(r^\mathrm{a}_{nm})_{;k^\mathrm{b}}
   \quad\mathrm{for}\quad n\ne m,
\end{align} 
where we used Eq \eqref{eli.13} and $(r^\mathrm{a}_{nm})_{;k^\mathrm{b}}$, from
Eq. \eqref{na_rgendevn}.

Likewise for the scissored term,
\begin{equation}\label{a.3b}
\mathcal{V}^{{\mathcal{S}},\mathrm{a},\ell}_{nm}
= \frac{1}{2}\left(
  v^{{\mathcal{S}},\mathrm{a}}\mathcal{C}^{\ell}
+ \mathcal{C}^{\ell} v^{{\mathcal{S}},\mathrm{a}}
\right)_{nm}
= \frac{1}{2}\sum_{q}\left(  
  v^{{\mathcal{S}},\mathrm{a}}_{nq}\mathcal{C}^{\ell}_{qm}
+ \mathcal{C}^{\ell}_{nq}v^{{\mathcal{S}},\mathrm{a}}_{qm}
\right)
\end{equation}
and
\begin{align}\label{a.3c2}
\left(\mathcal{V}^{{\mathcal{S}},\mathrm{a}}_{nm}\right)_{;k^\mathrm{b}}
&=
\frac{1}{2}\sum_{q}\left(  
v^{{\mathcal{S}},\mathrm{a}}_{nq}\mathcal{C}^{\ell}_{qm}+\mathcal{C}^{\ell}_{nq} v^{{\mathcal{S}},\mathrm{a}}_{qm}
\right)_{;k^\mathrm{b}}\nonumber\\
&= \frac{1}{2}\sum_{q}\left(
  (v^{{\mathcal{S}},\mathrm{a}}_{nq})_{;k^\mathrm{b}}\mathcal{C}^{\ell}_{qm}
+ v^{{\mathcal{S}},\mathrm{a}}_{nq}(\mathcal{C}^{\ell}_{qm})_{;k^\mathrm{b}}
+ (\mathcal{C}^{\ell}_{nq})_{;k^\mathrm{b}} v^{{\mathcal{S}},\mathrm{a}}_{qm}
+ \mathcal{C}^{\ell}_{nq} (v^{{\mathcal{S}},\mathrm{a}}_{qm})_{;k^\mathrm{b}}
\right),
\end{align}   
where $v^{{\mathcal{S}},\mathrm{a}}_{nm}(\mathbf{k})$ is given in Eq.
\eqref{chon.2} and $(v^{{\mathcal{S}},\mathrm{a}}_{nm})_{;k^\mathrm{b}}$ is
given in Eq. (A6) of Ref. \cite{cabellosPRB09} as
\begin{align}\label{choni.1}
(v^{{\mathcal{S}},\mathrm{a}}_{nm})_{;k^\mathrm{b}} = 
i\Delta f_{mn}(r^\mathrm{a}_{nm})_{;k^\mathrm{b}}.
\end{align}

To evaluate $(\mathcal{C}^{\ell}_{nm})_{;k^\mathrm{a}}$, we use the fact that as
$\mathcal{C}^{\ell}(z)$ is only a function of the $z$ coordinate, its commutator
with $\mathbf{r}$ is zero. Then,
\begin{align}\label{a.4}
\langle n\mathbf{k}\vert
\left[r^\mathrm{a},\mathcal{C}^{\ell}(z)\right]
\vert m\mathbf{k}'\rangle
&= \langle n\mathbf{k}\vert
\left[r_{e}^\mathrm{a},\mathcal{C}^{\ell}(z)\right]
\vert m\mathbf{k}'\rangle
+ \langle n\mathbf{k}\vert
\left[r_{i}^\mathrm{a},\mathcal{C}^{\ell}(z)\right]
\vert m\mathbf{k}'\rangle = 0.
\end{align} 
The interband part reduces to,
\begin{align}\label{a.5}
\left[r_{e}^\mathrm{a},\mathcal{C}^{\ell}(z)\right]_{nm}
&= \sum_{q\mathbf{k}''}
\left(
  \langle n\mathbf{k}\vert r_{e}^\mathrm{a}\vert q\mathbf{k}''\rangle
  \langle q\mathbf{k}''\vert\mathcal{C}^{\ell}(z)\vert m\mathbf{k}'\rangle
- \langle n\mathbf{k}\vert\mathcal{C}^{\ell}(z)\vert q\mathbf{k}''\rangle
  \langle q\mathbf{k}''\vert r_{e}^\mathrm{a}\vert m\mathbf{k}'\rangle
\right)\nonumber\\
&= \sum_{q\mathbf{k}''}
\delta(\mathbf{k}-\mathbf{k}'')
\delta(\mathbf{k}'-\mathbf{k}'')
\left(
  (1-\delta_{qn})\xi_{nq}^\mathrm{a}\mathcal{C}^{\ell}_{qm}
- (1-\delta_{qm})\mathcal{C}^{\ell}_{nq}\xi_{qm}^\mathrm{a}
\right)\nonumber\\
&= \delta(\mathbf{k}-\mathbf{k}')
\left(
\sum_{q}
\left(
  \xi_{nq}^\mathrm{a}\mathcal{C}^{\ell}_{qm}
- \mathcal{C}^{\ell}_{nq}\xi_{qm}^\mathrm{a}
\right)
+ \mathcal{C}^{\ell}_{nm}(\xi_{mm}^\mathrm{a}-\xi_{nn}^\mathrm{a})
\right),
\end{align}
where we used Eq. \eqref{rnme}, and the $\mathbf{k}$ and $z$ dependence is
implicitly understood. From Eq. \eqref{conmri3} the intraband part is,
\begin{equation}\label{a.6}
\langle n\mathbf{k}\vert
\left[\hat{\mathbf{r}}_{i},\mathcal{C}^{\ell}(z)\right]
\vert m\mathbf{k}'\rangle
= i\delta(\mathbf{k}-\mathbf{k}')(\mathcal{C}^{\ell}_{nm})_{;\mathbf{k}},
\end{equation}
then from Eq. \eqref{a.4}
\begin{equation}
\left(
(\mathcal{C}^{\ell}_{nm})_{;\mathbf{k}} - i\sum_{q}
\left(\xi_{nq}^\mathrm{a}\mathcal{C}^{\ell}_{qm}
     - \mathcal{C}^{\ell}_{nq}\xi_{qm}^\mathrm{a}\right)
- i\mathcal{C}^{\ell}_{nm}(\xi_{mm}^\mathrm{a}-\xi_{nn}^\mathrm{a})
\right)
i\delta(\mathbf{k}-\mathbf{k}')
= 0,
\end{equation}
which we can simplify,
\begin{align}\label{a.7}
\left(\mathcal{C}^{\ell}_{nm}\right)_{;\mathbf{k}}
&=
i\sum_{q}
\left(
  \xi_{nq}^\mathrm{a}\mathcal{C}^{\ell}_{qm}
- \mathcal{C}^{\ell}_{nq}\xi_{qm}^\mathrm{a}
\right)
+ i\mathcal{C}^{\ell}_{nm}(\xi_{mm}^\mathrm{a}-\xi_{nn}^\mathrm{a})\nonumber\\
%%%%%%%%%%%%%%%%%%%%%%%%%%%%%%%%%%%%%%%%%%%%%%
&= i\sum_{q\ne nm}
\left(
  \xi_{nq}^\mathrm{a}\mathcal{C}^{\ell}_{qm}
- \mathcal{C}^{\ell}_{nq}\xi_{qm}^\mathrm{a}
\right)
+ i\left(
  \xi_{nn}^\mathrm{a}\mathcal{C}^{\ell}_{nm}
- \mathcal{C}^{\ell}_{nn}\xi_{nm}^\mathrm{a}
\right)_{q=n}
+ i\left(
  \xi_{nm}^\mathrm{a}\mathcal{C}^{\ell}_{mm}
- \mathcal{C}^{\ell}_{nm}\xi_{mm}^\mathrm{a}
\right)_{q=m}
+ i\mathcal{C}^{\ell}_{nm}(\xi_{mm}^\mathrm{a}-\xi_{nn}^\mathrm{a})\nonumber\\
%%%%%%%%%%%%%%%%%%%%%%%%%%%%%%%%%%%%%%%%%%%%%%
&= i\sum_{q\ne nm}\left(
  \xi_{nq}^\mathrm{a}\mathcal{C}^{\ell}_{qm}
- \mathcal{C}^{\ell}_{nq}\xi_{qm}^\mathrm{a}
\right)
+ i\xi_{nm}^\mathrm{a}(\mathcal{C}^{\ell}_{mm}-\mathcal{C}^{\ell}_{nn})
\nonumber\\
%%%%%%%%%%%%%%%%%%%%%%%%%%%%%%%%%%%%%%%%%%%%%%
&= i\sum_{q\ne nm}
\left(
  r_{nq}^\mathrm{a}\mathcal{C}^{\ell}_{qm}
- \mathcal{C}^{\ell}_{nq}r_{qm}^\mathrm{a} 
\right) 
+ ir_{nm}^\mathrm{a}(\mathcal{C}^{\ell}_{mm}-\mathcal{C}^{\ell}_{nn})\nonumber\\
%%%%%%%%%%%%%%%%%%%%%%%%%%%%%%%%%%%%%%%%%%%%%%
&= i\left(
  \sum_{q\ne n}r_{nq}^\mathrm{a}\mathcal{C}^{\ell}_{qm}
- \sum_{q\ne m}\mathcal{C}^{\ell}_{nq}r_{qm}^\mathrm{a}
\right) 
+ ir_{nm}^\mathrm{a}(\mathcal{C}^{\ell}_{mm}-\mathcal{C}^{\ell}_{nn}),
\end{align} 
since in $\xi_{nm}^\mathrm{a}$ we have that $n\ne m$, and we can replace it with
$r^\mathrm{a}_{nm}$. The matrix elements $\mathcal{C}^{\ell}_{nm}(\mathbf{k})$
are calculated in Sec. \ref{app:calpcalc}.

For the general case of
\begin{equation}\label{a.8}
\langle n\mathbf{k}\vert
\left[\hat{r}^\mathrm{a},\hat{\mathcal{G}}(\mathbf{r},\mathbf{p})\right]
\vert m\mathbf{k}'\rangle
= \mathcal{U}_{nm}(\mathbf{k}),
\end{equation}
we can generalize our result to a more general expression, 
\begin{equation}\label{a.9}
({\mathcal{G}}_{nm}(\mathbf{k}))_{;k^\mathrm{a}}
= \mathcal{U}_{nm}(\mathbf{k})
+ i\sum_{q\ne(nm)}
\left(
  r_{nq}^\mathrm{a} (\mathbf{k}){\mathcal{G}}_{qm}(\mathbf{k})
- {\mathcal{G}}_{nq}(\mathbf{k})r_{qm}^\mathrm{a} (\mathbf{k})
\right)
+ ir_{nm}^\mathrm{a}(\mathbf{k})
({\mathcal{G}}_{mm}(\mathbf{k})-{\mathcal{G}}_{nn}(\mathbf{k})).
\end{equation}


%%%%%%%%%%%%%%%%%%%%%%%%%%%%%%%%%%%%%%%%%%%%%%%%%%%%%%%%%%%%%%%%%%%%%%%%%%%%%%%%
%%%%%%%%%%%%%%%%%%%%%%%%%%%%%%%%%%%%%%%%%%%%%%%%%%%%%%%%%%%%%%%%%%%%%%%%%%%%%%%%

\section{Generalized derivative 
\texorpdfstring{$(\omega_{n}(\mathbf{k}))_{;\mathbf{k}}$}{(wn);k}}
\label{app:gwk}

We obtain the generalized derivative $(\omega_{n}(\mathbf{k}))_{;\mathbf{k}}$.
We start from
\begin{equation}\label{a_conH0}
\langle n\mathbf{k}\vert \hat{H}^{\Sigma}_{0} \vert m\mathbf{k}'\rangle
= \delta_{nm}\delta(\mathbf{k}-\mathbf{k}')\hbar\omega^{\Sigma}_{m}(\mathbf{k}),
\end{equation}
then for $n = m$, Eq. \eqref{gendev} yields
\begin{align}\label{a_genderH0}
(H^{\Sigma}_{0,nn})_{;\mathbf{k}}
&= \nabla_{\mathbf{k}}H^{\Sigma}_{0,nn}(\mathbf{k})
 - iH^{\Sigma}_{0,nn}(\mathbf{k})
\left(\boldsymbol{\xi}_{nn}(\mathbf{k})
     -\boldsymbol{\xi}_{nn}(\mathbf{k})\right)\nonumber\\
&= \hbar\nabla_{\mathbf{k}}\omega^{\Sigma}_{m}(\mathbf{k}),
\end{align}
and from Eq. \eqref{conmri3}, 
\begin{equation}\label{a_rih0}
\langle n\mathbf{k}\vert
\left[\hat{\mathbf{r}}_i,\hat{H}_{0}\right]
\vert m\mathbf{k}\rangle
= i\delta_{nm}\hbar(\omega^{\Sigma}_{m}(\mathbf{k}))_{;\mathbf{k}}
= i\delta_{nm}\hbar\nabla_{\mathbf{k}}\omega^{\Sigma}_{m}(\mathbf{k}),
\end{equation}
so
\begin{equation}\label{a_wgendev}
\left(\omega^{\Sigma}_{n}(\mathbf{k})\right)_{;\mathbf{k}}
= \nabla_{\mathbf{k}}\omega^{\Sigma}_{n}(\mathbf{k}).
\end{equation}
From Eq. \eqref{vop},
\begin{equation}\label{a_hr}
\langle n\mathbf{k}\vert
\left[\hat{\mathbf{r}},\hat{H}^{\Sigma}_{0}\right]
\vert m\mathbf{k}\rangle
= i\hbar\mathbf{v}^{\Sigma}_{nm}(\mathbf{k}),
\end{equation}
and substituting Eqs. \eqref{a_rih0} and \eqref{a_hr} into
\begin{equation}\label{a_hrt}
\langle n\mathbf{k}\vert
\left[\hat{\mathbf{r}},\hat{H}^{\Sigma}_{0}\right]
\vert m\mathbf{k}\rangle 
= \langle n\mathbf{k}\vert
  \left[\hat{\mathbf{r}}_i,\hat{H}^{\Sigma}_{0}\right]
  \vert m\mathbf{k}\rangle
+ \langle n\mathbf{k}\vert
  \left[\hat{\mathbf{r}}_e,\hat{H}^{\Sigma}_{0}\right]
  \vert m\mathbf{k}\rangle,
\end{equation}
we get
\begin{equation}\label{a_hrt2}
i\hbar\mathbf{v}^{\Sigma}_{nm}(\mathbf{k})
= i\delta_{nm}\hbar\nabla_{\mathbf{k}}\omega^{\Sigma}_{m}(\mathbf{k})
  + \omega^{\Sigma}_{mn}\mathbf{r}_{e,nm}(\mathbf{k}).
\end{equation}
For $m = n$, we have that
\begin{align}\label{a_gradw}
\nabla_{\mathbf{k}}\omega^{\Sigma}_{n}(\mathbf{k})
    &= \mathbf{v}^{\Sigma}_{nn}(\mathbf{k})\nonumber\\
\nabla_{\mathbf{k}}(\omega{^\mathrm{LDA}}_{n}(\mathbf{k})
    + \frac{\Sigma}{\hbar}(1-f_{n}))
&= \nabla_{\mathbf{k}}\omega{^\mathrm{LDA}}_{n}(\mathbf{k})\nonumber\\
\nabla_{\mathbf{k}}\omega{^\mathrm{LDA}}_{n}(\mathbf{k})
    &= \mathbf{v}^{\Sigma}_{nn}(\mathbf{k}),
\end{align}
where we use Eq. \eqref{chon.78}. However, from Eq. \eqref{chon.2},
$v^{\mathcal{S}}_{nn} = 0$ so $\mathbf{v}^{\Sigma}_{nn}=v{^\mathrm{LDA}}_{nn}$.
Thus, from Eq. \eqref{a_wgendev}
\begin{align}\label{a_gradw2}
(\omega^{\Sigma}_{n}(\mathbf{k}))_{;k^{\mathrm{a}}}
= (\omega{^\mathrm{LDA}}_{n}(\mathbf{k}))_{;k^{\mathrm{a}}}
= v^{\mathrm{LDA},\mathrm{a}}_{nn}(\mathbf{k}),
\end{align}
which is the same for the LDA and scissored Hamiltonians;
$\mathbf{v}_{nn}{^\mathrm{LDA}}(\mathbf{k})$ are the LDA velocities of the
electron in state $\vert n\mathbf{k}\rangle$.


\stopcontents[chapters]
