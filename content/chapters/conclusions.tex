%!TEX root = ../../main.tex
\chapter{Final Remarks}\label{chap:conclusions}

I have presented a formulation to calculate the surface second-harmonic (SSH)
susceptibility tensor
$\boldsymbol{\chi}_{\mathrm{surface}}(-2\omega;\omega,\omega)$, using the length
gauge formalism and within the independent particle approximation (IPA). It
includes on equal footing: (i) the scissors correction, (ii) the contribution of
the non-local part of the pseudopotentials, and (iii) the cut function. I also
revised the 3-layer model for the SSHG yield including the effects of multiple
reflections from both the SH and fundamental fields. In this 3-layer model,  the
nonlinear polarization, $\boldsymbol{\mathcal{P}}(2\omega)$, and the fundamental
fields are taken within a small layer $\ell$ below the surface of the material.
This model reproduces key spectral features and yields an intensity closer to
the experiment for all cases of $\mathcal{R}_{\mathrm{iF}}$. I consider it an
improvement over the much reviewed 2-layer model \cite{mizrahiJOSA88}, and it
comes with very little added computational expense. Additionally, we have
compared these two models with other models where both
$\boldsymbol{\mathcal{P}}(2\omega)$ and the fundamental fields are evaluated at
different regions inside the material.

We have used a Si(001)$2\times 1$ surface to confirm that our scheme correctly
obtains the surface response as we confirm that
$\chi_{\mathrm{half-slab}}^{xxx}(-2\omega;\omega,\omega) \approx
\chi_{\mathrm{full-slab}}^{xxx}(-2\omega;\omega,\omega)$. Although one can in
principle increase the number of atomic layers, $\mathbf{k}$-points, etc. to
improve even further on the similarity of the half-slab and full-slab results,
we have chosen a good compromise between accuracy and the burden and time of the
computations. We describe the effect of the independent inclusion of the three
effects mentioned above in the calculation of
$\boldsymbol{\chi}_{\mathrm{surface}}(-2\omega;\omega,\omega)$. The scissors
correction shifts the spectrum to higher energies though the shifting is not
rigid and mixes the $1\omega$ and $2\omega$ resonances, and has a strong
influence in the line-shape, as for the case of bulk semiconductors
\cite{luppiJCP10, luppiPRB10, leitsmannPRB05}. The cut function allows us to
extract unequivocally $\chi^{xxx}_{2\times 1}(-2\omega;\omega,\omega)$. The
effects of the nonlocal part of the pseudopotentials keeps the same line-shape
of $|\chi^{xxx}_{2\times 1}(-2\omega;\omega,\omega)|$, but reduces the value of
by 15-20\%.

I then calculated the SSHG yield for the Si(111)(1$\times$1):H surface, which is
experimentally well characterized. This study affords us an interesting view of
both the theoretical and experimental aspects of SSHG studies. On the
theoretical side, we have shown the importance of using relaxed atomic positions
to more accurately calculate the nonlinear susceptibility tensor. The intensity
of these spectra is greatly improved when compared to previous works
\cite{mejiaPRB02}. Concerning the experiments, we show that surface preparation
and quality are important for better results. The approach for calculating the
SSHG yield presented here finds closer agreement with surfaces that are freshly
prepared with little or no oxidation, and with measurements taken at low
temperatures. We have neglected local field and excitonic effects. Although
these are important factors in the optical response of a semiconductor, their
efficient calculation is theoretically and numerically challenging and still
under debate \cite{beyond}. This merits further study but is beyond the scope of
this thesis. Nevertheless, the inclusion of aforementioned contributions in our
scheme opens the unprecedented possibility to study surface SHG with more
versatility and more accurate results. Overall, this newly implemented framework
for calculating $\boldsymbol{\chi}_{\mathrm{surface}}(-2\omega;\omega,\omega)$
and $\mathcal{R}$ focused on the Si(001)$2\times 1$ and Si(111)(1$\times$1):H
surfaces provides a compelling benchmark for SSHG studies. I am confident that
this work can be applied directly to many other surfaces of interest.

\stopcontents[chapters]
