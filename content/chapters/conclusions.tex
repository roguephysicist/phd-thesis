%!TEX root = ../../main.tex
\chapter{Final Remarks}\label{chap:conclusions}
\partialtoc


\section{Conclusions}


We have presented a formulation to calculate the surface second-harmonic (SSH)
susceptibility tensor $\boldsymbol{\chi}(-2\omega;\omega,\omega)$, using the
length gauge formalism and within the independent particle approximation (IPA).
It includes on equal footing: (i) the scissors correction, (ii) the contribution
of the non-local part of the pseudopotentials, and (iii) the cut function. We
have used a Si(001)$2\times 1$ surface to confirm that our scheme correctly
obtains the surface response as we confirm that
$\chi_{\mathrm{half-slab}}^{xxx}(-2\omega;\omega,\omega) \approx
\chi_{\mathrm{full-slab}}^{xxx}(-2\omega;\omega,\omega) . $ Although one can in
principle increase the number of atomic layers, $\mathbf{k}$-points, etc. to
improve even further on the similarity of the half-slab and full-slab results,
we have chosen a good compromise between accuracy and the burden and time of the
computations. We describe the effect of the independent inclusion of the three
effects mentioned above in the calculation of
$\boldsymbol{\chi}(-2\omega;\omega,\omega)$. The scissors correction shifts the
spectrum to higher energies though the shifting is not rigid and mixes the
$1\omega$ and $2\omega$ resonances, and has a strong influence in the
line-shape, as for the case of bulk
semiconductors.\cite{luppiJCP10,luppiPRB10,leitsmannPRB05} The cut function
allows us to extract unequivocally $\chi^{xxx}_{2\times
1}(-2\omega;\omega,\omega)$. The effects of the nonlocal part of the
pseudopotentials keeps the same line-shape of $|\chi^{xxx}_{2\times
1}(-2\omega;\omega,\omega)|$, but reduces the value of by 15-20\%. The $xxx$
component of $\boldsymbol{\chi}_{2\times 1}(-2\omega;\omega,\omega)$, can not be
experimentally isolated, however in a forthcoming publication we will compare
our formulation against experimental results. We have neglected local field and
excitonic effects. Although these are important factors in the optical response
of a semiconductor, their efficient calculation is theoretically and numerically
challenging and still under debate.\cite{beyond} This merits further study but
is beyond the scope of this paper. Nevertheless, the inclusion of aforementioned
contributions in our scheme opens the unprecedented possibility to study surface
SHG with more versatility and more accurate results.


We have presented new \emph{ab initio} LDA calculations for SSHG that are in
good quantitative agreement with experimental SSHG spectra for the
Si(111)(1$\times$1):H surface. These calculations include contributions not
previously considered in a single formulation, to wit, (i) the scissors
correction, (ii) the contribution of the nonlocal part of the pseudopotentials,
and (iii) the cut function used to extract the surface response, all within the
independent particle approximation. We also revised the 3-layer model for the
SSHG yield where the nonlinear polarization,
$\boldsymbol{\mathcal{P}}(2\omega)$, and the fundamental fields are taken within
a small layer $\ell$ below the surface of the material. This model reproduces
key spectral features and yields an intensity closer to the experiment for all
cases of $\mathcal{R}_{\mathrm{iF}}$. We consider it an upgrade over the much
reviewed 2-layer model\cite{mizrahiJOSA88}, and it comes with very little added
computational expense. Additionally, we have compared these two models with
another definition of the 2-layer model, where both
$\boldsymbol{\mathcal{P}}(2\omega)$ and the fundamental fields are considered
inside the bulk of the material. We found that this model yields an intensity
lower than the 3-layer model, but far closer than the 2-layer-vacuum model.
Lineshape is very similar between the 3-layer and 2-layer-bulk models.
Therefore, we consider that the 3-layer model offers the closest comparison to
experiment, while the 2-layer-bulk model offers a reasonable compromise between
the 3-layer and 2-layer-vacuum models.

This study affords us an interesting view of both the theoretical and
experimental aspects of SSHG studies. On the theoretical side, we have shown the
importance of using relaxed atomic positions to more accurately calculate the
nonlinear susceptibility tensor. The intensity of these spectra is greatly
improved when compared to previous works.\cite{mejiaPRB02} We also postulate
that the lack of local field effects in the theory is a serious shortcoming, but
in this case, it only affects two of the
$\boldsymbol{\chi}(-2\omega;\omega,\omega)$ components.

Concerning the experiments, we show that surface preparation and quality are
important for better results. The approach for calculating the SSHG yield
presented here finds closer agreement with surfaces that are freshly prepared
with little or no oxidation, and with measurements taken at low temperatures.

Overall, this newly implemented framework for calculating
$\boldsymbol{\chi}(-2\omega;\omega,\omega)$ and $\mathcal{R}$ focused on the
well-known Si(111)(1$\times$1):H surface provides a compelling benchmark for
SSHG studies. We are confident that this work can be applied directly to many
other surfaces of interest.

\section{Future Work}
I think that every work of experimental science has its fair share of setbacks,
complications, and difficulties. Sometimes the work itself can be very difficult
or even dangerous. Other times, the work is so cutting edge that problems have
to be solved as they come without the help of literature. Regardless of the
scope of the work, \emph{all} experimentation is very touch-and-go business --
you arm yourself with the best tools available for the job and hope for the
best. This work had its share of complications and setbacks, chief amongst these
was the constant breakdown of lasers in both countries. Then, the poor quality
of the samples which only came to light after they were in place and ready to be
measured. Lastly, the lack of information about the samples did not allow for
the systematic study needed to get the most out of this project.

Fortunately, Stephen Jay Gould once said that, ``Honorable errors do not count
as failures in science, but as seeds for progress in the quintessential activity
of correction.'' With that in mind I summarize what was learned from this.

First, the XP2SHG/SFG technique is fairly unique and specialized even amongst
groups that are dedicated to surface optics and nonlinear optical techniques.
Learning how this technique works and how it is used will be invaluable for
future work in this field. Actually having seen it in use, and then using it for
myself in the company of the people who pioneered it was a rewarding and
educational experience.

Second, while the results were inconclusive, the types of measurements done on
these types of samples are new and unexplored. There is much work to be done
with these kinds of materials and I hope that this work can serve as a starting
point for other interested scientists. I have no doubt in my mind that better
samples would have yielded excellent new results.

Lastly, this entire work helped broaden my knowledge of nonlinear optics in
general, as well as the many experimental techniques used everyday by scientists
everywhere. Even so, I only possess a very small portion of the ``big picture''
needed to understand every aspect of this work. There is still a lot to be
learned about surface optics and nonlinear techniques and I hope that this work,
at the very least, will pique the readers' interest on these topics.

\stopcontents[chapters]
