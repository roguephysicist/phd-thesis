%!TEX root = ../../main.tex
\chapter{Surface Second-Harmonic Generation Yield}
\minitoc

In this chapter I walk the reader through the considerations for developing the
three layer (3-layer) model for the SSHG yield, and then derive explicit
expressions for each of the four polarization configurations for the incoming
and outgoing fields.


%%%%%%%%%%%%%%%%%%%%%%%%%%%%%%%%%%%%%%%%%%%%%%%%%%%%%%%%%%%%%%%%%%%%%%%%%%%%%%%%
%%%%%%%%%%%%%%%%%%%%%%%%%%%%%%%%%%%%%%%%%%%%%%%%%%%%%%%%%%%%%%%%%%%%%%%%%%%%%%%%

\section{The three layer model for the SSHG yield}

In this section, we will derive the formulas required for the calculation of the
SSHG yield, defined by
\begin{equation}\label{eq:uno}
\mathcal{R}(\omega)=\frac{I(2\omega)}{I^2(\omega)},
\end{equation}
with the intensity in the MKS system is given by \cite{boyd, sutherland}
\begin{equation}\label{eq:dos}
I(\omega)=2n(\omega)\epsilon_{0}c|E(\omega)|^2,
\end{equation}
where $n(\omega)=\sqrt{\epsilon(\omega)}$ is the index of refraction with
$\epsilon(\omega)$ as the dielectric function, $\epsilon_{0}$ is the vacuum
permittivity, and $c$ the speed of light in vacuum.

There are several ways to calculate $R$, one of which is the procedure followed
by Cini \cite{ciniPRB91}. This approach calculates the nonlinear susceptibility
and at the same time the radiated fields. However, I present an alternative
derivation based on the work of Mizrahi and Sipe \cite{mizrahiJOSA88}, since the
derivation of the 3-layer model is straightforward. In this scheme, the surface
is represented by three regions or layers. The first layer is the vacuum region
(denoted by $v$) with a dielectric function $\epsilon_{v}(\omega)=1$ from where
the fundamental electric field $\mathbf{E}_{v}(\omega)$ impinges on the
material. The second layer is a thin layer (denoted by $\ell$) of thickness $d$
characterized by a dielectric function $\epsilon_{\ell}(\omega)$. It is in this
layer where the SHG takes place. The third layer is the bulk region denoted by
$b$ and characterized by $\epsilon_{b}(\omega)$. Both the vacuum and bulk layers
are semi-infinite (see Fig. \ref{fig:MR3layer2w}).

To model the electromagnetic response of the 3-layer model, we follow Ref.
\cite{mizrahiJOSA88} and assume a polarization sheet of the form
\begin{equation}\label{eq:m31}
\mathbf{P}(\mathbf{r},t) = \boldsymbol{\mathcal{P}}
  e^{i\boldsymbol{\kappa}\cdot\mathbf{R}}e^{-i\omega t}\delta(z - z_{\beta}) 
+ \mathrm{c.c.},
\end{equation}
where $\mathbf{R}=(x,y)$, $\boldsymbol{\kappa}$ is the component of the wave
vector $\boldsymbol{\nu}^{\phantom{a}}_{\beta}$ parallel to the surface, and
$z_{\beta}$ is the position of the sheet within medium $\beta$. In Ref.
\cite{sipeJOSAB87} they demonstrate that the solution of the Maxwell equations
for the radiated fields $E_{\beta,p\pm}$, and $E_{\beta,s}$ with
$\mathbf{P}(\mathbf{r},t)$ as a source can be written, at points $z\neq 0$, as
\begin{equation}\label{eq:r2}
(E_{\beta,p\pm},E_{\beta,s}) = 
(\frac{\gamma i\tilde{\omega}^2}{\tilde{w}_{\beta}}
\,\hat{\mathbf{p}}_{\beta\pm}\cdot\boldsymbol{\mathcal{P}},
\frac{\gamma i\tilde{\omega}^2}{\tilde{w}_{\beta}}
\,\hat{\mathbf{s}}\cdot\boldsymbol{\mathcal{P}}),
\end{equation} 
where $\gamma=2\pi$ in CGS units or $\gamma=1/2\epsilon_{0}$ in MKS units, and
$\tilde{\omega}=\omega/c$. Also, $\hat{\mathbf{s}}$ and
$\hat{\mathbf{p}}_{\beta\pm}$ are the unitary vectors for the $s$ and $p$
polarizations of the radiated field, respectively. The $\pm$ refers to upward
($+$) or downward ($-$) direction of propagation within medium $\beta$, as shown
in Fig. \ref{fig:MR3layer2w}. Also,
$\tilde{w}^{\phantom{a}}_{\beta}(\omega)=\tilde{\omega}w^{\phantom{a}}_{\beta}$,
where
\begin{equation}\label{eq:r3}
w^{\phantom{a}}_{\beta}(\omega) = 
\big(\epsilon^{\phantom{a}}_{\beta}(\omega) - \sin^{2}\theta_{0}\big)^{1/2},
\end{equation}
where $\theta_{0}$ is the angle of incidence of $\mathbf{E}_{v}(\omega)$, and
\begin{equation}\label{eq:r4}
\hat{\mathbf{p}}^{\phantom{A}}_{\beta\pm}(\omega) =
  \frac{\kappa(\omega)\hat{\mathbf{z}}\mp 
  \tilde{w}^{\phantom{A}}_{\beta}(\omega)\hat{\boldsymbol{\kappa}}} 
  {\tilde{\omega} n^{\phantom{A}}_{\beta}(\omega)}
= \frac{\sin\theta_{0}\hat{\mathbf{z}}\mp 
  w^{\phantom{A}}_{\beta}(\omega)\hat{\boldsymbol{\kappa}}} 
  {n^{\phantom{A}}_{\beta}(\omega)},
\end{equation}
where $\kappa(\omega)=|\boldsymbol{\kappa}|=\tilde{\omega}\sin\theta_{0}$,
$n^{\phantom{A}}_{\beta}(\omega)=\sqrt{\epsilon^{\phantom{A}}_{\beta}(\omega)}$
is the index of refraction of medium $\beta$, and $z$ is the direction
perpendicular to the surface that points towards the vacuum. If we consider the
plane of incidence along the $\boldsymbol{\kappa}z$ plane, then
\begin{equation}\label{mc1}
\hat{\boldsymbol{\kappa}} = \cos\phi\hat{\mathbf{x}} + \sin\phi\hat{\mathbf{y}},
\end{equation}
and
\begin{equation}\label{mmc2}
\hat{\mathbf{s}} = -\sin\phi\hat{\mathbf{x}} + \cos\phi\hat{\mathbf{y}},
\end{equation}
where $\phi$ the angle with respect to the $x$ axis.

In the three-layer model the nonlinear polarization responsible for the second
harmonic generation (SHG) is immersed in the thin $\beta=\ell$ layer, and is
given by
\begin{equation}\label{eq:tres}
\mathcal{P}_{i}(2\omega)=
\left\{
\begin{array}{cc}
\chi^{ijk}(2\omega)E_{j}(\omega)E_{k}(\omega)             & \text{(CGS units)}\\
\epsilon_{0}\chi^{ijk}(2\omega)E_{j}(\omega)E_{k}(\omega) & \text{(MKS units)}
\end{array}
\right.,
\end{equation}
where $\boldsymbol{\chi}(2\omega)$ is the dipolar surface nonlinear
susceptibility tensor, and the Cartesian indices $i,j,k$ are summed over if
repeated. Also, $\chi^{ijk}(2\omega) = \chi^{ikj}(2\omega)$ is the intrinsic
permutation symmetry due to the fact that SHG is degenerate in $E_{j}(\omega)$
and $E_{k}(\omega)$. As in Ref. \cite{mizrahiJOSA88}, we consider the
polarization sheet (Eq. \eqref{eq:m31}) to be oscillating at some frequency
$\omega$ for expressing Eqs. \eqref{eq:r2}-\eqref{mmc2}. However, in the following
we find it convenient to use $\omega$ exclusively to denote the fundamental
frequency and $\boldsymbol{\kappa}$ to denote the component of the incident wave
vector parallel to the surface. The generated nonlinear polarization is
oscillating at $\Omega = 2\omega$ and will be characterized by a wave vector
parallel to the surface $\mathbf{K} = 2\boldsymbol{\kappa}$. We can carry over
Eqs. \eqref{eq:m31}-\eqref{mmc2} simply by replacing the lowercase symbols
($\omega,\tilde{\omega},\boldsymbol{\kappa},n^{\phantom{A}}_{\beta},
\tilde{w}^{\phantom{A}}_{\beta},w^{\phantom{A}}_{\beta},
\hat{\mathbf{p}}_{\beta\pm},\hat{\mathbf{s}}$) with uppercase symbols 
($\Omega,\tilde{\Omega},\mathbf{K},N^{\phantom{A}}_{\beta},
\tilde{W}^{\phantom{A}}_{\beta},W^{\phantom{A}}_{\beta},
\hat{\mathbf{P}}_{\beta\pm},\hat{\mathbf{S}}$), all evaluated at $2\omega$. 
Of course, we always have that $\hat{\mathbf{S}}=\hat{\mathbf{s}}$.

\begin{figure}
\centering 
\includegraphics[scale=.5]{content/figures/diag-3layer_MR_2w}
\caption{Sketch of the three layer model for SHG. The vacuum region ($v$) is on
top with $\epsilon_{v}=1$; the layer $\ell$, of thickness $d = d_{1} + d_{2}$,
is characterized with $\epsilon_{\ell}(\omega)$, and it is where the SH
polarization sheet $\boldsymbol{\mathcal{P}}(2\omega)$ is located at $z_{\ell} =
d_{1}$; The bulk $b$ is described with $\epsilon_{b}(\omega)$. The arrows point
along the direction of propagation, and the $p$-polarization unit vector,
$\hat{\mathbf{P}}_{\ell -(+)}$, along the downward (upward) direction is denoted
with a thick arrow. The $s$-polarization unit vector $\hat{\mathbf{s}}$, points
out of the page. The fundamental field $\mathbf{E}(\omega)$ is incident from the
vacuum side along the $z\hat{\boldsymbol{\kappa}}$-plane, with $\theta_{0}$ its
angle of incidence and $\boldsymbol{\nu}_{v-}$ its wave vector.
$\Delta\varphi_{i}$ denote the phase difference of the multiple reflected beams
with respect to the first vacuum transmitted beam (dashed-red arrow), where the
dotted lines are perpendicular to this beam (see the text for details).}
\label{fig:MR3layer2w}
\end{figure}

From Fig. \ref{fig:MR3layer2w}, we observe the propagation of the SH field that
it is refracted at the layer-vacuum interface ($\ell v$), and  reflected
multiple times from the layer-bulk ($\ell b$) and layer-vacuum ($\ell v$)
interfaces. Thus, we can define
\begin{equation}\label{eq:r5}
\mathbf{T}^{\ell v}
= \hat{\mathbf{s}}T_{s}^{\ell v}\hat{\mathbf{s}} 
+ \hat{\mathbf{P}}_{v+}T_{p}^{\ell v} \hat{\mathbf{P}}_{\ell +},
\end{equation}
as the transmission tensor for the $\ell v$ interface,
\begin{equation}\label{eq:r6}
\mathbf{R}^{\ell b}
= \hat{\mathbf{s}}R_{s}^{\ell b}\hat{\mathbf{s}}
+ \hat{\mathbf{P}}_{\ell +}R_{p}^{\ell b} \hat{\mathbf{P}}_{\ell -},
\end{equation} 
as the reflection tensor for the $\ell b$ interface, and
\begin{equation}\label{eq:r6b}
\mathbf{R}^{\ell v}
= \hat{\mathbf{s}}R_{s}^{\ell v}\hat{\mathbf{s}}
+ \hat{\mathbf{P}}_{\ell -}R_{p}^{\ell v} \hat{\mathbf{P}}_{\ell +},
\end{equation} 
as the reflection tensor for the $\ell v$ interface. The Fresnel factors in
uppercase letters, $T^{ij}_{s,p}$ and $R^{ij}_{s,p}$, are evaluated at $2\omega$
from the following well known formulas
\begin{equation}\label{eq:e.f1}
\begin{split}
t_{s}^{ij}(\omega) &=
\frac{2k_{i}(\omega)}{k_{i}(\omega)+k_{j}(\omega)},
\quad\quad  
t_{p}^{ij}(\omega) =
\frac{2k_{i}(\omega)\sqrt{\epsilon_{i}(\omega)\epsilon_j(\omega)}}
     {k_{i}(\omega)\epsilon_{j}(\omega)+k_{j}(\omega)\epsilon_{i}(\omega)},\\
r_{s}^{ij}(\omega) &=
\frac{k_{i}(\omega) - k_{j}(\omega)}
     {k_{i}(\omega) + k_{j}(\omega)},
\quad\quad 
r_{p}^{ij}(\omega) =
\frac{k_{i}(\omega)\epsilon_{j}(\omega) - k_{j}\epsilon_{i}(\omega)}
     {k_{i}(\omega)\epsilon_{j}(\omega) + k_{j}(\omega)\epsilon_{i}(\omega)}. 
\end{split}
\end{equation}
With these expressions we can show that
\begin{equation}\label{eq:mf}
\begin{split}
1 + r^{\ell b}_{s} &= t^{\ell b}_{s},\\
1 + r^{\ell b}_{p} &= \frac{n_{b}}{n_{\ell}}t^{\ell b}_{p},\\
1 - r^{\ell b}_{p} &= \frac{n_{\ell}}{n_{b}}\frac{w_{b}}{w_{\ell}}t^{\ell b}_{p},\\
t^{\ell v}_{p} &= \frac{w_{\ell}}{w_{v}}t^{v\ell}_{p},\\
t^{\ell v}_{s} &= \frac{w_{\ell}}{w_{v}}t^{v\ell}_{s}.
\end{split}
\end{equation}


%%%%%%%%%%%%%%%%%%%%%%%%%%%%%%%%%%%%%%%%%%%%%%%%%%%%%%%%%%%%%%%%%%%%%%%%%%%%%%%%

\subsection{Multiple SHG reflections}

The SH field $\mathbf{E}(2\omega)$ radiated by the SH polarization
$\boldsymbol{\mathcal{P}}(2\omega)$ will radiate directly into the vacuum and
the bulk, where it will be reflected back at the layer-bulk interface into the
thin layer. This beam will be transmitted and reflected multiple times, as shown
in Fig. \ref{fig:MR3layer2w}. As the two beams propagate, a phase difference
will develop between them according to
\begin{equation}\label{eq:m99}
\begin{split}
\Delta\varphi_{m} 
&= \tilde{\Omega}
\Big(
(L_{3} + L_{4} + 2mL_{5})N_{\ell}
 - \big(L_{2}N_{\ell} + (L_{1} + mL_{6})N_{v}\big)
\Big)\\
&= \delta_{0} + m\delta\quad m=0,1,2,\ldots,
\end{split}
\end{equation}
where
\begin{equation}\label{delta0}
\delta_{0} =
8\pi\left(\frac{d_{2}}{\lambda_{0}}\right)
\sqrt{n^{2}_{\ell}(2\omega)-\sin^{2}\theta_{0}},
\end{equation}
and
\begin{equation}\label{delta}
\delta = 8\pi
\left(\frac{d}{\lambda_{0}}\right)
\sqrt{n^{2}_{\ell}(2\omega)-\sin^{2}\theta_{0}},
\end{equation}
where $\lambda_{0}$ is the wavelength of the fundamental field in the vacuum,
$d$ the thickness of layer $\ell$, and $d_{2}$ is the distance of
$\boldsymbol{\mathcal{P}}(2\omega)$ from the $\ell b$ interface (see Fig.
\ref{fig:MR3layer2w}). We see that $\delta_{0}$ is the phase difference of the
first and second transmitted beams, and $m\delta$ that of the first and third
($m = 1$), first and fourth ($m = 2$), and so on. Note that the thickness $d$ of
the layer $\ell$ enters through the phase $\delta$, and the position $d_{2}$ of
the nonlinear polarization sheet $\mathbf{P}(\mathbf{r},t)$, Eq. \eqref{eq:m31},
enters through $\delta_{0}$. In particular $d_{2}$ could be used as a variable
to study its effects on the SSHG yield $\mathcal{R}(2\omega)$.

To take into account the multiple reflections of the generated SH field in the
layer $\ell$, we proceed as follows. I include the algebra for the $p$-polarized
SH field, and the $s$-polarized field could be worked out along the same steps.
The multiple reflected $\mathbf{E}_{p}(2\omega)$ field is given by
\begin{equation*}\label{eq:E2wcomplete}
\begin{split}
\mathbf{E}(2\omega) 
&= E_{p+}(2\omega)\mathbf{T}^{\ell v}\cdot\hat{\mathbf{P}}_{\ell +}
 + E_{p-}(2\omega)\mathbf{T}^{\ell v}
\cdot\mathbf{R}^{\ell b}\cdot\hat{\mathbf{P}}_{\ell-}e^{i\Delta\varphi_{0}}\\
&+ E_{p-}(2\omega)\mathbf{T}^{\ell v}
\cdot\mathbf{R}^{\ell b}\cdot\mathbf{R}^{\ell v}
\cdot\mathbf{R}^{\ell b}\cdot\hat{\mathbf{P}}_{\ell-}e^{i\Delta\varphi_{1}}
\\
&+ E_{p-}(2\omega)\mathbf{T}^{\ell v}
\cdot\mathbf{R}^{\ell b}\cdot\mathbf{R}^{\ell v}
\cdot\mathbf{R}^{\ell b}\cdot\mathbf{R}^{\ell v}
\cdot\mathbf{R}^{\ell b}\cdot\hat{\mathbf{P}}_{\ell-}e^{i\Delta\varphi_{2}}
+\cdots
\end{split}
\end{equation*} 
\begin{equation}
= E_{p+}(2\omega)\mathbf{T}^{\ell v}\cdot\hat{\mathbf{P}}_{\ell +}
+ E_{p-}(2\omega) \mathbf{T}^{\ell v}
\cdot\sum_{m=0}^\infty  
\big(
\mathbf{R}^{\ell b}\cdot\mathbf{R}^{\ell v} 
e^{i\delta}\Big)^m 
\cdot\mathbf{R}^{\ell b}\cdot\hat{\mathbf{P}}_{\ell-}e^{i\delta_{0}}.
\end{equation}

From Eqs. \eqref{eq:r5} - \eqref{eq:r6b} it is easy to show that
\begin{equation*}\label{eq:m1}
\mathbf{T}^{\ell v}\cdot
\Big(\mathbf{R}^{\ell b}\cdot\mathbf{R}^{\ell v}\Big)^{n}\cdot
\mathbf{R}^{\ell b}
= \hat{\mathbf{s}}T^{\ell v}_{s}
  \Big(R^{\ell b}_{s}R^{\ell v}_{s}\Big)^{n}R^{\ell b}_{s}\hat{\mathbf{s}}
+ \hat{\mathbf{P}}_{v+}T^{\ell v}_{p}\Big(R^{\ell b}_{p}R^{\ell v}_{p}\Big)^n 
  R^{\ell b}_{p} 
\hat{\mathbf{P}}_{\ell-},
\end{equation*}
then,
\begin{equation}\label{eq:E2wreduced}
\mathbf{E}(2\omega) 
=  hat{\mathbf{P}}_{\ell +}T^{\ell v}_{p}
\Big(
E_{p+}(2\omega) +
\frac{R^{\ell b}_{p}e^{i\delta_{0}}}{1 + R^{v\ell}_{p}R^{\ell b}_{p}e^{i\delta}}
E_{p-}(2\omega) 
\Big),
\end{equation}
where we used $R^{ij}_{s,p}=-R^{ji}_{s,p}$. Using Eq. \eqref{eq:r2}, we can
readily write
\begin{equation}\label{eq:mr8}
\mathbf{E}(2\omega) =
\frac{\gamma i\tilde{\Omega}}{W_{\ell}}\mathbf{H}_{\ell}\cdot
\boldsymbol{\mathcal{P}}(2\omega),
\end{equation}
where
\begin{equation}\label{eq:mr9}
\mathbf{H}_{\ell}
= \hat{\mathbf{s}}\,T_{s}^{\ell v}
\left(1+ R^{M}_{s}\right)\hat{\mathbf{s}} + \hat{\mathbf{P}}_{v+}T_{p}^{\ell v}
\left(\hat{\mathbf{P}}_{\ell +} + R^{M}_{p}\hat{\mathbf{P}}_{\ell -}\right),
\end{equation}
and
\begin{align}\label{m61}
R^{M}_{l}\equiv
\frac{R^{\ell b}_{l}e^{i\delta_{0}}}
     {1+R^{v\ell}_{l} R^{\ell b}_{l}e^{i\delta}}
\end{align}
is defined as the multiple ($M$) reflection coefficient. $l$ can be either $s$
or $p$. This coefficient depends on the layer $\ell$ thickness $d$, and most
importantly on the position $d_{2}$ of $\boldsymbol{\mathcal P}(2\omega)$ within
this layer. The final results will depend on both $d$ and $d_{2}$, however we
could also define an average $\bar{R}^{M}_\mathrm{i}$ as
\begin{align}\label{eq:mcave}
\bar{R}^{M}_\mathrm{i}\equiv \frac{1}{d}\int_0^d R^{M}_\mathrm{i}(x)dx  
,
\end{align} 
where 
\begin{align}\label{eq:m16}
R^{M}_{\mathrm{i}} (x)
= 
\frac{R^{\ell b}_{\mathrm{i}}e^{i\alpha x}}
{1 + R^{v\ell}_{\mathrm{i}}R^{\ell b}_{\mathrm{i}}e^{i\delta}}
,
\end{align}
with $\alpha=8\pi W_{\ell}/\lambda_{0}$, to obtain that
\begin{align}\label{eq:mcave2}
\bar{R}^{M}_\mathrm{i}
= 
\frac{R^{\ell b}_{\mathrm{i}}e^{i\delta/2}}
{1 + R^{v\ell}_{\mathrm{i}}R^{\ell b}_{\mathrm{i}}e^{i\delta}}
\mathrm{sinc}(\delta/2) 
,
\end{align}
that only depends on $d$ through $\delta$ (Eq. \eqref{delta}).

To connect with the work in Ref. \cite{mizrahiJOSA88}, where
$\boldsymbol{\mathcal{P}}(2\omega)$ is located on top of the vacuum-surface
interface and only the vacuum radiated beam and the first (and only) reflected
beam need be considered, we take $\ell = v$ and $d_{2} = 0$, then $T^{\ell v} =
1$, $R^{v\ell} = 0$ and $\delta_{0} = 0$, with which $R^{M}_{l}=R^{vb}_{l}$.
Thus, Eq. \eqref{eq:mr9} coincides with Eq. (3.8) of Ref. \cite{mizrahiJOSA88}.


%%%%%%%%%%%%%%%%%%%%%%%%%%%%%%%%%%%%%%%%%%%%%%%%%%%%%%%%%%%%%%%%%%%%%%%%%%%%%%%%

\subsection{Multiple reflections for the linear field}

For a more complete formulation, we must also consider the multiple reflections
of the fundamental field $\mathbf{E}(\omega)$ inside the thin $\ell$ layer. In
Fig. \ref{fig:MR3layer2w} I present the situation where $\mathbf{E}(\omega)$
impinges from the vacuum side with an angle of incidence $\theta_{0}$. As the
first transmitted beam is multiply reflected from the $\ell b$ and the $\ell v$
interfaces, it accumulates a phase difference of $n\phi$, with $n=1,2,3,\ldots$,
given by
\begin{equation}\label{mphi}
\begin{split}
\phi &= \frac{\omega}{c}(2L_{1}n_{\ell} - L_{2}n_{v})\nonumber\\
&= 4\pi\left(\frac{d}{\lambda_{0}}\right)\sqrt{n^{2}_\ell-\sin^{2}\theta_{0}},
\end{split}
\end{equation}
where $n_{v}=1$. Besides the equivalent of Eqs. \eqref{eq:r6} and \eqref{eq:r6b}
for $\omega$, we also need
\begin{equation}\label{eq:mvv}
\mathbf{t}^{v\ell}
= \hat{\mathbf{s}}t_{s}^{v\ell}\hat{\mathbf{s}} 
+ \hat{\mathbf{p}}_{\ell -}t_{p}^{v\ell}\hat{\mathbf{p}}_{v-},
\end{equation}
to write
\begin{align}\label{eq:mcvew}
\mathbf{E}(\omega)
&= E_{0}
\Big[
\mathbf{t}^{v\ell} + \mathbf{r}^{\ell b}\cdot\mathbf{t}^{v\ell}e^{i\phi}
 + \mathbf{r}^{\ell b}\cdot\mathbf{r}^{\ell v}\cdot
   \mathbf{r}^{\ell b}\cdot\mathbf{t}^{v\ell} e^{i2\phi}\nonumber\\
&\qquad\qquad+ \mathbf{r}^{\ell b}\cdot\mathbf{r}^{\ell v}\cdot
   \mathbf{r}^{\ell b}\cdot\mathbf{r}^{\ell v}\cdot
   \mathbf{r}^{\ell b}\cdot\mathbf{t}^{v\ell} e^{i3\phi}
 + \cdots
\Big]\cdot\mathbf{e}^{\mathrm{in}}\nonumber\\
&= E_{0}
\Big[
1 + \Big(1 + \mathbf{r}^{\ell b}\cdot\mathbf{r}^{\ell v}e^{i\phi}
+ (\mathbf{r}^{\ell b}\cdot\mathbf{r}^{\ell v})^2e^{i2\phi}+\cdots\Big)\cdot
\mathbf{r}^{\ell b}e^{i\phi}
\Big]
\cdot\mathbf{t}^{v\ell}\cdot\mathbf{e}^{\mathrm{in}}\nonumber\\
&= E_{0}
\Big[
\hat{\mathbf{s}} t^{v\ell}_{s}(1+r^{M}_{s})\hat{\mathbf{s}} 
+ t^{v\ell}_{p}
\left(\hat{\mathbf{p}}_{\ell-}+\hat{\mathbf{p}}_{\ell+}r^{M}_{p}\right)
\hat{\mathbf{p}}_{v-}
\Big]\cdot\mathbf{e}^{\mathrm{in}},
\end{align}
where
\begin{align}\label{mvrm}
r^{M}_{l} =
\frac{r^{\ell b}_{l}e^{i\phi}}
     {1+r^{v\ell}_{l}r^{\ell b}_{l}e^{i\phi}}.
\end{align}
and $l$ can be either $s$ or $p$. We define $\mathbf{E}^{l}(\omega)\equiv
E_{0}\mathbf{e}^{\omega,l}_\ell$ ($l=s,p$), where using Eq. \eqref{eq:r4}, we
obtain that
\begin{equation}\label{eq:mcvep}
\mathbf{e}^{\omega,p}_{\ell}=\frac{t^{v\ell}_{p}}{n_{\ell}}
\left( 
  r^{M+}_{p}\sin\theta_{0}\hat{\mathbf{z}}
+ r^{M-}_{p}w_{\ell}\hat{\boldsymbol{\kappa}}
\right),
\end{equation} 
for $p$-input polarization with
$\mathbf{e}^{\mathrm{in}}=\hat{\mathbf{p}}_{v-}$, and
\begin{equation}\label{mcves}
\mathbf{e}^{\omega,s}_\ell=t^{v\ell}_{s}r^{M+}_{s}\hat{\mathbf{s}},
\end{equation}
for $s$-input polarization with $\mathbf{e}^{\mathrm{in}}=\hat{\mathbf{s}}$,
where
\begin{equation}\label{mvc}
r^{M\pm}_{l}=1\pm r^{M},\quad l = s,p.
\end{equation}

\begin{figure}
\centering 
\includegraphics[scale=.5]{content/figures/diag-3layer_MR_1w}
\caption{Sketch for the multiple reflected fundamental field
$\mathbf{E}(\omega)$, which impinges from the vacuum side along the
$\hat{\boldsymbol{\kappa}}z$-plane. $\theta_{0}$ and $\boldsymbol{\nu}_{v-}$ are
the angle of incidence and wave vector, respectively. The arrows point along the
direction of propagation. The $p$-polarization unit vectors
$\hat{\mathbf{p}}_{\beta\pm}$, point along the downward $(-)$ or upward $(+)$
directions and are denoted with thick arrows, where $\beta = v$ or $\ell$. The
$s$-polarization unit vector $\hat{\mathbf{s}}$ points out of the page.
$(1,2,3,\ldots)\phi$ denotes the phase difference for the multiple reflected
beams with respect to the incident field, where the dotted line is perpendicular
to this beam.}
\label{fig:MR3layer1w}
\end{figure}


%%%%%%%%%%%%%%%%%%%%%%%%%%%%%%%%%%%%%%%%%%%%%%%%%%%%%%%%%%%%%%%%%%%%%%%%%%%%%%%%

\subsection{Deriving the SSHG yield}

The magnitude of the radiated field is given by $E(2\omega) =
\hat{\mathbf{e}}^{\mathrm{out}}\cdot\mathbf{E}(2\omega)$, where
$\hat{\mathbf{e}}^{\mathrm{out}}$ is the polarization vector of the radiated
field such as $\hat{\mathbf{s}}$ or $\hat{\mathbf{P}}_{v+}$. Then, we write
\begin{equation}
\begin{split}
\hat{\mathbf{P}}_{\ell +} + R^{M}_{p}\hat{\mathbf{P}}_{\ell -}
&= \frac{\sin\theta_{0}\hat{\mathbf{z}} - W_{\ell}\hat{\boldsymbol{\kappa}}}
        {N_{\ell}}
 + R^{M}_{p}
   \frac{\sin\theta_{0}\hat{\mathbf{z}} + W_{\ell}\hat{\boldsymbol{\kappa}}}
        {N_{\ell}}\\
&= \frac{1}{N_{\ell}}
\left(
\sin\theta_{0}R^{M}_{p+}\hat{\mathbf{z}}
- K_{\ell}R^{M}_{p-}\hat{\boldsymbol{\kappa}}
\right),
\end{split}
\end{equation}
where
\begin{equation}\label{eq:rm}
R^{M\pm}_{l}\equiv 1 \pm R^{M}_{l} \qquad l=s,p.
\end{equation}
Using Eq. \eqref{eq:mf} we write Eq. \eqref{eq:mr8} as
\begin{equation}\label{eq:r10}
E(2\omega) = \frac{2\gamma i\omega}{cW_\ell}
\hat{\mathbf{e}}^{\mathrm{out}}\cdot\mathbf{H}_{\ell}\cdot
\boldsymbol{\mathcal{P}}(2\omega) 
= \frac{2\gamma i \omega}{cW_{v}}
\mathbf{e}^{\,2\omega}_{\ell}\cdot\boldsymbol{\mathcal{P}}(2\omega),
\end{equation}
where
\begin{equation}\label{eq:r12mm}
\mathbf{e}^{2\omega}_{\ell} =\hat{\mathbf{e}}^{\mathrm{out}}\cdot 
\Bigg[
\hat{\mathbf{s}}T_{s}^{v\ell}R^{M+}_{s}\hat{\mathbf{s}} + 
\hat{\mathbf{P}}_{v+}
\frac{T^{v\ell}_{p}}
     {N_{\ell}}
\left(
\sin\theta_{0}R^{M+}_{p}\hat{\mathbf{z}}
- W_{\ell}R^{M-}_{p}\hat{\boldsymbol{\kappa}}
\right) 
\Bigg]. 
\end{equation}  
We pause here to reduce this result to the case where the nonlinear polarization
$\mathbf{P}(2\omega)$ radiates from vacuum instead from the layer $\ell$. For
such case we simply take $\epsilon_{\ell}(2\omega) = 1$ and $\ell = v$ ($T^{\ell
v}_{s,p} = 1$), to get
\begin{equation}\label{eq:r13}
\mathbf{e}^{\,2\omega}_{v} = \hat{\mathbf{e}}^{\mathrm{out}}\cdot
\left[
\hat{\mathbf{s}}T_{s}^{v b}\hat{\mathbf{s}} + \hat{\mathbf{P}}_{v+}
\frac{T^{v b}_{p}}{\sqrt{\epsilon_{b}(2\omega)}}
\left(
  \epsilon_{b}(2\omega)\sin\theta_{0}\hat{\mathbf{z}}
- W_{b}\hat{\boldsymbol{\kappa}}
\right) 
\right],
\end{equation}
which agrees with Eq. (3.10) of Ref. \cite{mizrahiJOSA88}.

In the 3-layer model the SH polarization $\boldsymbol{\mathcal{P}}(2\omega)$ is
located in layer $\ell$, where we evaluate the fundamental field required in Eq.
\eqref{eq:tres}. We write
\begin{equation}\label{eq:m2}
\begin{split}
\mathbf{E}_{\ell}(\omega) 
&= E_{0}
\left(
  \hat{\mathbf{s}} t^{v\ell}_{s}(1+r^{\ell b}_{s})\hat{\mathbf{s}}
+ \hat{\mathbf{p}}_{\ell-}t^{v\ell}_{p}\hat{\mathbf{p}}_{v-}
+ \hat{\mathbf{p}}_{\ell+}t^{v\ell}_{p}r^{\ell b}_{p}\hat{\mathbf{p}}_{v-}
\right)
\cdot\hat{\mathbf{e}}^{\mathrm{in}}\\
&= E_{0}\mathbf{e}^\omega_{\ell},
\end{split}
\end{equation} 
where $\hat{\mathbf{e}}^{\mathrm{in}}$ is the $s$ ($\hat{\mathbf{s}}$) or $p$
($\hat{\mathbf{p}}_{v-}$) incoming polarization of the fundamental electric
field. This field is composed of the transmitted field and its first reflection
from the $\ell b$ interface for $s$ and $p$ polarizations. The fundamental
field, once inside the layer $\ell$ will be reflected multiple times at the
$\ell v$ and $\ell b$ interfaces. However, each reflection will diminish the
intensity of the fundamental field. As the SSHG yield scales with the square of
this field, the contribution of the subsequent reflections after the one
considered in Eq. \eqref{eq:m2} can be safely neglected. From Eq. \eqref{eq:mf}
we find that
\begin{equation}\label{eq:m12}
\mathbf{e}^{\omega}_{\ell} =
\left[
  \hat{\mathbf{s}}t_{s}^{v\ell}t_{s}^{\ell b}\hat{\mathbf{s}} 
+ \frac{t^{v\ell}_{p}t^{\ell b}_{p}}{n^{2}_{\ell} n_{b}}
\left(
  n^{2}_{b}\sin\theta_{0}\hat{\mathbf{z}} 
+ n^{2}_{\ell} w_b\hat{\boldsymbol{\kappa}}
\right)
\hat{\mathbf{p}}_{v-}
\right]
\cdot\hat{\mathbf{e}}^{\mathrm{in}}.  
\end{equation}  
To connect with the work in Ref. \cite{mizrahiJOSA88}, we evaluate the fields in
the bulk instead of the layer $\ell$and simply take $n_{\ell} = n_{b}$
$(t^{\ell b}_{s,p} = 1$), to obtain
\begin{equation}\label{eq:m13}
\mathbf{e}^{\omega}_{b} =
\left[
  \hat{\mathbf{s}}t_{s}^{vb}\hat{\mathbf{s}}
+ \frac{t^{vb}_{p}}{n_{b}}
\left(\sin\theta_{0}\hat{\mathbf{z}} + w_b\hat{\boldsymbol{\kappa}}\right) 
\hat{\mathbf{p}}_{v-}
\right]
\cdot\hat{\mathbf{e}}^{\mathrm{in}},  
\end{equation} 
that is in agreement with Eq. (3.5) of Ref. \cite{mizrahiJOSA88}. Then, we can
write Eq. \eqref{eq:tres} as
\begin{equation}\label{eq:m4}
\boldsymbol{\mathcal{P}}(2\omega) = 
\left\{
\begin{array}{cc}  
E^{2}_{0}\,
\boldsymbol{\chi}:\mathbf{e}^{\omega}_{\ell}\mathbf{e}^{\omega}_{\ell}
& \text{(CGS units)}\\
\epsilon_{0}E^{2}_{0}\,
\boldsymbol{\chi}:\mathbf{e}^{\omega}_{\ell}\mathbf{e}^{\omega}_{\ell}
& \text{(MKS units)}\\
\end{array}
\right.,
\end{equation}
where $E_{0}$ is the intensity of the fundamental electric field. Finally, with
this equation we rewrite Eq. \eqref{eq:r10} as
\begin{equation}\label{eq:mr10}
E(2\omega) 
= \frac{2\eta i \omega}{cW_{v}}
\mathbf{e}^{2\omega}_{\ell}\cdot
\boldsymbol{\chi}:\mathbf{e}^{\omega}_{\ell}\mathbf{e}^{\omega}_{\ell},
\end{equation}
where $\eta=2\pi$ in CGS units and $\eta=1/2$ in MKS units. For ease of
notation, we define
\begin{align}\label{eq:mc0}
\Upsilon_{\mathrm{iF}}
\equiv 
\mathbf{e}^{2\omega}_{\ell}\cdot
\boldsymbol{\chi}:\mathbf{e}^{\omega}_{\ell}\mathbf{e}^{\omega}_{\ell},
\end{align}
where i stands for the incoming polarization of the fundamental electric field
given by $\hat{\mathbf{e}}^{\mathrm{in}}$ in Eq. \eqref{eq:m12}, and O for the
outgoing polarization of the SH electric field given by
$\hat{\mathbf{e}}^{\mathrm{out}}$ in Eq. \eqref{eq:r12mm}.

From Eqs. \eqref{eq:uno} and \eqref{eq:dos} we obtain that in CGS units,
($\eta=2\pi$)
\begin{align}\label{eq:r01}
\vert E(2\omega)\vert^{2} &=
\vert E_{0}\vert^{4}\frac{16\pi^{2}\omega^{2}}{c^{2}W^2_{v}}
\vert\Upsilon_{\mathrm{iF}}\vert^{2}\nonumber\\
%%%%%%%%%%%%%%%%%%%%%%%%%%%%%%%%%%%%%%%%%%%%%%%%%%%%%
\frac{c}{2\pi}\vert\sqrt{N_{v}}E(2\omega)\vert^{2} &=
\frac{32\pi^{3}\omega^{2}}{c^{3}\cos^2\theta_{0}}
\left\vert\frac{\sqrt{N_{v}}}{n^{2}_{\ell}}\Upsilon_{\mathrm{iF}}\right\vert^{2} 
\left(\frac{c}{2\pi}\vert\sqrt{n_{\ell}}E_{0}\vert^{2}\right)^{2}\nonumber\\ 
%%%%%%%%%%%%%%%%%%%%%%%%%%%%%%%%%%%%%%%%%%%%%%%%%%%%%
I(2\omega) &=
\frac{32\pi^{3}\omega^{2}}{c^{3}\cos^2\theta_{0}}
\left\vert\frac{\sqrt{N_{v}}}{n^{2}_{\ell}}\Upsilon_{\mathrm{iF}}\right\vert^{2}
I^{2}(\omega)\nonumber\\
%%%%%%%%%%%%%%%%%%%%%%%%%%%%%%%%%%%%%%%%%%%%%%%%%%%%%
\mathcal{R}_{\mathrm{iF}}(2\omega) &=
\frac{32\pi^{3}\omega^{2}}{c^{3}\cos^2\theta_{0}}
\left\vert\frac{1}{n_{\ell}}\Upsilon_{\mathrm{iF}}\right\vert^{2},
\end{align} 
and in MKS units ($\eta=1/2$),
\begin{align}\label{r01m}
\vert E(2\omega)\vert^{2} &=
\vert E_{0}\vert^{4}\frac{\omega^{2}}{c^{2}W^{2}_{v}}\nonumber\\
%%%%%%%%%%%%%%%%%%%%%%%%%%%%%%%%%%%%%%%%%%%%%%%%%%%%%
2\epsilon_{0}c|\sqrt{N_{v}}E(2\omega)|^{2} &=
\frac{2\epsilon_{0}\omega^{2}}{c\cos^{2}\theta_{0}}
\left\vert\frac{\sqrt{N_{v}}}{n^{2}_{\ell}}\Upsilon_{\mathrm{iF}}\right\vert^{2} 
\frac{1}{4\epsilon^{2}_0c^{2}}
\left(2\epsilon_{0}c\vert\sqrt{n_{\ell}}E_{0}\vert^{2}\right)^{2}\nonumber\\
%%%%%%%%%%%%%%%%%%%%%%%%%%%%%%%%%%%%%%%%%%%%%%%%%%%%%
I(2\omega) &= 
\frac{\omega^{2}}{2\epsilon_{0}c^3\cos^{2}\theta_{0}}
\left\vert\frac{\sqrt{N_{v}}}{n^{2}_{\ell}}\Upsilon_{\mathrm{iF}}\right\vert^{2}
I^{2}(\omega)\nonumber\\
%%%%%%%%%%%%%%%%%%%%%%%%%%%%%%%%%%%%%%%%%%%%%%%%%%%%%
\mathcal{R}_{\mathrm{iF}}(2\omega) &=
\frac{\omega^{2}}{2\epsilon_{0}c^3\cos^{2}\theta_{0}}
\left\vert  \frac{1}{n_{\ell}}\Upsilon_{\mathrm{iF}}\right\vert^{2}.
\end{align}
Finally, we condense these results and establish the SSHG yield as
\begin{equation}\label{eq:mc6}
\mathcal{R}_{\mathrm{iF}}(2\omega) 
\left\{
\begin{array}{ r c } 
\frac{32\pi^{3}\omega^{2}}{c^{3}\cos^{2}\theta_{0}}
\left\vert\frac{1}{n_{\ell}}\Upsilon_{\mathrm{iF}}\right\vert^{2} 
& \text{(CGS units)} \\
\frac{\omega^{2}}{2\epsilon_{0}c^3\cos^{2}\theta_{0}}
\left\vert\frac{1}{n_{\ell}}\Upsilon_{\mathrm{iF}}\right\vert^{2} 
& \text{(MKS units)} 
\end{array}
\right.,
\end{equation}
where $N_{v}=1$ and $W_{v}=\cos\theta_{0}$. In the MKS unit system
$\boldsymbol{\chi}$ is given in m$^{2}$/V, since it is a surface second order
nonlinear susceptibility, and $\mathcal{R}_{\mathrm{iF}}$ is given in m$^2$/W.

It is worth mentioning that we can easily recover the results from Ref.
\cite{mizrahiJOSA88}, which are in turn equivalent to those in Ref.
\cite{sipePRB87}. We simply take
$\mathbf{e}^{2\omega}_{\ell}\to\mathbf{e}^{2\omega}_{v}$,
$\mathbf{e}^{\omega}_{\ell}\to\mathbf{e}^{\omega}_{b}$, and we have
\begin{equation}\label{eq:m69}
\mathcal{R}_{\mathrm{iF}}(2\omega) =
\frac{32\pi^{3}\omega^{2}}{c^{3}\cos^{2}\theta_{0}}
\left\vert\mathbf{e}^{\,2\omega}_{v}\cdot
\boldsymbol{\chi}:\mathbf{e}^{\omega}_{b}\mathbf{e}^{\omega}_{b}
\right\vert^{2}.
\end{equation}
This is the SSHG yield  of a nonlinear polarization sheet radiating from the
vacuum region above the surface, with the fundamental field evaluated below the
surface in the bulk of the material characterized by $\epsilon_{b}(\omega)$.

I include a full treatise on this exact procedure without considering the
effects of multiple reflections in Appendix \ref{app:shgyieldnomr}.


%%%%%%%%%%%%%%%%%%%%%%%%%%%%%%%%%%%%%%%%%%%%%%%%%%%%%%%%%%%%%%%%%%%%%%%%%%%%%%%%
%%%%%%%%%%%%%%%%%%%%%%%%%%%%%%%%%%%%%%%%%%%%%%%%%%%%%%%%%%%%%%%%%%%%%%%%%%%%%%%%

\section{\texorpdfstring{$\mathcal{R}_{\mathrm{iF}}$}{R} for different
polarization cases}\label{sec:rcases}

We now have everything we need to derive explicit expressions for
$\mathcal{R}_{\mathrm{iF}}$, Eq. \eqref{eq:mc6}, for the most commonly used
polarizations of incoming and outgoing fields (iF=$pP$, $pS$, $sP$, and $sS$).
For this, we must expand $\Upsilon_{\mathrm{iF}}$ from Eq. \eqref{eq:mc0} for each
case. By substituting Eqs. \eqref{mc1} and \eqref{mmc2} into Eq. \eqref{eq:r12mm},
we obtain
\begin{equation}\label{eq:e2wpmr}
\mathbf{e}^{2\omega,P}_{\ell} =
\frac{T^{v\ell}_{p}}{N_{\ell}}
\big(
  \sin\theta_{0}R^{M+}_{p}\hat{\mathbf{z}}
- W_{\ell}R^{M-}_{p}\cos\phi\hat{\mathbf{x}}
- W_{\ell}R^{M-}_{p}\sin\phi\hat{\mathbf{y}}
\big),
\end{equation}
for $P$ $(\hat{\mathbf{e}}^{\mathrm{F}} = \hat{\mathbf{P}}_{v+})$ outgoing
polarization, and
\begin{equation}\label{eq:e2wsmr}
\mathbf{e}^{2\omega,S}_{\ell} =
T_{s}^{v\ell}R^{M+}_{s}
\left(
- \sin\phi\hat{\mathbf{x}}
+ \cos\phi\hat{\mathbf{y}}
\right).
\end{equation}
for $S$ $(\hat{\mathbf{e}}^{\mathrm{F}}=\hat{\mathbf{s}})$ outgoing polarization.

Following a similar procedure, we use Eqs. \eqref{mc1} and \eqref{mmc2} with Eq.
\eqref{eq:mcvep}, and obtain
\begin{equation}\label{eq:ewewpmr}
\begin{split}
\mathbf{e}^{\omega,\mathrm{p}}_{\ell}\mathbf{e}^{\omega,\mathrm{p}}_{\ell} &=
\left(\frac{t^{v\ell}_{p}}{n_{\ell}}\right)^{2}
\bigg(
   \left(r^{M-}_{p}\right)^{2}w^{2}_{\ell}\cos^{2}\phi
   \hat{\mathbf{x}}\hat{\mathbf{x}}
 + 2\left(r^{M-}_{p}\right)^{2}w^{2}_{\ell}\sin\phi\cos\phi
   \hat{\mathbf{x}}\hat{\mathbf{y}}\\
&+ 2r^{M+}_{p}r^{M-}_{p}w_{\ell}\sin\theta_{0}\cos\phi
   \hat{\mathbf{x}}\hat{\mathbf{z}}
 + \left(r^{M-}_{p}\right)^{2}w^{2}_{\ell}\sin^{2}\phi
   \hat{\mathbf{y}}\hat{\mathbf{y}}\\
&+ 2r^{M+}_{p}r^{M-}_{p}w_{\ell}\sin\theta_{0}\sin\phi
   \hat{\mathbf{y}}\hat{\mathbf{z}}
 + \left(r^{M+}_{p}\right)^{2}\sin^{2}\theta_{0}
   \hat{\mathbf{z}}\hat{\mathbf{z}}
\bigg),
\end{split}
\end{equation}
for $p$ incoming polarization $(\hat{\mathbf{e}}^{\mathrm{i}} =
\hat{\mathbf{p}}_{v-})$, and with Eq. \eqref{mcves},
\begin{equation}\label{eq:ewewsmr}
\mathbf{e}^{\omega,\mathrm{s}}_{\ell}\mathbf{e}^{\omega,\mathrm{s}}_{\ell}
= \left(t^{v\ell}_{s}r^{M+}_{s}\right)^{2}
\big(
  \sin^{2}\phi\hat{\mathbf{x}}\hat{\mathbf{x}}
 + \cos^{2}\phi\hat{\mathbf{y}}\hat{\mathbf{y}}
 - 2\sin\phi\cos\phi\hat{\mathbf{x}}\hat{\mathbf{y}}
\big).
\end{equation}
for $s$ incoming polarization $(\hat{\mathbf{e}}^{\mathrm{i}} =
\hat{\mathbf{s}})$.

I have summarized the combination of equations needed to derive the expressions
or all four polarization cases of $\mathcal{R}_{\mathrm{iF}}$ in Table
\ref{tab:summary}. In the following subsections we will derive the explicit
expressions for $\Upsilon_{\mathrm{iF}}$ for the most general case where the
surface has no symmetry other than that of noncentrosymmetry. We will then
develop these expressions for particular cases of the most commonly investigated
surfaces, the (111), (100), and (110) crystallographic faces. For ease of
writing we split $\Upsilon_{\mathrm{iF}}$ as
\begin{equation}\label{eq:mc25}
\Upsilon_{\mathrm{iF}} = \Gamma_{\mathrm{iF}}\,r_{\mathrm{iF}}.
\end{equation} 

Lastly, in Table \ref{chis} I list the nonzero components of $\boldsymbol{\chi}$
for each surface symmetry \cite{sipePRB87, popovbook}.

\begin{table}
\centering
\begin{tabular}{| c | l | l | c | c |}
\hline
Case               & $\hat{\mathbf{e}}^{\mathrm{F}}$
                   & $\hat{\mathbf{e}}^{\mathrm{i}}$
                   & $\mathbf{e}^{2\omega,\mathrm{F}}_{\ell}$
                   & $\mathbf{e}^{\omega,\mathrm{i}}_{\ell}
                      \mathbf{e}^{\omega,\mathrm{i}}_{\ell}$ \\
\hline
$\mathcal{R}_{pP}$ & $\hat{\mathbf{P}}_{v+}$
                   & $\hat{\mathbf{p}}_{v-}$
                   &  Eq. \eqref{eq:e2wpmr} & Eq. \eqref{eq:ewewpmr} \\
$\mathcal{R}_{pS}$ & $\hat{\mathbf{S}}$
                   & $\hat{\mathbf{p}}_{v-}$
                   &  Eq. \eqref{eq:e2wsmr} & Eq. \eqref{eq:ewewpmr} \\
$\mathcal{R}_{sP}$ & $\hat{\mathbf{P}}_{v+}$
                   & $\hat{\mathbf{s}}$
                   &  Eq. \eqref{eq:e2wpmr} & Eq. \eqref{eq:ewewsmr} \\
$\mathcal{R}_{sS}$ & $\hat{\mathbf{S}}$
                   & $\hat{\mathbf{s}}$
                   &  Eq. \eqref{eq:e2wsmr} & Eq. \eqref{eq:ewewsmr} \\
\hline
\end{tabular}
\caption{Polarization unit vectors for $\hat{\mathbf{e}}^{\mathrm{F}}$ and
$\hat{\mathbf{e}}^{\mathrm{i}}$, and equations describing
$\mathbf{e}^{2\omega,\mathrm{F}}_{\ell}$ and
$\mathbf{e}^{\omega,\mathrm{i}}_{\ell}\mathbf{e}^{\omega,\mathrm{i}}_{\ell}$ for
each polarization case.}
\label{tab:summary}
\end{table}

\begin{table}
\centering
\begin{tabular}{| c | c | c |}
\hline 
(111)-$C_{3v}$     & (110)-$C_{2v}$  & (100)-$C_{4v}$ \\
\hline 
$\chi^{zzz}$ & $\chi^{zzz}$ & $\chi^{zzz}$\\
$\chi^{zxx}=\chi^{zyy}$ & $\chi^{zxx}\ne\chi^{zyy}$ & $\chi^{zxx}=\chi^{zyy}$\\
$\chi^{xxz}=\chi^{yyz}$ & $\chi^{xxz}\ne\chi^{yyz}$ & $\chi^{xxz}=\chi^{yyz}$\\
$\chi^{xxx}=-\chi^{xyy}=-\chi^{yyx}$ & &  \\
\hline 
\end{tabular}
\caption{Components of $\boldsymbol{\chi}$ for the (111), (110) and (100)
crystallographic faces, belonging to the $C_{3v}$, $C_{2v}$, and $C_{4v}$,
symmetry groups, respectively. For the (111) surface we choose the $x$ and $y$
axes along the [$11\bar{2}$] and [$1\bar{1}0$] directions, respectively. For the
(110) and (100) we consider the $y$ axis perpendicular to the plane of
symmetry.\cite{sipePRB87} We remark that in general
$\boldsymbol{\chi}^{(111)}\ne \boldsymbol{\chi}^{(110)} \ne
\boldsymbol{\chi}^{(100)}$.}
\label{chis}
\end{table}


%%%%%%%%%%%%%%%%%%%%%%%%%%%%%%%%%%%%%%%%%%%%%%%%%%%%%%%%%%%%%%%%%%%%%%%%%%%%%%%%

\subsection{\texorpdfstring{$\mathcal{R}_{pP}$}{RpP}}\label{sec:RpP} 

Per Table \ref{tab:summary}, $\mathcal{R}_{pP}$ requires Eqs. \eqref{eq:e2wpmr}
and \eqref{eq:ewewpmr}. After some algebra, we obtain that
\begin{equation}\label{eq:mc78}
\Gamma_{pP} =
\frac{T^{v\ell}_{p}}{N_{\ell}}
\left(\frac{t^{v\ell}_{p}}{n_{\ell}}\right)^{2}
,
\end{equation}
and
\begin{equation}
\begin{split}
r_{pP} =
&-R^{M-}_{p}\left(r^{M-}_{p}\right)^{2}w^{2}_{\ell}W_{\ell}\cos^{3}\phi
\chi^{xxx}
 -2R^{M-}_{p}\left(r^{M-}_{p}\right)^{2}w^{2}_{\ell}W_{\ell}\sin\phi\cos^{2}\phi
\chi^{xxy}\\
&-2R^{M-}_{p}r^{M+}_{p}r^{M-}_{p}w_{\ell}W_{\ell}\sin\theta_{0}\cos^{2}\phi
\chi^{xxz}
 -R^{M-}_{p}\left(r^{M-}_{p}\right)^{2}w^{2}_{\ell}W_{\ell}\sin^{2}\phi\cos\phi
\chi^{xyy}\\
&-2R^{M-}_{p}r^{M+}_{p}r^{M-}_{p}w_{\ell}W_{\ell}\sin\theta_{0}\sin\phi\cos\phi
\chi^{xyz}
 -R^{M-}_{p}\left(r^{M+}_{p}\right)^{2}W_{\ell}\sin^{2}\theta_{0}\cos\phi
\chi^{xzz}\\
%%%%%%%%%%%%%%%%%%%%%%%%%%%%%%%%%%%%%%%%%%%%%%%%%%%%%%%%%%%%
&-R^{M-}_{p}\left(r^{M-}_{p}\right)^{2}w^{2}_{\ell}W_{\ell}\sin\phi\cos^{2}\phi
\chi^{yxx}
 -2R^{M-}_{p}\left(r^{M-}_{p}\right)^{2}w^{2}_{\ell}W_{\ell}\sin^{2}\phi\cos\phi
\chi^{yxy}\\
&-2R^{M-}_{p}r^{M+}_{p}r^{M-}_{p}w_{\ell}W_{\ell}\sin\theta_{0}\sin\phi\cos\phi
\chi^{yxz}
 -R^{M-}_{p}\left(r^{M-}_{p}\right)^{2}w^{2}_{\ell}W_{\ell}\sin^{3}\phi
\chi^{yyy}\\
&-2R^{M-}_{p}r^{M+}_{p}r^{M-}_{p}w_{\ell}W_{\ell}\sin\theta_{0}\sin^{2}\phi
\chi^{yyz}
 -R^{M-}_{p}\left(r^{M+}_{p}\right)^{2}W_{\ell}\sin^{2}\theta_{0}\sin\phi
\chi^{yzz}\\
%%%%%%%%%%%%%%%%%%%%%%%%%%%%%%%%%%%%%%%%%%%%%%%%%%%%%%%%%%%%
&+R^{M+}_{p}\left(r^{M-}_{p}\right)^{2}w^{2}_{\ell}\sin\theta_{0}\cos^{2}\phi
\chi^{zxx}
 +2R^{M+}_{p}r^{M+}_{p}r^{M-}_{p}w_{\ell}\sin^{2}\theta_{0}\cos\phi
\chi^{zxz}\\
&+2R^{M+}_{p}\left(r^{M-}_{p}\right)^{2}w^{2}_{\ell}\sin\theta_{0}\sin\phi
\cos\phi\chi^{zxy}
 +R^{M+}_{p}\left(r^{M-}_{p}\right)^{2}w^{2}_{\ell}\sin\theta_{0}\sin^{2}\phi
\chi^{zyy}\\
&+2R^{M+}_{p}r^{M+}_{p}r^{M-}_{p}w_{\ell}\sin^{2}\theta_{0}\sin\phi
\chi^{zzy}
 +R^{M+}_{p}\left(r^{M+}_{p}\right)^{2}\sin^{3}\theta_{0}
\chi^{zzz}
,
\end{split}
\end{equation}
where all 18 independent components of $\boldsymbol{\chi}$ valid for a surface with no symmetries contribute to $\mathcal{R}_{pP}$. Recall that $\chi^{ijk}=\chi^{ikj}$. Using Table \ref{chis}, we present the expressions for each of the three surfaces being considered here. For the (111) surface we obtain
\begin{equation}\label{rpp111}
\begin{split}
r^{(111)}_{pP} &= 
R^{M+}_{p}\sin\theta_{0}
\Big(
  \left(r^{M+}_{p}\right)^{2}\sin^{2}\theta_{0}\chi^{zzz}
+ \left(r^{M-}_{p}\right)^{2}w^{2}_{\ell}\chi^{zxx}
\Big)\\
&- R^{M-}_{p}w_{\ell}W_{\ell}
\Big(
  2r^{M+}_{p}r^{M-}_{p}\sin\theta_{0}\chi^{xxz}
+ \left(r^{M-}_{p}\right)^{2}w_{\ell}\chi^{xxx}\cos3\phi
\Big),
\end{split}
\end{equation}
where the three-fold azimuthal symmetry of the SHG signal, typical of the $C_{3v}$ symmetry group, is seen in the $3\phi$ argument of the cosine function. For the (110) we have that
\begin{equation}\label{eq:final-rpp.mr.110}
\begin{split}
r^{(110)}_{pP} &= 
R^{M+}_{p}\sin\theta_{0}
\bigg(
  \left(r^{M+}_{p}\right)^{2}\sin^{2}\theta_{0}\chi^{zzz}
+ \left(r^{M-}_{p}\right)^{2}w^{2}_{\ell}
\left(
\frac{\chi^{zyy} + \chi^{zxx}}{2} + \frac{\chi^{zyy} - \chi^{zxx}}{2}\cos2\phi 
\right) 
\bigg)\\
&- 2R^{M-}_{p}r^{M+}_{p}r^{M-}_{p}w_{\ell}W_{\ell}\sin\theta_{0}
\left(
\frac{\chi^{yyz} + \chi^{xxz}}{2} + \frac{\chi^{yyz} - \chi^{xxz}}{2}\cos2\phi 
\right). 
\end{split}
\end{equation}
The two-fold azimuthal symmetry of the SHG signal, typical of the $C_{2v}$ symmetry group, is seen in the $2\phi$ argument of the cosine function. For the (100) surface we simply make $\chi^{zxx}=\chi^{zyy}$ and $\chi^{xxz}=\chi^{yyz}$, as seen from Table \ref{chis}, and above expression reduces to
\begin{equation}\label{rpp100}
r^{(100)}_{pP} = 
R^{M+}_{p}\sin\theta_{0}
\bigg(
  \left(r^{M+}_{p}\right)^{2}\sin^{2}\theta_{0}\chi^{zzz}
+ \left(r^{M-}_{p}\right)^{2}w^{2}_{\ell}\chi^{zxx}
\bigg)
-
2R^{M-}_{p}r^{M+}_{p}r^{M-}_{p}w_{\ell}W_{\ell}\sin\theta_{0}\chi^{xxz}
.
\end{equation}
where we mention that the azimutal $4\phi$ symmetry for the $C_{4v}$ group of the (100) surface is absent in above expresion since such contribution is only related to the bulk nonlinear quadrupolar SH term,\cite{sipePRB87} that is neglected in this work.


%%%%%%%%%%%%%%%%%%%%%%%%%%%%%%%%%%%%%%%%%%%%%%%%%%%%%%%%%%%%%%%%%%%%%%%%%%%%%%%%

\subsection{\texorpdfstring{$\mathcal{R}_{pS}$}{RpS}}\label{sec:RpS}

Per Table \ref{tab:summary}, $\mathcal{R}_{pS}$ requires Eqs. \eqref{eq:e2wsmr} and \eqref{eq:ewewpmr}. After some algebra, we obtain that
\begin{equation}\label{mcv}
\Gamma_{pS} =
T_{s}^{v\ell}R^{M+}_{s}
\left(\frac{t^{v\ell}_{p}}{n_{\ell}}\right)^{2},
\end{equation}
and
\begin{equation}
\begin{split}
r_{pS}=
&- \left(r^{M-}_{p}\right)^{2}w^{2}_{\ell}\sin\phi\cos^{2}\phi\chi^{xxx}
 - 2\left(r^{M-}_{p}\right)^{2}w^{2}_{\ell}\sin^{2}\phi\cos\phi\chi^{xxy}
 - 2r^{M+}_{p}r^{M-}_{p}w_{\ell}\sin\theta_{0}\sin\phi\cos\phi\chi^{xxz}\\
&- \left(r^{M-}_{p}\right)^{2}w^{2}_{\ell}\sin^{3}\phi\chi^{xyy}
 - 2r^{M+}_{p}r^{M-}_{p}w_{\ell}\sin\theta_{0}\sin^{2}\phi\chi^{xzy}
 - \left(r^{M+}_{p}\right)^{2}\sin^{2}\theta_{0}\sin\phi\chi^{xzz}\\
%%%%%%%%%%%%%%%%%%%%%%%%%%%%%%%%%%%%%%%%%%%%%%%%%%%%%%%%%%%%%%%%%%%%%%%%%%%%%%%%
&+ \left(r^{M-}_{p}\right)^{2}w^{2}_{\ell}\cos^{3}\phi\chi^{yxx}
 + 2\left(r^{M-}_{p}\right)^{2}w^{2}_{\ell}\sin\phi\cos^{2}\phi\chi^{yxy}
 + 2r^{M+}_{p}r^{M-}_{p}w_{\ell}\sin\theta_{0}\cos^{2}\phi\chi^{yxz}\\
&+ \left(r^{M-}_{p}\right)^{2}w^{2}_{\ell}\sin^{2}\phi\cos\phi\chi^{yyy}
 + 2r^{M+}_{p}r^{M-}_{p}w_{\ell}\sin\theta_{0}\sin\phi\cos\phi\chi^{yzy}
 + \left(r^{M+}_{p}\right)^{2}\sin^{2}\theta_{0}\cos\phi\chi^{yzz}.
\end{split}
\end{equation}
In this case 12 out of the 18 components of $\boldsymbol{\chi}$ valid for a surface with no symmetries, contribute to $\mathcal{R}_{pS}$. This is so, because there is no $\mathcal{P}_{z}$ component, as the outgoing polarization is $S$. From Table \ref{chis} we obtain,
\begin{equation}\label{r111ps}
r^{(111)}_{pS} = - \left(r^{M-}_{p}\right)^{2}w^{2}_{\ell}\chi^{xxx}\sin3\phi,
\end{equation}
for the (111) surface,
\begin{equation}\label{r110ps}
r^{(110)}_{sP} =
r^{M+}_{p}r^{M-}_{p}w_{\ell}\sin\theta_{0}(\chi^{yyz} - \chi^{xxz})\sin2\phi,
\end{equation}
for the (110) surface, 
finally,
\begin{equation}\label{r100ps}
r^{(100)}_{pS} = 0,
\end{equation}
for the (100) surface, where again, the zero value is only surface related as we neglect  the bulk nonlinear quadrupolar contribution.

%%%%%%%%%%%%%%%%%%%%%%%%%%%%%%%%%%%%%%%%%%%%%%%%%%%%%%%%%%%%%%%%%%%%%%%%%%%%%%%%
%%%%%%%%%%%%%%%%%%%%%%%%%%%%%%%%%%%%%%%%%%%%%%%%%%%%%%%%%%%%%%%%%%%%%%%%%%%%%%%%


\subsection{\texorpdfstring{$\mathcal{R}_{sP}$}{RsP}}\label{sec:RsP}

Per Table \ref{tab:summary}, $\mathcal{R}_{sP}$ requires Eqs. \eqref{eq:e2wpmr} and \eqref{eq:ewewsmr}. After some algebra, we obtain that
\begin{equation}\label{mcv4}
\Gamma_{sP}=
\frac{T^{v\ell}_{p}}{N_{\ell}}
\left(t^{v\ell}_{s}r^{M+}_{s}\right)^{2},
\end{equation}
and
\begin{equation}
\begin{split}
r_{sP} = 
& R^{M-}_{p}W_{\ell}
\big(
- \sin^{2}\phi\cos\phi\chi^{xxx}
+ 2\sin\phi\cos^{2}\phi\chi^{xxy}
- \cos^{3}\phi\chi^{xyy}
\big)\\
& R^{M-}_{p}W_{\ell}
\big(
- \sin^{3}\phi\chi^{yxx}
+ 2\sin^{2}\phi\cos\phi\chi^{yxy}
- \sin\phi\cos^{2}\phi\chi^{yyy}
\big)\\
& R^{M+}_{p}\sin\theta_{0}
\big(
  \sin^{2}\phi\chi^{zxx}
- 2\sin\phi\cos\phi\chi^{zxy}
+ \cos^{2}\phi\chi^{zyy}
\big).
\end{split}
\end{equation}
In this case 9 out of the 18 components of $\boldsymbol{\chi}(2\omega)$ valid for a surface with no symmetries, contribute to $\mathcal{R}_{sP}$. This is so, because there is no $E_z(\omega)$ component, as the incoming polarization is $s$. From Table \ref{chis} we get,
\begin{equation}
r^{(111)}_{sP} = 
R^{M+}_{p}\sin\theta_{0}\chi^{zxx} +
R^{M-}_{p}W_{\ell}\chi^{xxx}\cos3\phi,
\end{equation}
for the (111) surface,
\begin{equation}
r^{(110)}_{sP} = 
R^{M+}_{p}\sin\theta_{0}
\left(
\frac{\chi^{zxx} + \chi^{zyy}}{2} + \frac{\chi^{zyy} - \chi^{zxx}}{2}\cos2\phi
\right),
\end{equation}
for the (110) surface, and
\begin{equation}
r^{(100)}_{sP} = R^{M+}_{p}\sin\theta_{0}\chi^{zxx},
\end{equation}
for the (100) surface.


%%%%%%%%%%%%%%%%%%%%%%%%%%%%%%%%%%%%%%%%%%%%%%%%%%%%%%%%%%%%%%%%%%%%%%%%%%%%%%%%

\subsection{\texorpdfstring{$\mathcal{R}_{sS}$}{RsS}}\label{sec:RsS}

Per Table \ref{tab:summary}, $\mathcal{R}_{sS}$ requires Eqs. \eqref{eq:e2wsmr} and \eqref{eq:ewewsmr}. After some algebra, we obtain that
\begin{equation}
\Gamma_{sS} = 
T_{s}^{v\ell}R^{M+}_{s}\left(t^{v\ell}_{s}r^{M+}_{s}\right)^{2},
\end{equation}
and
\begin{equation}
\begin{split}
r_{sS} = 
&- \sin^{3}\phi\chi^{xxx}
 + 2\sin^{2}\phi\cos\phi\chi^{xxy}
 - \sin\phi\cos^{2}\phi\chi^{xyy}\\
&+ \sin^{2}\phi\cos\phi\chi^{yxx}
 + \cos^{3}\phi\chi^{yyy}
 - 2\sin\phi\cos^{2}\phi\chi^{yxy}
.
\end{split}
\end{equation}
In this case 6 out of the 18 components of $\boldsymbol{\chi}(2\omega)$ valid for a surface with no symmetries, contribute to $\mathcal{R}_{sS}$. This is so, because there is neither an $E_{z}(\omega)$ component, as the incoming polarization is $s$, nor a $\mathcal{p}_{z}$ component, as the outgoing polarization is $S$. From Table \ref{chis}, we get
\begin{equation}
r^{(111)}_{sS} = \chi^{xxx}\sin3\phi,
\end{equation}
for the (111) surface,
\begin{equation}
r^{(110)}_{sS} = 0,
\end{equation}
and
\begin{equation}
r^{(100)}_{sS} = 0,
\end{equation}
for the (110) and (100) surfaces, respectively, both being zero as the bulk nonlinear quadrupolar contribution is not considered here.


%%%%%%%%%%%%%%%%%%%%%%%%%%%%%%%%%%%%%%%%%%%%%%%%%%%%%%%%%%%%%%%%%%%%%%%%%%%%%%%%
%%%%%%%%%%%%%%%%%%%%%%%%%%%%%%%%%%%%%%%%%%%%%%%%%%%%%%%%%%%%%%%%%%%%%%%%%%%%%%%%

\section{Different scenarios}\label{sec:scenarios}

In this section we present five different scenarios, alternative to the three-layer model presented above, for the placement of the nonlinear polarization $\boldsymbol{\mathcal{P}}(2\omega)$ and the fundamental electric field $\mathbf{E}(\omega)$. In these scenarios we neglect the SH multiple reflections contained in $R^{M\pm}_{l}$ through $R^{M}_{l}$, Eq. \eqref{eq:rm} and \eqref{m61}, respectively, for which we take $R^{M}_{l}\to R^{\ell b}_{l}$. This is equivalent of taking only one single reflection from the $\ell b$ interface. Within the three-layer model we neglect multiple reflections, as yet another scenario, by the same $R^{M}_{l}\to R^{\ell b}_{l}$ replacement in the formulae shown in the previous section. In what follows, we confine ourselves only to the the (111) surface and the $pP$ combination of incoming-outgoing polarizations, since this is the case where the proposed scenarios differ the most. However, the other $pS$, $sP$ and $sS$ polarization cases, and (100) and (110) surfaces could be worked out along the same lines described below. For all the scenarios we have that the omission of multiple SH reflections by taking $R^{M\pm}_{p}\to 1\pm R^{\ell b}_{p}$ (Eq. \eqref{eq:rm}) reduces to
\begin{align}\label{mvc89}
R^{M+}_{p}&\to\frac{N_{b}}{N_{\ell}}T^{\ell b}_{p}\nonumber\\
R^{M-}_{p}&\to\frac{N_{\ell}}{N_{b}}\frac{W_{b}}{W_{\ell}}T^{\ell b}_{p},
\end{align}
after using the expressions in Eq. \eqref{eq:mf}.


%%%%%%%%%%%%%%%%%%%%%%%%%%%%%%%%%%%%%%%%%%%%%%%%%%%%%%%%%%%%%%%%%%%%%%%%%%%%%%%%

\subsection{Three layer model: without multiple reflections}\label{sec:nomr}

Using Eq. \eqref{mvc89} in Eq. \eqref{rpp111} we obtain
\begin{equation}\label{m79}
\Gamma_{pP}=
\frac{T_{p}^{\ell v}T^{\ell b}_{p}}
     {N^{2}_{\ell}N_{b}}
\left(
\frac{t_{p}^{v\ell}t^{\ell b}_{p}}
     {n^{2}_{\ell}n_{b}}
\right)^{2},  
\end{equation}
and
\begin{equation}\label{m81}
r^{(111)}_{pP} =
N^{2}_{b}\sin\theta_{0}
\Big(
  n^{4}_{b}\sin^{2}\theta_{0}\chi^{zzz}
+ n^{4}_{\ell}w^2_{b}\chi^{zxx}
\Big)
- N^{2}_{\ell}n^{2}_{\ell}w_{b}W_{b}
\Big(
  2n^{2}_{b}\sin\theta_{0}\chi^{xxz}
+ n^{2}_{\ell}w_{b}\chi^{xxx}\cos(3\phi) 
\Big).
\end{equation}
Now that we have neglected multiple SH reflections, we can use above two expressions for $\Gamma_{pP}$ and $r_{pP}$ to obtain the following four scenarios, by using the choices as described in each subsection bellow. We mention that by neglecting the multiple reflections the thickness $d$ of layer $\ell$ disappears from the formulation, and the location of the nonlinear polarization sheet $\mathbf{P}(\mathbf{r},t)$ (Eq. \eqref{eq:m31}) at $d_{2}$ (see Fig. \ref{fig:MR3layer2w}), is inmaterial.

%%%%%%%%%%%%%%%%%%%%%%%%%%%%%%%%%%%%%%%%%%%%%%%%%%%%%%%%%%%%%%%%%%%%%%%%%%%%%%%%
%%%%%%%%%%%%%%%%%%%%%%%%%%%%%%%%%%%%%%%%%%%%%%%%%%%%%%%%%%%%%%%%%%%%%%%%%%%%%%%%


\subsection{Two layer model}\label{sec:2layer}

Historically, this is the model most used in the literature, and our three-layer model with multiple reflections, as mentioned in the introduction, is a clear improvement upon the simple two layer model. In the two layer model, one considers that $\boldsymbol{\mathcal{P}}(2\omega)$, is evaluated in the vacuum region, while the fundamental fields are evaluated in the bulk region.\cite{sipePRB87, mizrahiJOSA88} To do this, we take the $2\omega$ radiations factors for vacuum by taking $\ell=v$, thus $\epsilon_{\ell}(2\omega)=1$, $T^{\ell v}_{p}=1$, $T^{\ell b}_{p} = T^{vb}_{p}$, and the fundamental field inside medium $b$ by taking $\ell=b$, thus $\epsilon_{\ell}(\omega)=\epsilon_{b}(\omega)$, $t^{v\ell}_{p}=t^{vb}_{p}$, and $t^{\ell b}_{p}=1$. With these choices Eqs. \eqref{m79} and \eqref{m81} reduce to
\begin{equation}\label{eq:m78}
\Gamma_{pP}
= \frac{T^{v b}_{p}(t^{vb}_{p})^2}
       {n^{2}_{b}N_{b}}, 
\end{equation}
and
\begin{equation}\label{eq:m82}
r^{(111)}_{pP} =
N^{2}_{b}\sin\theta_{0}
\Big(
\sin^{2}\theta_{0}\chi^{zzz} + w^{2}_{b}\chi^{zxx}
\Big)
- w_{b}W_{b}
\Big(
2\sin\theta_{0}\chi^{xxz} + w_{b}\chi^{xxx}\cos(3\phi)
\Big),
\end{equation}
and these expressions are in agreement with Refs. \cite{sipePRB87} and
\cite{mizrahiJOSA88}.


%%%%%%%%%%%%%%%%%%%%%%%%%%%%%%%%%%%%%%%%%%%%%%%%%%%%%%%%%%%%%%%%%%%%%%%%%%%%%%%%

\subsection{Taking the nonlinear polarization and the fundamental fields in the
bulk}\label{sec:bulk}

We follow the same procedure as above considering that both the $2\omega$ and $1\omega$ terms will be evaluated in the bulk taking $\ell = b$, thus $\epsilon_{\ell}(2\omega)=\epsilon_{b}(2\omega)$, $T^{v\ell}_{p}=T^{vb}_{p}$, $T^{\ell b}_{p} = 1$, and $\epsilon_{\ell}(\omega)=\epsilon_{b}(\omega)$, $t^{v\ell}_{p} = t^{vb}_{p}$, and $t^{\ell b}_{p} = 1$. With these choices Eqs. \eqref{m79} and \eqref{m81} reduce to
\begin{equation}
\Gamma_{pP} =
\frac{T_{p}^{vb}\left(t^{vb}_{p}\right)^{2}}
     {n^{2}_{b}N_{b}}, 
\end{equation}
and
\begin{equation}
r^{(111)}_{pP} =
  \sin^{3}\theta_{0}\chi^{zzz}
 + w^{2}_{b}\sin\theta_{0}\chi^{zxx} 
 - 2w_{b}W_{b}\sin\theta_{0}\chi^{xxz}
 - w^{2}_{b}W_{b}\chi^{xxx}\cos3\phi.
\end{equation}


%%%%%%%%%%%%%%%%%%%%%%%%%%%%%%%%%%%%%%%%%%%%%%%%%%%%%%%%%%%%%%%%%%%%%%%%%%%%%%%%

\subsection{Taking the nonlinear polarization in \texorpdfstring{$\ell$}{l} and
the fundamental fields in the bulk}\label{sec:hybrid}

Again, we follow the same procedure as above considering that $2\omega$ terms are evaluated in the thin layer $\ell$, and the $1\omega$ terms will be evaluated in the bulk by taking $\ell = b$, thus $\epsilon_{\ell}(\omega)=\epsilon_{b}(\omega)$, $t^{v\ell}_{p} = t^{vb}_{p}$, and $t^{\ell b}_{p} = 1$. With these choices Eqs. \eqref{m79} and \eqref{m81} reduce to
\begin{equation}
\Gamma^{\ell b}_{pP}=
\frac{T^{v\ell}_{p}T^{\ell b}_{p}\left(t^{vb}_{p}\right)^{2}}
  {N^{2}_{\ell}n^{2}_{b}N_{b}}
,
\end{equation}
and
\begin{equation}
r^{(111)}_{pP} = 
  N^{2}_{b}\sin^{3}\theta_{0}\chi^{zzz}
+ N^{2}_{b}k^{2}_{b}\sin\theta_{0}\chi^{zxx}
- 2N^{2}_{\ell}w_{b}W_{b}\sin\theta_{0}\chi^{xxz}
- N^{2}_{\ell}w^{2}_{b}W_{b}\chi^{xxx}\cos3\phi.
\end{equation}


%%%%%%%%%%%%%%%%%%%%%%%%%%%%%%%%%%%%%%%%%%%%%%%%%%%%%%%%%%%%%%%%%%%%%%%%%%%%%%%%

\subsection{Taking the nonlinear polarization and the fundamental fields in the
vacuum}\label{sec:vacuum}

Our last scenario considers both the $\boldsymbol{\mathcal{P}}(2\omega)$ and fundamental fields evaluated in the vacuum. We take $\ell = v$, thus $\epsilon_{\ell}(2\omega)=1$, $T^{\ell v}_{p}=1$, $T^{\ell b}_{p} = T^{vb}_{p}$, and $\epsilon_{\ell}(\omega) = 1$, $t^{v\ell}_{p} = 1$, and $t^{\ell b}_{p} = t^{vb}_{p}$. With these choices Eqs. \eqref{m79} and \eqref{m81} reduce to
\begin{equation}
\Gamma_{pP} =
\frac{T^{v b}_{p}\left(t^{v b}_{p}\right)^{2}}
     {n^{2}_{b}N_{b}}
,
\end{equation}
and
\begin{equation}
r^{(111)}_{pP} =
  n^{4}_{b}N^{2}_{b}\sin^{3}\theta_{0}\chi^{zzz}
+ N^{2}_{b}w^{2}_{b}\sin\theta_{0}\chi^{zxx}
- 2n^{2}_{b}w_{b}W_{b}\sin\theta_{0}\chi^{xxz}
- w^{2}_{b}W_{b}\chi^{xxx}\cos3\phi.
\end{equation}

We summarize all these scenarios in Table \ref{tab:models} for quick reference.

\begin{table}[t]
\centering
\begin{tabular}{| l | c | c |}
\hline 
Label           &  $\boldsymbol{\mathcal{P}}(2\omega)$  &  $\mathbf{E}(\omega)$ \\
\hline 
3-layer         &          $\ell$           &      $\ell$   \\
2-layer-fresnel &            $v$            &        $b$    \\
2-layer-bulk    &            $b$            &        $b$    \\
3-layer-hybrid  &          $\ell$           &        $b$    \\
2-layer-vacuum  &            $v$            &        $v$    \\
\hline 
\end{tabular}
\caption{Summary of SSHG yield models. ``Label'' is the name used in subsequent figures, while the remaining columns show in which medium we will consider the specified quantity. $\ell$ is the thin layer below the surface of the material, $v$ is the vacuum region, and $b$ is the bulk region of the material.}
\label{tab:models}
\end{table}
