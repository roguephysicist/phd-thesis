%!TEX root = ../../thesis.tex

\null
\vfill

\begin{abstract}
In this thesis we formulate a theoretical approach of surface second-harmonic
generation (SSHG) from semiconductor surfaces based on the length gauge and the
electron density operator. Within the independent particle approximation, the
surface nonlinear second-order surface susceptibility tensor is calculated. We
include, in one unique formulation: (i) the scissors correction, needed to have
the correct value of the energy band gap, (ii) the contribution of the nonlocal
part of the pseudopotentials, and (iii) the derivation for the inclusion of the
cut function, used to extract the surface response. The first two contributions
are described by spatially nonlocal quantum mechanical operators and are fully
taken into account in the present formulation.

We also revisit the three layer model for the SSHG yield and demonstrate that it
provides more accurate results over several, more common, two layer models. We
fully derive the expressions for the SSHG yield, including the effect of
multiple reflections from both the second-harmonic and fundamental waves into
the final expressions. These detailed derivations are applicable to any surface,
regardless of symmetry considerations.

This entire framework is implemented in the TINIBA software suite, which was
developed over the course of this doctoral project. We will apply this framework
for the clean Si(001)(2$\times$1) and Si(111)(1$\times$1):H surfaces, and
compare with various experimental spectra from several different sources. These
surfaces provide an excellent platform for comparison with theory, and allows us
to offer this study as an efficient benchmark for this type of calculation.
Finally, we can conclude that this new approach to the calculation of the
second-harmonic spectra is versatile and accurate within this level of
approximation.
\end{abstract}

\vfill

\clearpage
