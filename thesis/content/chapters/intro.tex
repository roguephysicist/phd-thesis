%!TEX root = ../../thesis.tex
\chapter{Introduction}\label{chap:intro}
\partialtoc


%%%%%%%%%%%%%%%%%%%%%%%%%%%%%%%%%%%%%%%%%%%%%%%%%%%%%%%%%%%%%%%%%%%%%%%%%%%%%%%%
%%%%%%%%%%%%%%%%%%%%%%%%%%%%%%%%%%%%%%%%%%%%%%%%%%%%%%%%%%%%%%%%%%%%%%%%%%%%%%%%

\section{A Review of Nonlinear Optics}


%%%%%%%%%%%%%%%%%%%%%%%%%%%%%%%%%%%%%%%%%%%%%%%%%%%%%%%%%%%%%%%%%%%%%%%%%%%%%%%%

\subsection{Historical Overview}\label{chap_theory_hist}

The discovery of the optical maser by Townes \cite{PhysRev.112.1940} and the
construction of the laser by Maiman in the late 1950s and early 1960s ushered a
new age of optical discoveries. The ability to produce optical beams with these
devices automatically lead to very highly focused energies distributed over very
small areas. These concentrated energies allowed scientists to finally move into
the optical nonlinear regime for many different materials. The laser allowed for
the first recorded observation of optical second-harmonic generation (SHG) by
Franken et al. in 1961 \cite{PhysRevLett.7.118}. They produced a second beam of
light at twice the frequency of the original by exciting a piece of crystalline
quartz. This frequency doubling effect was dubbed SHG and was observed to be
much less intense than the exciting beam. There is a humorous anecdote about
this experiment. Apparently, the editor of Physical Review Letters thought that
the second harmonic dot on the photographic plate was a speck of dust, which he
edited out. The image found in the article has an arrow pointing at the empty
spot where it should be. However, this did not detract from the importance of
the find.

Other developments followed promptly. In 1962, Bloembergen et al.
\cite{PhysRev.127.1918, PhysRev.128.606} developed the mathematical framework to
explain nonlinear optical phenomena. That same year, Terhune et al.
\cite{PhysRevLett.8.404} observed SHG in calcite. These discoveries were amongst
others \cite{lax1962nonlinear} that lead to further research into the
geometrical dependence of nonlinear effects, and helped verify that the majority
of the SHG signal produced in a centrosymmetric material comes from surface
contribution, where inversion symmetry is broken. In the late 1960s, Bloembergen
\cite{PhysRev.174.813} and others \cite{PhysRev.178.1218} studied SHG in a
variety of centrosymmetric materials and semiconductors. The advent of pulsed
lasers during the 1970s \cite{nla.cat-vn2583352} allowed for even greater
intensities to be obtained. Dye lasers came to prominence during these years,
offering very large bandwidths and relatively short picosecond pulses. However,
these lasers were very difficult to maintain and the dyes used were typically
very toxic and presented serious health risks. Interest began to form around
using SHG to study surfaces and interfaces, since it had been proven
\cite{PhysRevLett.46.145} to be exclusive to the surface area of a
centrosymmetric material in the dipole approximation. Shen et al. published
\cite{PhysRevB.38.7985} that there is also a quadrupole bulk contribution for
this kind of material, and in 1989 \cite{shen89nature} published a review
article summarizing most of the trends in surface spectroscopy using SHG.
Theoretical work also played an important role in the 1990s, with new
theoretical models by Sipe \cite{PhysRevB.53.10751} and others
\cite{PhysRevB.53.4999, PhysRevB.60.14334, PhysRevB.55.2489, PhysRevB.57.2569}.
Downer et al. \cite{downer2001optical} and L\"upke \cite{Lupke199975} both
produced very thorough and referenced texts on SHG surface spectroscopy of
semiconductors in the late 1990s and early 2000s. This period of time provided
the foundations for surface nonlinear optics today.

At around the same time, the first Ti:sapphire lasers were being produced and
analyzed \cite{Moulton:86}. These early ultrafast lasers were capable of
producing femtosecond pulses via mode-locked oscillators. Since the active
medium is in solid state form, they present none of the risks of using dyes.
These lasers were considerably more compact than dye lasers since they no longer
needed external dye control systems. These lasers became commercial in the early
1990s. Chirped pulse amplification (CPA) was invented in 1985 by Mourou and
Strickland \cite{Strickland1985447}. This technique allowed Ti:sapphire lasers
to achieve much higher peak energy without compromising the ultrashort pulse
duration. During the 1990s, CPA became the prominent method for increasing
energy output in Ti:sapphire lasers. At this point, Ti:sapphire lasers using the
CPA technique were both compact, efficient, and cost effective. These factors
would only improve over the following decade as the Ti:sapphire laser became the
standard for high energy, ultrashort pulse applications.


%%%%%%%%%%%%%%%%%%%%%%%%%%%%%%%%%%%%%%%%%%%%%%%%%%%%%%%%%%%%%%%%%%%%%%%%%%%%%%%%

\subsection{Nonlinear Polarization and Susceptibility}

So what happens when very intense light coincides on a given material? Let us
talk about the dipole moment per unit volume, or polarization $\mathbf{P}(t)$.
This polarization describes the effect light has on a material and vice versa;
it represents the optical response of a material. Taking Maxwell's equations
with the usual considerations of zero charge density ($\rho=0$) and no free
currents ($\mathbf{J}=0$), we have
\begin{align}
\nabla\cdot\mathbf{D} &= 0,\label{eq_max_1}\\
\nabla\cdot\mathbf{H} &= 0,\\
\nabla\times\mathbf{E} &= -\mu_{0}\frac{\partial\mathbf{H}}{\partial t},\\
\nabla\times\mathbf{H} &= \frac{\partial\mathbf{D}}{\partial t}.
\end{align}
We take into account the nonlinearity of the material by relating the \textbf{D}
and \textbf{E} fields with the total (linear and nonlinear) polarization
\textbf{P},
\begin{equation}
\mathbf{D} = \epsilon_{0}\mathbf{E} + \mathbf{P}.
\end{equation}
Proceeding in the usual manner for deriving the wave equation, we obtain
\begin{equation}
\nabla\times\nabla\times\mathbf{E} 
   + \frac{1}{c^{2}}\frac{\partial^{2}}{\partial t^{2}}\mathbf{E}
= -\frac{1}{\epsilon_{0}c^{2}}\frac{\partial^{2}\mathbf{P}}{\partial t^{2}},
\end{equation}
which can be considerably simplified thanks to the identity
\begin{equation}
\nabla\times\nabla\times\mathbf{E}
= \nabla\left(\nabla\cdot\mathbf{E}\right)-\nabla^{2}\mathbf{E},
\end{equation}
so we can finally express the inhomogenous wave equation as
\begin{equation}
\nabla^{2}\mathbf{E} 
- \frac{1}{c^{2}}\frac{\partial^{2}}{\partial t^{2}}\mathbf{E}
= \frac{1}{\epsilon_{0}c^{2}}\frac{\partial^{2}\mathbf{P}}{\partial t^{2}}.
\end{equation}
In this form, it is clear that the polarization acts as a source for this
differential equation, analogous to a simple harmonic oscillator. The
polarization can be expressed by a power series of the form
\begin{align}
{P}(t)
&= \epsilon_{0}\left[\boldsymbol{\chi}^{(1)}{E}(t) 
 + \boldsymbol{\chi}^{(2)}{E}^{2}(t)
 + \boldsymbol{\chi}^{(3)}{E}^{3}(t)
 + \ldots\right]\label{eq_power}\\
&\equiv {P}^{(1)}(t) + {P}^{(2)}(t) + {P}^{(3)}(t) + \ldots,
\end{align}
where $\boldsymbol{\chi}^{(n)}$ is the $n^{\mathrm{th}}$-order susceptibility of the
material. We can define the susceptibility as a constant of proportionality
that describes the degree of polarizability a material has in terms of the
strength of an incoming optical electric field. The first term
\begin{equation}
P(t) = \epsilon_{0}\boldsymbol{\chi}^{(1)}E(t),
\end{equation}
is the linear term that describes most everyday interactions between light and
matter. When taking into account that the incoming fields are vectorial in
nature, the linear susceptibility $\boldsymbol{\chi}^{(1)}$ becomes a second-rank tensor.
$\boldsymbol{\chi}^{(2)}$, the second-order nonlinear optical susceptibility is a third-rank
tensor \cite{boyd2003nonlinear}. The nonlinear susceptibilities are very small
in nature. If $\boldsymbol{\chi}^{(1)}$ is unity, $\boldsymbol{\chi}^{(2)}$ is on the order of $\approx
10^{-12}\,\text{m/V}$. This explains why such high intensity fields are needed
to produce nonlinear interactions; each term in equation \eqref{eq_power}
depends on a higher power of the incoming field but has a much smaller value for
the corresponding susceptibility.

A more general definition of the nonlinear polarization can be found when
treating the input field as a superposition of plane waves. We assume that the
electric field vector is of the form
\begin{equation}
\mathbf{E}(\mathbf{r},t) = \sum_{n}\mathbf{E}_{n}(\mathbf{r},t),
\end{equation}
where
\begin{equation}
\mathbf{E}_{n}(\mathbf{r},t)
= \mathbf{E}_{n}(\mathbf{r})e^{-i\omega_{n}t} + \text{c.c.}.
\end{equation}
If we look at the form of equation \eqref{eq_power}, we can express the
nonlinear polarization in its full form as
\begin{equation}
\mathbf{P}(\mathbf{r},t) = \sum_{n}\mathbf{P}(\omega_{n})e^{-i\omega_{n}t}.
\end{equation}
Since we are only interested in second-order effects we can define the
corresponding nonlinear polarization in terms of the second order susceptibility
as
\begin{equation}
P^{\mathrm{a}}(\omega_{n} + \omega_{m})
= \epsilon_{0}\sum_{\mathrm{bc}}\sum_{(nm)}
\boldsymbol{\chi}^{(2),\mathrm{abc}}(-(\omega_{n}+\omega_{m});\omega_{n},\omega_{m})
E^{\mathrm{b}}(\omega_{n})E^{\mathrm{c}}(\omega_{m}),\label{eq_nonlin_p}
\end{equation}
where the indices $\mathrm{abc}$ refer to the Cartesian components of the
fields, and $(nm)$ denotes that $n$ and $m$ can be varied while the sum
$\omega_{n} + \omega_{m}$ remains fixed. We can study the generalized case when
we have two incoming scalar fields with frequencies $\omega_{1}$ and
$\omega_{2}$. We can represent this in the following form
\begin{equation}
E(t) = E_{1}e^{-i\omega_{1}t} + E_{2}e^{-i\omega_{2}t}
     + \text{c.c.}.\label{eq_sfg_form}
\end{equation}
Assuming the form of equation \eqref{eq_power}
\begin{equation}
P^{(2)} = \epsilon_{0}\boldsymbol{\chi}^{(2)}E^{2}(t),
\end{equation}
and substituting expression \eqref{eq_sfg_form} we get
\begin{align}
P^{(2)}(t)
&= \epsilon_{0}\boldsymbol{\chi}^{(2)}\left[E^{2}_{1}e^{-i2\omega_{1}t}
 + E^{2}_{2}e^{-i2\omega_{2}t}\right.\nonumber\\
&+ \left. 2E_{1}E_{2}e^{-i(\omega_{1}+\omega_{2})t}
 + 2E_{1}E^{\ast}_{2}e^{-i(\omega_{1}-\omega_{2})t}
 + \text{c.c.}\right]\nonumber\\
&+ 2\epsilon_{0}\boldsymbol{\chi}^{(2)}\left[E_{1}E^{\ast}_{1}
 + E_{2}E^{\ast}_{2}\right].\label{eq_second_order}
\end{align}
We separate this expression into its components and the nonlinear effect
represented, in the following manner (abbreviations defined in table
\ref{tab:janner}),
\begin{equation}\label{eq_list}
\begin{split}
P(2\omega_{1}) 
&= \epsilon_{0}\boldsymbol{\chi}^{(2)}E^{2}_{1}e^{-i2\omega_{1}t}
 + \text{c.c.}\quad\text{(SHG)},\\
%%%%%%%%%%%%%%%%%%%%%%%%%%%%%%%%%%%%%%%%%%%%%%%%%
P(2\omega_{2})
&= \epsilon_{0}\boldsymbol{\chi}^{(2)}E^{2}_{2}e^{-i2\omega_{2}t}
 + \text{c.c.}\quad\text{(SHG)},\\
%%%%%%%%%%%%%%%%%%%%%%%%%%%%%%%%%%%%%%%%%%%%%%%%%
P(\omega_{1}+\omega_{2})
&= 2\epsilon_{0}\boldsymbol{\chi}^{(2)}E_{1}E_{2}e^{-i(\omega_{1}+\omega_{2})t}
 + \text{c.c.}\quad\text{(SFG)},\\
%%%%%%%%%%%%%%%%%%%%%%%%%%%%%%%%%%%%%%%%%%%%%%%%%
P(\omega_{1}-\omega_{2})
&= 2\epsilon_{0}\boldsymbol{\chi}^{(2)}E_{1}E^{\ast}_{2}e^{-i(\omega_{1}-\omega_{2})t}
 + \text{c.c.}\quad\text{(DFG)},\\
%%%%%%%%%%%%%%%%%%%%%%%%%%%%%%%%%%%%%%%%%%%%%%%%%
P(0)
&= 2\epsilon_{0}\boldsymbol{\chi}^{(2)}\left(E_{1}E^{\ast}_{1}
 + E_{2}E^{\ast}_{2}\right) + \text{c.c.}\quad\text{(OR)}.
\end{split}
\end{equation}
Janner \cite{janner1998exciton} has a wonderfully formatted table in her
dissertation that summarizes the first few optical processes, reproduced here in
Table \ref{tab:janner}. From this point forward we will only be concerned with
second-order effects.

\begin{table}[b]
\caption{Optical processes described with $\boldsymbol{\chi}^{(n)}$.\label{tab:janner}}
\centering
\scalebox{0.9}{
\begin{tabular}{| c c c | p{7cm} | c |}
\hline
\multicolumn{3}{|c|}
{\,\,\,
$\boldsymbol{\chi}^{(n)}(-(\omega_{1} + \ldots + \omega_{n});\omega_{1},\ldots,\omega_{n})$}
& Process & Order \\ \hline
$-\omega_{1}$ & ; & $\omega_{1}$
& Linear absorption / emission and refractive index & 1 \\ \hline
$0$ & ; & $\omega_{1},-\omega_{1}$
& Optical rectification (OR) & 2 \\ \hline
$-\omega_{1}$ & ; & $0,\omega_{1}$
& Pockels effect & 2 \\ \hline
$-2\omega_{1}$ & ; & $\omega_{1},\omega_{1}$
& Second-harmonic generation (SHG) & 2 \\ \hline
$-(\omega_{1}+\omega_{2})$ & ; & $\omega_{1},\omega_{2}$
& Sum-frequency generation (SFG) & 2 \\ \hline
$-(\omega_{1}-\omega_{2})$ & ; & $\omega_{1},\omega_{2}$
& Difference-frequency generation (DFG) / 
  Parametric amplification and oscillation & 2 \\ \hline
$-\omega_{1}$ & ; & $0,0,\omega_{1}$
& d.c. Kerr effect & 3 \\ \hline
$-2\omega_{1}$ & ; & $0,\omega_{1},\omega_{1}$
& Electric Field induced SHG (EFISH) & 3 \\ \hline
$-3\omega_{1}$ & ; & $\omega_{1},\omega_{1},\omega_{1}$
& Third-harmonic generation (THG) & 3 \\ \hline
$-\omega_{1}$ & ; & $\omega_{1},-\omega_{1},\omega_{1}$
& Degenerate four-wave mixing (DFWM) & 3 \\ \hline
$-\omega_{1}$ & ; & $-\omega_{2},\omega_{2},\omega_{1}$
& Two-photon absorption (TPA) / ionization / emission & 3 \\ \hline
\end{tabular}}
\end{table}


%%%%%%%%%%%%%%%%%%%%%%%%%%%%%%%%%%%%%%%%%%%%%%%%%%%%%%%%%%%%%%%%%%%%%%%%%%%%%%%%

\subsection{Symmetry Considerations for Centrosymmetric Materials}
\label{chap_theory_sym}

As mentioned previously, $\boldsymbol{\chi}^{(2)}$ is a third-rank tensor with
27 elements. The amount of non-zero elements varies with the symmetry properties
of the medium. SHG has intrinsic permutation symmetry for the incoming fields,
such that $\chi^{(2),\mathrm{abc}} = \chi^{(2),\mathrm{acb}}$; this reduces the
total components from 27 to 18. Knowledge of the symmetry properties of the
material can help us reduce the amount of unknown elements to calculate.

A centrosymmetric material, or a material with an inversion center, is a
material that for every point at coordinates $(x,y,z)$, there is an identical
point located at $(-x,-y,-z)$. For instance, many crystals are centrosymmetric.
If we assume that we are in the bulk of a centrosymmetric material, we can write
the nonlinear polarization as
\begin{equation}\label{eq_regular}
{P}(\mathbf{r},t) = \epsilon_{0}\boldsymbol{\chi}^{(2)}{E}^{2}(t).
\end{equation}
If the medium is centrosymmetric, a sign change on the coordinates must affect
both the electric field and the polarization since they are polar vectors
\cite{jacksonbook}. So,
\begin{align}\label{eq_centro}
-{P}(\mathbf{r},t)
&= \epsilon_{0}\boldsymbol{\chi}^{(2)}\left[-{E}(t)\right]^{2},\\
&= \epsilon_{0}\boldsymbol{\chi}^{(2)}{E}^{2}(t).
\end{align}
However, substituting \eqref{eq_centro} into \eqref{eq_regular} we get
${P}(\mathbf{r},t) = -{P}(\mathbf{r},t)$. We can finally deduce that
\begin{equation}
\boldsymbol{\chi}^{(2)} = 0.
\end{equation}
Therefore, all second-order processes are forbidden in the bulk of
centrosymmetric materials in the dipole approximation. We will talk about
another important approximation in section \ref{chap_theory_quad}. This property
is broken at the surface since that region no longer presents an inversion
center.  This very special property is what enables second-order nonlinearities
to be so effective for surface and interface measurements. Likewise, any other
mechanism that breaks the symmetry, such as an electric field or mechanical
stress will also allow a second-order signal to be produced. See Bloembergen's
\cite{bloembergen1999surface} excellent review about second-order effects for
surface spectroscopy for further reading.


%%%%%%%%%%%%%%%%%%%%%%%%%%%%%%%%%%%%%%%%%%%%%%%%%%%%%%%%%%%%%%%%%%%%%%%%%%%%%%%%

\subsection{Bulk Quadrupolar and Other Contributions}\label{chap_theory_quad}

We assume that the nonlinear polarization can take the form of a multipole
expansion, as we expressed in Eq. \eqref{eq_power}. This work is interested
exclusively in the \emph{dipole approximation}, that assumes that the dipolar
contribution is significantly greater than the quadrupolar and higher order
contributions. This is not necessarily the case in many materials. In
particular, we find that there can be a non-negligible electric quadrupole
contribution from the bulk of centrosymmetric materials. Bloembergen et al.
\cite{PhysRev.174.813} elaborate on this as early as the 1960s. This adds a
severe complication to the use of second-order nonlinearities as surface probes
since signal is actually produced from both surface and bulk. Sipe et al.
\cite{sipe1987fundamental} go into some detail about this problem, stating that
it is very difficult to separate the surface and bulk contributions as the
various nonlinear coefficients cannot be measured separately. Guyot-Sionnest and
Shen \cite{PhysRevB.38.7985} go one step further and state that the
contributions are impossible to separate. They suggest that the best way to
distinguish one from the other is by taking measurements before and after
altering the surface and observing the overall changes to the produced signal.
About a decade later, Shen et al. \cite{shen1999surface} state that bulk
contributions not only come from the electric quadrupole, but also from the
magnetic dipole, although the latter is typically much less intense than either
of the former. They express the bulk polarization as a multipole series as
follows,
\begin{equation}
\mathbf{P}^{B}(\omega)
= \mathbf{P}_{D}(\omega)
- \nabla\cdot\mathbf{Q}(\omega)
- \left(\frac{c}{i\omega}\right)\nabla\times\mathbf{M}(\omega)
+ \ldots,
\end{equation}
where $\mathbf{P}_{D}(\omega)$ is the dipolar polarization, $\mathbf{Q}(\omega)$
is the electric quadrupole polarization, and $\mathbf{M}(\omega)$ is the
magnetic dipole polarization. Indeed, if only the dipolar contribution is
forbidden for centrosymmetric materials then there will be a contribution from
the other two in addition to the dipolar contribution at the surface. The group
does however go on to explain that there are a few experimental ways that can
help distinguish between surface and bulk contributions.

If $\mathbf{Q}(\omega)$ is assumed to take some form similar to
\begin{equation}
\mathbf{Q}(\omega_{1}+\omega_{2})\approx \boldsymbol{\chi}^{(2)}_{q}(\omega_{1}
+ \omega_{2})\mathbf{E}(\omega_{1})\nabla\mathbf{E}(\omega_{2}),
\end{equation}
then $\boldsymbol{\chi}^{(2)}_{q}$ is a fourth-rank tensor with 81 independent
elements. Clearly this adds considerable complication to our problem; hence the
importance of selecting the appropriate symmetry. In summary, bulk electric
quadrupole and magnetic dipole contributions to second-order surface effects may
not be negligible and need to be taken into account. However, in this work we
study the nonlinear optical properties of Silicon, and it is known that the
quadrupolar effects are quite small \cite{aktsipetrovJETP86, sipePRB87,
xuJVST97, guyotPRB88, downerSIA01, shenAPB99}. Therefore, we will neglect these
effects for the remainder of this thesis.


%%%%%%%%%%%%%%%%%%%%%%%%%%%%%%%%%%%%%%%%%%%%%%%%%%%%%%%%%%%%%%%%%%%%%%%%%%%%%%%%

\subsection{SFG and SHG}\label{chap_theory_sum}

We call the third process in expression \eqref{eq_list} sum-frequency generation
(SFG). It is a second-order process that involves two photons, of frequencies
$\omega_{1}$ and $\omega_{2}$ that combine to form one photon of frequency
$\omega_{3} = \omega_{1} + \omega_{2}$. This is represented mathematically as
\begin{equation}
P(\omega_{1}+\omega_{2})
= 2\epsilon_{0}\chi^{(2)}E_{1}E_{2}e^{-i(\omega_{1}+\omega_{2})t} + \text{c.c.},
\end{equation}
where the term is explicitly stated in the exponential. A special case of
sum-frequency generation is when both incoming field frequencies are the same,
i.e. $\omega_{1} = \omega_{2}$; this case is also known as the degenerate case.
The resulting frequency is then exactly twice that of the input frequency.

As mentioned previously, second-order nonlinear processes are prohibited in the
bulk of centrosymmetric materials (in the dipole approximation). Since it has a
very strong surface contribution (where the inversion symmetry is broken), it
can be used as a very precise diagnostic tool for surface and interface regions.
The use of these second-order nonlinearites for surface studies had gained
momentum in the 1990s. McGilp wrote a review about using SHG and SFG as surface
and interface probes in 1996 \cite{mcgilp1996review}. He added experimental
confirmation to his theories in 1999 \cite{mcgilp1999second} in a thorough
review about using SHG on almost any surface. Aktsipetrov et al.
\cite{aktsipetrov1997dc} followed a different approach by establishing what they
call electric field induced second-harmonic generation, or EFISH. In this paper
they elaborate how the sensitivity of SHG to surfaces can be enhanced by
applying an electric field across the interface. More recently a book in the
field of second-order nonlinear optics has been published with a wealth of
useful information \cite{aktsipetrovbook}.

The theoretical side of things was further developed in a paper by Maytorena et
al. \cite{PhysRevB.57.2569} discussing the formalities of SFG from surfaces by
finding the exact expressions for the susceptibility based on modeling
conductors and dielectrics. These models include fluid based, classical dynamics
in addition to the wave equation treatment. A couple of interesting review
papers by Downer et al. \cite{downer2001optical} and Scheidt et al.
\cite{scheidt2004optical} exist, where they report results of SHG spectroscopies
from a variety of different surfaces and interfaces including nanocrystals.


%%%%%%%%%%%%%%%%%%%%%%%%%%%%%%%%%%%%%%%%%%%%%%%%%%%%%%%%%%%%%%%%%%%%%%%%%%%%%%%%
%%%%%%%%%%%%%%%%%%%%%%%%%%%%%%%%%%%%%%%%%%%%%%%%%%%%%%%%%%%%%%%%%%%%%%%%%%%%%%%%

\section{The Nonlinear Surface Susceptibility}\label{sec:introchi2}

In recent years surface nonlinear optical spectroscopies, particulary surface
second-harmonic generation (SSHG), have evolved as useful nondestructive and
noninvasive tools to study surface and interface properties. These properties
include atomic structure, phase transitions, adsorption of atoms, and many
others.\cite{daumPRL93, mcgilpOE94, meyerPRL95, powerPRL95, godefroyAPL96,
hoferAPA96, dadapPRB97, bloembergenAPB99, mcgilpSRL99, suzukiAPB99,
mitchellSS01, hughesPRB96, guyotPRB88, downerPSSA01, shenAPB99, shenNAT89,
chenPRL81, mendozaPRL98, downerSIA01} Nowadays, SSHG spectroscopy is a crucial
tool for research and development in microelectronics \cite{zheltikovLP00},
semiconductors \cite{lupkeSSR99}, nanomaterials and bidimensional materials
\cite{deanPRB14, malardPRB13}, and many more recent areas of great scientific
and commercial interest.\cite{cazzanelliNM14} The high surface sensitivity of
SSHG spectroscopy is due to the fact that within the dipole approximation the
bulk SHG signal of centrosymmetric materials is identically zero. However, the
SHG process can occur only at the surface where the inversion symmetry is
broken. As mentioned previously, the bulk quadrupole contribution for
centrosymmetric materials is different from zero, but usually it is very small
\cite{downerSIA01}, and can be neglected. Much of the foundation of surface
science has been built from experiments involving emission or scattering of
electrons from surfaces. These require ultrahigh vacuum (UHV) environments and
provide no access to buried interfaces. However, SSHG is compatible with non-UHV
conditions and has access to interfaces buried beneath transparent overlayers.
Even when applied to surfaces in UHV, the light source and detectors can be
aligned and used outside the vacuum chamber.

The usefulness of SSHG could be limited by the lack of microscopic theoretical
understanding of the nonlinear spectra. The macroscopic phenomenological theory
of SSHG, which relates the intensity, phase, and polarization of detected fields
to nonlinear and linear susceptibilities at the material interface is now fully
developed.\cite{downerSIA01} However, microscopic theory that relates
electronic-level structure to the nonlinear source polarization is still being
developed. \cite{butcherPOPS63, aspnesPRB72, sipePRB93, levinePRB94,
aversaPRB95, hughesPRB96, rashkeevPRB98, beyond,mendozaPRL98, arzatePRB01,
mendozaPRB01, mejiaPRB02, sanoPRB02, mejiaRMF04, trollePRB14} Within the
independent particle approximation (IPA) some frameworks for bulk SHG have been
developed to study the nonlinear optical response of bulk materials.
\cite{butcherPOPS63, aspnesPRB72, sipePRB93, levinePRB94,aversaPRB95,
hughesPRB96, rashkeevPRB98} In this thesis we put forward an approach to
calculate the microscopic second-harmonic surface susceptibility that
encompasses several theoretical features not taken into account before in the
case of a surface.

The most used framework for \emph{ab initio} calculations, Density Functional
Theory (DFT) within the Local Density Approximation (LDA),
\cite{kohnPR65} underestimates the energy band gap of semiconductors. It is well
understood that one has to include the many-body interaction to correct for this
underestimation of the gap. In this context, the so-called GW
approximation\cite{onidaRMP02} is known to correct the electronic gap of most
semiconductors\cite{luceroJPCM12}. However this can be a very expensive
calculation and thus one uses the much simpler scissors operator scheme.
\cite{levinePRL89,levinePRL91,delsolePRB93} This allows us to ``open'' the
DFT-LDA gap to its correct experimental or GW value for most bulk
semiconductors. This approximation has already been used in linear optical
calculations for surfaces,\cite{kippPRL96} thus improving the agreement with
experimental results. In this context, to correct for the underestimation of the
energy band gap of semiconductors Nastos et al.\cite{nastosPRB05} used the
``length gauge'' or ``$\mathbf{r}\cdot\mathbf{E}$'' gauge to show how to
correctly include the many-body corrections through the scissors operator in the
SH susceptibility. Later, Cabellos \textit{et al}.\cite{cabellosPRB09}
elaborated a derivation of the ``velocity gauge'' or
``$\mathbf{A}\cdot\mathbf{v}$'' gauge properly including the scissors operator
and proved gauge invariance with respect to the length gauge. From these works
it is clear the length gauge is a much better starting point to obtain the
surface second-harmonic (SSH) susceptibility, as will be elaborated in this
thesis. However, these considerations are only valid for bulk semiconductors.

Concerning the optical response of surfaces and interfaces, Reining \textit{et
al.}\cite{reiningPRB94} introduced the concept of a cut function in order to
obtain the surface SH susceptibility tensor. This cut function is required since
one usually uses a slab approach when treating semi-infinite surface
systems.\cite{reiningPRB94} If the slab is centrosymmetric the susceptibility
tensor will be identically zero. The cut function is such that it separates the
nonlinear response for the two surfaces of the slab avoiding the destructive
interference between them giving a finite value that one identifies with the SSH
susceptibility tensor. If the slab is not centrosymmetric the cut function can
be used to separate the different signals coming from either surface of the
slab. Indeed, one of the results of this thesis is to show that the SSH
susceptibility tensor obtained by using the cut function is correctly extracted
from the slab. After Reining \textit{et al.},\cite{reiningPRB94} Refs.
\cite{mendozaPRL98,arzatePRB01,mendozaPRB01,mejiaPRB02,sanoPRB02} followed upon
this work and in particular Ref. \cite{arzatePRB01} went into a detailed
analysis of the different contributions to the SHG spectra of a surface and the
nuanced relationship between bulk, surface, interband, intraband, $1\omega$ and
$2\omega$ terms, and Ref. \cite{mejiaRMF04} developed a layer-by-layer analysis
for the nonlinear responses of semiconductor systems, within a tight-binding
framework. This model allows for obtaining results from selected regions of a
system including the surface. However, in these references the scissors operator
is either excluded or incorrectly implemented. In the works that include this
operator, the velocity gauge was used to derive the expressions for the
nonlinear second-order susceptibility,
$\chi^{\mathrm{abc}}(-2\omega;\omega,\omega)$. Nevertheless, a term in the
time-dependent perturbation scheme necessary to satisfy the gauge invariance of
$\chi^{\mathrm{abc}}(-2\omega;\omega,\omega)$  was omitted. This was
demonstrated in Ref. \cite{cabellosPRB09} where a comparison between the
velocity and the length gauge was carried out.

Finally, DFT-LDA calculations are often based on the use of pseudopotentials. As
it will be discussed in this thesis, the presence of a nonlocal part of the
pseudopotential introduces corrections to the momentum operator of the electron
that have to be included with care in the SSH susceptibility. For the bulk
counterpart see for instance Refs. \cite{ismailPRL01,luppiPRB08}. Therefore,
within the IPA the most complete approach for the calculation of the SSH
susceptibility is one which includes (i) the scissors correction, (ii) the
contribution of the nonlocal part of the pseudopotential, and (iii) the cut
function. One of the goals of this thesis is to derive a new expression within
the length gauge for the SSH susceptibility tensor
$\chi^{\mathrm{a}\mathrm{b}\mathrm{c}}(-2\omega;\omega,\omega)$ that includes
the aforementioned contributions. The inclusion of these three contributions
makes our scheme unprecedented and opens the possibility to study surface SHG
with more versatility and providing accurate results.


%%%%%%%%%%%%%%%%%%%%%%%%%%%%%%%%%%%%%%%%%%%%%%%%%%%%%%%%%%%%%%%%%%%%%%%%%%%%%%%%
%%%%%%%%%%%%%%%%%%%%%%%%%%%%%%%%%%%%%%%%%%%%%%%%%%%%%%%%%%%%%%%%%%%%%%%%%%%%%%%%

\section{The SSHG Yield}

For calculating the SSHG yield, we will use the nonlinear surface susceptiblity
tensor produced with the framework mentioned in the previous section, and
featured in Ref. \cite{andersonPRB15}. This formulation includes three features
not previously found in a single formulation: (i) the scissors correction, (ii)
the contribution of the nonlocal part of the pseudopotentials, and (iii) the cut
function used to extract the surface response, all within the independent
particle approximation. The inclusion of these three contributions opens the
possibility to study SSHG with more versatility and accuracy than was previously
available at this level of approximation. We also use the three layer (3-layer)
model for the SSHG yield, which considers that the SH conversion takes place in
a thin layer just below the surface that lies under the vacuum region and above
the bulk of the material. Validating these improvements is difficult, however,
without experimental data for comparison.

SSHG experiments focusing on semiconductor surfaces are available, but they are
often reported over very limited energy ranges and lacking units and scale for
the intensity. This lack of comprehensive experimental data has made comparison
between theory and experiment difficult. However, the Si(111)(1$\times$1):H
surface offers some respite in this area. This surface can be prepared to a high
degree of structural quality and has been experimentally characterized with SHG
to a great degree of accuracy \cite{mitchellSS01, mejiaPRB02}. The added H
saturates the surface Si dangling bonds and eliminates any surface-related
electronic states in the band gap. We consider that this surface represents an
ideal benchmark for \emph{ab initio} SSHG studies. More specifically, SSHG from
the Si(111)(1$\times$1):H surface was treated in detail in Ref.
\cite{mejiaPRB02}, and their approach yielded good qualitative results.
However, the expressions presented for the nonlinear susceptibility tensor,
$\boldsymbol{\chi}(-2\omega;\omega,\omega)$, which is required for the SSHG
yield, are derived in the velocity gauge. This method incorrectly implements the
scissors quasiparticle correction and diverges for low energies
\cite{cabellosPRB09}. We mention that the formulas presented in Ref.
\cite{mejiaPRB02}, where the 3-layer model was introduced for the first time,
have some minor mistakes that have been corrected in Ref.
\cite{andersonARXIV16}. They also propose a two layer (2-layer) model for SSHG
which does not accurately represent the real physical process for surfaces. We
consider that the theoretical and computational aspects of this subject have
evolved considerably since then, making this topic ripe for revision.

In Chapter \ref{chap:results}, I will present a comparison between theory and
experiment by presenting the improved theoretical calculations against
experimental SSHG spectra from several sources, namely Refs. \cite{hoferAPA96,
bergfeldPRL04, mejiaPRB02, mitchellSS01}, with two-photon energies ranging from
2.5\,eV to 5\,eV covering both the E$_{1}$ and E$_{2}$ critical point
transitions for bulk Si. These SHG experiments were carried out with different
polarizations of incoming and outgoing beams which are taken into account in the
theoretical analysis. We find that the new formalism compares favorably with
experiment and permits insight into the physics behind SSHG. In spite of the
advances mentioned, our treatment neglects local field and excitonic effects
that are challenging from both a theoretical and a computational standpoint.
This topic merits further review and may prove to be crucial for more accurate
SSHG theory.


%%%%%%%%%%%%%%%%%%%%%%%%%%%%%%%%%%%%%%%%%%%%%%%%%%%%%%%%%%%%%%%%%%%%%%%%%%%%%%%%
%%%%%%%%%%%%%%%%%%%%%%%%%%%%%%%%%%%%%%%%%%%%%%%%%%%%%%%%%%%%%%%%%%%%%%%%%%%%%%%%

\section{Other Optical Nonlinear Phenomena of Interest}

Besides the phenomena mentioned above, there are several other nonlinear effects
that have become particularly interesting with the advent of bidimensional
materials, such as MoS$_{2}$, MoSe$_{2}$, WS$_{2}$, WSe$_{2}$, and Graphene.
In particular, Graphene is an allotrope of carbon with a planar, hexagonal,
two-dimensional honeycomb structure with one carbon atom at each vertex. It has
attracted a great deal of interest due to its distinctive properties, such as
the fractional quantum Hall effect at room temperature, and excellent thermal
transport properties \cite{geimNM07, reinaNL08, novoselov2S7, balandinNL08}. It
behaves like a metal, but can be modified to semiconductor behavior by tuning
the band gap. This can be achieved by changing the surface area \cite{hanPRL07},
applying an electric field \cite{zhangN09}, applying uniaxial strain
\cite{niACSN08}, or by doping, amongst other methods. Previous works have
explored doping with boron, nitrogen \cite{guoIJ11}, and hydrogen
\cite{eliasS09, guisingerNL09, samarakoonACSN10}. Hydrogenated graphene can
achieve different spatial configurations by varying the amount and location of
the hydrogen bonds. When a hydrogen atom is bonded to a carbon atom in graphene,
it pulls the atom away from the plane. This modifies the carbon-carbon bond
length resulting in an opening of the band gap \cite{eliasS09, boukhvalovPRB08}.
During my doctoral research, we carried out a theoretical study of three optical
nonlinear phenomena for two hydrogenated graphane structures: optical spin
injection, optical current injection and second-harmonic generation (SHG). I
describe the former two as follows.


%%%%%%%%%%%%%%%%%%%%%%%%%%%%%%%%%%%%%%%%%%%%%%%%%%%%%%%%%%%%%%%%%%%%%%%%%%%%%%%%

\subsection{Optical Spin Injection}

The injection and detection of spin polarized electrons in nonmagnetic materials
is at the core of spintronics \cite{vzuticRMP04, fertRMP08} and an important
problem in condensed matter theory. The idea of creating and detecting spin
polarized electrons from light originates in the 1960s with Ref.
\cite{lampelPRL68}. Following that work, it was later demonstrated that
converting the angular momentum of light into electron spin is very efficient in
III-IV semiconductors \cite{dyakonovOO84}. Optical spin injection is
characterized through the dimensionless degree of spin polarization (DSP),
$\boldsymbol{\mathcal{D}}(\omega)$. DSP quantifies the fraction of injected
electrons in the conduction bands that are spin polarized. This effect occurs
when circularly polarized light is incident on a semiconducting material
\cite{dyakonovOO84}, thus allowing electrons to move from the valence to the
conduction bands. The resulting polarization is produced by the interaction
between the electron spin and its motion caused by the spin-orbit coupling in
the material. DSP can be calculated with a full band structure method as shown
in Refs. \cite{nastosPRB07,cabellosPRB09}. There are theoretical reports of DSP
calculations for bulk media \cite{nastosPRB07, cabellosPRB09} and also for
surfaces \cite{mendozaPRB12, arzatePRB14}.


%%%%%%%%%%%%%%%%%%%%%%%%%%%%%%%%%%%%%%%%%%%%%%%%%%%%%%%%%%%%%%%%%%%%%%%%%%%%%%%%

\subsection{Optical Current Injection}

The optical current injection is a second-order optical nonlinear effect that
has been the subject of research in recent years
\cite{arzatePRB14,bhatPRB05,fraserPRL99,hachePRL97,lamanAPL99}. A
photocurrent, $\mathbf{\dot{J}}(\omega)$, can be injected with a single
optical beam into noncentrosymmetric materials or at the surface of bulk
centrosymmetric materials where the inversion symmetry has been broken
\cite{arzatePRB14}. This phenomenon results from the interference of one-photon 
absorption processes associated with different linear polarizations of light. In
the process of current injection, the energy increase of the injected carriers
is provided by the electromagnetic field, while the increase in momentum is
provided by the crystal lattice \cite{arzatePRB14}. One-photon current injection
is characterized by the current injection tensor, $\eta^{abc}(\omega)$, and
since it is generated with circularly polarized light, this phenomenon is also
called the circular photovoltaic effect \cite{sturmanCRCP92}. This effect has
been studied in bulk semiconductors \cite{hachePRL97, sipePRB00},
two-dimensional systems \cite{melePRB00, cabellosPRB11}, and one-dimensional
nanotubes \cite{melePRB00}. Bidimensional materials are quite often
noncentrosymmetric and present an optical current injection response.


%%%%%%%%%%%%%%%%%%%%%%%%%%%%%%%%%%%%%%%%%%%%%%%%%%%%%%%%%%%%%%%%%%%%%%%%%%%%%%%%
%%%%%%%%%%%%%%%%%%%%%%%%%%%%%%%%%%%%%%%%%%%%%%%%%%%%%%%%%%%%%%%%%%%%%%%%%%%%%%%%

\section{Outline}
This thesis is divided into 5 chapters including this introduction. Chapter
\ref{chap:chi2} presents the derivation of explicit expressions for the
nonlinear surface susceptibility,
$\boldsymbol{\chi}_{\mathrm{surface}}(-2\omega;\omega,\omega)$, featuring the
developments mentioned in Sec. \ref{sec:introchi2}. Chapter \ref{chap:sshgyield}
presents the derivation of the expressions for the SSHG yield that utilize the
calculated components of the nonlinear surface susceptibility. Chapter
\ref{chap:results} presents the spectra for both
$\boldsymbol{\chi}_{\mathrm{surface}}(-2\omega;\omega,\omega)$ and the SSHG
yield for two surfaces, Si(001)(2$\times$1) and Si(111)(1$\times$1):H. I present
comparisons with experimental data, and we will find that the results from the
newly derived theory compares quite favorably to the experimental spectra.
Chapter \ref{chap:conclusions} is dedicated to the final observations and
remarks. Appendix \ref{app:chi2deriv} contains the derivations for several
necessary terms needed for
$\boldsymbol{\chi}_{\mathrm{surface}}(-2\omega;\omega,\omega)$, and Appendix
\ref{app:sshg_explicit_expressions_rif} presents the complete, step-by-step
derivations for the SSHG yield. Finally, the complete bibliography is located at
the end of the document for easy reference.

\stopcontents[chapters]
