\documentclass[aps,prb,10pt,letterpaper,notitlepage]{revtex4-1}
\usepackage{amsmath}
\usepackage{fullpage}

% \usepackage[colorlinks=true]{hyperref}

\begin{document}

\title{On rotating $\boldsymbol{\chi}(-2\omega;\omega,\omega)$}
\author{Sean M. Anderson}
\email{sma@cio.mx}
\affiliation{Centro de Investigaciones en \'Optica, A.C., Le\'on 37150, Mexico}
\date{\today}

\maketitle

To take the components of $\boldsymbol{\chi}(-2\omega;\omega,\omega)$ from the
crystallographic frame to the lab frame, we can simply apply a standard
rotational matrix,
\begin{equation}
R =
\begin{pmatrix}
R_{Ii} & R_{Ij} & R_{Ik} \\
R_{Ji} & R_{Jj} & R_{Jk} \\
R_{Ki} & R_{Kj} & R_{Kk} \\
\end{pmatrix}
=
\begin{pmatrix}
\sin\phi & -\cos\phi & 0 \\
\cos\phi &  \sin\phi & 0 \\
    0    &      0    & 1
\end{pmatrix},
\end{equation}
such that
\begin{equation}
\chi^{IJK} = \sum_{ijk}R_{Ii}R_{Jj}R_{Kk}\chi^{ijk},
\end{equation}
where $I$, $J$, and $K$ ($i$, $j$, $k$) cycle through $X$, $Y$, or $Z$ ($x$,
$y$, $z$). Thus, our $\chi^{\mathrm{abc}}$ components in the original
coordinates are
\begin{equation}
\boldsymbol{\chi} = 
\left( 
\begin{array}{c@{}}
\left(\begin{array}{c c c}
\chi^{xxx} & \chi^{xxy} & \chi^{xxz} \\
\chi^{xyx} & \chi^{xyy} & \chi^{xyz} \\
\chi^{xzx} & \chi^{xzy} & \chi^{xzz} 
\end{array}\right)\\[20pt]
\left(\begin{array}{c c c}
\chi^{yxx} & \chi^{yxy} & \chi^{yxz} \\
\chi^{yyx} & \chi^{yyy} & \chi^{yyz} \\
\chi^{yzx} & \chi^{yzy} & \chi^{yzz} 
\end{array}\right)\\[20pt]
\left(\begin{array}{c c c}
\chi^{zxx} & \chi^{zxy} & \chi^{zxz} \\
\chi^{zyx} & \chi^{zyy} & \chi^{zyz} \\
\chi^{zzx} & \chi^{zzy} & \chi^{zzz} 
\end{array}\right)
\end{array}\right)
\end{equation}

\begin{equation}
\boldsymbol{\chi} =
\left(
\begin{array}{c c c c c c c c c c c}
\chi^{xxx}& \chi^{xyy}& \chi^{xzz}& |& \chi^{xyz}& \chi^{xxz}& \chi^{xxy}& |& \chi^{xzy}& \chi^{xzx}& \chi^{xyx} \\[3pt]
\chi^{yxx}& \chi^{yyy}& \chi^{yzz}& |& \chi^{yyz}& \chi^{yxz}& \chi^{yxy}& |& \chi^{yzy}& \chi^{yzx}& \chi^{yyx} \\[3pt]
\chi^{zxx}& \chi^{zyy}& \chi^{zzz}& |& \chi^{zyz}& \chi^{zxz}& \chi^{zxy}& |& \chi^{zzy}& \chi^{zzx}& \chi^{zyx}
\end{array}
\right)
,
\end{equation}

\begin{equation}
R\boldsymbol{\chi} =
\begin{pmatrix}
\sin\phi & -\cos\phi & 0 \\
\cos\phi &  \sin\phi & 0 \\
    0    &      0    & 1
\end{pmatrix}
\begin{pmatrix}
\chi^{xxx} & \chi^{xxy} & \chi^{xxz} \\
\chi^{xyx} & \chi^{xyy} & \chi^{xyz} \\
\chi^{xzx} & \chi^{xzy} & \chi^{xzz} 
\end{pmatrix}
=
\begin{pmatrix}
\sin\phi\chi^{xxx} - \cos\phi\chi^{xyx} & \sin\phi\chi^{xxy} - \cos\phi\chi^{xyy} & \sin\phi\chi^{xxz} - \cos\phi\chi^{xyz}\\
\cos\phi\chi^{xxx} + \sin\phi\chi^{xyx} & \cos\phi\chi^{xxy} + \sin\phi\chi^{xyy} & \cos\phi\chi^{xxz} + \sin\phi\chi^{xyz}\\
\chi^{xzx} & \chi^{xzy} & \chi^{xzz}  
\end{pmatrix}
\end{equation}


\end{document}
