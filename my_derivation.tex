\documentclass[11pt]{article}
\usepackage{amsfonts}
\usepackage{amsmath}
\usepackage{parskip}
\usepackage{fullpage}
\usepackage{showkeys}
\usepackage[colorlinks,linkcolor={blue},citecolor={red}]{hyperref}
\allowdisplaybreaks[1]%biutiful equation breaker!!!
%\usepackage{graphicx}
%\usepackage{ulem}
%\usepackage{subfigure}
%\usepackage{xmpincl}% For including licensing XMP metadata
\usepackage{amsmath}% AMS Math
%\usepackage{amssymb}% AMS Symbol
%\usepackage{parskip}% No indent, line skipping paragraph breaks
\usepackage{float}% For better control of figure placement and additional floats

\usepackage{fancyhdr}% Fancy headers and footers.
\pagestyle{empty}% Empty for \frontmatter

\usepackage{graphicx}% You know it, baby!
%\graphicspath{{.}{content/images/}}

\usepackage{color}% For link colors
\definecolor{linkcol}{rgb}{0,0,0.4} 
\definecolor{citecol}{rgb}{0.5,0,0}
\definecolor{emerald}{rgb}{0,0.5,0.5}

\usepackage[bookmarksdepth=2,colorlinks=true,linkcolor=linkcol,citecolor=citecol,filecolor=magenta,urlcolor=linkcol,breaklinks=true]{hyperref}
\urlstyle{same}% For all links within the document and URLs

\usepackage{minitoc}% Mini table of contents for each chapter
\setcounter{minitocdepth}{2}

\usepackage[nottoc,notlof,notlot]{tocbibind}% Fine tuning table of contents
\setcounter{tocdepth}{2}% Number of levels shown in TOC
\setcounter{secnumdepth}{3}% Levels of section numbering
\addtocontents{toc}{\protect\thispagestyle{empty}}% Ensures no page
\addtocontents{lof}{\protect\thispagestyle{empty}}% numbers on toc,
\addtocontents{lot}{\protect\thispagestyle{empty}}% lof, or lot

%------------------------------ Definitions --------------------------------%

% Centered page environment for frontmatter.
% Credit to Matthieu Herrb (matthieu@laas.fr)
\newenvironment{vcenterpage}
{\newpage\vspace*{\fill}\thispagestyle{empty}\renewcommand{\headrulewidth}{0pt}}
{\vspace*{\fill}}

% Changes line spacing to be slightly easier to read
%\renewcommand{\baselinestretch}{1.05}

% Modifies book class defaults for better looking
% chapter pages with small caps titles
\makeatletter
\def\@makechapterhead#1{%
  \vspace*{1\p@}%
  {\parindent \z@ \raggedright \normalfont
    \ifnum \c@secnumdepth >\m@ne
      \if@mainmatter
        \LARGE\sc \@chapapp\space \thechapter
        \par\nobreak
        \vskip 10\p@
      \fi
    \fi
    \interlinepenalty\@M
    \Huge \sc #1\par\nobreak
    \vskip 25\p@
  }}
\def\@makeschapterhead#1{%
  \vspace*{1\p@}%
  {\parindent \z@ \raggedright
    \normalfont
    \interlinepenalty\@M
    \Huge \sc  #1\par\nobreak
    \vskip 30\p@
  }}
\makeatother

%\usepackage{bm}

\begin{document}

We start with the expression for the susceptibility for the intraband transtitions,

\begin{equation}\label{chii}
\chi_{i,\text{a}\text{b}\text{c}}^{s,\ell}=-\frac{e^3}{\Omega\hbar^2\omega_3}\sum_{mn\mathbf{k}}\frac{\mathcal{V}_{mn}^{\Sigma,\text{a},\ell}}{\omega^S_{nm}-\omega_3}\left(\frac{f_{mn}r_{nm}^{\text{b}}}{\omega^S_{nm}-\omega_\beta}\right)_{;k^{\text{c}}},
\end{equation} 

where \emph{s} denotes \emph{surface} and \emph{S} refers to the \emph{scissors} correction. This expression diverges as $\omega_{3} \rightarrow 0$. To eliminate this divergence we take the partial fraction expansion,

\begin{eqnarray}\label{pfi} 
I &=& C \left[-\frac{1}{2(\omega^{S}_{nm})^{2}}\frac{1}{\omega^{S}_{nm}-\omega}+\frac{2}{(\omega^{S}_{nm})^{2}}\frac{1}{\omega^{S}_{nm}-2\omega}+\frac{1}{2(\omega^{S}_{nm})^{2}}\frac{1}{\omega}\right]\nonumber\\
&-& D \left[-\frac{3}{2(\omega^{S}_{nm})^{2}}\frac{1}{\omega^{S}_{nm}-\omega}+\frac{4}{(\omega^{S}_{nm})^{3}}\frac{1}{\omega^{S}_{nm}-2\omega}+\frac{1}{2(\omega^{S}_{nm})^{3}}\frac{1}{\omega}-\frac{1}{2(\omega^{S}_{nm})^{2}}\frac{1}{(\omega^{S}_{nm}-\omega)^2}\right],
\end{eqnarray} 

where $C = f_{mn}\mathcal{V}^{\Sigma,\text{a}}_{mn}(r^{\text{LDA},\text{b}}_{nm})_{;k^{\text{c}}}$, and $D=f_{mn}\mathcal{V}^{\Sigma,\text{a}}_{mn}r^{\text{b}}_{nm}\Delta^{\text{c}}_{nm}$.

Time-reversal symmetry leads to the following relationships:

\begin{eqnarray*}
\mathbf{r}_{mn}(\mathbf{k})                                 &=  \mathbf{r}_{nm}(-\mathbf{k}),                               \\
\mathbf{r}_{mn;\mathbf{k}}(\mathbf{k})                      &=  -\mathbf{r}_{nm;\mathbf{k}}(-\mathbf{k}),                   \\
\mathcal{V}^{\Sigma,\text{a}}_{mn}(-\mathbf{k})             &=  -\mathbf{\mathcal{V}}_{nm}^{\Sigma,\text{a}}(\mathbf{k}),   \\
\omega_{mn}^{S}(-\mathbf{k})                                &=  \omega_{mn}^{S}(\mathbf{k}),                                \\
\Delta^a_{nm}(-\mathbf{k})                                  &=  -\Delta^a_{nm}(\mathbf{k}).
\end{eqnarray*}

For a clean cold semiconductor, $f_{n} = 1$ for an occupied or valence $(n = v)$ band, and $f_{n} = 0$ for an empty or conduction $(n = c)$ band independent of $\mathbf{k}$, and $f_{nm}=-f_{mn}$.

The $\frac{1}{\omega}$ terms cancel each other out. We notice that the energy denominators are invariant under $\mathbf{k} \rightarrow - \mathbf{k}$, and then we only look at the numerators, then
\begin{eqnarray}\label{ct}
C \rightarrow f_{mn}\mathcal{V}^{\Sigma,\text{a}}_{mn}(r^{\text{b}}_{nm})_{;k^{\text{c}}}|_{\mathbf{k}} + f_{mn}\mathcal{V}^{\Sigma,\text{a}}_{mn}(r^{\text{b}}_{nm})_{;k^{\text{c}}}|_{-\mathbf{k}}
&=& f_{mn}\left[\mathcal{V}^{\Sigma,\text{a}}_{mn}(r^{\text{b}}_{nm})_{;k^{\text{c}}}|_{\mathbf{k}} + (-\mathcal{V}^{\Sigma,\text{a}}_{nm})(-(r^{\text{b}}_{mn})_{;k^{\text{c}}})|_{\mathbf{k}}\right]\nonumber\\
&=& f_{mn}\left[\mathcal{V}^{\Sigma,\text{a}}_{mn}(r^{\text{b}}_{nm})_{;k^{\text{c}}} + \mathcal{V}^{\Sigma,\text{a}}_{nm}(r^{\text{b}}_{mn})_{;k^{\text{c}}}\right]\nonumber\\
&=& f_{mn}\left[\mathcal{V}^{\Sigma,\text{a}}_{mn}(r^{\text{b}}_{nm})_{;k^{\text{c}}} + (\mathcal{V}^{\Sigma,\text{a}}_{mn}(r^{\text{b}}_{nm})_{;k^{\text{c}}})^*\right]\nonumber\\
%&=& m_ef_{mn}\omega_{mn}\left[i\mathcal{R}^{\text{a}}_{mn}(r^{\text{b}}_{nm})_{;k^{\text{c}}} + (i\mathcal{R}^{\text{a}}_{mn}(r^{\text{b}}_{nm})_{;k^{\text{c}}})^*\right]\nonumber\\
%&=& im_ef_{mn}\omega_{mn}\left[\mathcal{R}^{\text{a}}_{mn}(r^{\text{b}}_{nm})_{;k^{\text{c}}} - (\mathcal{R}^{\text{a}}_{mn}(r^{\text{b}}_{nm})_{;k^{\text{c}}})^*\right]\nonumber\\
%&=& -2m_ef_{mn}\omega_{mn}\mathrm{Im}[\mathcal{R}^{\text{a}}_{mn}(r^{\text{b}}_{nm})_{;k^{\text{c}}}],
\end{eqnarray}

The last term in the second line of \eqref{pfi} is dealt with as follows,

\begin{eqnarray}\label{dresn}
\frac{D}{2(\omega^S_{nm})^2}\frac{1}{(\omega^S_{nm}-\omega)^2} 
&=& \frac{f_{mn}}{2}\frac{\mathcal{V}^{\Sigma,\text{a}}_{mn}r^{\text{b}}_{nm}}{(\omega^S_{nm})^2}\frac{\Delta^{\text{c}}_{nm}}{(\omega^S_{nm}-\omega)^2} = \frac{f_{mn}}{2}\frac{\mathcal{V}^{\Sigma,\text{a}}_{mn}r^{\text{b}}_{nm}}{(\omega^S_{nm})^2}\left(\frac{1}{\omega^S_{nm}-\omega}\right)_{;k^{\text{c}}}\nonumber\\
&=& -\frac{f_{mn}}{2}\left(\frac{\mathcal{V}^{\Sigma,\text{a}}_{mn}r^{\text{b}}_{nm}}{(\omega^S_{nm})^2}\right)_{;k^{\text{c}}}\frac{1}{\omega^S_{nm}-\omega}.
\end{eqnarray} 

We use the fact that

\begin{equation}\label{wk}
(\omega^S_{nm})_{;k^{\text{c}}}=(\omega^\text{LDA}_{nm})_{;k^{\text{c}}} = \frac{p_{nn}^{\text{c}}-p_{mm}^{\text{c}}}{m_{e}} \equiv \Delta_{nm}^{\text{c}},
\end{equation}

and for the last line, we performed an integration by parts over the Brillouin zone, where the contribution from the edges vanishes.

\section{Generalized Derivative}

Using the chain rule we obtain

\begin{equation}\label{chrn}
\left(\frac{\mathcal{V}^{\Sigma,\text{a}}_{mn}r^{\text{b}}_{nm}}{(\omega^S_{nm})^2}\right)_{;k^{\text{c}}} = \frac{r^{\text{b}}_{nm}}{(\omega^S_{nm})^2}\left(\mathcal{V} ^{\Sigma,\text{a}}_{mn}\right)_{;k^{\text{c}}} + \frac{\mathcal{V}^{\Sigma,\text{a}}_{mn}}{(\omega^S_{nm})^2}\left(r^{\text{b}}_{nm}\right)_{;k^{\text{c}}} - \frac{\mathcal{V}^{\Sigma,\text{a}}_{mn}r^{\text{b}}_{nm}}{2(\omega^S_{nm})^3}\left(\omega^S_{nm}\right)_{;k^{\text{c}}}.
\end{equation}

The individual terms for this expression can be expanded as follows. First,

\begin{equation}\label{eli.1}
\left(\omega^S_{nm}\right)_{;k^{\text{c}}} = \Delta^{\text{LDA},\text{c}}_{nm},
\end{equation}

and,

\begin{equation}\label{eli.2}
(r^{\text{b}}_{nm})_{;k^{\text{a}}} \approx \frac{r^{\text{a}}_{nm}\Delta^{\text{LDA},\text{b}}_{mn} + r^{\text{b}}_{nm}\Delta^{\text{LDA},\text{a}}_{mn}}{\omega^\text{LDA}_{nm}} + \frac{i}{\omega^\text{LDA}_{nm}}\sum_{\ell}\left(\omega^\text{LDA}_{\ell m}r^{\text{a}}_{n\ell}r^{\text{b}}_{\ell m} - \omega^\text{LDA}_{n\ell}r^{\text{b}}_{n\ell}r^{\text{a}}_{\ell m}\right).
\end{equation}

\subsection{Generalized derivative for \texorpdfstring{$\mathcal{V}^{\Sigma,\text{a},\ell}_{nm}$}{V-Sigma}}

We must include the generalized derivative for $\mathcal{V}^{\Sigma,\text{a},\ell}_{nm}$. We can seperate the expression into its components,

\begin{eqnarray}\label{a.1}
\left(\mathcal{V}^{\Sigma,\text{a},\ell}_{nm}\right)_{;k^\text{b}} = \left(\mathcal{V}^{\text{LDA},\text{a},\ell}_{nm}\right)_{;k^\text{b}} + \left(\mathcal{V}^{S,\text{a},\ell}_{nm}\right)_{;k^\text{b}},
\end{eqnarray}

where, %find out about scissors

\begin{equation}\label{a.2}
\left(\mathcal{V}^{\text{LDA},\text{a}}_{nm}\right)_{;k^\text{b}} = \frac{1}{2}\sum_q\left((v^{\text{LDA},\text{a}}_{nq})_{;k^\text{b}}\mathcal{F}^\ell_{qm} + v^{\text{LDA},\text{a}}_{nq}(\mathcal{F}^\ell_{qm})_{;k^\text{b}} + (\mathcal{F}^\ell_{nq})_{;k^\text{b}} v^{\text{LDA},\text{a}}_{qm} + \mathcal{F}^\ell_{nq} (v^{\text{LDA},\text{a}}_{qm})_{;k^\text{b}}\right),
\end{equation} 

and $\left(v^{\text{LDA},\text{a}}_{nn}\right)_{;k^\text{b}}$ is given by

\begin{equation}
\left(v^{\text{LDA},\text{a}}_{nn}\right)_{;k^\text{b}} = \frac{\hbar}{m_e}\delta_{\text{a}\text{b}} - \sum_{\ell\ne n}\omega^\text{LDA}_{\ell n}\left(r^{\text{a}}_{n\ell}r^\text{b}_{\ell n} + r^\text{b}_{n\ell}r^{\text{a}}_{\ell n}\right).
\end{equation}

Lastly,

\begin{equation}\label{a.3b}
\mathcal{V}^{S,\text{a},\ell}_{nm} = \frac{1}{2}\sum_{q}\left((v^{S,\text{a}}_{nq})_{;k^\text{b}}\mathcal{F}_{qm} + v^{S,\text{a}}_{nq}(\mathcal{F}_{qm})_{;k^\text{b}} + (\mathcal{F}_{nq})_{;k^\text{b}}v_{qm}^{S,\text{a}} + \mathcal{F}_{nq} (v_{qm}^{S,\text{a}})_{;k^\text{b}}\right),
\end{equation}

where $\left(v^{S,\text{a}}_{nm}\right)_{;k^\text{b}}$ is given by

\begin{eqnarray}\label{choni.1}
\left(v^{S,\text{a}}_{nm}\right)_{;k^\text{b}}=i\Delta f_{mn}(r^\text{a}_{nm})_{;k^\text{b}}.
\end{eqnarray}

\end{document}
