\documentclass[11pt]{article}
\usepackage{amsfonts}
\usepackage{amsmath}
\usepackage{parskip}
\usepackage{fullpage}
\usepackage{showkeys}
\usepackage[colorlinks,linkcolor={blue},citecolor={red}]{hyperref}
%\allowdisplaybreaks[1]%biutiful equation breaker!!!
%\usepackage{graphicx}
%\usepackage{ulem}
%\usepackage{subfigure}
%\usepackage{xmpincl}% For including licensing XMP metadata
\usepackage{amsmath}% AMS Math
%\usepackage{amssymb}% AMS Symbol
%\usepackage{parskip}% No indent, line skipping paragraph breaks
\usepackage{float}% For better control of figure placement and additional floats

\usepackage{fancyhdr}% Fancy headers and footers.
\pagestyle{empty}% Empty for \frontmatter

\usepackage{graphicx}% You know it, baby!
%\graphicspath{{.}{content/images/}}

\usepackage{color}% For link colors
\definecolor{linkcol}{rgb}{0,0,0.4} 
\definecolor{citecol}{rgb}{0.5,0,0}
\definecolor{emerald}{rgb}{0,0.5,0.5}

\usepackage[bookmarksdepth=2,colorlinks=true,linkcolor=linkcol,citecolor=citecol,filecolor=magenta,urlcolor=linkcol,breaklinks=true]{hyperref}
\urlstyle{same}% For all links within the document and URLs

\usepackage{minitoc}% Mini table of contents for each chapter
\setcounter{minitocdepth}{2}

\usepackage[nottoc,notlof,notlot]{tocbibind}% Fine tuning table of contents
\setcounter{tocdepth}{2}% Number of levels shown in TOC
\setcounter{secnumdepth}{3}% Levels of section numbering
\addtocontents{toc}{\protect\thispagestyle{empty}}% Ensures no page
\addtocontents{lof}{\protect\thispagestyle{empty}}% numbers on toc,
\addtocontents{lot}{\protect\thispagestyle{empty}}% lof, or lot

%------------------------------ Definitions --------------------------------%

% Centered page environment for frontmatter.
% Credit to Matthieu Herrb (matthieu@laas.fr)
\newenvironment{vcenterpage}
{\newpage\vspace*{\fill}\thispagestyle{empty}\renewcommand{\headrulewidth}{0pt}}
{\vspace*{\fill}}

% Changes line spacing to be slightly easier to read
%\renewcommand{\baselinestretch}{1.05}

% Modifies book class defaults for better looking
% chapter pages with small caps titles
\makeatletter
\def\@makechapterhead#1{%
  \vspace*{1\p@}%
  {\parindent \z@ \raggedright \normalfont
    \ifnum \c@secnumdepth >\m@ne
      \if@mainmatter
        \LARGE\sc \@chapapp\space \thechapter
        \par\nobreak
        \vskip 10\p@
      \fi
    \fi
    \interlinepenalty\@M
    \Huge \sc #1\par\nobreak
    \vskip 25\p@
  }}
\def\@makeschapterhead#1{%
  \vspace*{1\p@}%
  {\parindent \z@ \raggedright
    \normalfont
    \interlinepenalty\@M
    \Huge \sc  #1\par\nobreak
    \vskip 30\p@
  }}
\makeatother

%\usepackage{bm}

\begin{document}

We start with the expression for the susceptibility for the intraband transtitions,

\begin{equation}\label{chii}
\chi_{i,\text{a}\text{b}\text{c}}^{s,\ell}=-\frac{e^3}{\Omega\hbar^2\omega_3}\sum_{mn\mathbf{k}}\frac{\mathcal{V}_{mn}^{\Sigma,\text{a},\ell}}{\omega^S_{nm}-\omega_3}\left(\frac{f_{mn}r_{nm}^{\text{b}}}{\omega^S_{nm}-\omega_\beta}\right)_{;k^{\text{c}}},
\end{equation} 

where \emph{s} denotes \emph{surface} and \emph{S} refers to the \emph{scissors} correction. This expression diverges as $\omega_{3} \rightarrow 0$. To eliminate this divergence we take the partial fraction expansion,

\begin{align}\label{pfi} 
I &= C \left[-\frac{1}{2(\omega^{S}_{nm})^{2}}\frac{1}{\omega^{S}_{nm}-\omega}+\frac{2}{(\omega^{S}_{nm})^{2}}\frac{1}{\omega^{S}_{nm}-2\omega}+\frac{1}{2(\omega^{S}_{nm})^{2}}\frac{1}{\omega}\right]\nonumber\\
&- D \left[-\frac{3}{2(\omega^{S}_{nm})^{2}}\frac{1}{\omega^{S}_{nm}-\omega}+\frac{4}{(\omega^{S}_{nm})^{3}}\frac{1}{\omega^{S}_{nm}-2\omega}+\frac{1}{2(\omega^{S}_{nm})^{3}}\frac{1}{\omega}-\frac{1}{2(\omega^{S}_{nm})^{2}}\frac{1}{(\omega^{S}_{nm}-\omega)^2}\right],
\end{align} 

where $C = f_{mn}\mathcal{V}^{\Sigma,\text{a}}_{mn}(r^{\text{LDA},\text{b}}_{nm})_{;k^{\text{c}}}$, and $D=f_{mn}\mathcal{V}^{\Sigma,\text{a}}_{mn}r^{\text{b}}_{nm}\Delta^{\text{c}}_{nm}$.

Time-reversal symmetry leads to the following relationships:

\begin{align}\label{time_reversal}
\mathbf{r}_{mn}(\mathbf{k})|_{-\mathbf{k}}                       &=  \mathbf{r}_{nm}(\mathbf{k})|_{\mathbf{k}},                                 \nonumber\\
(\mathbf{r}_{mn})_{;\mathbf{k}}(\mathbf{k})|_{-\mathbf{k}}       &=  (-\mathbf{r}_{nm})_{;\mathbf{k}}(\mathbf{k})|_{\mathbf{k}},                \nonumber\\
\mathcal{V}^{\Sigma,\text{a}}_{mn}(\mathbf{k})|_{-\mathbf{k}}    &=  -\mathbf{\mathcal{V}}_{nm}^{\Sigma,\text{a}}(\mathbf{k})|_{\mathbf{k}},             \\
\omega_{mn}^{S}(\mathbf{k})|_{-\mathbf{k}}                       &=  \omega_{mn}^{S}(\mathbf{k})|_{\mathbf{k}},                                 \nonumber\\
\Delta^a_{nm}(\mathbf{k})|_{-\mathbf{k}}                         &=  -\Delta^a_{nm}(\mathbf{k})|_{\mathbf{k}}.                                  \nonumber
\end{align}

For a clean cold semiconductor, $f_{n} = 1$ for an occupied or valence $(n = v)$ band, and $f_{n} = 0$ for an empty or conduction $(n = c)$ band independent of $\mathbf{k}$, and $f_{nm}=-f_{mn}$.

The $\frac{1}{\omega}$ terms cancel each other out. We notice that the energy denominators are invariant under $\mathbf{k} \rightarrow - \mathbf{k}$, and then we only look at the numerators, then

\begin{align}\label{ct}
C \rightarrow f_{mn}\mathcal{V}^{\Sigma,\text{a}}_{mn}\left(r^{\text{LDA},\text{b}}_{nm}\right)_{;k^{\text{c}}}|_{\mathbf{k}}
&+ f_{mn}\mathcal{V}^{\Sigma,\text{a}}_{mn}\left(r^{\text{LDA},\text{b}}_{nm}\right)_{;k^{\text{c}}}|_{-\mathbf{k}}\nonumber\\
&= f_{mn}\left[\mathcal{V}^{\Sigma,\text{a}}_{mn}\left(r^{\text{LDA},\text{b}}_{nm}\right)_{;k^{\text{c}}}|_{\mathbf{k}} + \left(-\mathcal{V}^{\Sigma,\text{a}}_{nm}\right)\left(-r^{\text{LDA},\text{b}}_{mn}\right)_{;k^{\text{c}}}|_{\mathbf{k}}\right]\nonumber\\
&= f_{mn}\left[\mathcal{V}^{\Sigma,\text{a}}_{mn}\left(r^{\text{LDA},\text{b}}_{nm}\right)_{;k^{\text{c}}} + \mathcal{V}^{\Sigma,\text{a}}_{nm}\left(r^{\text{LDA},\text{b}}_{mn}\right)_{;k^{\text{c}}}\right]\nonumber\\
&= f_{mn}\left[\mathcal{V}^{\Sigma,\text{a}}_{mn}\left(r^{\text{LDA},\text{b}}_{nm}\right)_{;k^{\text{c}}} + \left(\mathcal{V}^{\Sigma,\text{a}}_{mn}\left(r^{\text{LDA},\text{b}}_{nm}\right)_{;k^{\text{c}}}\right)^*\right]\nonumber\\
&= 2f_{mn}\,\mathrm{Re}\left[\mathcal{V}^{\Sigma,\text{a}}_{mn}\left(r^{\text{LDA},\text{b}}_{nm}\right)_{;k^{\text{c}}}\right],
\end{align}

and likewise,

\begin{align}\label{dt}
D \rightarrow f_{mn}\mathcal{V}^{\Sigma,\text{a}}_{mn}r^{\text{b}}_{nm}\Delta^{\text{c}}_{nm}|_{\mathbf{k}} 
&+ f_{mn}\mathcal{V}^{\Sigma,\text{a}}_{mn}r^{\text{b}}_{nm}\Delta^{\text{c}}_{nm}|_{-\mathbf{k}}\nonumber\\
&= f_{mn}\left[\mathcal{V}^{\Sigma,\text{a}}_{mn}r^{\text{b}}_{nm}\Delta^{\text{c}}_{nm}|_{\mathbf{k}} + \left(-\mathcal{V}^{\Sigma,\text{a}}_{nm}\right)r^{\text{b}}_{mn}\left(-\Delta^{\text{c}}_{nm}\right)|_{\mathbf{k}}\right]\nonumber\\
&= f_{mn}\left[\mathcal{V}^{\Sigma,\text{a}}_{mn}r^{\text{b}}_{nm} + \mathcal{V}^{\Sigma,\text{a}}_{nm}r^{\text{b}}_{mn}\right]\Delta^{\text{c}}_{nm}\nonumber\\
&= f_{mn}\left[\mathcal{V}^{\Sigma,\text{a}}_{mn}r^{\text{b}}_{nm} + \left(\mathcal{V}^{\Sigma,\text{a}}_{mn}r^{\text{b}}_{nm}\right)^*\right]\Delta^{\text{c}}_{nm}\nonumber\\
&= 2f_{mn}\,\mathrm{Re}\left[\mathcal{V}^{\Sigma,\text{a}}_{mn}r^{\text{b}}_{nm}\right]\Delta^{\text{c}}_{nm}.
\end{align}

The last term in the second line of \eqref{pfi} is dealt with as follows,

\begin{align}\label{dresn}
\frac{D}{2(\omega^S_{nm})^2}\frac{1}{(\omega^S_{nm}-\omega)^2} 
&= \frac{f_{mn}}{2}\frac{\mathcal{V}^{\Sigma,\text{a}}_{mn}r^{\text{b}}_{nm}}{(\omega^S_{nm})^2}\frac{\Delta^{\text{c}}_{nm}}{(\omega^S_{nm}-\omega)^2} = \frac{f_{mn}}{2}\frac{\mathcal{V}^{\Sigma,\text{a}}_{mn}r^{\text{b}}_{nm}}{(\omega^S_{nm})^2}\left(\frac{1}{\omega^S_{nm}-\omega}\right)_{;k^{\text{c}}}\nonumber\\
&= -\frac{f_{mn}}{2}\left(\frac{\mathcal{V}^{\Sigma,\text{a}}_{mn}r^{\text{b}}_{nm}}{(\omega^S_{nm})^2}\right)_{;k^{\text{c}}}\frac{1}{\omega^S_{nm}-\omega}.
\end{align} 

We use the fact that

\begin{equation}\label{wk}
(\omega^S_{nm})_{;k^{\text{c}}}=(\omega^\text{LDA}_{nm})_{;k^{\text{c}}} = \frac{p_{nn}^{\text{c}}-p_{mm}^{\text{c}}}{m_{e}} \equiv \Delta_{nm}^{\text{c}},
\end{equation}

and for the last line, we performed an integration by parts over the Brillouin zone, where the contribution from the edges vanishes. Using the chain rule, we obtain

\begin{equation}\label{chr}
\left(\frac{\mathcal{V}^{\Sigma,\text{a}}_{mn}r^{\text{b}}_{nm}}{(\omega^{S}_{nm})^2}\right)_{;k^{\text{c}}} = \frac{r^{\text{b}}_{nm}}{(\omega^{S}_{nm})^2}\left(\mathcal{V} ^{\Sigma,\text{a}}_{mn}\right)_{;k^{\text{c}}} + \frac{\mathcal{V}^{\Sigma,\text{a}}_{mn}}{(\omega^{S}_{nm})^2}\left(r^{\text{b}}_{nm}\right)_{;k^{\text{c}}} - \frac{\mathcal{V}^{\Sigma,\text{a}}_{mn}r^{\text{b}}_{nm}}{2(\omega^{S}_{nm})^3}\left(
\omega^{S}_{nm}\right)_{;k^{\text{c}}}.
\end{equation}

%% verify that the generalized derivative actually behaves this way. you might need to look into older versions when we were still using calR.
We will check each term of \eqref{chr} over $\mathbf{k} \rightarrow - \mathbf{k}$ using the relations in \eqref{time_reversal}. The first term is reduced to

\begin{align}\label{first_term_gen_deriv}
\frac{r^{\text{b}}_{nm}}{(\omega^{S}_{nm})^{2}}\left(\mathcal{V}^{\Sigma,\text{a}}_{mn}\right)_{;k^{\text{c}}}|_{\mathbf{k}} + \frac{r^{\text{b}}_{nm}}{(\omega^{S}_{nm})^{2}}\left(\mathcal{V}^{\Sigma,\text{a}}_{mn}\right)_{;k^{\text{c}}}|_{-\mathbf{k}}
&= \frac{r^{\text{b}}_{nm}}{(\omega^{S}_{nm})^{2}}\left(\mathcal{V}^{\Sigma,\text{a}}_{mn}\right)_{;k^{\text{c}}}|_{\mathbf{k}} - \frac{r^{\text{b}}_{mn}}{(\omega^{S}_{nm})^{2}}\left(\mathcal{V}^{\Sigma,\text{a}}_{nm}\right)_{;k^{\text{c}}}|_{\mathbf{k}}\nonumber\\
&= \frac{1}{(\omega^{S}_{nm})^{2}}\left[r^{\text{b}}_{nm}\left(\mathcal{V}^{\Sigma,\text{a}}_{mn}\right)_{;k^{\text{c}}} - \left(r^{\text{b}}_{nm}\left(\mathcal{V}^{\Sigma,\text{a}}_{mn}\right)_{;k^{\text{c}}}\right)^*\right]\nonumber\\
&= \frac{2i}{(\omega^{S}_{nm})^{2}}\mathrm{Im}\left[r^{\text{b}}_{nm}\left(\mathcal{V}^{\Sigma,\text{a}}_{mn}\right)_{;k^{\text{c}}}\right],
\end{align}

the second term is reduced to

\begin{align}\label{second_term_gen_deriv}
\frac{\mathcal{V}^{\Sigma,\text{a}}_{mn}}{(\omega^{S}_{nm})^{2}}\left(r^{\text{b}}_{nm}\right)_{;k^{\text{c}}}|_{\mathbf{k}} + \frac{\mathcal{V}^{\Sigma,\text{a}}_{mn}}{(\omega^{S}_{nm})^{2}}\left(r^{\text{b}}_{nm}\right)_{;k^{\text{c}}}|_{-\mathbf{k}}
&= \frac{\mathcal{V}^{\Sigma,\text{a}}_{mn}}{(\omega^{S}_{nm})^{2}}\left(r^{\text{b}}_{nm}\right)_{;k^{\text{c}}}|_{\mathbf{k}} + \frac{\mathcal{V}^{\Sigma,\text{a}}_{nm}}{(\omega^{S}_{nm})^{2}}\left(r^{\text{b}}_{mn}\right)_{;k^{\text{c}}}|_{\mathbf{k}}\nonumber\\
&= \frac{1}{(\omega^{S}_{nm})^{2}}\left[\mathcal{V}^{\Sigma,\text{a}}_{mn}\left(r^{\text{b}}_{nm}\right)_{;k^{\text{c}}} + \left(\mathcal{V}^{\Sigma,\text{a}}_{mn}\left(r^{\text{b}}_{nm}\right)_{;k^{\text{c}}}\right)^*\right]\nonumber\\
&= \frac{2}{(\omega^{S}_{nm})^{2}}\mathrm{Re}\left[\mathcal{V}^{\Sigma,\text{a}}_{mn}\left(r^{\text{b}}_{nm}\right)_{;k^{\text{c}}}\right],
\end{align}

%% the generalized derivative of \omega^{S}_{nm} is equivalent to \Delta_{nm}^{\text{c}} from \eqref{wk}.
and by using \eqref{wk}, the third term is reduced to

\begin{align}\label{third_term_gen_deriv}
\frac{\mathcal{V}^{\Sigma,\text{a}}_{mn}r^{\text{b}}_{nm}}{2(\omega^{S}_{nm})^{3}}\left(\omega^{S}_{nm}\right)_{;k^{\text{c}}}|_{\mathbf{k}} + \frac{\mathcal{V}^{\Sigma,\text{a}}_{mn}r^{\text{b}}_{nm}}{2(\omega^{S}_{nm})^{3}}\left(\omega^{S}_{nm}\right)_{;k^{\text{c}}}|_{-\mathbf{k}}
&= \frac{\mathcal{V}^{\Sigma,\text{a}}_{mn}r^{\text{b}}_{nm}}{2(\omega^{S}_{nm})^{3}}\Delta_{nm}^{\text{c}}|_{\mathbf{k}} + \frac{\mathcal{V}^{\Sigma,\text{a}}_{mn}r^{\text{b}}_{nm}}{2(\omega^{S}_{nm})^{3}}\Delta_{nm}^{\text{c}}|_{-\mathbf{k}}\nonumber\\
&= \frac{\mathcal{V}^{\Sigma,\text{a}}_{nm}r^{\text{b}}_{mn}}{2(\omega^{S}_{nm})^{3}}\Delta_{nm}^{\text{c}}|_{\mathbf{k}} + \frac{\mathcal{V}^{\Sigma,\text{a}}_{mn}r^{\text{b}}_{nm}}{2(\omega^{S}_{nm})^{3}}\Delta_{nm}^{\text{c}}|_{\mathbf{k}}\nonumber\\
&= \frac{1}{2(\omega^{S}_{nm})^{3}}\left[\mathcal{V}^{\Sigma,\text{a}}_{nm}r^{\text{b}}_{mn} + \left(\mathcal{V}^{\Sigma,\text{a}}_{nm}r^{\text{b}}_{mn}\right)^{*}\right]\Delta_{nm}^{\text{c}}\nonumber\\
&= \frac{1}{(\omega^{S}_{nm})^{3}}\mathrm{Re}\left[\mathcal{V}^{\Sigma,\text{a}}_{nm}r^{\text{b}}_{mn}\right]\Delta_{nm}^{\text{c}}
\end{align}

\end{document}
