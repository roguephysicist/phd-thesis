We derive a series of results that follow from Dirac's notation and
that are  useful in the various derivations.

Let's start with the Fourier transform of the wave function written in
the \sch~ representation, i.e.
\begin{equation}\label{ap_ft}
\psi(\bfr) = \frac{1}{(2\pi\hbar)^{3/2}}\int d\bfp \psi(\bfp)e^{i\bfp\cdot\bfr/\hbar}
,
\end{equation}  
and inversely 
\begin{equation}\label{ap_tf}
\psi(\bfp) = \frac{1}{(2\pi\hbar)^{3/2}}\int d\bfr \psi(\bfr)
e^{-i\bfp\cdot\bfr/\hbar}
.
\end{equation}  
Now,
\begin{equation}\label{rpsi}
\braket{\bfr}{\psi}=\psi(\bfr)=\int d\bfp \braket{\bfr}{\bfp}\braket{\bfp}{\psi}=
\int d\bfp \braket{\bfr}{\bfp}\psi(\bfp)
,
\end{equation}
that when compared with Eq.~\eqref{ap_ft} allow us to identify,
\begin{equation}\label{rp2}
\braket{\bfr}{\bfp}=\frac{1}{(2\pi\hbar)^{3/2}}e^{i\bfp\cdot\bfr/\hbar}
.
\end{equation}
By the same token,
\begin{equation}\label{rpsi2}
\braket{\bfp}{\psi}=\psi(\bfp)=\int d\bfr \braket{\bfp}{\bfr}\braket{\bfr}{\psi}=
\int d\bfr \braket{\bfp}{\bfr}\psi(\bfr)
,
\end{equation}
that when compared with Eq.~\eqref{ap_tf} allow us to identify,
\begin{equation}\label{rp}
\braket{\bfp}{\bfr}=\frac{1}{(2\pi\hbar)^{3/2}}e^{-i\bfp\cdot\bfr/\hbar}
,
\end{equation}
where
\begin{equation}\label{ap_good}
\braket{\bfr}{\bfp}=(\braket{\bfp}{\bfr})^*
,
\end{equation}
is succinctly verified.
 
We calculate the matrix elements of $\bfp$ in the $\bfr$
representation,
\begin{align}\label{ap_matp}
\bra{\bfr}\hat{p}_x\ket{\bfr'}&=
\int d\bfp
\bra{\bfr}\hat{p}_x\ket{\bfp}\braket{\bfp}{\bfr'} \\ \nonumber
&=
\int d\bfp
p_x\braket{\bfr}{\bfp}\braket{\bfp}{\bfr'} \\ \nonumber
&=
\frac{1}{(2\pi\hbar)^3}
\int d\bfp
p_x
e^{i\bfp\cdot(\bfr-\bfr')/\hbar}
\\ \nonumber
&=
\frac{1}{(2\pi\hbar)^3}
\int dp_x
p_x
e^{ip_x(x-x')/\hbar}
\int dp_y
e^{ip_y(y-y')/\hbar}
\int dp_z
e^{ip_z(z-z')/\hbar}
\\ \nonumber
&=
\frac{1}{2\pi\hbar}
\int dp_x
p_x
e^{ip_x(x-x')/\hbar}
\gd(y-y')
\gd(z-z')
,
\end{align}
where we used the fact that
\begin{equation}\label{ap_otra}
\hat\bfp\ket{\bfp}=\bfp\ket{\bfp}
,
\end{equation}
and that
\begin{equation}\label{ap_delta}
\gd(q-q')=\frac{1}{2\pi\hbar}
\int dp
e^{ip(q-q')/\hbar}
.
\end{equation}
Now,
\begin{equation}\label{ap_mas}
\frac{1}{2\pi\hbar}
\int dp_x
p_x
e^{ip_x(x-x')/\hbar}
=
-i\hbar\frac{\partial}{\partial x}
\int
\frac{dp_x}{2\pi\hbar}
e^{ip_x(x-x')/\hbar}
=-i\hbar\frac{\partial}{\partial x}\gd(x-x')
,
\end{equation}
from where we finally get 
\begin{equation}\label{ap_fin}
\bra{\bfr}\hat{p}_x\ket{\bfr'}=
(-i\hbar\frac{\partial}{\partial x}\gd(x-x'))
\gd(y-y')
\gd(z-z')
,
\end{equation}
with similar results for $\hat{p}_y$ and $\hat{p}_z$.
Now we can calculate
\begin{align}\label{ap_psi}
\bra{\bfr}\hat{p}_x\ket{\psi}&=
\int d\bfr' \bra{\bfr}\hat{p}_x\ket{\bfr'}\braket{\bfr'}{\psi}
\\ \nonumber
&=
\int dx' (-i\hbar\frac{\partial}{\partial x}\gd(x-x'))
\int dy' \gd(y-y')
\int dz' \gd(z-z')
\psi(x',y',z')
\\ \nonumber
&=
-i\hbar
\int dx' (\frac{\partial}{\partial x}\gd(x-x'))
\psi(x',y,z)
=
-i\hbar
\frac{\partial}{\partial x}
\int dx' 
\gd(x-x')
\psi(x',y,z)
\\ \nonumber
&=
-i\hbar
\frac{\partial}{\partial x}
\psi(x,y,z)
,
\end{align}
which confirms that in the $\bfr$ representation,
the $\hat\bfp$ operator is replaced with the differential operator
$-i\hbar\bfgnabla$. 
