\chapter{Derived expressions for the SHG yield}
%%%%%%%%%%%%%%%%%%%%%%%%%%%%%%%%%%%%%%%%%%%%%%%%%%%%%%%%%%%%%%%%%%%%%%%%%%%%%%%%
%%%%%%%%%%%%%%%%%%%%%%%%%%%%%%%%%%%%%%%%%%%%%%%%%%%%%%%%%%%%%%%%%%%%%%%%%%%%%%%%


\section{Some useful expressions}
We are interested in finding
\begin{equation*}
\Upsilon = 
\mathbf{e}^{2\omega}_{\ell}\cdot\boldsymbol{\chi}:
\mathbf{e}^{\omega}_{\ell}\mathbf{e}^{\omega}_{\ell}
\end{equation*}
for each different polarization case. We choose the plane of incidence
along the $\boldsymbol{\kappa}z$ plane, and define 
\begin{equation}\label{eq:kappavec}
\hat{\boldsymbol{\kappa}}
= \cos\phi\hat{\mathbf{x}} + \sin\phi\hat{\mathbf{y}},
\end{equation}
and
\begin{equation}\label{eq:svec}
\hat{\mathbf{s}} = -\sin\phi\hat{\mathbf{x}} + \cos\phi\hat{\mathbf{y}},
\end{equation}
where $\phi$ the angle with respect to the $x$ axis.


\subsection{\texorpdfstring{$2\omega$}{2w} terms}

Including multiple reflecions, the $\mathbf{e}^{2\omega}_{\ell}$ term is
\begin{equation}\label{eq:e2wellmr}
\mathbf{e}^{2\omega}_{\ell} = \hat{\mathbf{e}}^{\mathrm{out}}\cdot
\Bigg[
\hat{\mathbf{s}}T_{s}^{v\ell}R^{M+}_{s}\hat{\mathbf{s}} + 
\hat{\mathbf{P}}_{v+}\frac{T^{v\ell}_{p}}{N_{\ell}}
\left(
\sin\theta_{0}R^{M+}_{p}\hat{\mathbf{z}} 
- W_{\ell}R^{M-}_{p}\hat{\boldsymbol{\kappa}}
\right)
\Bigg],
\end{equation}
and neglecting the multiple reflections reduces this expression to
\begin{equation}\label{eq:e2well}
\begin{split}
\mathbf{e}^{2\omega}_{\ell} = 
\hat{\mathbf{e}}^{\mathrm{out}}\cdot
\Bigg[
\hat{\mathbf{s}}T_{s}^{v\ell}T_{s}^{\ell b}\hat{\mathbf{s}} + 
\hat{\mathbf{P}}_{v+}
\frac{T^{v\ell}_{p}T^{\ell b}_{p}}
     {N^{2}_{\ell}N_{b}}
\left(
N^{2}_{b}\sin\theta_{0}\hat{\mathbf{z}} - N^{2}_{\ell}W_{b}\hat{\mathbf{x}}
\right)
\Bigg].
\end{split}
\end{equation}  

We first expand these equations for clarity. Substituting Eqs.
\eqref{eq:kappavec} and \eqref{eq:svec} into Eq. \eqref{eq:e2wellmr},
\begin{equation*}\label{eq:e2wexpmr}
\begin{split}
\mathbf{e}^{2\omega}_{\ell} = \hat{\mathbf{e}}^{\mathrm{out}}&\cdot
\Bigg[
\hat{\mathbf{s}}T_{s}^{v\ell}R^{M+}_{s}
\left(
- \sin\phi\hat{\mathbf{x}}
+ \cos\phi\hat{\mathbf{y}}
\right)\\
&+ \hat{\mathbf{P}}_{v+}\frac{T^{v\ell}_{p}}{N_{\ell}}
\left(
  \sin\theta_{0}R^{M+}_{p}\hat{\mathbf{z}}
- W_{\ell}R^{M-}_{p}\cos\phi\hat{\mathbf{x}}
- W_{\ell}R^{M-}_{p}\sin\phi\hat{\mathbf{y}}
\right)
\Bigg].
\end{split}
\end{equation*}
We now have $\mathbf{e}^{2\omega}_{\ell}$ in terms of
$\hat{\mathbf{P}}_{v+}$,
\begin{equation}\label{eq:e2wpmr}
\mathbf{e}^{2\omega}_{\ell} =
\frac{T^{v\ell}_{p}}{N_{\ell}}
\left(
  \sin\theta_{0}R^{M+}_{p}\hat{\mathbf{z}}
- W_{\ell}R^{M-}_{p}\cos\phi\hat{\mathbf{x}}
- W_{\ell}R^{M-}_{p}\sin\phi\hat{\mathbf{y}}
\right),
\end{equation}
and in terms of $\hat{\mathbf{s}}$,
\begin{equation}\label{eq:e2wsmr}
\mathbf{e}^{2\omega}_{\ell} =
T_{s}^{v\ell}R^{M+}_{s}
\left(
- \sin\phi\hat{\mathbf{x}}
+ \cos\phi\hat{\mathbf{y}}
\right).
\end{equation}
Likewise, we do the exact same for Eq. \eqref{eq:e2well}, and get the
following term for $\hat{\mathbf{P}}_{v+}$,
\begin{equation}\label{eq:e2wp}
\mathbf{e}^{2\omega}_{\ell} =
\frac{T^{v\ell}_{p}T^{\ell b}_{p}}
     {N^{2}_{\ell}N_{b}}
\left(
  N^{2}_{b}\sin\theta_{0}\hat{\mathbf{z}}
- N^{2}_{\ell}W_{b}\cos\phi\hat{\mathbf{x}}
- N^{2}_{\ell}W_{b}\sin\phi\hat{\mathbf{y}}
\right),
\end{equation}
and $\hat{\mathbf{s}}$,
\begin{equation}\label{eq:e2ws}
\mathbf{e}^{2\omega}_{\ell} 
= T^{v\ell}_{s}T^{\ell b}_{s}
\left[-\sin\phi\hat{\mathbf{x}} + \cos\phi\hat{\mathbf{y}}\right].
\end{equation}


\subsection{\texorpdfstring{$1\omega$}{1w} terms}

We posit that the effects of the multiple reflections can be neglected for the
$\mathbf{e}^{\omega}_{\ell}$ term. This term is
\begin{equation*}\label{eq:ewell}
\mathbf{e}^{\omega}_{\ell} = 
\left[
\hat{\mathbf{s}}t_{s}^{v\ell}t_{s}^{\ell b}\hat{\mathbf{s}} 
+ \frac{t^{v\ell}_{p}t^{\ell b}_{p}}{n^{2}_{\ell}n_{b}}
\left(
  n^{2}_{b}\sin\theta_{0}\hat{\mathbf{z}} 
+ n^{2}_{\ell}w_{b}\hat{\boldsymbol{\kappa}}
\right)
\hat{\mathbf{p}}_{v-}
\right]
\cdot\hat{\mathbf{e}}^{\mathrm{in}}.
\end{equation*}
For all cases, we require a
$\mathbf{e}^{\omega}_{\ell}\mathbf{e}^{\omega}_{\ell}$ product. For brevity, we
will directly list these terms for both polarizations. For
$\hat{\mathbf{e}}^{\mathrm{in}} = \hat{\mathbf{p}}_{v-}$,
\begin{equation}\label{eq:ewewp}
\begin{split}
\mathbf{e}^{\omega}_{\ell}\mathbf{e}^{\omega}_{\ell}
= \left(\frac{t^{v\ell}_{p}t^{\ell b}_{p}}
{n^{2}_{\ell}n_{b}}\right)^{2}
\big(
  &n^{4}_{\ell}w^{2}_{b}\cos^{2}\phi
\hat{\mathbf{x}}\hat{\mathbf{x}}
+ 2n^{4}_{\ell}w^{2}_{b}\sin\phi\cos\phi
\hat{\mathbf{x}}\hat{\mathbf{y}}\\
&+ 2n^{2}_{\ell}n^{2}_{b}w_{b}\sin\theta_{0}\cos\phi
\hat{\mathbf{x}}\hat{\mathbf{z}}
+ n^{4}_{\ell}w^{2}_{b}\sin^{2}\phi
\hat{\mathbf{y}}\hat{\mathbf{y}}\\
&+ 2n^{2}_{\ell}n^{2}_{b}w_{b}\sin\theta_{0}\sin\phi
\hat{\mathbf{y}}\hat{\mathbf{z}}
+ n^{4}_{b}\sin^{2}\theta_{0}
\hat{\mathbf{z}}\hat{\mathbf{z}}
\big),
\end{split}
\end{equation}
and for $\hat{\mathbf{e}}^{\mathrm{in}} = \hat{\mathbf{s}}$,
\begin{equation}\label{eq:ewews}
\mathbf{e}^{\omega}_{\ell}\mathbf{e}^{\omega}_{\ell}
= \left(t^{v\ell}_{s}t^{\ell b}_{s}\right)^{2}
\left(
  \sin^{2}\phi\hat{\mathbf{x}}\hat{\mathbf{x}}
+ \cos^{2}\phi\hat{\mathbf{y}}\hat{\mathbf{y}} 
- 2\sin\phi\cos\phi\hat{\mathbf{x}}\hat{\mathbf{y}}
\right).
\end{equation}

We summarize these expressions in Table \ref{tab:review}. In order to derive the
equations for a given polarization case, we refer to the equations listed there.
Then it is simply a matter of multiplying the terms correctly and obtaining the
appropriate components of $\boldsymbol{\chi}(-2\omega; \omega, \omega)$.

\begin{table}[b]
\centering
\begin{tabular}{ | c l l | c | c | }
\hline
Case               & $\hat{\mathbf{e}}^{\mathrm{out}}$
                   & $\hat{\mathbf{e}}^{\mathrm{in}}$
                   & $\mathbf{e}^{2\omega}_{\ell}$
                   & $\mathbf{e}^{\omega}_{\ell}\mathbf{e}^{\omega}_{\ell}$ \\
\hline
$\mathcal{R}_{pP}$ & $\hat{\mathbf{P}}_{v+}$
                   & $\hat{\mathbf{p}}_{v-}$
                   &  Eq. \eqref{eq:e2wpmr} or \eqref{eq:e2wp}
                   & Eq. \eqref{eq:ewewp} \\
$\mathcal{R}_{pS}$ & $\hat{\mathbf{s}}$
                   & $\hat{\mathbf{p}}_{v-}$
                   &  Eq. \eqref{eq:e2wsmr} or \eqref{eq:e2ws}
                   & Eq. \eqref{eq:ewewp} \\
$\mathcal{R}_{sP}$ & $\hat{\mathbf{P}}_{v+}$
                   & $\hat{\mathbf{s}}$
                   &  Eq. \eqref{eq:e2wpmr} or \eqref{eq:e2wp}
                   & Eq. \eqref{eq:ewews} \\
$\mathcal{R}_{sS}$ & $\hat{\mathbf{s}}$
                   & $\hat{\mathbf{s}}$
                   &  Eq. \eqref{eq:e2wsmr} or \eqref{eq:e2ws}
                   & Eq. \eqref{eq:ewews} \\
\hline
\end{tabular}
\caption{Polarization unit vectors for $\hat{\mathbf{e}}^{\mathrm{out}}$ and
$\hat{\mathbf{e}}^{\mathrm{in}}$, and equations describing
$\mathbf{e}^{2\omega}_{\ell}$ and
$\mathbf{e}^{\omega}_{\ell}\mathbf{e}^{\omega}_{\ell}$ for each polarization
case. When there are two equations to choose from, the former includes the
effects of multiple reflections, and the latter neglects
them.\label{tab:review}}
\end{table}


\subsection{Nonzero components of \texorpdfstring{$\boldsymbol{\chi}(-2\omega;
\omega, \omega)$}{X(2w;-w,w)}}


\subsubsection{The (111) surface}
For a (111) surface with $C_{3v}$ symmetry, we have the following nonzero
components: 
\begin{equation}\label{eq:nonzero111}
\begin{split}
\chi_{xxx}&=-\chi_{xyy}=-\chi_{yyx},\\
\chi_{xxz}&=\chi_{yyz},\\
\chi_{zxx}&=\chi_{zyy},\\
\chi_{zzz}&.
\end{split}
\end{equation}


\subsubsection{The (110) surface}
For a (110) surface with $C_{2v}$ symmetry, we have the following nonzero
components: 
\begin{equation}\label{eq:nonzero110}
\chi_{xxz}, \chi_{yyz}, \chi_{zxx}, \chi_{zyy}, \chi_{zzz}.
\end{equation}


\subsubsection{The (001) surface}
For a (001) surface with $C_{4v}$ symmetry, we have the following nonzero
components: 
\begin{equation}\label{eq:nonzero001}
\begin{split}
\chi_{xxz}&=\chi_{yyz},\\
\chi_{zxx}&=\chi_{zyy},\\
\chi_{zzz}&.
\end{split}
\end{equation}

%%%%%%%%%%%%%%%%%%%%%%%%%%%%%%%%%%%%%%%%%%%%%%%%%%%%%%%%%%%%%%%%%%%%%%%%%%%%%%%%
%%%%%%%%%%%%%%%%%%%%%%%%%%%%%%%%%%%%%%%%%%%%%%%%%%%%%%%%%%%%%%%%%%%%%%%%%%%%%%%%


\section{\texorpdfstring{$\mathcal{R}_{pP}$}{RpP}}

In this section, we derive the expresions for $\mathcal{R}_{pP}$ for different
limiting cases. We evaluate $\mathcal{P}(2\omega)$ and the fundamental fields in
different regions. It is worth noting that the first case, the three layer
model, can be reduced to any of the other cases by simply considering where we
want to evaluate the $1\omega$ and $2\omega$ terms.

Per Table \ref{tab:review}, $\mathcal{R}_{pP}$ requires Eqs. \eqref{eq:e2wp}
and \eqref{eq:ewewp}. After some algebra, we obtain that
\begin{equation}\label{eq:rppfullmult}
\begin{split}
\Upsilon^{\mathrm{MR}}_{pP} =
\Gamma^{\mathrm{MR}}_{pP}
\bigg[
- R^{M-}_{p}&W_{\ell}\big(
   n^{4}_{\ell}w^{2}_{b}\cos^{3}\phi\chi_{xxx}
 + 2n^{4}_{\ell}w^{2}_{b}\sin\phi\cos^{2}\phi\chi_{xxy}\\
&+ 2n^{2}_{\ell}n^{2}_{b}w_{b}\sin\theta_{0}\cos^{2}\phi\chi_{xxz}
 + n^{4}_{\ell}w^{2}_{b}\sin^{2}\phi\cos\phi\chi_{xyy}\\
&+ 2n^{2}_{\ell}n^{2}_{b}w_{b}\sin\theta_{0}\sin\phi\cos\phi\chi_{xyz}
 + n^{4}_{b}\sin^{2}\theta_{0}\cos\phi\chi_{xzz}
 \big)\\\\
%%%%%%%%%%%%%%%%%%%%%%%%%%%%%%%%%%%%%%%%%%%%%%%%%%%%%%%%%%%%%
- R^{M-}_{p}&W_{\ell}\big(
  n^{4}_{\ell}w^{2}_{b}\sin\phi\cos^{2}\phi\chi_{yxx}
+ 2n^{4}_{\ell}w^{2}_{b}\sin^{2}\phi\cos\phi\chi_{yxy}\\
&+ 2n^{2}_{\ell}n^{2}_{b}w_{b}\sin\theta_{0}\sin\phi\cos\phi\chi_{yxz}
 + n^{4}_{\ell}w^{2}_{b}\sin^{3}\phi\chi_{yyy}\\
&+ 2n^{2}_{\ell}n^{2}_{b}w_{b}\sin\theta_{0}\sin^{2}\phi\chi_{yyz}
 + n^{4}_{b}\sin^{2}\theta_{0}\sin\phi\chi_{yzz}
 \big)\\\\
%%%%%%%%%%%%%%%%%%%%%%%%%%%%%%%%%%%%%%%%%%%%%%%%%%%%%%%%%%%%%
+ R^{M+}_{p}&\sin\theta_{0}\big(
  n^{4}_{\ell}w^{2}_{b}\cos^{2}\phi\chi_{zxx}
 + 2n^{4}_{\ell}w^{2}_{b}\sin\phi\cos\phi\chi_{zxy}\\
&+ n^{4}_{\ell}w^{2}_{b}\sin^{2}\phi\chi_{zyy}
 + 2n^{2}_{\ell}n^{2}_{b}w_{b}\sin\theta_{0}\cos\phi\chi_{zzx}\\
&+ 2n^{2}_{\ell}n^{2}_{b}w_{b}\sin\theta_{0}\sin\phi\chi_{zzy}
 + n^{4}_{b}\sin^{2}\theta_{0}\chi_{zzz}
 \big)
\bigg].
\end{split}
\end{equation}
We take this opportunity to introduce a quantity that will be repeated
throughout this section,
\begin{equation}\label{eq:gammamr}
\Gamma^{\mathrm{MR}}_{pP} =
\frac{T^{v\ell}_{p}}{N_{\ell}}
\left(\frac{t^{v\ell}_{p}t^{\ell b}_{p}}
{n^{2}_{\ell}n_{b}}\right)^{2}.
\end{equation}

If we neglect the multiple reflections, as described in the manuscript, we have
that
\begin{equation}\label{eq:rppfull}
\begin{split}
\Upsilon_{pP} =
\Gamma_{pP}
%%%%%%%%%%%%%%%%%%%%%%%%%%%%%%%%%%%%%%%%%%%%
\bigg[
- N^{2}_{\ell}W_{b}\big(
&+ n^{4}_{\ell}w^{2}_{b}\cos^{3}\phi\chi_{xxx}
 + 2n^{4}_{\ell}w^{2}_{b}\sin\phi\cos^{2}\phi\chi_{xxy}\\
&+ 2n^{2}_{b}n^{2}_{\ell}w_{b}\sin\theta_{0}\cos^{2}\phi\chi_{xxz}
 + n^{4}_{\ell}w^{2}_{b}\sin^{2}\phi\cos\phi\chi_{xyy}\\
&+ 2n^{2}_{b}n^{2}_{\ell}w_{b}\sin\theta_{0}\sin\phi\cos\phi\chi_{xyz}
 + n^{4}_{b}\sin^{2}\theta_{0}\cos\phi\chi_{xzz}
  \big)\\\\
%%%%%%%%%%%%%%%%%%%%%%%%%%%%%
- N^{2}_{\ell}W_{b}\big(
&+ n^{4}_{\ell}w^{2}_{b}\sin\phi\cos^{2}\phi\chi_{yxx}
 + 2n^{4}_{\ell}w^{2}_{b}\sin^{2}\phi\cos\phi\chi_{yxy}\\
&+ 2n^{2}_{b}n^{2}_{\ell}w_{b}\sin\theta_{0}\sin\phi\cos\phi\chi_{yxz}
 + n^{4}_{\ell}w^{2}_{b}\sin^{3}\phi\chi_{yyy}\\
&+ 2n^{2}_{b}n^{2}_{\ell}w_{b}\sin\theta_{0}\sin^{2}\phi\chi_{yyz}
 + n^{4}_{b}\sin^{2}\theta_{0}\sin\phi\chi_{yzz}
  \big)\\\\
%%%%%%%%%%%%%%%%%%%%%%%%%%%%%%
+ N^{2}_{b}\sin\theta_{0}\big(
&+ n^{4}_{\ell}w^{2}_{b}\cos^{2}\phi\chi_{zxx}
 + 2n^{4}_{\ell}w^{2}_{b}\sin\phi\cos\phi\chi_{zxy}\\
&+ n^{4}_{\ell}w^{2}_{b}\sin^{2}\phi\chi_{zyy}
 + 2n^{2}_{\ell}n^{2}_{b}w_{b}\sin\theta_{0}\cos\phi\chi_{zzx}\\
&+ 2n^{2}_{\ell}n^{2}_{b}w_{b}\sin\theta_{0}\sin\phi\chi_{zzy}
 + n^{4}_{b}\sin^{2}\theta_{0}\chi_{zzz}
  \big)
\bigg],
\end{split}
\end{equation}
and again we introduce a quantity that will be repeated throughout this section,
\begin{equation}
\Gamma_{pP} =
\frac{T^{v\ell}_{p}T^{\ell b}_{p}}
     {N^{2}_{\ell}N_{b}}
\left(\frac{t^{v\ell}_{p}t^{\ell b}_{p}}
{n^{2}_{\ell}n_{b}}\right)^{2}.
\end{equation}


\subsection{For the (111) surface}

We take Eqs. \eqref{eq:rppfullmult} and \eqref{eq:nonzero111}, eliminate the
components that do not contribute, and apply the the symmetry relations as
follows,
\begin{equation*}
\begin{split}
\Upsilon^{\mathrm{MR},(111)}_{pP} =
\Gamma^{\mathrm{MR}}_{pP}
\big[
&+ R^{M+}_{p}n^{4}_{b}\sin^{3}\theta_{0}\chi_{zzz}\\
&+ R^{M+}_{p}n^{4}_{\ell}w^{2}_{b}\sin\theta_{0}\cos^{2}\phi\chi_{zxx}\\
&+ R^{M+}_{p}n^{4}_{\ell}w^{2}_{b}\sin\theta_{0}\sin^{2}\phi\chi_{zxx}\\
&- 2R^{M-}_{p}n^{2}_{\ell}n^{2}_{b}w_{b}W_{\ell}\sin\theta_{0}\cos^{2}\phi
   \chi_{xxz}\\
&- 2R^{M-}_{p}n^{2}_{\ell}n^{2}_{b}w_{b}W_{\ell}\sin\theta_{0}\sin^{2}\phi
   \chi_{xxz}\\
&- R^{M-}_{p}n^{4}_{\ell}w^{2}_{b}W_{\ell}\cos^{3}\phi\chi_{xxx}\\
&+ R^{M-}_{p}n^{4}_{\ell}w^{2}_{b}W_{\ell}\sin^{2}\phi\cos\phi\chi_{xxx}\\
&+ 2R^{M-}_{p}n^{4}_{\ell}w^{2}_{b}W_{\ell}\sin^{2}\phi\cos\phi\chi_{xxx}
\big].
\end{split}
\end{equation*}
We reduce terms,
\begin{equation*}
\begin{split}
\Upsilon^{\mathrm{MR},(111)}_{pP} &=
\Gamma^{\mathrm{MR}}_{pP}
\big[
+ R^{M+}_{p}n^{4}_{b}\sin^{3}\theta_{0}\chi_{zzz}\\
&\qquad\qquad+ R^{M+}_{p}n^{4}_{\ell}w^{2}_{b}\sin\theta_{0}
   (\sin^{2}\phi+\cos^{2}\phi)\chi_{zxx}\\
&\qquad\qquad- 2R^{M-}_{p}n^{2}_{\ell}n^{2}_{b}w_{b}W_{\ell}\sin\theta_{0}
   (\sin^{2}\phi+\cos^{2}\phi)\chi_{xxz}\\
&\qquad\qquad+ R^{M-}_{p}n^{4}_{\ell}w^{2}_{b}W_{\ell}
   (3\sin^{2}\phi\cos\phi - \cos^{3}\phi)\chi_{xxx}
\big]\\\\
&=
\Gamma^{\mathrm{MR}}_{pP}
\big[
R^{M+}_{p}\sin\theta_{0}(n^{4}_{b}\sin^{2}\theta_{0}\chi_{zzz} 
+ n^{4}_{\ell}w^{2}_{b}\chi_{zxx})\\
&\qquad\qquad- R^{M-}_{p}n^{2}_{\ell}w_{b}W_{\ell}(2n^{2}_{b}\sin\theta_{0}
\chi_{xxz}
+ n^{2}_{\ell}w_{b}\chi_{xxx}\cos3\phi)
\big]\\\\
& = \Gamma^{\mathrm{MR}}_{pP}\,r^{\mathrm{MR},(111)}_{pP},
\end{split}
\end{equation*}
where
\begin{equation}
\boxed{
\begin{split}
r^{\mathrm{MR},(111)}_{pP} = 
&R^{M+}_{p}\sin\theta_{0}(n^{4}_{b}\sin^{2}\theta_{0}\chi_{zzz} 
+ n^{4}_{\ell}w^{2}_{b}\chi_{zxx})\\
&- R^{M-}_{p}n^{2}_{\ell}w_{b}W_{\ell}(2n^{2}_{b}\sin\theta_{0}\chi_{xxz}
+ n^{2}_{\ell}w_{b}\chi_{xxx}\cos3\phi).
\end{split}
}
\end{equation}

If we wish to neglect the effects of the multiple reflections, we follow the
exact same procedure but starting with Eq. \eqref{eq:rppfull},
\begin{equation*}
\begin{split}
\Upsilon^{(111)}_{pP}
= \Gamma_{pP}
\big[
&+ n^{4}_{b}N^{2}_{b}\sin^{3}\theta_{0}\chi_{zzz}\\
&+ n^{4}_{\ell}N^{2}_{b}w^{2}_{b}\sin\theta_{0}\cos^{2}\phi\chi_{zxx}\\
&+ n^{4}_{\ell}N^{2}_{b}w^{2}_{b}\sin\theta_{0}\sin^{2}\phi\chi_{zxx}\\
&- 2n^{2}_{b}n^{2}_{\ell}N^{2}_{\ell}w_{b}W_{b}\sin\theta_{0}\cos^{2}\phi
  \chi_{xxz}\\
&- 2n^{2}_{b}n^{2}_{\ell}N^{2}_{\ell}w_{b}W_{b}\sin\theta_{0}\sin^{2}\phi
  \chi_{xxz}\\
&- n^{4}_{\ell}N^{2}_{\ell}w^{2}_{b}W_{b}\cos^{3}\phi\chi_{xxx}\\
&+ n^{4}_{\ell}N^{2}_{\ell}w^{2}_{b}W_{b}\sin^{2}\phi\cos\phi\chi_{xxx}\\
&+ 2n^{4}_{\ell}N^{2}_{\ell}w^{2}_{b}W_{b}\sin^{2}\phi\cos\phi\chi_{xxx}
\big],
\end{split}
\end{equation*}
and reduce,
\begin{equation*}
\begin{split}
\Upsilon^{(111)}_{pP} &=
\Gamma_{pP}
\big[
+ n^{4}_{b}N^{2}_{b}
   \sin^{3}\theta_{0}\chi_{zzz}\\
&\qquad\qquad+ n^{4}_{\ell}N^{2}_{b}w^{2}_{b}
   \sin\theta_{0}(\sin^{2}\phi + \cos^{2}\phi)\chi_{zxx}\\
&\qquad\qquad- 2n^{2}_{b}n^{2}_{\ell}N^{2}_{\ell}w_{b}W_{b}
   \sin\theta_{0}(\sin^{2}\phi + \cos^{2}\phi)\phi\chi_{xxz}\\
&\qquad\qquad+ n^{4}_{\ell}N^{2}_{\ell}w^{2}_{b}W_{b}
   (3\sin^{2}\phi\cos\phi - \cos^{3}\phi)\chi_{xxx}
\big]\\\\
&=
\Gamma_{pP}
\big[
N^{2}_{b}\sin\theta_{0}(n^{4}_{b}\sin^{2}\theta_{0}\chi_{zzz} 
+ n^{4}_{\ell}w^{2}_{b}\chi_{zxx})\\
&\qquad\qquad- n^{2}_{\ell}N^{2}_{\ell}w_{b}W_{b}(2n^{2}_{b}\sin\theta_{0}\phi
\chi_{xxz}
 + n^{2}_{\ell}w_{b}\chi_{xxx}\cos3\phi)
\big]\\\\
&= \Gamma_{pP}\,r^{(111)}_{pP},
\end{split}
\end{equation*}
where
\begin{equation}
\boxed{
\begin{split}
r^{(111)}_{pP} = 
&N^{2}_{b}\sin\theta_{0}(n^{4}_{b}\sin^{2}\theta_{0}\chi_{zzz} 
+ n^{4}_{\ell}w^{2}_{b}\chi_{zxx})\\
&- n^{2}_{\ell}N^{2}_{\ell}w_{b}W_{b}(2n^{2}_{b}\sin\theta_{0}\phi\chi_{xxz}
 + n^{2}_{\ell}w_{b}\chi_{xxx}\cos3\phi).
\end{split}
}
\end{equation}


\subsection{For the (110) surface}

We take Eqs. \eqref{eq:rppfullmult} and \eqref{eq:nonzero110}, eliminate the
components that do not contribute, and apply the the symmetry relations as
follows,
\begin{equation*}
\begin{split}
\Upsilon^{\mathrm{MR},(110)}_{pP} &=
\Gamma^{\mathrm{MR}}_{pP}
\big[
R^{M+}_{p}\sin\theta_{0}\\
&\qquad\qquad(n^{4}_{b}\sin^{2}\theta_{0}\chi_{zzz} 
 + n^{4}_{\ell}w^{2}_{b}\cos^{2}\phi\chi_{zxx}
 + n^{4}_{\ell}w^{2}_{b}\sin^{2}\phi\chi_{zyy})\\
&\qquad\qquad\quad- 2R^{M-}_{p}n^{2}_{\ell}n^{2}_{b}w_{b}W_{\ell}\sin\theta_{0}
 (\cos^{2}\phi\chi_{xxz} + \sin^{2}\phi\chi_{yyz})
\big]\\
&= \Gamma^{\mathrm{MR}}_{pP}\,r^{\mathrm{MR},(110)}_{pP},
\end{split}
\end{equation*}
where
\begin{equation}
\boxed{
\begin{split}
r^{\mathrm{MR},(110)}_{pP} &= 
R^{M+}_{p}\sin\theta_{0}(n^{4}_{b}\sin^{2}\theta_{0}\chi_{zzz} 
 + n^{4}_{\ell}w^{2}_{b}\cos^{2}\phi\chi_{zxx}
 + n^{4}_{\ell}w^{2}_{b}\sin^{2}\phi\chi_{zyy})\\
&\qquad- 2R^{M-}_{p}n^{2}_{\ell}n^{2}_{b}w_{b}W_{\ell}\sin\theta_{0}
 (\cos^{2}\phi\chi_{xxz} + \sin^{2}\phi\chi_{yyz}).
\end{split}
}
\end{equation}

If we wish to neglect the effects of the multiple reflections, we follow the
exact same procedure but starting with Eq. \eqref{eq:rppfull},
\begin{equation*}
\begin{split}
\Upsilon^{(110)}_{pP} &=
\Gamma_{pP}
\big[
N^{2}_{b}\sin\theta_{0}(n^{4}_{b}\sin^{2}\theta_{0}\chi_{zzz}
 + n^{4}_{\ell}w^{2}_{b}\cos^{2}\phi\chi_{zxx}
 + n^{4}_{\ell}w^{2}_{b}\sin^{2}\phi\chi_{zyy})\\
&\qquad- 2n^{2}_{b}n^{2}_{\ell}N^{2}_{\ell}w_{b}W_{b}\sin\theta_{0}
(\cos^{2}\phi\chi_{xxz} + \sin^{2}\phi\chi_{yyz})
\big]\\
&= \Gamma_{pP}\,r^{(110)}_{pP},
\end{split}
\end{equation*}
where
\begin{equation}
\boxed{
\begin{split}
r^{(110)}_{pP} &= 
N^{2}_{b}\sin\theta_{0}(n^{4}_{b}\sin^{2}\theta_{0}\chi_{zzz}
 + n^{4}_{\ell}w^{2}_{b}\cos^{2}\phi\chi_{zxx}
 + n^{4}_{\ell}w^{2}_{b}\sin^{2}\phi\chi_{zyy})\\
&\qquad- 2n^{2}_{b}n^{2}_{\ell}N^{2}_{\ell}w_{b}W_{b}\sin\theta_{0}
(\cos^{2}\phi\chi_{xxz} + \sin^{2}\phi\chi_{yyz}).
\end{split}
}
\end{equation}


\subsection{For the (001) surface}

We take Eqs. \eqref{eq:rppfullmult} and \eqref{eq:nonzero110}, eliminate the
components that do not contribute, and apply the the symmetry relations as
follows,
\begin{equation*}
\begin{split}
\Upsilon^{\mathrm{MR},(001)}_{pP} &=
\Gamma^{\mathrm{MR}}_{pP}
\big[
   R^{M+}_{p}n^{4}_{b}\sin^{3}\theta_{0}\chi_{zzz}\\
&\qquad\qquad+ R^{M+}_{p}n^{4}_{\ell}w^{2}_{b}\sin\theta_{0}\cos^{2}\phi
\chi_{zxx}\\
&\qquad\qquad+ R^{M+}_{p}n^{4}_{\ell}w^{2}_{b}\sin\theta_{0}\sin^{2}\phi
\chi_{zxx}\\
&\qquad\qquad- 2R^{M-}_{p}n^{2}_{\ell}n^{2}_{b}w_{b}W_{\ell}\sin\theta_{0}
\cos^{2}\phi\chi_{xxz}\\
&\qquad\qquad- 2R^{M-}_{p}n^{2}_{\ell}n^{2}_{b}w_{b}W_{\ell}\sin\theta_{0}
\sin^{2}\phi\chi_{xxz}
\big]\\\\
&=
\Gamma^{\ell}_{pP}
\big[
R^{M+}_{p}\sin\theta_{0}(n^{4}_{b}\sin^{2}\theta_{0}\chi_{zzz}
+ n^{4}_{\ell}w^{2}_{b}\chi_{zxx})\\
&\qquad\qquad- 2R^{M-}_{p}n^{2}_{\ell}n^{2}_{b}w_{b}W_{\ell}\sin\theta_{0}
\chi_{xxz}
\big]\\\\
&= \Gamma^{\mathrm{MR}}_{pP}\,r^{\mathrm{MR},(001)}_{pP},
\end{split}
\end{equation*}
where
\begin{equation}
\boxed{
\begin{split}
r^{\mathrm{MR},(001)}_{pP} = &R^{M+}_{p}\sin\theta_{0}(n^{4}_{b}\sin^{2}\theta_{0}\chi_{zzz}
+ n^{4}_{\ell}w^{2}_{b}\chi_{zxx})\\
&- 2R^{M-}_{p}n^{2}_{\ell}n^{2}_{b}w_{b}W_{\ell}\sin\theta_{0}\chi_{xxz},
\end{split}
}
\end{equation}

If we wish to neglect the effects of the multiple reflections, we follow the
exact same procedure but starting with Eq. \eqref{eq:rppfull},
\begin{equation*}
\begin{split}
\Upsilon^{(001)}_{pP} &=
\Gamma_{pP}
\big[
N^{2}_{b}\sin\theta_{0}(n^{4}_{b}\sin^{2}\theta_{0}\chi_{zzz}
+ n^{4}_{\ell}w^{2}_{b}\chi_{zxx})\\
&\qquad\qquad- 2n^{2}_{b}n^{2}_{\ell}N^{2}_{\ell}w_{b}W_{b}\sin\theta_{0}
\chi_{xxz}
\big]\\
&= \Gamma_{pP}\,r^{(001)}_{pp},
\end{split}
\end{equation*}
where
\begin{equation}
\boxed{
\begin{split}
r^{(001)}_{pP} = 
&N^{2}_{b}\sin\theta_{0}(n^{4}_{b}\sin^{2}\theta_{0}\chi_{zzz}
+ n^{4}_{\ell}w^{2}_{b}\chi_{zxx})\\
&- 2n^{2}_{b}n^{2}_{\ell}N^{2}_{\ell}w_{b}W_{b}\sin\theta_{0}\chi_{xxz}.
\end{split}
}
\end{equation}


%%%%%%%%%%%%%%%%%%%%%%%%%%%%%%%%%%%%%%%%%%%%%%%%%%%%%%%%%%%%%%%%%%%%%%%%%%%%%%%%
%%%%%%%%%%%%%%%%%%%%%%%%%%%%%%%%%%%%%%%%%%%%%%%%%%%%%%%%%%%%%%%%%%%%%%%%%%%%%%%%


\section{\texorpdfstring{$\mathcal{R}_{pS}$}{RpS}}

Per Table \ref{tab:summary}, $\mathcal{R}_{pS}$ requires Eqs. \eqref{eq:e2ws}
and \eqref{eq:ewewp}. After some algebra, we obtain that
\begin{equation*}
\begin{split}
\mathbf{e}^{2\omega}_{\ell}\cdot\boldsymbol{\chi}:
\mathbf{e}^{\omega}_{\ell}\mathbf{e}^{\omega}_{\ell}
= T_{s}^{v\ell}R^{M+}_{s}
\left(\frac{t^{v\ell}_{p}t^{\ell b}_{p}}
{n^{2}_{\ell}n_{b}}\right)^{2}
\big[
&- n^{4}_{b}\sin^{2}\theta_{0}\sin\phi\chi_{xzz}\\
&- n^{4}_{\ell}w^{2}_{b}\sin\phi\cos^{2}\phi\chi_{xxx}\\
&- n^{4}_{\ell}w^{2}_{b}\sin^{3}\phi\chi_{xyy}\\
&- 2n^{2}_{\ell}n^{2}_{b}w_{b}\sin\theta_{0}\sin\phi\cos\phi\chi_{xzx}\\
&- 2n^{2}_{\ell}n^{2}_{b}w_{b}\sin\theta_{0}\sin^{2}\phi\chi_{xzy}\\
&- 2n^{4}_{\ell}w^{2}_{b}\sin^{2}\phi\cos\phi\chi_{xxy}\\
%%%%%%%%%%%%%%%%%%%%%%%%%%%%%%%%%%%%%%%%%%%
&+ n^{4}_{b}\sin^{2}\theta_{0}\cos\phi\chi_{yzz}\\
&+ n^{4}_{\ell}w^{2}_{b}\cos^{3}\phi\chi_{yxx}\\
&+ n^{4}_{\ell}w^{2}_{b}\sin^{2}\phi\cos\phi\chi_{yyy}\\
&+ 2n^{2}_{\ell}n^{2}_{b}w_{b}\sin\theta_{0}\cos^{2}\phi\chi_{yzx}\\
&+ 2n^{2}_{\ell}n^{2}_{b}w_{b}\sin\theta_{0}\sin\phi\cos\phi\chi_{yzy}\\
&+ 2n^{4}_{\ell}w^{2}_{b}\sin\phi\cos^{2}\phi\chi_{yxy}
\big].
\end{split}
\end{equation*}

And neglectiong multiple refs,
\begin{equation*}
\begin{split}
\mathbf{e}^{2\omega}_{\ell}
\cdot\boldsymbol{\chi}:
\mathbf{e}^{\omega}_{\ell}\mathbf{e}^{\omega}_{\ell} =
\qquad\qquad\qquad&\\\\
T^{v\ell}_{s}T^{\ell b}_{s}\left(\frac{t^{v\ell}_{p}t^{\ell b}_{p}}
      {\epsilon_{\ell}(\omega)\sqrt{\epsilon_{b}(\omega)}}\right)^{2}
\big[
- &\epsilon_{b}^{2}(\omega)\sin^{2}\theta_{\mathrm{in}}\sin\phi\chi_{xzz}\\
-&2\epsilon_{b}(\omega)\epsilon_{\ell}(\omega)k_{b}\sin\theta_{0}
   \cos\phi\sin\phi\chi_{xxz}\\
- &\epsilon^{2}_{\ell}(\omega)k^{2}_{b}\cos^{2}\phi\sin\phi\chi_{xxx}\\
-&2\epsilon^{2}_{\ell}(\omega)k^{2}_{b}\cos\phi\sin^{2}\phi\chi_{xxy}\\
- &\epsilon^{2}_{\ell}(\omega)k^{2}_{b}\sin^{3}\phi\chi_{xyy}\\
-&2\epsilon_{b}(\omega)\epsilon_{\ell}(\omega)k_{b}\sin\theta_{0}
   \sin^{2}\phi\chi_{xyz}\\
+ &\epsilon_{b}^{2}(\omega)\sin^{2}\theta_{\mathrm{in}}\cos\phi\chi_{yzz}\\
+&2\epsilon_{b}(\omega)\epsilon_{\ell}(\omega)k_{b}\sin\theta_{0}
   \cos^{2}\phi\chi_{yxz}\\
+ &\epsilon^{2}_{\ell}(\omega)k^{2}_{b}\cos^{3}\phi\chi_{yxx}\\
+&2\epsilon^{2}_{\ell}(\omega)k^{2}_{b}\cos^{2}\phi\sin\phi\chi_{yxy}\\
+ &\epsilon^{2}_{\ell}(\omega)k^{2}_{b}\cos\phi\sin^{2}\phi\chi_{yyy}\\
+&2\epsilon_{b}(\omega)\epsilon_{\ell}(\omega)k_{b}\sin\theta_{0}
   \cos\phi\sin\phi\chi_{yyz}
\big],
\end{split}
\end{equation*}
and taking into account that $\chi_{xzz} = \chi_{xxy} = \chi_{xyz} = \chi_{yzz}
= \chi_{yxz} = \chi_{yxx} = \chi_{yyy} = 0$, we have
\begin{equation*}
\begin{split}
= \Gamma^{\ell}_{pS}
\big[
+ &\epsilon^{2}_{\ell}(\omega)k^{2}_{b}\sin^{3}\phi\chi_{xxx}\\
-&2\epsilon^{2}_{\ell}(\omega)k^{2}_{b}\cos^{2}\phi\sin\phi\chi_{xxx}\\
- &\epsilon^{2}_{\ell}(\omega)k^{2}_{b}\cos^{2}\phi\sin\phi\chi_{xxx}\\
+&2\epsilon_{b}(\omega)\epsilon_{\ell}(\omega)k_{b}\sin\theta_{0}
   \cos\phi\sin\phi\chi_{xxz}\\
-&2\epsilon_{b}(\omega)\epsilon_{\ell}(\omega)k_{b}\sin\theta_{0}
   \cos\phi\sin\phi\chi_{xxz}
\big]\\\\
= \Gamma^{\ell}_{pS}
\big[
&\epsilon^{2}_{\ell}(\omega)k^{2}_{b}
(\sin^{3}\phi - 3\cos^{2}\phi\sin\phi)\chi_{xxx}
\big]\\\\
= \Gamma^{\ell}_{pS}
\big[
&-\epsilon^{2}_{\ell}(\omega)k^{2}_{b}
\sin3\phi\chi_{xxx}
\big].
\end{split}
\end{equation*}
We summarize as follows,
\begin{equation*}
\mathbf{e}^{\,2\omega}_{\ell}\cdot
\boldsymbol{\chi}:\mathbf{e}^\omega_{\ell}\mathbf{e}^\omega_{\ell}
\equiv\Gamma^{\ell}_{pS}\, r^{\ell}_{pS},
\end{equation*}
where
\begin{equation*}
r^{\ell}_{pS}
= -\epsilon^{2}_{\ell}(\omega)k^{2}_{b}\sin3\phi\chi_{xxx},
\end{equation*} 
and  
\begin{equation*}
\Gamma^{\ell}_{pS} =
T^{v\ell}_{s}T^{\ell b}_{s}\left(\frac{t^{v\ell}_{p}t^{\ell b}_{p}}
      {\epsilon_{\ell}(\omega)\sqrt{\epsilon_{b}(\omega)}}\right)^{2}
\end{equation*} 
In order to reduce above result to that of Ref. \cite{mizrahiJOSA88} and
\cite{sipePRB87},  we take the 2-$\omega$ radiations factors for vacuum by
taking $\ell=v$, thus $\epsilon_{\ell}(2\omega)=1$, $T^{v\ell}_{s}=1$,
$T^{\ell b}_{s}=T^{vb}_{s}$, and the fundamental field inside medium $b$ by
taking $\ell=b$, thus $\epsilon_{\ell}(\omega)=\epsilon_{b}(\omega)$,
$t^{v\ell}_{p}=t^{vb}_{p}$, and $t^{\ell b}_{p}=1$. With these choices,
\begin{equation*}
r^{b}_{pS} = -k^{2}_{b}\sin3\phi\chi_{xxx},
\end{equation*} 
and 
\begin{equation*}
\Gamma^{b}_{pS} =
T^{vb}_{s}
\left(
\frac{t^{vb}_{p}}{\sqrt{\epsilon_{b}(\omega)}}
\right)^{2}.  
\end{equation*} 


\subsection{For the (111) surface}

For this surface we have that
$\chi_{xzz}=\chi_{xzy}=\chi_{xxy}=\chi_{yzz}=\chi_{yxx}=\chi_{yyy}=
\chi_{yzx}=0$. We use the symmetry relations where necessary,
\begin{equation*}
\begin{split}
\Upsilon^{\ell\ell}_{pP} =
\Gamma^{\ell}_{pS}
\big[
&- n^{4}_{\ell}w^{2}_{b}\sin\phi\cos^{2}\phi\chi_{xxx}\\
&+ n^{4}_{\ell}w^{2}_{b}\sin^{3}\phi\chi_{xxx}\\
&- 2n^{2}_{\ell}n^{2}_{b}w_{b}\sin\theta_{0}\sin\phi\cos\phi\chi_{xxz}\\
&+ 2n^{2}_{\ell}n^{2}_{b}w_{b}\sin\theta_{0}\sin\phi\cos\phi\chi_{xxz}\\
&- 2n^{4}_{\ell}w^{2}_{b}\sin\phi\cos^{2}\phi\chi_{xxx}
\big],
\end{split}
\end{equation*}
and reduce
\begin{equation*}
\begin{split}
\Upsilon^{\ell\ell}_{pP} &=
\Gamma^{\ell}_{pS}
\big[
n^{4}_{\ell}w^{2}_{b}(\sin^{3}\phi - 3\sin\phi\cos^{2}\phi)\chi_{xxx}
\big]\\
&=
\Gamma^{\ell}_{pS}
\big[
-n^{4}_{\ell}w^{2}_{b}\chi_{xxx}\sin3\phi
\big].
\end{split}
\end{equation*}
Finally, we express this final result in a more compact form,
\begin{equation}
\begin{split}
\Upsilon^{\ell\ell}_{pP} &=
\Gamma^{\ell}_{pS}
r^{\ell}_{pS},
\end{split}
\end{equation}
where
\begin{equation}
r^{\ell}_{pS} = -n^{4}_{\ell}w^{2}_{b}\chi_{xxx}\sin3\phi,
\end{equation}
and
\begin{equation}
\Gamma^{\ell}_{pS} = T_{s}^{v\ell}R^{M+}_{s}
\left(\frac{t^{v\ell}_{p}t^{\ell b}_{p}}{n^{2}_{\ell}n_{b}}\right)^{2}.
\end{equation}

\subsection{For the (110) surface}


\subsection{For the (001) surface}


%%%%%%%%%%%%%%%%%%%%%%%%%%%%%%%%%%%%%%%%%%%%%%%%%%%%%%%%%%%%%%%%%%%%%%%%%%%%%%%%
%%%%%%%%%%%%%%%%%%%%%%%%%%%%%%%%%%%%%%%%%%%%%%%%%%%%%%%%%%%%%%%%%%%%%%%%%%%%%%%%


\section{\texorpdfstring{$\mathcal{R}_{sP}$}{RsP}}

Per Table \ref{tab:summary}, $\mathcal{R}_{sP}$ requires Eqs. \eqref{eq:e2wp}
and \eqref{eq:ewews}. After some algebra, we obtain that

\begin{equation*}
\begin{split}
\mathbf{e}^{2\omega}_{\ell}\cdot\boldsymbol{\chi}:
\mathbf{e}^{\omega}_{\ell}\mathbf{e}^{\omega}_{\ell} =
\frac{T^{v\ell}_{p}}{N_{\ell}}
\left(t^{v\ell}_{s}t^{\ell b}_{s}\right)^{2}
\big[
&+ R^{M+}_{p}\sin\theta_{0}\sin^{2}\phi
   \chi_{zxx}\\
&+ R^{M+}_{p}\sin\theta_{0}\cos^{2}\phi
   \chi_{zyy}\\
&- 2R^{M+}_{p}\sin\theta_{0}\sin\phi\cos\phi
   \chi_{zxy}\\
%%%%%%%%%%%%%%%%%%%%%%%%%%%%%%%%%%%%%%%%%%%%%%%%%%%%%%
&- R^{M-}_{p}W_{\ell}\sin^{2}\phi\cos\phi
   \chi_{xxx}\\
&- R^{M-}_{p}W_{\ell}\cos^{3}\phi
   \chi_{xyy}\\
&+ 2R^{M-}_{p}W_{\ell}\sin\phi\cos^{2}\phi
   \chi_{xxy}\\
%%%%%%%%%%%%%%%%%%%%%%%%%%%%%%%%%%%%%%%%%%%%%%%%%%%%%%
&- R^{M-}_{p}W_{\ell}\sin^{3}\phi
   \chi_{yxx}\\
&- R^{M-}_{p}W_{\ell}\sin\phi\cos^{2}\phi
   \chi_{yyy}\\
&+ 2R^{M-}_{p}W_{\ell}\sin^{2}\phi\cos\phi
   \chi_{yxy}
\big].
\end{split}
\end{equation*}
We know that $\chi_{zxy}=\chi_{xxy}=\chi_{yxx}=\chi_{yyy}=0$, so we can
eliminate and reduce components,
\begin{equation*}
\begin{split}
\mathbf{e}^{2\omega}_{\ell}\cdot\boldsymbol{\chi}:
\mathbf{e}^{\omega}_{\ell}\mathbf{e}^{\omega}_{\ell} =
\Gamma^{\ell}_{sP}
\big[
&+ R^{M+}_{p}\sin\theta_{0}\sin^{2}\phi\chi_{zxx}\\
&+ R^{M+}_{p}\sin\theta_{0}\cos^{2}\phi\chi_{zxx}\\
&- R^{M-}_{p}W_{\ell}\sin^{2}\phi\cos\phi\chi_{xxx}\\
&+ R^{M-}_{p}W_{\ell}\cos^{3}\phi\chi_{xxx}\\
&- 2R^{M-}_{p}W_{\ell}\sin^{2}\phi\cos\phi\chi_{xxx}
\big]\\
= \Gamma^{\ell}_{sP}
\big[
&+ R^{M+}_{p}\sin\theta_{0}(\sin^{2}\phi + \cos^{2}\phi)\chi_{zxx}\\
&+ R^{M-}_{p}W_{\ell}(\cos^{3}\phi - 3\sin^{2}\phi\cos\phi)\chi_{xxx}
\big]\\
= \Gamma^{\ell}_{sP}
\big[
&+ R^{M+}_{p}\sin\theta_{0}\chi_{zxx} + R^{M-}_{p}W_{\ell}\chi_{xxx}\cos3\phi
\big].
\end{split}
\end{equation*}
Finally, we express this final result in a more compact form,
\begin{equation}
\mathbf{e}^{2\omega}_{\ell}\cdot\boldsymbol{\chi}:
\mathbf{e}^{\omega}_{\ell}\mathbf{e}^{\omega}_{\ell} =
\Gamma^{\ell}_{sP}r^{\ell}_{sP},
\end{equation}
where
\begin{equation}
r^{\ell}_{sP} = 
R^{M+}_{p}\sin\theta_{0}\chi_{zxx} + R^{M-}_{p}W_{\ell}\chi_{xxx}\cos3\phi,
\end{equation}
and
\begin{equation}
\Gamma^{\ell}_{sP} = 
\frac{T^{v\ell}_{p}}{N_{\ell}}
\left(t^{v\ell}_{s}t^{\ell b}_{s}\right)^{2}.
\end{equation}

To obtain $R_{sP}(2\omega)$ we use
$\hat{\mathbf{e}}^{\mathrm{in}}=\hat{\mathbf{s}}$ in Eq. \eqref{m12}, and
$\hat{\mathbf{e}}^{\mathrm{out}}=\hat{\mathbf{P}}_{v+}$ in Eq. \eqref{r12}. We
also use the unit vectors defined in Eqs. \eqref{eq:kappavec} and
\eqref{eq:svec}. Substituting, we get
\begin{equation*}
\mathbf{e}^{2\omega}_{\ell} 
= \frac{T^{v\ell}_{p}T^{\ell b}_{p}}
       {\epsilon_{\ell}(2\omega)\sqrt{\epsilon_{b}(2\omega)}}
\left[
  \epsilon_{b}(2\omega)\sin\theta_{0}\hat{\mathbf{z}} 
- \epsilon_{\ell}(2\omega)K_{b}\cos\phi\hat{\mathbf{x}} 
- \epsilon_{\ell}(2\omega)K_{b}\sin\phi\hat{\mathbf{y}}
\right],
\end{equation*}
for $2\omega$, and for the fundamental fields,
\begin{equation*}
\mathbf{e}^{\omega}_{\ell}\mathbf{e}^{\omega}_{\ell}
= \left(t^{v\ell}_{s}t^{\ell b}_{s}\right)^{2}
\left(
  \sin^{2}\phi\hat{\mathbf{x}}\hat{\mathbf{x}}
+ \cos^{2}\phi\hat{\mathbf{y}}\hat{\mathbf{y}} 
- 2\sin\phi\cos\phi\hat{\mathbf{x}}\hat{\mathbf{y}}
\right).
\end{equation*}
Therefore,
\begin{equation*}
\begin{split}
\mathbf{e}^{2\omega}_{\ell}
\cdot\boldsymbol{\chi}:&
\mathbf{e}^{\omega}_{\ell}\mathbf{e}^{\omega}_{\ell} =\\
&\frac{T^{v\ell}_{p}T^{\ell b}_{p}\left(t^{v\ell}_{s}t^{\ell b}_{s}\right)^{2}}
      {\epsilon_{\ell}(2\omega)\sqrt{\epsilon_{b}(2\omega)}}
\big[
   \epsilon_{b}(2\omega)\sin\theta_{0}\sin^{2}\phi\chi_{zxx} 
 + \epsilon_{b}(2\omega)\sin\theta_{0}\cos^{2}\phi\chi_{zyy}\\
&- 2\epsilon_{b}(2\omega)\sin\theta_{0}\sin\phi\cos\phi\chi_{zxy}
 - \epsilon_{\ell}(2\omega)K_{b}\cos\phi\sin^{2}\phi\chi_{xxx}\\
&- \epsilon_{\ell}(2\omega)K_{b}\cos\phi\cos^{2}\phi\chi_{xyy}
 + 2\epsilon_{\ell}(2\omega)K_{b}\cos\phi\sin\phi\cos\phi\chi_{xxy}\\
&- \epsilon_{\ell}(2\omega)K_{b}\sin\phi\sin^{2}\phi\chi_{yxx}
 - \epsilon_{\ell}(2\omega)K_{b}\sin\phi\cos^{2}\phi\chi_{yyy}\\
&+ 2\epsilon_{\ell}(2\omega)K_{b}\sin\phi\sin\phi\cos\phi\chi_{yxy}
\big],
\end{split}
\end{equation*}
and taking into account that $\chi_{zxy} = \chi_{xxy} = \chi_{yxx} = \chi_{yyy}
= 0$, we have
\begin{equation*}
\begin{split}
= \Gamma^{\ell}_{sP}
&\big[
   \epsilon_{b}(2\omega)\sin\theta_{0}\sin^{2}\phi\chi_{zxx}
 + \epsilon_{b}(2\omega)\sin\theta_{0}\cos^{2}\phi\chi_{zxx}\\
&- \epsilon_{\ell}(2\omega)K_{b}\cos\phi\sin^{2}\phi\chi_{xxx}
 + \epsilon_{\ell}(2\omega)K_{b}\cos^{3}\phi\chi_{xxx}\\
&- 2\epsilon_{\ell}(2\omega)K_{b}\sin^{2}\phi\cos\phi\chi_{xxx}
\big]\\\\
= \Gamma^{\ell}_{sP}
&\big[
\epsilon_{b}(2\omega)\sin\theta_{0}
(\sin^{2}\phi + \cos^{2}\phi)\chi_{zxx}\\
&- \epsilon_{\ell}(2\omega)K_{b}
(\cos\phi\sin^{2}\phi - \cos^{3}\phi + 2\sin^{2}\phi\cos\phi)\chi_{xxx}
\big]\\\\
= \Gamma^{\ell}_{sP}
&\big[
  \epsilon_{b}(2\omega)\sin\theta_{0}\chi_{zxx} 
+ \epsilon_{\ell}(2\omega)K_{b}(\cos^{3}\phi - 3\sin^{2}\phi\cos\phi)\chi_{xxx}
\big]\\\\
= \Gamma^{\ell}_{sP}
&\big[
  \epsilon_{b}(2\omega)\sin\theta_{0}\chi_{zxx}
+ \epsilon_{\ell}(2\omega)K_{b}\cos3\phi\chi_{xxx}
\big].
\end{split}
\end{equation*}
We summarize as follows,
\begin{equation*}
\mathbf{e}^{\,2\omega}_{\ell}\cdot
\boldsymbol{\chi}:\mathbf{e}^\omega_{\ell}\mathbf{e}^\omega_{\ell}
\equiv\Gamma^{\ell}_{sP}\, r^{\ell}_{sP},
\end{equation*}
where
\begin{equation*}
r^{\ell}_{sP}
= \epsilon_{b}(2\omega)\sin\theta_{0}\chi_{zxx}
+ \epsilon_{\ell}(2\omega)K_{b}\chi_{xxx}\cos3\phi,
\end{equation*} 
and  
\begin{equation*}
\Gamma^{\ell}_{sP}=
\frac{T_{p}^{v\ell}T^{\ell b}_{p}\left(t_s^{v\ell}t^{\ell b}_s\right)^2}
     {\epsilon_{\ell}(2\omega)\sqrt{\epsilon_{b}(2\omega)}}.  
\end{equation*} 
In order to reduce above result to that of Ref. \cite{mizrahiJOSA88} and
\cite{sipePRB87},  we take the 2-$\omega$ radiations factors for vacuum by
taking $\ell=v$, thus $\epsilon_{\ell}(2\omega)=1$, $T^{v\ell}_{p}=1$,
$T^{\ell b}_{p}=T^{vb}_{p}$, and the fundamental field inside medium $b$ by
taking $\ell=b$, thus $\epsilon_{\ell}(\omega)=\epsilon_{b}(\omega)$,
$t^{v\ell}_s=t^{vb}_s$, and $t^{\ell b}_s=1$. With these choices,
\begin{equation*}
r^{b}_{sP} = \epsilon_{b}(2\omega)\sin\theta_{0}\chi_{zxx}
+ K_{b}\chi_{xxx}\cos3\phi,
\end{equation*} 
and 
\begin{equation*}
\Gamma^{b}_{sP} =
\frac{T^{v b}_{p}(t_s^{vb})^{2}}{\sqrt{\epsilon_{b}(2\omega)}}.  
\end{equation*}


\subsection{For the (111) surface}


\subsection{For the (110) surface}


\subsection{For the (001) surface}


%%%%%%%%%%%%%%%%%%%%%%%%%%%%%%%%%%%%%%%%%%%%%%%%%%%%%%%%%%%%%%%%%%%%%%%%%%%%%%%%
%%%%%%%%%%%%%%%%%%%%%%%%%%%%%%%%%%%%%%%%%%%%%%%%%%%%%%%%%%%%%%%%%%%%%%%%%%%%%%%%


\section{\texorpdfstring{$\mathcal{R}_{sS}$}{RsS}}

Per Table \ref{tab:summary}, $\mathcal{R}_{sS}$ requires Eqs. \eqref{eq:e2ws}
and \eqref{eq:ewews}. After some algebra, we obtain that
\begin{equation*}
\begin{split}
\mathbf{e}^{2\omega}_{\ell}\cdot\boldsymbol{\chi}:
\mathbf{e}^{\omega}_{\ell}\mathbf{e}^{\omega}_{\ell} = 
T_{s}^{v\ell}R^{M+}_{s}
\left(t^{v\ell}_{s}t^{\ell b}_{s}\right)^{2}
\big[
&- \sin^{3}\phi\chi_{xxx}\\
&- \sin\phi\cos^{2}\phi\chi_{xyy}\\
&+ 2\sin^{2}\phi\cos\phi\chi_{xxy}\\
&+ \sin^{2}\phi\cos\phi\chi_{yxx}\\
&+ \cos^{3}\phi\chi_{yyy}\\
&- 2\sin\phi\cos^{2}\phi\chi_{yxy}
\big].
\end{split}
\end{equation*}
As before, we know that $\chi_{xxy}=\chi_{yxx}=\chi_{yyy}=0$, so we eliminate
and reduce components,
\begin{equation*}
\begin{split}
\mathbf{e}^{2\omega}_{\ell}\cdot\boldsymbol{\chi}:
\mathbf{e}^{\omega}_{\ell}\mathbf{e}^{\omega}_{\ell} &= 
\Gamma^{\ell}_{sS}
\big[
- \sin^{3}\phi\chi_{xxx}
+ \sin\phi\cos^{2}\phi\chi_{xxx}
+ 2\sin\phi\cos^{2}\phi\chi_{xxx}
\big]\\
&= 
\Gamma^{\ell}_{sS}
\big[
(3\sin\phi\cos^{2}\phi- \sin^{3}\phi)\chi_{xxx}
\big]\\
&= \Gamma^{\ell}_{sS}\chi_{xxx}\sin3\phi.
\end{split}
\end{equation*}
Finally, we express this final result in a more compact form,
\begin{equation}
\mathbf{e}^{2\omega}_{\ell}\cdot\boldsymbol{\chi}:
\mathbf{e}^{\omega}_{\ell}\mathbf{e}^{\omega}_{\ell} = 
\Gamma^{\ell}_{sS}r^{\ell}_{sS},
\end{equation}
where
\begin{equation}
r^{\ell}_{sS} = \chi_{xxx}\sin3\phi,
\end{equation}
and
\begin{equation}
\Gamma^{\ell}_{sS} = 
T_{s}^{v\ell}R^{M+}_{s}\left(t^{v\ell}_{s}t^{\ell b}_{s}\right)^{2}.
\end{equation}

For $\mathcal{R}_{sS}$ we have that
$\hat{\mathbf{e}}^{\mathrm{in}}=\hat{\mathbf{s}}$ and
$\hat{\mathbf{e}}^{\mathrm{out}}=\hat{\mathbf{S}}$. This leads to
\begin{align*}
\mathbf{e}^{2\omega}_{\ell} 
&= T^{v\ell}_{s}T^{\ell b}_{s}
\left[-\sin\phi\hat{\mathbf{x}} + \cos\phi\hat{\mathbf{y}}\right],\\
\mathbf{e}^{\omega}_{\ell}\mathbf{e}^{\omega}_{\ell}
&= \left(t^{v\ell}_{s}t^{\ell b}_{s}\right)^{2}
\left(
  \sin^{2}\phi\hat{\mathbf{x}}\hat{\mathbf{x}}
+ \cos^{2}\phi\hat{\mathbf{y}}\hat{\mathbf{y}} 
- 2\sin\phi\cos\phi\hat{\mathbf{x}}\hat{\mathbf{y}}
\right).
\end{align*}
Therefore,
\begin{equation*}
\begin{split}
\mathbf{e}^{2\omega}_{\ell}
\cdot\boldsymbol{\chi}:
\mathbf{e}^{\omega}_{\ell}\mathbf{e}^{\omega}_{\ell} =
T^{v\ell}_{s}T^{\ell b}_{s}\left(t^{v\ell}_{s}t^{\ell b}_{s}\right)^{2}
\big[&
-  \sin^{3}\phi\chi_{xxx}
-  \sin\phi\cos^{2}\phi\chi_{xyy}
+ 2\sin^{2}\phi\cos\phi\chi_{xxy}\\
&+ \sin^{2}\phi\cos\phi\chi_{yxx}
+  \cos^{3}\phi\chi_{yyy}
- 2\sin\phi\cos^{2}\phi\chi_{yxy}
\big]\\
=
T^{v\ell}_{s}T^{\ell b}_{s}\left(t^{v\ell}_{s}t^{\ell b}_{s}\right)^{2}
\big[&
-  \sin^{3}\phi\chi_{xxx}
+ 3\sin\phi\cos^{2}\phi\chi_{xxx}
\big]\\\\
=T^{v\ell}_{s}T^{\ell b}_{s}\left(t^{v\ell}_{s}t^{\ell b}_{s}\right)^{2}
\,\,\,&\chi_{xxx}\sin3\phi
\end{split}
\end{equation*}
Summarizing,
\begin{equation*}
\mathbf{e}^{\,2\omega}_{\ell}\cdot
\boldsymbol{\chi}:\mathbf{e}^\omega_{\ell}\mathbf{e}^\omega_{\ell}
\equiv\Gamma^{\ell}_{sS}\, r^{\ell}_{sS},
\end{equation*}
where
\begin{equation*}
r^{\ell}_{sS} = \chi_{xxx}\sin3\phi,
\end{equation*}
and
\begin{equation*}
\Gamma^{\ell}_{sS}=
T^{v\ell}_{s}T^{\ell b}_{s}\left(t^{v\ell}_{s}t^{\ell b}_{s}\right)^{2}.
\end{equation*} 
In order to reduce above result to that of Ref. \cite{mizrahiJOSA88} and
\cite{sipePRB87}, we take the $2\omega$ radiations factors for vacuum by
taking $\ell=v$, thus $\epsilon_{\ell}(2\omega)=1$, $T^{v\ell}_{s}=1$,
$T^{\ell b}_{s}=T^{vb}_{s}$, and the fundamental field inside medium $b$ by
taking $\ell=b$, thus $\epsilon_{\ell}(\omega)=\epsilon_{b}(\omega)$,
$t^{v\ell}_{s}=t^{vb}_{s}$, and $t^{\ell b}_{s}=1$. With these choices,
\begin{equation*}
r^{b}_{sS} = \chi_{xxx}\sin3\phi,
\end{equation*}
and 
\begin{equation*}
\Gamma^{b}_{sS} = T^{vb}_{s}\left(t^{vb}_{s}\right)^{2}.
\end{equation*} 

\subsection{For the (111) surface}


\subsection{For the (110) surface}


\subsection{For the (001) surface}


%%%%%%%%%%%%%%%%%%%%%%%%%%%%%%%%%%%%%%%%%%%%%%%%%%%%%%%%%%%%%%%%%%%%%%%%%%%%%%%%
%%%%%%%%%%%%%%%%%%%%%%%%%%%%%%%%%%%%%%%%%%%%%%%%%%%%%%%%%%%%%%%%%%%%%%%%%%%%%%%%


\section{Some limiting cases of interest}

\subsection{The two layer model}

In order to reduce above result to that of Ref. \cite{mizrahiJOSA88} and
\cite{sipePRB87}, we now consider that $\mathcal{P}(2\omega)$ is evaluated in
the vacuum region, while the fundamental fields are evaluated in the bulk
region. To do this, we take the $2\omega$ radiations factors for vacuum by
taking $\ell=v$, thus $\epsilon_{\ell}(2\omega)=1$, $T^{\ell v}_{p}=1$, $T^{\ell
b}_{p}=T^{vb}_{p}$, and the fundamental field inside medium $b$ by taking
$\ell=b$, thus $\epsilon_{\ell}(\omega)=\epsilon_{b}(\omega)$,
$t^{v\ell}_{p}=t^{vb}_{p}$, and $t^{\ell b}_{p}=1$. With these choices
\begin{equation*}\label{m800}
\mathbf{e}^{\,2\omega}_{v}\cdot\boldsymbol{\chi}:
\mathbf{e}^\omega_{b}\mathbf{e}^\omega_{b}
\equiv\Gamma^{vb}_{pP}\,r^{vb}_{pP}
,
\end{equation*}
where,
\begin{equation*}\label{m82}
\begin{split}
r^{vb}_{pP}
&= \epsilon_{b}(2\omega)\sin\theta_{0}
\Big(
\sin^2\theta_{0}\chi_{zzz} + k^{2}_{b}\chi_{zxx}
\Big)\\
&- k_{b}K_{b}
\Big(
2\sin\theta_{0}\chi_{xxz} + k_{b}\chi_{xxx}\cos(3\phi) 
\Big) 
,
\end{split}
\end{equation*}
and 
\begin{equation*}\label{m78}
\Gamma^{vb}_{pP}
= \frac{T^{v b}_{p}(t^{vb}_{p})^2}
       {\epsilon_{b}(\omega)\sqrt{\epsilon_{b}(2\omega)}}.
\end{equation*}


\subsection{Taking \texorpdfstring{$\mathcal{P}(2\omega)$}{P(2w)} and the
fundamental fields in the bulk}

To consider the $2\omega$ fields in the bulk, we start with Eq. \eqref{r9} but
substitute $\ell\rightarrow b$, thus
\begin{equation*}
\mathbf{H}_{b}
= \hat{\mathbf{s}}\,T_s^{b v}\left(1+R_{s}^{b b}\right)\hat{\mathbf{s}}
+ \hat{\mathbf{P}}_{v+}T_{p}^{b v}
\left(
\hat{\mathbf{P}}_{b+} + R_{p}^{b b}\hat{\mathbf{P}}_{b-}
\right).
\end{equation*}
$R_{p}^{b b}$ and $R_{s}^{b b}$ are zero, so we are left with
\begin{equation*}
\begin{split}
\mathbf{H}_{b}
&= \hat{\mathbf{s}}\,T_s^{b v}\hat{\mathbf{s}}
 + \hat{\mathbf{P}}_{v+}T_{p}^{b v}\hat{\mathbf{P}}_{b+}\\
&= \frac{K_{b}}{K_{v}}\left(\hat{\mathbf{s}}\,T_s^{vb}\hat{\mathbf{s}}
 + \hat{\mathbf{P}}_{v+}T_{p}^{vb}\hat{\mathbf{P}}_{b+}\right)\\
&= \frac{K_{b}}{K_{v}}
   \left[
   \hat{\mathbf{s}}\,T_s^{vb}\hat{\mathbf{s}}
 + \hat{\mathbf{P}}_{v+}
   \frac{T_{p}^{vb}}{\sqrt{\epsilon_{b}(2\omega)}}
   (\sin\theta_{0}\hat{\mathbf{z}}
 - K_{b}\cos\phi\hat{\mathbf{x}} 
 - K_{b}\sin\phi\hat{\mathbf{y}})
   \right],
\end{split}
\end{equation*}
and we define
\begin{equation*}
\mathbf{e}^{\,2\omega}_{b}
= \frac{K_{b}}{K_{v}}\,\hat{\mathbf{e}}^{\mathrm{out}}\cdot
\left[
   \hat{\mathbf{s}}\,T_s^{vb}\hat{\mathbf{s}}
 + \hat{\mathbf{P}}_{v+}
   \frac{T_{p}^{vb}}{\sqrt{\epsilon_{b}(2\omega)}}
   (\sin\theta_{0}\hat{\mathbf{z}}
 - K_{b}\cos\phi\hat{\mathbf{x}} 
 - K_{b}\sin\phi\hat{\mathbf{y}})
   \right].
\end{equation*}
For $\mathcal{R}_{pP}$, we require
$\hat{\mathbf{e}}^{\mathrm{out}}=\hat{\mathbf{P}}_{v+}$, so we have that
\begin{equation*}
\mathbf{e}^{\,2\omega}_{b}
= \frac{K_{b}}{K_{v}}
  \frac{T_{p}^{vb}}{\sqrt{\epsilon_{b}(2\omega)}}
  (\sin\theta_{0}\hat{\mathbf{z}}
- K_{b}\cos\phi\hat{\mathbf{x}} 
- K_{b}\sin\phi\hat{\mathbf{y}}).
\end{equation*}

The $1\omega$ fields will still be evaluated inside the bulk, so we have Eq.
\eqref{m13}
\begin{equation*}
\mathbf{e}^{\omega}_{b}
= \left[
\hat{\mathbf{s}}t_{s}^{vb}\hat{\mathbf{s}}
+ \frac{t^{vb}_{p}}{\sqrt{\epsilon_{b}(\omega)}}
\left(
  \sin\theta_{0}\hat{\mathbf{z}}
+ k_{b}\cos\phi\hat{\mathbf{x}}
+ k_{b}\sin\phi\hat{\mathbf{y}}
\right) 
\hat{\mathbf{p}}_{v-}
\right]
\cdot\hat{\mathbf{e}}^{\mathrm{in}},  
\end{equation*}
and for our particular case of
$\hat{\mathbf{e}}^{\mathrm{in}}=\hat{\mathbf{p}}_{v-}$,
\begin{equation*}
\mathbf{e}^{\omega}_{b}
= \frac{t^{vb}_{p}}{\sqrt{\epsilon_{b}(\omega)}}
\left(
  \sin\theta_{0}\hat{\mathbf{z}}
+ k_{b}\cos\phi\hat{\mathbf{x}}
+ k_{b}\sin\phi\hat{\mathbf{y}}
\right),
\end{equation*}
and
\begin{equation*}
\begin{split}
\mathbf{e}^{\omega}_{b}\mathbf{e}^{\omega}_{b}
&= \frac{\left(t^{vb}_{p}\right)^{2}}{\epsilon_{b}(\omega)}
\left(
  \sin\theta_{0}\hat{\mathbf{z}}
+ k_{b}\cos\phi\hat{\mathbf{x}}
+ k_{b}\sin\phi\hat{\mathbf{y}}
\right)^{2}\\
&= \frac{\left(t^{vb}_{p}\right)^{2}}{\epsilon_{b}(\omega)}
\big(
  \sin^{2}\theta_{\mathrm{in}}\hat{\mathbf{z}}\hat{\mathbf{z}}
+ k^{2}_{b}\cos^{2}\phi\hat{\mathbf{x}}\hat{\mathbf{x}}
+ k^{2}_{b}\sin^{2}\phi\hat{\mathbf{y}}\hat{\mathbf{y}}\\
&\qquad\qquad
+ 2k_{b}\sin\theta_{0}\cos\phi\hat{\mathbf{z}}\hat{\mathbf{x}}
+ 2k_{b}\sin\theta_{0}\sin\phi\hat{\mathbf{z}}\hat{\mathbf{y}}
+ 2k^{2}_{b}\sin\phi\cos\phi\hat{\mathbf{x}}\hat{\mathbf{y}}
\big)
\end{split}
\end{equation*}

So lastly, we have that
\begin{equation*}
\begin{split}
\mathbf{e}^{\,2\omega}_{b}\cdot
\boldsymbol{\chi}:\mathbf{e}^{\omega}_{b}\mathbf{e}^{\omega}_{b} =
\frac{K_{b}}{K_{v}}
\frac{T_{p}^{vb}\left(t^{vb}_{p}\right)^{2}}
     {\epsilon_{b}(\omega)\sqrt{\epsilon_{b}(2\omega)}}
&\big(
   \sin^{3}\theta_{\mathrm{in}}\chi_{zzz}\\
&+ k^{2}_{b}\sin\theta_{0}\cos^{2}\phi\chi_{zxx}\\
&+ k^{2}_{b}\sin\theta_{0}\sin^{2}\phi\chi_{zyy}\\
&+ 2k_{b}\sin^{2}\theta_{\mathrm{in}}\cos\phi\chi_{zzx}\\
&+ 2k_{b}\sin^{2}\theta_{\mathrm{in}}\sin\phi\chi_{zzy}\\
&+ 2k^{2}_{b}\sin\theta_{0}\sin\phi\cos\phi\chi_{zxy}\\
&- K_{b}\sin^{2}\theta_{\mathrm{in}}\cos\phi\chi_{xzz}\\
&- k^{2}_{b}K_{b}\cos^{3}\phi\chi_{xxx}\\
&- k^{2}_{b}K_{b}\sin^{2}\phi\cos\phi\chi_{xyy}\\
&- 2k_{b}K_{b}\sin\theta_{0}\cos^{2}\phi\chi_{xzx}\\
&- 2k_{b}K_{b}\sin\theta_{0}\sin\phi\cos\phi\chi_{xzy}\\
&- 2k^{2}_{b}K_{b}\sin\phi\cos^{2}\phi\chi_{xxy}\\
&- K_{b}\sin^{2}\theta_{\mathrm{in}}\sin\phi\chi_{yzz}\\
&- k^{2}_{b}K_{b}\sin\phi\cos^{2}\phi\chi_{yxx}\\
&- k^{2}_{b}K_{b}\sin^{3}\phi\chi_{yyy}\\
&- 2k_{b}K_{b}\sin\theta_{0}\sin\phi\cos\phi\chi_{yzx}\\
&- 2k_{b}K_{b}\sin\theta_{0}\sin^{2}\phi\chi_{yzy}\\
&- 2k^{2}_{b}K_{b}\sin^{2}\phi\cos\phi\chi_{yxy}
\big),
\end{split}
\end{equation*}
and we can eliminate many terms since
$\chi_{zzx}=\chi_{zzy}=\chi_{zxy}=\chi_{xzz}=\chi_{xzy}=\chi_{xxy}=\chi_{yzz}
=\chi_{yxx}=\chi_{yyy}=\chi_{yzx}=0$, and substituting the equivalent
components of $\boldsymbol{\chi},$
\begin{equation*}
\begin{split}
=
\frac{K_{b}}{K_{v}}
\Gamma^{b}_{pP}
&\big(
   \sin^{3}\theta_{\mathrm{in}}\chi_{zzz}\\
&+ k^{2}_{b}\sin\theta_{0}\cos^{2}\phi\chi_{zxx}\\
&+ k^{2}_{b}\sin\theta_{0}\sin^{2}\phi\chi_{zxx}\\
&- 2k_{b}K_{b}\sin\theta_{0}\cos^{2}\phi\chi_{xxz}\\
&- 2k_{b}K_{b}\sin\theta_{0}\sin^{2}\phi\chi_{xxz}\\
&- k^{2}_{b}K_{b}\cos^{3}\phi\chi_{xxx}\\
&+ k^{2}_{b}K_{b}\sin^{2}\phi\cos\phi\chi_{xxx}\\
&+ 2k^{2}_{b}K_{b}\sin^{2}\phi\cos\phi\chi_{xxx}
\big),
\end{split}
\end{equation*}
and reducing,
\begin{equation*}
\begin{split}
=
\frac{K_{b}}{K_{v}}
\Gamma^{b}_{pP}
&\big(
   \sin^{3}\theta_{\mathrm{in}}\chi_{zzz}\\
&+ k^{2}_{b}\sin\theta_{0}(\sin^{2}\phi + \cos^{2}\phi)\chi_{zxx}\\
&- 2k_{b}K_{b}\sin\theta_{0}(\sin^{2}\phi + \cos^{2}\phi)\chi_{xxz}\\
&+ k^{2}_{b}K_{b}(3\sin^{2}\phi\cos\phi - \cos^{3}\phi)\chi_{xxx}
\big)\\\\
=
\frac{K_{b}}{K_{v}}
\Gamma^{b}_{pP}
&\big(
  \sin^{3}\theta_{\mathrm{in}}\chi_{zzz} 
+ k^{2}_{b}\sin\theta_{0}\chi_{zxx}
- 2k_{b}K_{b}\sin\theta_{0}\chi_{xxz}
- k^{2}_{b}K_{b}\chi_{xxx}\cos3\phi
\big),
\end{split}
\end{equation*}
where,
\begin{equation*}
\Gamma^{b}_{pP} =
\frac{T_{p}^{vb}\left(t^{vb}_{p}\right)^{2}}
     {\epsilon_{b}(\omega)\sqrt{\epsilon_{b}(2\omega)}}.
\end{equation*}

We find the equivalent expression for $\mathcal{R}$ evaluated inside the bulk
as
\begin{equation*}
R(2\omega) =
\frac{32\pi^{3} \omega^{2}}{c^{3}K^{2}_{b}}
\left\vert
\mathbf{e}^{\,2\omega}_{b}\cdot\boldsymbol{\chi}:
\mathbf{e}^{\omega}_{b}\mathbf{e}^{\omega}_{b}
\right\vert^{2} 
,
\end{equation*}
and we can remove the $K_{b}/K_{v}$ factor completely and reduce to the
standard form of
\begin{equation*}
R(2\omega) =
\frac{32\pi^{3} \omega^{2}}{c^{3}\cos^{2}\theta_{\mathrm{in}}}
\left\vert
\mathbf{e}^{\,2\omega}_{b}\cdot\boldsymbol{\chi}:
\mathbf{e}^{\omega}_{b}\mathbf{e}^{\omega}_{b}
\right\vert^{2}.
\end{equation*}


\subsection{Taking \texorpdfstring{$\mathcal{P}(2\omega)$}{P(2w)} and the
fundamental fields in the vacuum}

To consider the $1\omega$ fields in the vacuum, we start with Eq. \eqref{m2}
but substitute $\ell\rightarrow v$, thus
\begin{equation*}
\mathbf{E}_{v}(\omega)= E_{0}
\left[
  \hat{\mathbf{s}} t^{vv}_s(1+r^{v b}_s)\hat{\mathbf{s}}
+ \hat{\mathbf{p}}_{v-}t^{vv}_{p}\hat{\mathbf{p}}_{v-}
+ \hat{\mathbf{p}}_{v+}t^{vv}_{p}r^{v b}_{p}\hat{\mathbf{p}}_{v-}
\right]
\cdot\hat{\mathbf{e}}^{\mathrm{in}}
,
\end{equation*}
$t_{p}^{vv}$ and $t_{s}^{vv}$ are one, so we are left with
\begin{equation*}
\begin{split}
\mathbf{e}^{\omega}_{v}
&= \left[
  \hat{\mathbf{s}}(1+r^{v b}_s)\hat{\mathbf{s}}
+ \hat{\mathbf{p}}_{v-}\hat{\mathbf{p}}_{v-}
+ \hat{\mathbf{p}}_{v+}r^{v b}_{p}\hat{\mathbf{p}}_{v-}
\right]
\cdot\hat{\mathbf{e}}^{\mathrm{in}}\\
&= \left[
  \hat{\mathbf{s}}(t^{vb}_s)\hat{\mathbf{s}}
+ (\hat{\mathbf{p}}_{v-} + \hat{\mathbf{p}}_{v+}r^{v b}_{p})
  \hat{\mathbf{p}}_{v-}
\right]
\cdot\hat{\mathbf{e}}^{\mathrm{in}}\\
&= \left[
  \hat{\mathbf{s}}(t^{vb}_s)\hat{\mathbf{s}}
+ \frac{1}{\sqrt{\epsilon_{v}(\omega)}}
\big(
  k_{v}(1 - r^{vb}_{p})\hat{\boldsymbol{\kappa}}
+ \sin\theta_{0}(1 + r^{vb}_{p})
\hat{\mathbf{z}}
\big)
  \hat{\mathbf{p}}_{v-}
\right]\\
&= \left[
  \hat{\mathbf{s}}(t^{vb}_s)\hat{\mathbf{s}}
+ \left(
  \frac{k_{b}}{\sqrt{\epsilon_{b}(\omega)}}
  t^{v b}_{p}\hat{\boldsymbol{\kappa}} 
+ \sqrt{\epsilon_{b}(\omega)}\sin\theta_{0}
  t^{v b}_{p}\hat{\mathbf{z}}
  \right)
  \hat{\mathbf{p}}_{v-}
\right]
\cdot\hat{\mathbf{e}}^{\mathrm{in}}\\
&= \left[
  \hat{\mathbf{s}}(t^{vb}_s)\hat{\mathbf{s}}
+ \frac{t^{v b}_{p}}{\sqrt{\epsilon_{b}(\omega)}}\left(
  k_{b}\cos\phi\hat{\mathbf{x}}
+ k_{b}\sin\phi\hat{\mathbf{y}}
+ \epsilon_{b}(\omega)\sin\theta_{0}\hat{\mathbf{z}}
  \right)
  \hat{\mathbf{p}}_{v-}
\right]
\cdot\hat{\mathbf{e}}^{\mathrm{in}}.
\end{split}
\end{equation*}
For $\mathcal{R}_{pP}$ we require that $\hat{\mathbf{e}}^{\mathrm{in}} =
\hat{\mathbf{p}}_{v-}$, so
\begin{equation*}
\mathbf{e}^{\omega}_{v} =
\frac{t^{v b}_{p}}{\sqrt{\epsilon_{b}(\omega)}}
\left(
  k_{b}\cos\phi\hat{\mathbf{x}}
+ k_{b}\sin\phi\hat{\mathbf{y}}
+ \epsilon_{b}(\omega)\sin\theta_{0}\hat{\mathbf{z}}
  \right),
\end{equation*}
and
\begin{equation*}
\begin{split}
\mathbf{e}^{\omega}_{v}\mathbf{e}^{\omega}_{v} =
\left(\frac{t^{v b}_{p}}{\sqrt{\epsilon_{b}(\omega)}}\right)^{2}
&\big[
   k^{2}_{b}\cos^{2}\phi\hat{\mathbf{x}}\hat{\mathbf{x}}\\
&+ k^{2}_{b}\sin^{2}\phi\hat{\mathbf{y}}\hat{\mathbf{y}}\\
&+ \epsilon^{2}_{b}(\omega)\sin^{2}\theta_{\mathrm{in}}
   \hat{\mathbf{z}}\hat{\mathbf{z}}\\
&+ 2k^{2}_{b}\sin\phi\cos\phi\hat{\mathbf{x}}\hat{\mathbf{y}}\\
&+ 2\epsilon_{b}(\omega)k_{b}\sin\theta_{0}\sin\phi
   \hat{\mathbf{y}}\hat{\mathbf{z}}\\
&+ 2\epsilon_{b}(\omega)k_{b}\sin\theta_{0}\cos\phi
   \hat{\mathbf{x}}\hat{\mathbf{z}}
\big].
\end{split}
\end{equation*}

We also require the $2\omega$ fields evaluated in the vacuum, which is Eq.
\eqref{r13},
\begin{equation}
\mathbf{e}^{\,2\omega}_{v} = \hat{\mathbf{e}}^{\mathrm{out}}
\cdot\left[
\hat{\mathbf{s}}T_s^{v b}\hat{\mathbf{s}} + \hat{\mathbf{P}}_{v+}
\frac{T^{v b}_{p}}{\sqrt{\epsilon_{b}(2\omega)}}
\left(
  \epsilon_{b}(2\omega)\sin\theta_{0}\hat{\mathbf{z}}
  - K_{b}\hat{\boldsymbol{\kappa}}
\right) 
\right],
\end{equation}
and with $\hat{\mathbf{e}}^{\mathrm{out}} = \hat{\mathbf{P}}_{v+}$ we have
\begin{equation}
\mathbf{e}^{\,2\omega}_{v} =
\frac{T^{v b}_{p}}{\sqrt{\epsilon_{b}(2\omega)}}
\left(
\epsilon_{b}(2\omega)\sin\theta_{0}\hat{\mathbf{z}}
- K_{b}\cos\phi\hat{\mathbf{x}}
- K_{b}\sin\phi\hat{\mathbf{y}}
\right).
\end{equation}

So lastly, we have that
\begin{equation*}
\begin{split}
\mathbf{e}^{\,2\omega}_{v}\cdot
\boldsymbol{\chi}:\mathbf{e}^{\omega}_{v}\mathbf{e}^{\omega}_{v}
=\qquad\qquad&\\
\frac{T^{v b}_{p}}{\sqrt{\epsilon_{b}(2\omega)}}
\left(\frac{t^{v b}_{p}}{\sqrt{\epsilon_{b}(\omega)}}\right)^{2}
&\big[
    \epsilon_{b}(2\omega)k^{2}_{b}
    \sin\theta_{0}\cos^{2}\phi\chi_{zxx}\\
&+  \epsilon_{b}(2\omega)k^{2}_{b}
    \sin\theta_{0}\sin^{2}\phi\chi_{zyy}\\
&+  \epsilon^{2}_{b}(\omega)\epsilon_{b}(2\omega)
    \sin^{3}\theta_{\mathrm{in}}\chi_{zzz}\\
&+ 2\epsilon_{b}(2\omega)k^{2}_{b}\sin\theta_{0}
    \sin\phi\cos\phi\chi_{zxy}\\
&+ 2\epsilon_{b}(\omega)\epsilon_{b}(2\omega)k_{b}
    \sin^{2}\theta_{\mathrm{in}}\sin\phi\chi_{zyz}\\
&+ 2\epsilon_{b}(\omega)\epsilon_{b}(2\omega)k_{b}
    \sin^{2}\theta_{\mathrm{in}}\cos\phi\chi_{zxz}\\
&-  k^{2}_{b}K_{b}\cos^{3}\phi\chi_{xxx}\\
&-  k^{2}_{b}K_{b}\sin^{2}\phi\cos\phi\chi_{xyy}\\
&-  \epsilon^{2}_{b}(\omega)K_{b}
    \sin^{2}\theta_{\mathrm{in}}\cos\phi\chi_{xzz}\\
&- 2k^{2}_{b}K_{b}\sin\phi\cos^{2}\phi\chi_{xxy}\\
&- 2\epsilon_{b}(\omega)k_{b}K_{b}
    \sin\theta_{0}\sin\phi\cos\phi\chi_{xyz}\\
&- 2\epsilon_{b}(\omega)k_{b}K_{b}
    \sin\theta_{0}\cos^{2}\phi\chi_{xxz}\\
&-  k^{2}_{b}K_{b}\sin\phi\cos^{2}\phi\chi_{yxx}\\
&-  k^{2}_{b}K_{b}\sin^{3}\phi\chi{yyy}\\
&-  \epsilon^{2}_{b}(\omega)K_{b}
    \sin^{2}\theta_{\mathrm{in}}\sin\phi\chi_{yzz}\\
&- 2k^{2}_{b}K_{b}\sin^{2}\phi\cos\phi\chi_{yxy}\\
&- 2\epsilon_{b}(\omega)k_{b}K_{b}
    \sin\theta_{0}\sin^{2}\phi\chi_{yyz}\\
&- 2\epsilon_{b}(\omega)k_{b}K_{b}
    \sin\theta_{0}\sin\phi\cos\phi\chi_{yxz}
\big],
\end{split}
\end{equation*}
and after eliminating components,
\begin{equation*}
\begin{split}
= \Gamma^{v}_{pP}&\big[
    \epsilon^{2}_{b}(\omega)\epsilon_{b}
    (2\omega)\sin^{3}\theta_{\mathrm{in}}\chi_{zzz}\\
&+  \epsilon_{b}(2\omega)k^{2}_{b}
    \sin\theta_{0}\cos^{2}\phi\chi_{zxx}\\
&+  \epsilon_{b}(2\omega)k^{2}_{b}
    \sin\theta_{0}\sin^{2}\phi\chi_{zxx}\\
&- 2\epsilon_{b}(\omega)k_{b}K_{b}
    \sin\theta_{0}\cos^{2}\phi\chi_{xxz}\\
&- 2\epsilon_{b}(\omega)k_{b}K_{b}
    \sin\theta_{0}\sin^{2}\phi\chi_{xxz}\\
&+ 3k^{2}_{b}K_{b}\sin^{2}\phi\cos\phi\chi_{xxx}\\
&-  k^{2}_{b}K_{b}\cos^{3}\phi\chi_{xxx}
\big]\\\\
= \Gamma^{v}_{pP}&\big[
    \epsilon^{2}_{b}(\omega)\epsilon_{b}(2\omega)
    \sin^{3}\theta_{\mathrm{in}}\chi_{zzz}
 +  \epsilon_{b}(2\omega)k^{2}_{b}\sin\theta_{0}\chi_{zxx}\\
&- 2\epsilon_{b}(\omega)k_{b}K_{b}\sin\theta_{0}\chi_{xxz}
 -  k^{2}_{b}K_{b}\chi_{xxx}\cos3\phi
\big],
\end{split}
\end{equation*}
where
\begin{equation*}
\Gamma^{v}_{pP} =
\frac{T^{v b}_{p}\left(t^{v b}_{p}\right)^{2}}
     {\epsilon_{b}(\omega)\sqrt{\epsilon_{b}(2\omega)}}.
\end{equation*}


\subsection{Taking \texorpdfstring{$\mathcal{P}(2\omega)$}{P(2w)} in
\texorpdfstring{$\ell$}{l} and the fundamental fields in the bulk}

For this scenario with $\hat{\mathbf{e}}^{\mathrm{in}}=\hat{\mathbf{p}}_{v-}$
and $\hat{\mathbf{e}}^{\mathrm{out}}=\hat{\mathbf{P}}_{v+}$, we obtain from Eq.
\eqref{r12},
\begin{equation*}\label{ri12}
\mathbf{e}^{2\omega}_{\ell} =
\frac{T^{v\ell}_{p}T^{\ell b}_{p}}
     {\epsilon_{\ell}({2\omega})\sqrt{\epsilon_{b}(2\omega)}}
\left(
  \epsilon_{b}(2\omega)\sin\theta_{0}\hat{\mathbf{z}}
- \epsilon_{\ell}(2\omega)K_{b}\cos\phi\hat{\mathbf{x}}
- \epsilon_{\ell}(2\omega)K_{b}\sin\phi\hat{\mathbf{y}}
\right),
\end{equation*}
and Eq. \eqref{m13},
\begin{equation*}
\begin{split}
\mathbf{e}^{\omega}_{b}\mathbf{e}^{\omega}_{b}
&= \frac{\left(t^{vb}_{p}\right)^{2}}{\epsilon_{b}(\omega)}
\big(
  \sin^{2}\theta_{\mathrm{in}}\hat{\mathbf{z}}\hat{\mathbf{z}}
+ k^{2}_{b}\cos^{2}\phi\hat{\mathbf{x}}\hat{\mathbf{x}}
+ k^{2}_{b}\sin^{2}\phi\hat{\mathbf{y}}\hat{\mathbf{y}}\\
&\qquad\qquad
+ 2k_{b}\sin\theta_{0}\cos\phi\hat{\mathbf{z}}\hat{\mathbf{x}}
+ 2k_{b}\sin\theta_{0}\sin\phi\hat{\mathbf{z}}\hat{\mathbf{y}}
+ 2k^{2}_{b}\sin\phi\cos\phi\hat{\mathbf{x}}\hat{\mathbf{y}}
\big).
\end{split}
\end{equation*}
Thus,
\begin{equation*}
\begin{split}
\mathbf{e}^{\,2\omega}_{\ell}\cdot
\boldsymbol{\chi}:\mathbf{e}^{\omega}_{b}\mathbf{e}^{\omega}_{b} = 
\frac{T^{v\ell}_{p}T^{\ell b}_{p}\left(t^{vb}_{p}\right)^{2}}
     {\epsilon_{\ell}({2\omega})\epsilon_{b}(\omega)\sqrt{\epsilon_{b}(2\omega)}}
\bigg[
&+ \epsilon_{b}(2\omega)\sin^{3}\theta_{\mathrm{in}}\chi_{zzz}\\
&+ \epsilon_{b}(2\omega)k^{2}_{b}\sin\theta_{0}\cos^{2}\phi\chi_{zxx}\\
&+ \epsilon_{b}(2\omega)k^{2}_{b}\sin\theta_{0}\sin^{2}\phi\chi_{zyy}\\
&+ 2\epsilon_{b}(2\omega)k_{b}\sin^{2}\theta_{\mathrm{in}}\cos\phi\chi_{zzx}\\
&+ 2\epsilon_{b}(2\omega)k_{b}\sin^{2}\theta_{\mathrm{in}}\sin\phi\chi_{zzy}\\
&+ 2\epsilon_{b}(2\omega)k^{2}_{b}\sin\theta_{0}\sin\phi\cos\phi\chi_{zxy}\\
%%%%%%%%%%%%%%%%%%%%%%%%%%%%%%%%%%%%%%%%%%%%%%%%%%%%%%%%%%%%%%%%%%%
&- \epsilon_{\ell}(2\omega)\sin^{2}\theta_{\mathrm{in}}K_{b}\cos\phi\chi_{xzz}\\
&- \epsilon_{\ell}(2\omega)k^{2}_{b}K_{b}\cos^{3}\phi\chi_{xxx}\\
&- \epsilon_{\ell}(2\omega)k^{2}_{b}K_{b}\sin^{2}\phi\cos\phi\chi_{xyy}\\
&- 2\epsilon_{\ell}(2\omega)k_{b}K_{b}\sin\theta_{0}\cos^{2}\phi\chi_{xzx}\\
&- 2\epsilon_{\ell}(2\omega)k_{b}K_{b}\sin\theta_{0}\sin\phi\cos\phi\chi_{xzy}\\
&- 2\epsilon_{\ell}(2\omega)k^{2}_{b}K_{b}\sin\phi\cos^{2}\phi\chi_{xxy}\\
%%%%%%%%%%%%%%%%%%%%%%%%%%%%%%%%%%%%%%%%%%%%%%%%%%%%%%%%%%%%%%%%%%%
&- \epsilon_{\ell}(2\omega)K_{b}\sin^{2}\theta_{\mathrm{in}}\sin\phi\chi_{yzz}\\
&- \epsilon_{\ell}(2\omega)k^{2}_{b}K_{b}\cos^{2}\phi\sin\phi\chi_{yxx}\\
&- \epsilon_{\ell}(2\omega)k^{2}_{b}K_{b}\sin^{3}\phi\chi_{yyy}\\
&- 2\epsilon_{\ell}(2\omega)k_{b}K_{b}\sin\theta_{0}\cos\phi\sin\phi\chi_{yzx}\\
&- 2\epsilon_{\ell}(2\omega)k_{b}K_{b}\sin\theta_{0}\sin^{2}\phi\chi_{yzy}\\
&- 2\epsilon_{\ell}(2\omega)k^{2}_{b}K_{b}\sin^{2}\phi\cos\phi\chi_{yxy}
\bigg].
\end{split}
\end{equation*}
We eliminate and replace components,
\begin{equation*}
\begin{split}
\mathbf{e}^{\,2\omega}_{\ell}\cdot
\boldsymbol{\chi}:\mathbf{e}^{\omega}_{b}\mathbf{e}^{\omega}_{b} = 
\Gamma^{\ell b}_{pP}
\bigg[
&+ \epsilon_{b}(2\omega)\sin^{3}\theta_{\mathrm{in}}\chi_{zzz}\\
&+ \epsilon_{b}(2\omega)k^{2}_{b}\sin\theta_{0}\cos^{2}\phi\chi_{zxx}\\
&+ \epsilon_{b}(2\omega)k^{2}_{b}\sin\theta_{0}\sin^{2}\phi\chi_{zxx}\\
&- 2\epsilon_{\ell}(2\omega)k_{b}K_{b}\sin\theta_{0}\cos^{2}\phi\chi_{xxz}\\
&- 2\epsilon_{\ell}(2\omega)k_{b}K_{b}\sin\theta_{0}\sin^{2}\phi\chi_{xxz}\\
&- \epsilon_{\ell}(2\omega)k^{2}_{b}K_{b}\cos^{3}\phi\chi_{xxx}\\
&+ \epsilon_{\ell}(2\omega)k^{2}_{b}K_{b}\sin^{2}\phi\cos\phi\chi_{xxx}\\
&+ 2\epsilon_{\ell}(2\omega)k^{2}_{b}K_{b}\sin^{2}\phi\cos\phi\chi_{xxx}
\bigg],
\end{split}
\end{equation*}
so lastly
\begin{equation*}
\begin{split}
\mathbf{e}^{\,2\omega}_{\ell}\cdot
\boldsymbol{\chi}:\mathbf{e}^{\omega}_{b}\mathbf{e}^{\omega}_{b} = 
\Gamma^{\ell b}_{pP}&
\bigg[
  \epsilon_{b}(2\omega)\sin^{3}\theta_{\mathrm{in}}\chi_{zzz}
+ \epsilon_{b}(2\omega)k^{2}_{b}\sin\theta_{0}\chi_{zxx}\\
&- 2\epsilon_{\ell}(2\omega)k_{b}K_{b}\sin\theta_{0}\chi_{xxz}
- \epsilon_{\ell}(2\omega)k^{2}_{b}K_{b}\chi_{xxx}\cos3\phi
\bigg],
\end{split}
\end{equation*}
where
\begin{equation*}
\Gamma^{\ell b}_{pP}=
\frac{T^{v\ell}_{p}T^{\ell b}_{p}\left(t^{vb}_{p}\right)^{2}}
  {\epsilon_{\ell}({2\omega})\epsilon_{b}(\omega)\sqrt{\epsilon_{b}(2\omega)}}.
\end{equation*}
